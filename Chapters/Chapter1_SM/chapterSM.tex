%\dropchapter{0.4in}
\chapter{title} \label{chp:SM}
%\epigraphhead[70]{\epigraph{\textit{If I could remember the names of all these particles, I'd be a botanist.}}{Enrico Fermi}}
%\undodrop

%The ultimate goal of any particle physics is to \\
%Within experimental particle physics, the ultimate goal is to understand \\
Ever since the beginning of physics, and especially of particle physics, the ultimate goal is to progress and move forward in the understanding of the particles and their interactions observed around us. 
In order to achieve a new breakthrough on the level of elementary particle physics our current knowledge should continously be questioned and no detail, no matter how small, should be overlooked.
%** no stone should be left unturned **.
In this perspective the Standard Model of elementary particle physics should be seen as a first step towards a grand unification of all fundamental interactions which has met every test endured up to now.
%endured is ``tested to the limit'' time after time, but for the moment has met every test.
\\
\textit{Now shortly summarize what will be discussed in this chapter!}

\section{Standard Model}
The Standard Model of elementary particle physics (SM) is a theoretical framework designed in 1978 which contains three of the four fundamental interactions.

\subsection{Theory framework}
Try to explain the framework starting from the theory (the particles should follow automatically ... \textit{But maybe you first need to mention which are these fundamental interactions and building blocks ...})

\subsection{Particle content}

\subsection{Successes and shortcomings of the Standard Model}

\section{Top-quark physics}

\subsection{Production and decay}

\subsection{Anomalous couplings in the top-quark sector}