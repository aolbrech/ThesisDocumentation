%\dropchapter{0.4in}
\chapter{Anomalous couplings in the top quark sector} \label{chp::SM}
%\epigraphhead[70]{\epigraph{\textit{If I could remember the names of all these particles, I'd be a botanist.}}{Enrico Fermi}}
%\undodrop

%The ultimate goal of any particle physics is to \\
%Within experimental particle physics, the ultimate goal is to understand \\
(Ever since the beginning of physics, and especially of particle physics, the ultimate goal is to progress and move forward in the understanding of the particles and their interactions observed around us. )
In order to achieve a new breakthrough on the level of elementary particle physics our current knowledge should continously be questioned and no detail, no matter how small, should be overlooked.
%** no stone should be left unturned **.
In this perspective the Standard Model of elementary particle physics should be seen as a first step towards a grand unification of all fundamental interactions which has met every test endured up to now.
%endured is ``tested to the limit'' time after time, but for the moment has met every test.
\\
\textit{Now shortly summarize what will be discussed in this chapter!}\\
\textbf{More focus ont he role of top in SM searches!}

\section{Standard Model of elementary particle physics}
The Standard Model of elementary particle physics (SM) is a theoretical framework designed in 1978 which contains three of the four fundamental interactions.

\subsection{Particle content}

The extensive search for elementary particles and the chain of discoveries during the 20$^{th}$ century continously altered the understanding of the fundamental theory describing them.
%It almost appears that every new particle discovery resulted in a complete destruction of the existing theory assumptions.
Every new discovery divided the physics community and often required the development of a completely new framework capable of describing the observations.
Hence the Standard Model, officially established in the early 1970s, actually consists of many ingenious contributions from many renown physicists~\cite{MandlAndShaw, PeskinAndSchroeder, Paschos:2007pi}. 
For the last decades the general belief on elementary particles is conceived rather stable, especially since every new discovery validated the theory described by the Standard Model. % and summarized in Table~\ref{table::ElemParticles}.
\\

Within the Standard Model elementary particles are categorized based on their spin. Fermions, consisting of both leptons and quarks, have half-integer spin while bosons, also called force mediators, have integer spin. 
The collection of fermions can be stored into three separate generations, characterized by increasing mass, and is summarized in Table~\ref{table::ElemParticles}.  Each fermion $f$ has an antiparticle, which is defined to have identical mass but opposite electrical charge and is generally denoted as $\bar{f}$. 
Only for the charged leptons, $l^{-}$, the notation $l^{+}$ is used for their respective antiparticle.
\\
Even though the Standard Model consists of three fermion generations the first one is sufficient to describe all stable matter around us.
This because a single atom consists of an electron circulating around a proton-neutron nucleus, which are bound states of up- and down-quarks with respective quark-content $uud$ and $udd$. From such an atom every known chemical element can be formed. 
%This because the protons and neutrons are a bound state of up- and down-quarks, respectively with quark-content $uud$ and $udd$. Together with the electron circulating around the proton-neutron nucleus this gives rise to a simple atom from which every known chemical element can be formed. 
\setlength\extrarowheight{5pt}
\begin{table}[h!t]
 \centering
 \caption{Overview of the fermions in the Standard Model and their corresponding electrical charge.} \label{table::ElemParticles}
 \begin{tabular}{|c|cr|cc|cc|cc|c|}
  \hline
  \textbf{Generation} 		& \multicolumn{4}{c|}{\textbf{Quarks}} 				& \multicolumn{4}{c|}{\textbf{Leptons}} 				\\
  \hline
  1$^{st}$ 			& up 		& $u$ 		& down 		& $d$ 		& electron neutrino	& $\nu_{e}$ 	& electron	& $e^{-}$ 	\\
  \hline
  2$^{nd}$ 			& charm 	& $c$ 		& strange 	& $s$		& muon neutrino		& $\nu_{\mu}$ 	& muon		& $\mu^{-}$ 	\\
  \hline
  3$^{rd}$ 			& top		& $t$ 		& bottom 	& $b$ 		& tau neutrino 		& $\nu_{\tau}$ 	& tau		& $\tau^{-}$ 	\\
  \hline
  \hline
  \textbf{Electrical charge} 	& \multicolumn{2}{c|}{+2/3} 	& \multicolumn{2}{c|}{-1/3} 	& \multicolumn{2}{c|}{0} 		& \multicolumn{2}{c|}{1}	\\
  \hline
 \end{tabular}
\end{table}

Separating fermions into leptons and quarks is motivated by the different fundamental forces they interact with.
The Standard Model comprises three of the four fundamental interactions: the electromagnetic force responsible for ..., the weak force used for describing ... and finally the strong interaction ... .
The only missing piece of the puzzle is gravity which is unfortunately not yet included.

The leptons only interact through the weak and electromagnetic force although the neutral ones, the neutrinos, obviously are not influenced by the latter one. The strong interaction only interacts with the quarks.
The fundamental forces described by the Standard Model are each represented by a spin-1 boson or force mediator.
For the weak interaction this is a massive boson, as can be seen from Table~\ref{table::ForceCarriers}.
\\
The number of force mediators is different for each interaction: a single one for the electromagnetic interaction, three for the weak and even eight for the strong one. These numbers follow from the charge they exchange, which is the called colour charge in the case of the strong interaction. Since each quark occurs in three different colours (red, green and blue) eight different interaction combinations are possible.
%bosons for each force is allowed to vary since the electromagnetic one is provided by one single photon, the weak force by three massive bosons and the strong force even has 8 gluons. 
%The only difference between the different gluons is the colour charge they carry and which is exchanged with the quarks.
%Within the Standard Model 11 of these force mediators exist, of which 8 are gluons with only a different color charge. \textbf{BETTER!!}

\begin{table}[h!t]
 \centering
 \caption{Overview of the spin-1 force-carriers in the Standard Model and their mass~\cite{WMass,ZMass}.} \label{table::ForceCarriers}
 \begin{tabular}{|c|cc|c|}%c|}
  \hline
  \textbf{Force} 		&\multicolumn{2}{c|}{\textbf{Boson}} 	& \textbf{Mass ($\GeV$)}	\\%& \textbf{Spin}	\\
  \hline
  Strong force 			& gluon 	& g 			& 0 				\\%& 1		\\
  \hline
  Electromagnetic force		& photon 	& $\gamma$ 		& 0 				\\%& 1 		\\
  \hline
  \multirow{2}{*}{Weak force} 	& W-boson 	& W$^{\pm}$ 		& $\pm$ 			\\%& 1 		\\
				& Z-boson 	& Z$^{0}$ 		& $\pm$ 			\\%& 1 		\\
  \hline
 \end{tabular}
\end{table}

A final, but definitely not less important, boson which is incorporated in the Standard Model is the spin-0 Brout-Englert-Higgs (BEH) boson. This particle is responsible for providing mass to all other particles through the mechanism of electroweak symmetry breaking, as will be explained in Section~\ref{sec::EWSB}. Its existence was postulated in 1964 but was only discovered rather recently in 2012~\cite{HiggsDiscCMS, HiggsDiscAtlas}.

\subsection{Interactions through gauge invariance}
The Standard Model is much more than a mere collection of elementary particles, its theoretical framework is that of a relativistic quantum field theory.
With this purely mathematical description, based on gauge invariance under each of the three included forces, the fermion interactions follow automatically. 
\\
This statement will now be illustrated for invariance under a general local gauge transformation. Since the fermions are half-integer spin particles they can be represented by a Dirac spinor field:
%Hence the particles and their interactions can be described in a purely mathematical manner and is based on gauge invariance under each of the three included forces.
%The most powerful aspect of the Standard Model is that it is able to describe the interactions of the particles as a relativistic quantum field theory. The basic property on which it is based is gauge invariance under each of the three included interactions. From this the interactions between the different particles follows automatically (\textbf{or only the case for fermions?}).
%Since the fermions are half integer spin particles they are represented by a Dirac spinor field which is described by the Dirac Lagrangian:
\begin{equation} \label{eq::DiracL}
 \mathcal{L}_{Dirac} = i \bar{\psi} \gamma^{\mu} \partial_{\mu} \psi - m \bar{\psi} \psi
\end{equation}
The imposed local gauge invariance requires the fermion fields, and the overall Lagrangian, to be invariant under the following general transformation:
%The gauge invariance requires the fields to be invariant under the corresponding transformation
\begin{equation} \label{eq::GaugeTransf}
 \psi \rightarrow U(x) \psi =  \exp \left( -i \vec{\alpha}(x) \cdot \frac{\vec{\tau}}{2} \right) \psi
\end{equation}
where $\vec{\alpha}$ are the rotation parameters in the symmetry group represented by the Lie group generators $\vec{\tau}$. The fact that these rotation parameters depend on $x$ is crucial and emphasizes that it is a local gauge invariance and not just a global. It is just the invariance under these local gauge transformation which introduces the fermion interactions.
\\
\\
Invariance of the Dirac Lagrangian under the transformation given in Equation (\ref{eq::GaugeTransf}) can only be accomplished by replacing the partial derivative $\partial_{\mu}$ by a covariant derivative $D_{\mu}$. This however comes at the price of introducing new gauge fields $A_{\mu}$ which will interact with the fermion fields with coupling strenght $g$.
% which transforms the same way as the matter field $\psi$. 
\begin{equation} \label{eq::CovDer}
 D_{\mu} = \partial_{\mu} -i g \vec{A}_{\mu} \cdot \frac{\vec{\tau}}{2}
\end{equation}
Inserting this covariant derivative results in an additional term in the Dirac Lagrangian, which describes the interaction between the fermion fields $\psi$ mediated by the gauge fields $A_{\mu}$. 
Since the covariant derivative should transform under the gauge transformation as the fermion fields, the local changes are incorporated by this vector field.
\begin{equation} \label{eq::DiracLInter}
 \mathcal{L}_{Dirac} = i \bar{\psi} \gamma^{\mu} \partial_{\mu} \psi - m \bar{\psi} \psi + g \bar{\psi} \gamma^{\mu} \psi \vec{A}_{\mu} \cdot \frac{\vec{\tau}}{2}
\end{equation}
%Requiring that the covariant derivative transforms in the same way as the matter fields $\psi$ in order to ensure the Lagrangian to remain invariant under the considered gauge transformation, this new vector field should incorporate the local changes and transform in the following way (\textbf{is this not general enough? why using the matrix U?}):
%\begin{equation}
% A_{\mu}^{'} =  A_{\mu} - \frac{1}{g} \partial_{\mu} (\vec{\alpha}\cdot\frac{\vec{\tau}}{2})
%\end{equation}
\paragraph{Remark: } Is this $\cdot$ and vector arrow always necessary??

\subsubsection{Elementary fermion interactions in the Standard Model}
In the above explanation of gauge invariance the introduced matrix $U(x)$ has been defined in order to represent the symmetry group SU(N). 
This procedure can however easily be simplified in order to obtain the three gauge interactions of the Standard Model, which will each introduce a number of vector fields describing the interactions between the fermions.
%The theory of gauge invariance has been explained in a general way with the introduced matrix $U(x)$ being the most general rotation matrix of the symmetry group SU(N). This can however easily be simplified in order to obtain the three gauge interactions for which the Standard Model is invariant, which each introduce a number of vector fields responsible for the interactions between the fermions. 

\begin{myindentpar}
  \begin{description}
    \item[Quantum chromodynamics gauge transformations] \hfill \\
    As mentioned before the strong interaction is represented by the quantum number colour and thus each quark has three equivalent states. Therefore the fermion fields should be seen as a three-component column vector implying that the symmetry group for quantum chromodynamics (QCD) is SU(3) with eight gauge fields $G_{\mu}^{a}$. 
    %This explains the existence of 8 gluons, which are introduced as the gauge fields $G_{\mu}^{a}$ in order for the Lagrangian to remain invariant under the gauge transformations. 
    The generators $\tau$ in Equation (\ref{eq::GaugeTransf}) are in this case the Gell-Mann matrices $\lambda_{i}^{a}$. As a result the covariant derivative of the strong interaction takes the form:
    \begin{equation}
      D_{\mu} = \partial_{\mu} - i g_{S} \frac{\lambda^{a}}{2} G_{\mu}^a
    \end{equation}
    where $g_{S}$ is the coupling constant of the strong interaction. \\
    The three-component or triplet representation is only valid for particles carrying colour charge, otherwise they should be represented as singlets in $SU(3)_{C}$. 
    %Hence only the triplets will be able to interact by exchanging colour.
    
    \item[Electroweak gauge theory] \hfill \\
    The electroweak interaction combines the electromagnetic and weak theory and should be able to explain the parity violation observed in the weak interaction. The smallest group capable of doing so is $SU(2)_{L} \times U(1)_{Y}$ where the subscript $L$ stands for left-handed\footnote{
      Left-handed and right-handed fermions can be distinghuished using the left-handed and right-handed operator $P_{L,R}$ = $(1 - \gamma_{5})$ with $\gamma_5$ defined as the fifth gamma matrix ($\gamma_5$ = i$\gamma_0 \gamma_1 \gamma_2 \gamma_3$). 
    }
    and $Y$ for the weak hypercharge.
    The overall covariant derivative which should be used for the electroweak interaction is thus:
    \begin{equation}
     D_{\mu} = \partial_{\mu} - i g \frac{\tau}{2} W_{\mu}^{i} - i g^{'} \frac{Y}{2} B_{\mu}
    \end{equation}
    where $g$ and $g^{'}$ are the respective coupling strengths of the weak and electromagnetic interaction and $\tau_{i}$ the Pauli matrices. 
    
    This gauge invariances introduces a total of four gauge fields, three from the $SU(2)_L$ transformations and one from the $U(1)_Y$ one.
    However these gauge fields are not directly identifiable as the electromagnetic photon $A_{\mu}$ and the weak vector bosons, $W^{\pm}$ and $Z^{0}$.
    %the gauge bosons observed for the electromagnetic interaction, , and the weak interaction, the  bosons. 
    These gauge bosons are linear combinations of the four introduced gauge fields in the following way: %as is shown in Equation (\ref{eq::EWGaugeBosons}).
    \begin{eqnarray}
     A_{\mu} & = & W_{\mu}^{3} \sin \theta_{W} + B_{\mu} \cos \theta_{W} \nonumber \\
     W_{\mu}^{\pm} & = & \frac{1}{\sqrt{2}} \left( W_{\mu}^{1} \mp i W_{\mu}^{2} \right) \label{eq::EWGaugeBosons} \\
     Z_{\mu} & = & W_{\mu}^{3} \cos \theta_{W} - B_{\mu} \sin \theta_{W} \nonumber
    \end{eqnarray}
    The angle $\theta_{W}$ used in these equations is the weak mixing or Weinberg angle, defined as:
    \begin{equation}
     \tan \theta_{W} = \frac{g^{'}}{g}
    \end{equation}
    %The structure of this symmetry group implies that 
    Only the left-handed fermions can be represented as a doublet in SU(2) while all other fermions are mere singlets and therefore do not interact with the gauge fields $W_{\mu}^{i}$. 
   \end{description}
\end{myindentpar}

An important property of the introduced gauge fields follows from the fact whether the underlying gauge group is abelian or non-abelian. Only in the latter case self-interactions among the gauge fields themselves are allowed, as is the case for the gluons and the three vector bosons. The photon on the other hand is not able to have any self-interactions. (\textbf{Only the gauge fields can commute, not the related bosons (so this would mean that ZZ is also not possible since it contains a $B_{\mu}$ in the relation .. ?)})

\subsubsection{Electroweak symmetry breaking} \label{sec::EWSB}
The mathematical framework of gauge invariance explains in detail the interactions of the fermions and bosons, their mass acquirement however remains a big mystery. Simply introducing a bosonic mass term of the form $m^{2} A_{\mu}A^{\mu}$ would violate gauge invariance.
The same even holds for a fermionic mass term, $m_{f} \psi \psi$, which would violate the SU(2)$\times$U(1) symmetry because of the different transformation rules for right- and left-handed fermions.
\\
Nevertheless observations of massive fermions and bosons indicate that the Standard Model, in order to remain thrustworthy, should be expanded in a way to accommodate mass terms for both the fermions and bosons.
%Hence at first sight it appears that the theoretical framework of the Standard Model only allows for massless fermions and bosons, which is in contradiction with the observation of the weak interaction's massive vector bosons. % of the weak interaction.
A solution is given by the principle of spontaneous symmetry breaking, known as the Brout-Englert-Higgs (BEH) mechanism~\cite{Englert, Higgs, Kibble}, as postulated in 1964.
It introduces a single scalar doublet which leaves the Lagrangian invariant but breaks the ground state of the vacuum.
%The gauge field in Equation (\ref{eq::DiracLInter}) is allowed to have a kinetic term however the introduction of a mass term of the form $m^{2} A_{\mu}A^{\mu}$ would violate gauge invariance. Hence the gauge bosons are required to be massless, which is in contradiction with the known massive electroweak vector bosons $W^{\pm}$ and $Z^0$. Additionally for the electroweak interaction the different behaviour of the right-handed and left-handed fermions implies that the fermion mass term, $m_{f} \psi \psi$, violates the SU(2)$\times$U(1) gauge invariance. Therefore mechanism should be developed which gives mass to both the massive gauge bosons and the fermions.
%\\
%Within the Standard Model this mechanism, denoted the Brout-Englert-Higgs (BEH) mechanism, is based on spontaneous symmetry breaking of SU(2)$\times$U(1). It has been developed in 1964 and introduces a single scalar doublet which leaves the Lagrangian invariant but breaks the ground state of the vacuum.
\begin{equation}
 \phi = \begin{pmatrix}
            \phi^{+} \\
            \phi^{0}
           \end{pmatrix}
\end{equation}
The Lagrangian of this BEH field can take the following gauge-invariant terms:
%For this doublet or Higgs field with non-zero hypercharge (\textbf{Mention something more about this U(1) part?}) the following terms can be added to the Lagrangian without violating gauge invariance: 
\begin{eqnarray} \label{eq::HiggsL}
 \mathcal{L}_{BEH} & = & (D^{\mu} \phi)^{\dagger}(D_{\mu} \phi) - V(\phi) \nonumber \\
                   & = & (D^{\mu} \phi)^{\dagger}(D_{\mu} \phi) - \mu^{2} (\phi^{\dagger} \phi) - \lambda (\phi^{\dagger} \phi)^{2}
\end{eqnarray}
where $\mu^{2}$ and $\lambda$ ($>$ 0) are two real values representing a mass parameter and the scalar's self-interaction strenght, respectively.
\\
In case the mass parameter is positive the potential only has the trivial minimum at $\phi$ = 0 and Equation (\ref{eq::HiggsL}) simply describes a massive scalar particle with mass $\mu$ and quartic coupling strenght $\lambda$. However if the mass parameter is negative the situation is much less trivial since a non-unique vacuum state is retrieved for the potential resulting in spontaneous symmetry breaking once a vacuum expectation value is chosen.
\begin{equation}
 \left< \phi^{\dagger} \phi \right> = v^{2} = \frac{\vert \mu^{2} \vert}{\lambda} \qquad \textrm{\textbf{Correct?? (different in each book)}}
\end{equation}
%The selected vacuum is chosen to be neutral and only has one scalar real field H(x) remaining defined as the BEH field. 
In order to study the particle spectrum in the theory small perturbations around this minimum should be considered:
%which is neutral, hence leaving U(1)$_{EM}$ invariant. 
%A particular vacuum is then chosen (is this the same as the unitary gauge or still something different?) and an expansion about this minimum is performed:
\begin{equation}
 \phi_{0} = \frac{1}{\sqrt{2}}\begin{pmatrix}
             0 \\
             v + H(x)
            \end{pmatrix}
\end{equation}
%Implementing the covariant derivative of Equation (\ref{eq::CovDer}) in the BEH Lagrangian and evaluating at the scalar field vacuum expectation value indicates that the three dissapearing scalar fields of $\phi$ transform three originally massless vector fields into massive ones, corresponding to the intermediate vector bosons of the weak interaction.
From the four original fields of the scalar doublet only one remains: the BEH field H. The three other real fields have been ``eaten'' by the massless vector fields of the weak interaction making them massive.
The BEH boson $H^{0}$, originating from the BEH field, itself acquires a mass $m_{H}$ = $\sqrt{2 \lambda}v$ while the photon remains massless. The mass of the three vector bosons is given by: %can easily be determined from the following relation:
%gives the mass term for the three vector bosons of the weak interaction and keeps the photon massless. The BEH field itself is associated to a new boson, the  The masses of the three vector bosons are related as can be seen from Equation (\ref{eq::VectorBosonMasses}).
\begin{equation}\label{eq::VectorBosonMasses}
 M_W = \frac{1}{2} v g \qquad \qquad M_Z = \frac{1}{2} v \sqrt{g^2 + g^{'2}}
\end{equation}
%The different weak-isospin doublets are:
%\begin{equation}
% \begin{pmatrix} u \\ d \end{pmatrix}_{L}~, \qquad \begin{pmatrix} c \\ s \end{pmatrix}_{L}~, \qquad \begin{pmatrix} t \\ b \end{pmatrix}_{L}
%\end{equation}
%and:
%\begin{equation}
% \begin{pmatrix} \nu_{e} \\ e^{-} \end{pmatrix}_{L}~, \qquad \begin{pmatrix} \nu_{\mu} \\ \mu^{-} \end{pmatrix}_{L}~, \qquad \begin{pmatrix} \nu_{\tau} \\ \tau^{-} \end{pmatrix}_{L}
%\end{equation}
%where the $L$ subscript denotes the left-handed structure.

The principle of electroweak symmetry breaking illustrates elegantly how the bosons acquire mass within the Standard Model, but no mass term for the fermions is yet included.
Their mass, however, also follows from the same BEH mechanism but in a slightly less trivial manner. %The presence of the additional scalar doublet $\phi$ allows to add terms of the form $\bar{\psi}_{L}\phi \psi_{R}$, which are defined as Yukawa couplings and are gauge-invariant under SU(2)$\times$U(1).
\\
The existence of the additional BEH field $\phi$ allows for the introduction of the following gauge-invariant terms in the Lagrangain:
%With the presence of the additional BEH field $\phi$ it is now permitted to introduce the following gauge-invariant terms to the Lagrangian:
%The presence of the additional scalar doublet in the model now allows to add the following gauge-invariant terms to the Lagrangian:
\begin{equation}
 \mathcal{L}_{Yukawa} = - Y_{ij} \bar{\psi}_{L,i} \phi \psi_{R,j} + h.c. 
\end{equation}
with $Y_{ij}$ the unknown Yukawa matrices. Hence the fermion masses arise from the Yukawa interactions describing the couplings of the fermions with the BEH field.
\\

For the quarks the weak-interaction eigenstates, considered up to now, have been observed to differ slightly from the mass eigenstates. Hence a matrix conversion is required which diagonalizes the mass matrix. This is done by the $3 \times3$ Cabibbo-Kobayashi-Maskawa (CKM) matrix~\cite{?}, which represents the probability of a transition from a quark $q$ into a quark $q^{'}$ by the matrix element $\vert V_{q^{'}q} \vert$.
\begin{equation}
 \begin{pmatrix}
  d^{weak} \\ s^{weak} \\ b^{weak} 
 \end{pmatrix}
 = \begin{pmatrix}
    V_{ud} & V_{us} & V_{ub} \\ V_{cd} & V_{cs} & V_{cb} \\ V_{td} & V_{ts} & V_{tb}
   \end{pmatrix}
   \begin{pmatrix}
    d \\ s \\ b
   \end{pmatrix}
\end{equation}

\subsection{Unanswered questions in the Standard Model} \label{sec::QuestionsSM}
The Standard Model has been deemed as very succesful and experimentally verified to the percent level. However it still bares some important shortcomings which cannot be ignored and should be understood in order to denote the Standard Model as a ``theory of everything''. 

\begin{myindentpar}
  \begin{description}
    \item[Grand Unified Theory] \hfill \\
    The successful unification of the weak and electromagnetic interaction into the electroweak one sparked hope of one day respresenting the three forces of the Standard Model by a single one.
    %The hope of representing the three forces of the Standard Model by one general force with a single coupling strenght stems from the succesful unification of the weak and electromagnetic interaction into the electroweak one.
    Such a(n) unification of the elementary forces is currently not yet explicable by the Standard Model since it requires new physics at a very high energy scale ($\Lambda_{GUT} \sim 10^{16} \GeV$).
    %included in the Standard Model, and new physics at a very high scale  is required in order to do so.
    Such a Grand Unified Theory (GUT) is believed to be a first step to the incorporation of gravity in the Standard Model.
    %Once this first step has been taken it is hoped that this unified theory can easily be related to gravity. (\textit{in order to establish a theory of everything (TOE).})
    
    \item[Hierarchy problem] \hfill \\
    The observed vector boson masses indicate that the principle of electroweak symmetry breaking should occur at an energy scale of $\mu^{2} \sim (100 \GeV)^{2}$.
    %The principle of electroweak symmetry breaking is supposed to occur at an energy scale of $\mu^{2} \sim (100 \GeV)^{2}$ in order to explain the vector boson masses.
    The large energy gap up to the GUT or Planck scale ($\Lambda_{Planck} \sim 10^{19} \GeV$), energy regime where the gravitional attraction becomes comparable to the other elementary interactions, implies a significant fine-tuning of at least 28 orders of magnitude is required. 
    A scale difference of this size, also known as the hierarchy problem, is far from desirable for a complete theory.
    %This large scale difference is also known as the hierarchy problem. 
    %Hence up to the GUT or Planck scale ($\Lambda_{Planck} \sim 10^{19} \GeV$), where the gravitional attraction becomes comparable to the other elementary forces, a wide energy regime exists without any new physics. 
    %This enormeous scale difference or hierarchy problem implies a fine-tuning of at least 28 orders of magnitude, way too large to be comfortable. 
    %has devastating effects on the mass of the scalar BEH field since it will receive additive radiative corrections 
    %(Is this full explanation with radiative corrections and \Lambda cut-off necessary??)
    %Not really clear what comes first and what follows from what ... The hierarchy problem follows from the fact that gravity lies in a complete different energy regime, but the actual scale used for the corrections seems to be related to the SU(5) breaking in order to explain a unification of the three other forces ...
    
    \item[Dark matter and energy] \hfill \\
    Cosmological observations have pointed out that the matter described by the Standard Model only constitutes about $4.8 \%$ of the matter in the universum~\cite{PlanckResults}. The remaining part is occupied by dark matter ($25.8 \%$) and dark energy ($69.4 \%$), two cosmological concepts which cannot be detected in a direct manner.
    
    %\item[Flavor problem] \hfill \\
    %Is this not relevant for my analysis? Should go through the notes of the Fermilab summer school!
    
   \end{description}
\end{myindentpar}
The shortcomings outlined above are not the full story but merely a selected list of unexplained issues within the Standard Model. 
However, this does not weighs up to the numerous successes of the Standard Model such that the underlying theoretical framework has not been abandoned but rather an extension is searched for.
\\
Supersymmetry (SUSY) is one of the more widely accepted suggestions%( solving a couple of the discussed shortcomings)
, which introduces an additional symmetry relating bosons and fermions.
In this framework each particle has a superpartner with identical quantum numbers, except for the spin parameter differing by a half-integer.
SUSY gives a possible explanation for the hierarchy problem because of the large variety of new supersymmetric particles required. 
%However, their masses should be of the order of $100 \GeV$ implying that they should be observable with the current particle colliders. 
As a bonus the lightest supersymmetric particle is a possible dark matter candidate.
\\
Other extensions are studied by as well such as theories including additional dimensions or theories where particles are replaced by strings. These type of theories require a significant number cosmological observations, which is rather challenging due to its complex nature and lower detection probability. 
%The main difficulty that arises with studying cosmological aspects is the limited number of experimental data due to its complex nature and detection probability. 
Hence in order to decide upon the correct Standard Model extension much more experimental data of both cosmological and elementary particle physics processes are required. The latter can be achieved by constructing state-of-the art particle colliders.

\section{Importance of the top-quark interaction vertex}
One of the regions of interest to look for new physics phenomena in elementary particle physics is the top quark sector. This heaviest elementary particle has been discovered more than 20 years ago, in 1995, but still remains a challenging research subject due to its important role in physics theories beyond the Standard Model. 
\\
The high mass of the top quark has rendered its observation very arduous due to the extreme energy conditions required to produce such a quark. 
However it is exactly this property that makes the top quark such an interesting particle to investigate.
Since it is the only elementary particle for which the Yukawa coupling is of the order of $1$ it is believed that the top quark might shed some light on the principle of electroweak symmetry breaking.
\\
\textit{Link with anomalous couplings in Wtb required here? Yes, say something about the left-right handed properties which can be studied in the interaction vertex ...}
%The top quark sector is extremely relevant to study new physics phenomena since it is the heaviest quark and is the only one for which the Yukawa coupling is of the order $1$. This section will start with some general information about top quark physics and emphasis will be set on the production and decay mechanisms together with a concise overview of the latest observations. Afterwards the presence of anomalous couplings in the top quark decay vertex and their importance will be explained.
%
%\textit{The different transformation between right-and left-handed fermions in the weak interaction has already shortly been mentioned in the previous section. However here it will come clear what are the consequences of this parity violation, and emphasis will be set on the top quark sector.}

\subsection{Top quark physics}
The energy regime required to produce the heavy top quarks was only reachable at the Tevatron~\cite{Tevatron}, where they were finally discovered in 1995 by the CDF~\cite{CDF} and D$\varnothing$~\cite{D0} experiments. However since a couple of years the Tevatron has been superseded by the LHC~\cite{} at CERN~\cite{CERN} as ``top-quark machine'', which can produce top quarks in ample amounts due to the higher centre-of-mass energy. %(8 $\TeV$ in stead of 1.96 $\TeV$).
At hadron colliders top quarks can be produced either in pairs or singly, although the former one is the more dominant production method.
\\
The top quark pair production cross-section can be determined theoretically in a very precise manner and compared to the measured cross-sections at the LHC. These theoretical and experimental cross-section values have been summarized in Table~\ref{table::XSTopPair}, containing the final 8 $\TeV$ and even the first 13 $\TeV$ results. 
All experimental observations have provided for good agreement with the theory predictions.
%This is governed by the strong interaction and originates at the LHC in most of the cases from gluon fusion. (interesting to mention?)
\begin{table}[h!t]
 \centering
 \caption{Comparison between the theoretical predictions~\cite{CzakonTopPairXS, CzakonGluonPDF} and experimental measurements~\cite{TevatronTTbarXS, CMSTTbarXS} of the $\ttbar$ production cross-section $\sigma_{\ttbar}$.} \label{table::XSTopPair}
 \begin{tabular}{|cl|c|c|}
  \hline
		&						& Theory prediction (pb) 	& Measured $\sigma_{\ttbar}$ (pb) 	\\
  \hline						
  Tevatron: 	& $p\bar{p}$ at $\sqrt{s}$ = 1.96 $\TeV$ 	& $7.164^{+0.391}_{-0.475}$	& 7.60 $\pm$ 0.41			\\
  LHC: 		& $pp$ at $\sqrt{s}$ = 7 $\TeV$ 		& $172.0^{+12.1}_{-13.4}$	& 174.5 $\pm$ 6.2			\\
  LHC: 		& $pp$ at $\sqrt{s}$ = 8 $\TeV$ 		& $245.8^{+16.6}_{-18.7}$	& 245.6 $\pm$ 9.3			\\
  %LHC: 		& $pp$ at $\sqrt{s}$ = 13 $\TeV$ 		& 				& 				\\
  \hline
 \end{tabular}
\end{table}

Top quarks can also be produced individually, denoted as single top quark processes, but the lower probability and significantly higher background makes detection much more challenging. Single top quark production can occur in three different channels: s- and t-channel through exchange of a W-boson or associated tW channel (\textit{Need more detail ... Check slides of Top2015!}). The t-channel production is the dominant one ... (First evidence of s-channel production at the LHC only obtained in 2015 by ATLAS).
\\
Results in measurement of the $V_{tb}$ component of the CKM matrix. Combination of the 7 and 8 $\TeV$ results gives~\cite{CMSVtbResult}:
\begin{equation}
 \vert V_{tb} \vert = 0.998 \pm 0.038 (exp.) \pm 0.016 (theo.) \quad \Rightarrow \quad \vert V_{tb} \vert > 0.92 ~ @ ~ 95 \% ~ CL
\end{equation}

Even though the production of the top quark can be governed by either strong interactions, describing $\ttbar$ pair production, or by electroweak interactions, responsible for the single top quark production, the decay of the top quark is purely characterized by electroweak processes. This is indeed confirmed by observations which also indicate a decay probability of almost 100 $\%$ of a top quark into a W-boson and b-quark.
\\

Another consequence of its high mass is that the top quark is the only elementary particle that decays before hadronisation (\textit{But this is only explained in the next chapter ...})

%\subsubsection{Production and decay mechanisms}

%\subsubsection{Important experimental results}

\subsection{Anomalous couplings}