\chapter{Event selection (change)} \label{chp:labelTitle}

In the previous section it was thoroughly described how the different physics objects can be identified and reconstructed.
However, in order to study a specific type of interactions, semi-leptonically decaying top-quark events in this thesis, special attention should be devoted in order to select exactly these event topologies.
This ... occurs in a sequential manner, starting from the general and centrally managed trigger system which aims to only store the events matching some predefined requirements and ranges up to an analysis-specific event optimisation to reject particular event signatures.
\\
\\
This multi-layered event selection procedure will be explained in detail here, first focusing in Section~\ref{sec::MainSelec} on the more general steps such as triggering and cleaning of the relevant events together with the main selection criteria.
Afterwards the additional selection constraints applied in order to single out the specific characterisations of this analysis will be discussed in Section~\ref{sec::SpecificSelec}.

\section{Main selection}\label{sec::MainSelec}

\subsection{Triggering and cleaning of events}\label{subsec::Trigger}
\textit{Where can you find information about this trigger path and what is actually applied?}\\
As was already briefly mentioned in Chapter~\ref{chp:CERN}, CMS possesses a complex trigger system consisting of two/three levels.
\textit{Briefly mention something about the trigger ...}
However since every analysis focusses on a different event topology the event filtering system is used to select only the desired final state products, drastically reducing the event rate.

\subsection{Jet selection criteria}

\subsection{Lepton selection criteria}

\subsection{Data-MC agreement/Kinematics of selected events}

\section{Analysis-specific selection}\label{sec::SpecificSelec}
\subsection{B-tagging}

\subsection{Signal optimisation}

\subsection{Data-MC agreement}