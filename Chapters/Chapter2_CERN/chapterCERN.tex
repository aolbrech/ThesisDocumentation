%\dropchapter{0.4in}
\chapter{The CMS experiment at CERN's accelerator complex} \label{chp:CERN}
%\epigraphhead[70]{\epigraph{\textit{If I could remember the names of all these particles, I'd be a botanist.}}{Enrico Fermi}}
%\undodrop

The Standard Model of elementary particle physics, for which its main successes and shortcomings has been discussed extensively in Chapter~\ref{chp::SM}, has proven to result in very precise predictions. However it is only acknowledged as an effective theory up to an energy scale of about 1 $\TeV$. Physics beyond this energy range is studied with specific high-energetic particle colliders including, for example, the Large Hadron Collider (LHC) located at CERN (European Council for Nuclear Research) near Geneva. The LHC provides proton-proton collisions at a record-breaking energy and is currently the world's most energetic particle collider.\\
Many different experiments surround the Large Hadron Collider each with a specific physics goal ranging from general high-luminosity physics to dedicated plasma-studies and even long-lifetime neutrino interactions (and even medical ...??).
In this chapter attention will mainly be devoted to the CMS experiment, which is the LHC general-purpose experiment used for collecting data processed within this thesis.

\section{The Large Hadron Collider}
The need for the construction of a particle collider with the enormous dimensions of the LHC was driven by a quest to understand the nature of the electroweak symmetry breaking, for which the Higgs mechanism is presumed to be responsible, and investigate the $\TeV$ scale.
\\
\textit{\textbf{Useful to mention here again?}}
\\

When the design of the LHC machine was approved in 1994 it was decided to reuse the existing 26.7 $\km$ Large Electron Positron (LEP) tunnel, previously excavated in the 1980's and positioned between 45m and 150m below the Earth's surface.
Avoiding the excavation of a new tunnel was a huge cost-saver but presented some stringent limitations on the machine's design. For example the space limitation in the tunnel compelled the use of so-called ``twin-bore'' magnets where both proton rings are contained within a single magnet structure.
\\
The LHC is designed to provide proton-proton collisions with a beam energy of 7 $\TeV$ each, resulting in a centre-of-mass energy of 14 $\TeV$. This is a seven-fold energy increase compared to the previous most energetic particle collider: the Tevatron~\cite{} which yielded proton-antiproton collisions between 19.. up to 2014 (?). In order to reach these extreme energy conditions the LHC exploits the presence of the extensive accelerator complex present at CERN to increase the beam energy gradually. This adopted accelerator sequence is denominated as the injection chain of the LHC and will be discussed in detail further in the text.
\\
When the proton beams are circulating within the LHC at the desired beam energy they can be forced to collide head-on in the dedicated interaction regions where beam crossings are provided. Of the eight interaction regions existing in the LEP tunnel only four have been equipped with particle detectors for the LHC run. The ATLAS~\cite{} and CMS~\cite{} experiments are the two largest ones and are intended as general-purpose detectors studying a broad range of high-luminosity physics while the ALICE~\cite{} and LHCb~\cite{} experiments search for a specific type of physics interactions. The former one serves mainly as a heavy-ion detector while the latter one is dedicated to heavy-flavour physics.
\\
Within this thesis data collected at the CMS detector during the first era of data-taking has been analyzed, which started in March 2010 and continued until December 2012. These collisions did not take place at the design beam energy of 7 $\TeV$ but at a reduced energy of 3.5 $\TeV$ and 4 $\TeV$ for the 2010-2011 and 2012 data-taking, respectively. \textit{Maybe just put some of the main characteristics corresponding to this run to have similar information of all four parts!}

\subsubsection{LHC design, driven by the LEP legacy}
Start with choice of protons (because anti-protons can not be produced in adequate quantities to attain the design luminosity), then mention this requires two separate magnetic beams since the particles need to be bend in opposite directions and then explain this required the development of twin-bore magnets which contain two beampipes. Also mention that the required magnetic field \textit{``implies'' (need better word)} the need of superconducting coils at extremely low temperatures (1.9 K, never done before). End with given a ``cross-section'' of the LHC dipole structure.

\subsubsection{The LHC injection chain}
The LHC does not only reuse the existing LEP tunnel it also benefits from the entire accelerator complex existing at CERN in order to reach a record energy of 7 $\TeV$. As a first step of the injection scheme a linear accelerator (Linac2) is used which accelerates protons that were electrically stripped from hydrogen atoms up to an energy of 50 $\MeV$. They are then injected in the first circular accelerator, the Proton Synchrotron Booster (PSB), until they reach an energy of 1.4 $\GeV$ and can be passed on further in the injection scheme. The next accelerator in line is the Proton Synchrotron (PS) -- here bunches are formed!

\subsubsection{Particle detectors}

\subsubsection{2010-2012 data taking}
These collisions occured not at the design beam energy but at a reduced energy of only 3.5 and 4 $\TeV$ resulting in a luminosity of ..., a fraction .. lower than the design luminosity.

%\begin{myindentpar}
%  \begin{description}
%    \item[LHC design, driven by the LEP legacy] \hfill \\    
%    Start with choice of protons (because anti-protons can not be produced in adequate quantities to attain the design luminosity), then mention this requires two separate magnetic beams since the particles need to be bend in opposite directions and then explain this required the development of twin-bore magnets which contain two beampipes. Also mention that the required magnetic field \textit{``implies'' (need better word)} the need of superconducting coils at extremely low temperatures (1.9 K, never done before). End with given a ``cross-section'' of the LHC dipole structure.\\
%    \textit{?Is there another pp collider which exists or existed before? Otherwise ``The LHC machine is the first ever to collide protons with protons instead of the previously adopted approach of ppbar interactions''...}
%    \item[The LHC injection chain] \hfill \\
%    The LHC does not only reuse the existing LEP tunnel it also benefits from the entire accelerator complex existing at CERN in order to reach beam-energies up to 7 $\TeV$.
%    \item[Particle detectors] \hfill \\
%    \item[2010-2012 data taking] \hfill \\
%    These collisions occured not at the design beam energy but at a reduced energy of only 3.5 and 4 $\TeV$ resulting in a luminosity of ..., a fraction .. lower than the design luminosity.
%  \end{description}
%\end{myindentpar}

\section{The Compact Muon Solenoid detector} 

One of the two main-purpose particle detectors of the Large Hadron Collider is the Compact Muon Solenoid (CMS) experiment which was constructed to perform a wide variety of physics measurements. Hence the design of the CMS experiment was led by the LHC physics programme goals which required good identification and momentum resolution throughout the entire detector.
In order to efficiently detect all the different particles emerging from the interaction point, the CMS apparatus consists of four separate subdetectors which are all designed in order to identify specific types of particles: a tracker detector, an electromagnetic and hadronic calorimeter, and a muon system.

The main distinguishing feature of the CMS experiment (which even drives the design of each of the subdetectors,) is the high-field solenoid magnet of $3.8$ Tesla as depicted in Figure \ref{fig::CMSFig}. This to ensure good momentum resolution within a compact spectrometer without the need to make use of heavily restricted muon chambers. The other three subdetectors are required to be placed within this magnet coil (because ...?) which significantly restricts their size. The tracker detector is placed closely around the beam pipe and actually consists of a silicon-based pixel and strip detector to guarantee track reconstruction in the high density environment close to the interaction point. Since also the electromagnetic calorimeter (ECAL) and hadronic calorimeter (HCAL) are located within the solenoid they are designed to be as compact as possible without any loss of granularity. The former calorimeter is a scintillating crystal-based type while the latter is a brass/scintillator sampling one.
\begin{figure}[h!t]
 \centering
 %\includegraphics[width = 0.4 \textwidth]{Afbeeldingen/CMS.png}
 \caption{CMS Figure with all subdetectors clearly visible} \label{fig::CMSFig}
\end{figure}

The design of the CMS experiment is \textbf{also} optimized for the reconstruction of neutrino's, which cannot be measured directly and hence only appear in the form of missing energy, by ensuring a hermetically closed detector. This resulted\footnote{Is this really the motivation why a barrel and endcap design has been adopted??} in the construction of a cylindrical barrel part located centrally with respect to the interaction point and an endcap part represented by a sort of disklike closure componenents for each of the different subdetectors.
Even though the CMS experiment is denominated as ``compact'', its overal dimensions are a total length of $21.6$ m and a diameter of $14.6$ m resulting in a total weight of $12500$ tons.

The CMS experiment has adopted a proper coordinate system for which the origin is centered at the nominal collision point within the detector. The $y$-axis (ordinate) is pointing upwards and the $x$-axis (abscissa) radially inwards toward the center of the LHC. Hence, according to the right-hand rule, the $z$-axis follows along the anticlockwise-beam direction. This coordinate system can easily be converted into a spherical coordinate system where the azimuthal angle $\phi$ is measured in the $x$-$y$ plane and the polar angle $\theta$ from the $z$-axis. 
A very important variable, which is used widely in accelerator physics, can now be derived from this coordinate system. The pseudorapidity $\eta$ is used to describe the angle of a particle with respect to the beam axis (or the $z$-axis) and is defined in Equation (\ref{eq::PseudoRapidity}). The key reason why this variable is so crucial in particle detectors is its invariance with respect to Lorentz boosts along the beam axis.

\begin{equation} \label{eq::PseudoRapidity}
 \eta = - \ln \tan \frac{\theta}{2}
\end{equation}

Another variable that is closely related is the rapidity, also denoted using the symbol $y$. Since this variable needs both the energy and total momentum of a particle it is much more challenging to correctly determine, but in the case of high energy collisions both quantities are almost identical.

\begin{equation}
 y = \frac{1}{2} \ln \left( \frac{E+p_{z}c}{E - p_{z}c} \right)
\end{equation}

\subsection{The silicon tracking apparatus}

\subsubsection*{Hit and track reconstruction in the pixel and strip tracker}

The tracking detector is responsible for translating the measured energy deposits in the pixel and strip tracker into charged particles' tracks. This process is done using a computationally challenging track reconstruction algorithm which proceeds in an iterative manner in order to first identify the straightforward prompt tracks.
The dense environment in the inner tracker is the main challenge for the track reconstruction implying the need of an efficient search for hits during the pattern recognition stage and a fast propagation of trajectory candidates. Hence the use of a Combinatiorial Track Finder (CTF), an extension of the Kalman Filter technique which allows the combination of track fitting and pattern recognition.

The local reconstruction is performed prior to the iterative tracking and clusters energy deposits by combining neighbouring pixels or strips which fullfill specific signal over noise (S/N) requirements. The cluster position is determined from the charge-weighted average or from the cluster edges in the case of the strip or pixel detector, respectively.

The track reconstruction algorithm can be decomposed into four separate steps: seed generation, pattern recognition, ambiguity resolution and track fitting.
The benefit of using an iterative process is that during the first iteration the easist to find tracks are identified such that their associated hits can be removed from the list in order to reduce combinatorial complexity for the following iterations.

\begin{myindentpar}
  \begin{description}
    \item[Seed generation] \hfill \\
    This step of the track reconstruction provides initial trajectory candidates from pairs of pixel hits. The track finding starts from trajectory seeds created in the innermost region of the tracker because the high granularity of the pixel detector ensures lower channel occupancy in the inner pixel layer than in the outer strip layer. 
    %Hence optimal efficiency is retrieved when the tracks are built in the outward direction.
    The starting trajectory parameters and their uncertainty in the quasi-uniform magnetic field of the tracker can be determined from five parameters. Hence either three 3-D hits or two 2D-hits combined with a beam spot constraint should be extracted in order to construct the seed trajectory.
    \item[Pattern recognition] \hfill \\
    This module of the CTF algorithm is basically a Kalman Filter which proceeds iteratively from the seed layer until the outermost tracker layer is reached. % taking into account the effects of multiple scattering and energy loss. 
    %Combining the information of successive layers significantly improves the precision of the track parameters for each layer. 
    First a dedicated navigation (step) is performed in order to identify the layers possibly intersected by the trajectory of the seed. Then, for each hit on this layer, a new trajectory candidate is created and its track parameters are recalculated using the Kalman Fiter formalism by combining the predicted trajectory state with the added hits in a weighted mean.
    \item[Ambiguity resolution] \hfill \\
    Since the track of a single charged particle can be reconstructed more than once, either originating from different seeds or when one single seed resulted in more than one trajectory candidate, double-counting is possible and should be resolved. Hence exclusion of specific tracks is performed based on the number of hits shared between two trajectories. 
    %The track with the fewest hits is removed and the trajectory cleaner is applied again on the remaining list of trajectory candidates. 
    \item[Track fitting] \hfill \\
    During the final step of the iterative tracking process the trajectory parameters are refitted using all hits in order to exclude any bias introduced during the seeding stage. The Kalman Filter used here starts from the innermost hit and proceeds outwards. Afterwards a so-called smoothing stage is applied in the form of an outside-in Kalman Fitler which uses the result of the first one. This approach yields optimal estimates of the parameters.
  \end{description}
\end{myindentpar}

%\textit{? Is this hit reconstruction (with fast and template-based algorithms) relevant?}

%The energy deposits detected in the pixel and strip tracker will be translated into actual charged particles' tracks by a track reconstruction algorithm. This is a computationally challenging tasks which is performed by a Combinatorial Track Finder (CTF), an extension of the Kalman Filter technique allowing the combined use of track fitting and pattern recognition. The process is performed iteratively, where each iteration proceeds in a similar manner:
%\begin{itemize}
% \item \textbf{Seed generation} \\
% During the first step of the CTF algorithm initial track candidates are identified which will serve as starting point for the reconstruction of the actual track parameters. The five parameters needed for representing the particle's trajectory in the tracker's magnetic field are retrieved by either three 3-D hits or two 2D-hits combined with a beam-spot constraint. The search for the seeds starts from the most inner layers of the pixel detector for efficiency optimization.
% \item \textbf{Track finding} \\
% In this step an inside-out Kalman Filter is applied in order to identify hits in successive detector layers compatible with the trajectory seed. After each layer the track parameters are updated taking into account the effects of multiple scattering and energy loss. This iterative process continues until the outermost layer of the tracker has been reached.
% \item \textbf{Track fitting} \\
% In order to obtain the most optimal track parameters, including the 
% \item \textbf{Track selection} \\
%\end{itemize}

\subsection{The calorimetry subdetectors}

\subsection{The muon system}

\subsection{Trigger and data acquisition}
