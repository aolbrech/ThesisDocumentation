\documentclass[a4paper,12pt]{book}
\usepackage[T1]{fontenc}
\usepackage[utf8]{inputenc}
\usepackage{lmodern}
\usepackage[margin=1in]{geometry}
\usepackage{graphicx}
\usepackage{lscape}
\usepackage{multirow}
\usepackage[labelfont=bf]{caption}
\usepackage{color}
\usepackage{hyperref}
\usepackage{amsmath}
%\hypersetup{colorlinks,urlcolor=blue}
\usepackage{color}
\usepackage{bm,array}      %Needed for changing width of boxes in tables!
%----------------------------------------------------------------------------------
\usepackage[centersections]{vubtitlepage}
\usepackage[title,titletoc,toc]{appendix}

\faculty{Faculteit Wetenschappen en bio-ingenieurswetenschappen}
\date{Started 4 April 2014 \\ Version: \today}
\sectone{Anomalous Couplings Project}
\sectthree{Annik Olbrechts }
\secttwo{Progress Report and documentation of Tools }

\sectonesize{35pt}
\sectoneskip{40pt}
\secttwosize{30pt}
\secttwoskip{10pt}
\sectthreesize{24pt}
\sectthreeskip{28pt}
%----------------------------------------------------------------------------------
\usepackage{hyperref}
\begin{document}
\newcommand{\VL}{V_{L}}
\newcommand{\VR}{V_{R}}
\newcommand{\gR}{g_{R}}
\newcommand{\mT}{m_{top}}

\newcommand{\scd}{2^{nd}}
\newcommand{\fth}{4^{th}}
\newcommand{\chisq}{\chi^{2}}

\newcommand{\loglik}{\ln(\mathcal{L})}

\newcommand{\csTh}{\cos \theta^{*}}

\newcommand{\NegLL}{-$\ln(\mathcal{L})$~}
\newcommand{\ttbar}{$t\bar{t}$~}

\newcommand{\pT}{p_{\textrm T}}

\newcommand{\unit}[1]{\, \mathrm{#1}}	% units: roman + small space before
\newcommand{\GeV}{\unit{GeV}}

\pagenumbering{roman}
\setcounter{secnumdepth}{3} %Number up to the subsubsections!
\setcounter{tocdepth}{3}    %Don't want the subsubsections appearing in the table of contents (-> then change back to 2!)

\maketitle
\tableofcontents
\newpage

\pagenumbering{arabic}
%°°°°°°°°°°°°°°°°°°°°°°°°°°°°°°°°°°°°°°°°°°°°°°°°°°°°°°°°°°°°°°°°°°°°°°°°°°°°°°°°°°°°°°°°°°°°°°°°°°°°°°°°°°°°°°°°°°°°°°°°°°°°°°°°°°°°°°°°°°°°°°°°°°°°°°°°°°°°°°
\chapter{Pending issues}
\begin{itemize}
  \item Does this created FeynRules model still contain Effective Field Theory?
  \item If the kinematics doesn't change for coupling parameters larger than $1$, how is it than possible to differentiate the different configurations which will be studied?
  %\item Access to Mathematica for Width studies ... ?
  %\item XS values which are used are the ones calculated using the MadGraph$\_$v155 version. Should this be calculated again using the new MadGraphv5$\_$aMC@NLO since normally this shouldn't change ...?
  %\item Create branch for localgrid scripts! (Now already a branch exists for the documentation !)
  \item Understand \textit{CorrectPhi} comment in MadWeight name \\ $\rightarrow$ Maybe related with phi issues of neutrino's ...
  \item EventWeight calculated in analyzer should be integrated into MadWeight output (Should MadWeight weight just be multiplied with the EventWeight for each event?) \\ \textbf{Otherwise all the effort to include JES and PU has no influence at all.							}
  \item Update EventNumberInformation file to only cout the information when the event has passed the eventSelection requirements. Otherwise just discard this event from the output file in order to avoid a long .txt file and long running time ...
  \item Construct nTuple in such a way that the likelihood and weight distributions can be calculated for specific $p_T$ cuts without having to run the entire base code again. So perhaps also necessary to include some of the different python scripts which have been created. Hence avoid spending too much time on improving them.
  \item Need to understand how MadWeight deals with the permutations ... Should this permutation of the light jets be done within MadWeight or should two .lhco files be created and sent separately to MadWeight. If the latter is the case, how should they be combined afterwards ?
  %\item Be careful with used Beam Energy when using MadGraph. In the newest MadGraph directory this still seems to be set to 7TeV in stead of the required 4 GeV. \\ \textbf{Could this have an influence on the obtained MadWeight results when using the new MadWeight version (connected to the newest MadGraph version) ...}
  \item Specific MadWeight question for the latest version: \\ \textit{The 'refine' option allows you to realunch the computation of the weights which have a precision lower than X. \textbf{But how can you find this precision?}}
  \item Should a ScaleFactor correction be applied for the Branching ratio ? \\ \textit{This is not the case in code of James, but should find recommendations somewhere.}
  \item \textbf{To Check:} Does MadWeight give a different weight when the Neutrino Mass changes?\\ $\rightarrow$ Currently mass is manually set to zero ...
  %\item Running the analyzer code with the PileUp and JES/JER influences correct and active, the number of preselected events in the Event Selection Table has changed. Need to understand if it is expected that these corrections have an influence (lowering the number) on the number of preselected events ...
  \item Check whether the created root file can be trusted when the number of considered b-tags is greater than 1 ... \textit{Why this comment??}
\end{itemize}

\newpage
%°°°°°°°°°°°°°°°°°°°°°°°°°°°°°°°°°°°°°°°°°°°°°°°°°°°°°°°°°°°°°°°°°°°°°°°°
%	CHAPTER: Used Tools and Techniques
\chapter{Used Tools and Techniques}

\section{MadGraph}

\subsection{MadGraph$\_$v155 version}
Currently the MadGraph version which should be used is the $MadGraph\_v155$ since this is the only one which was compatible with the used FeynRules version at the time of the creation of the model. According to a mail of Olivier Mattelaere on 6 February 2014 this issue should now be fixed.\\
\\
In this project MadGraph is only used to create .lhe files in order to understand the created model. Especially for Cross Section studies it is extremely useful.\\
Hence two python scripts have been created in the following directory:
\begin{eqnarray*}
  & & /user/aolbrech/AnomalousCouplings/MadGraph\_v155/MassiveLeptons/\\ & & MadGraph5\_v1\_5\_5/Wtb\_ttbarSemiElMinus/RVL\_RVR\_XSGrid003.py \\
  & & /user/aolbrech/AnomalousCouplings/MadGraph\_v155/MassiveLeptons/\\ & & MadGraph5\_v1\_5\_5/XSScript.py
\end{eqnarray*}
The first one automatically creates all the desired configurations, for different RVL, RVR, Lepton Pt Cut and Jet Pt Cut values. The desired output directory can be specified in this file as well.\\
In order to save some disk space the events.lhe file is deleted using this script and only the unweighted$\_$events.lhe are kept for analysis. These are the relevant events (\textit{according to Alexis so understand what is the difference}).\\
\\
The second file calculates the cross section of the desired configurations. It again loops over all the different variables and creates a .txt table and a .pdf table.\\
\textbf{Maybe this can be created automatically when executing the first python script! This to avoid copying the considered variables from one file to another!}

\subsection{New MadGraph5$\_$aMC@NLO version}
This new MadWeight version was released in the spring of 2014 and is accompanied with a new MadWeight version as well.\\
Normally the cross section calculations shouldn't differ between the different MadGraph versions, but for at least one of the decay channels this should be checked!\\
\\
In order to use MadWeight and calculate the cross section values for the desired processes and decay channels the following commands should be used:
\begin{eqnarray}
 ./bin/mg5_aMC \\
 import model MassiveLeptons \\
 generate p p > t t~ , ( t > b w+ , w+ > mu+ vm ) , ( t~ > b~ w- , w- > j j ) @1 QED = 2 \\
 output dirName \\
 exit
\end{eqnarray}
Once this directory is created the \textit{index.html} file contains all the Feynmann Diagrams which correspond to the considered process. This file can be opened using $firefox$ on mtop.\\
Since the FeynRules diagram is created in such a way that it contains a \textbf{NP}, \textbf{QCD}, \textbf{QED} and \textbf{TEST} variable representing the different interaction vertices. Hence asking the variable \textbf{QED} to be equal to $2$ results in the 16 independent diagrams which are required to represent the anomalies in the Wtb vertex. Currently the decay of the top quark is altered to contain all the different coupling constants while the other interaction vertices represent the Standard Model expectations.\\
\\
When the cross section should be calculated for a specific process the following configuration files in the Cards directory should be updated.
\begin{itemize}
 \item \textbf{run$\_$card.dat} where the number of events which should be generated is defined together with the beam energy of the considered collision process. Also should be ensured that no kinematic cuts are active on the particles since these cuts will have a different effect then the cuts applied in the event selection of the analysis. These cuts are applied on the generator level instead of the jet-level which results in much more stringent and tighter effects when using the same $p_{T}$ value.
 \item \textbf{param$\_$card.dat} where all the parameters of the considered model should be defined.
 \item \textbf{proc$\_$card.dat} contains the event which will be generated and the models which have been imported. This file shouldn't be changed but is particularly useful to ensure that the correct event is considered for generation.
\end{itemize}

All these files can still be changed once the \textit{./bin/generate$\_$events} command has been executed. The MadGraph software always asks whether one or more of the configuration files should be adapted before the command is submitted and allows a waiting time of 60s. Hence in order to run MadGraph continously using a script the \textbf{me5$\_$configuration.txt} file should be adapted to enable this waiting time since otherwise when using nohup to run the script results in a crash and the script is terminated.
%**************************************************

\section{MadWeight}

The running of the MadWeight event generator has to be done on localgrid since the events .lhco file has to be splitted in multiple jobs. The number of jobs which can be submitted is not limited, hence the events per job should be taken as low as possible. Running MadWeight with a large number of events per job implies that it will take a very long time since the submitted jobs have a long walltime which reduces their priority. \\
Another important factor for the MadWeight event generator is defining the number of interaction points which have to be considered. This number has to be rather large (around 10 000 - 30 000 according to Lieselotte \textbf{TO CHECK MYSELF}) to ensure the uncertainty on the weight to be smaller than the weight itself.\\
\\
Currently two different MadWeight event generator versions exist, and up to now (May 6 2014) only the oldest one has been tested and used extensively. The newer one was only installed beginning of May but should be less CPU intensive and allows the user to split the interaction points in two distinct steps. The versions are:
\begin{itemize}
  \item madweight\_mc\_perm
  \item MadGraph5\_aMC@NLO
\end{itemize}

The installation commands of these two versions are the following (the bzr command package has been installed on the m-machines, but not on mtop):
\begin{itemize}
  \item bzr branch lp:~maddevelopers/madgraph5/madweight 
  \item bzr branch lp:~maddevelopers/mg5amcnlo/madweight
\end{itemize}

Whenever MadWeight is first used a personal directory should be initalized. This can be done in the following way:
\begin{itemize}
  \item Import the created model (called MassiveLeptons) in the model directory of MadGraph
  \item Activating the MadWeight event program and initializing a personal directory.
  \begin{enumerate}
    \item ./bin/mg5$\_$aMC (The mg5 executable should not be used!)
    \item import model MassiveLeptons
    \item generate p p > t t~ , ( t > b w+ , w+ > mu+ vm ) , ( t~ > b~ w- , w- > j j ) @1 QED = 2 
    \item output madweight
    \item exit
  \end{enumerate}
\end{itemize}

\subsection{MC$\_$PERM MadWeight use}
The most important steps which have to be executed to use this MadWeight version are the following ones~\footnote{Full detailed explanation about the use of this madweight version can be found in the documentation of Bettina.}:
\begin{enumerate}
  \item Initialize MadWeight to run on localgrid!
  \item Update the MadWeight$\_$card.dat
  \item Set the correct transfer functions
  \item Run MadWeight
\end{enumerate}

For this first step the following two files have to be changed in the /blablaMadWeight/bin/internal directory:
\begin{itemize}
  \item madweight$\_$interface.py
  \item cluster.py
\end{itemize}

\subsection{Checking MadWeight on localgrid}
Different command which should be used to check whether jobs are running are running on localgrid and how they should be killed when something went wrong:
\begin{eqnarray}
 qstat \; \; @cream02 \; | \; grep \; \; aolbrech \\
 qdel \; \; 394402.cream02
\end{eqnarray}

\subsection{Influence of the used Transfer Function}

\subsection{Adapting MadWeight to run continously on localgrid}
A single test with the following configuration (5000 events - 20 events/job - 10 000 initPoints) took almost 16h to finish so the full generator event sample ($>$ 2 470 000 events) would take more than 329 days!\footnote{From this can be concluded that the events on generator level should never be all run. Only a limited selection of these events should be considered.}\\
Even the full reconstructed event sample ($>$ 210 000 events) requires more than 28 days of running this way.\\
\textbf{TODO: Time of running should be compared against the new MadWeight version!!}\\
\\
A possible solution to reduce the CPU time to process these events is in stead of sending 2000 jobs, waiting until they are all finished and only then submitting the next bunch of 2000 jobs trying to adapt the MadWeight script to send continously 2000 jobs. This would imply that the script should check whether one of the jobs have been finished and immediatly submitting a new one.\\
For this the above mentioned scripts should be adapted!\\
\\
According to Olivier (D) this can't be changed by the grid admins, but should be adapted by the MadGraph developers. 
%**************************************************

\section{MadAnalysis}
\subsection{How to run MadAnalysis in expert mode}
Version v112 should be used since the expert mode in this version works exactly as explained in the manual (arXiv: 1206.1599). 
In the more recent version v115 the expert mode doesn't work out-of-the-box and the python files need to be adapted ...\\
\\
Starting with a new analysis directory in MadAnalysis in expert mode can be done by typing the following command:
\begin{equation}
  ./bin/ma5 \; \; \; --expert
\end{equation}
In the questions asked by MadAnalysis after executing this command the name of the directory which needs to be created and the name of the analysis has to be given.\\
The name of this analysis shouldn't be made too complex since it has to be typed everytime when creating plots with MadAnalysis.\\
\\
After the correct directory is created, the $Name$/SampleAnalyzer directory should be initialized by executing the following two commands:
\begin{eqnarray}
  source \; \; setup.sh \\
  make
\end{eqnarray}

The actual analysis should be created in the Analysis directory, and a similar approach to user.cpp and user.h should be adopted.\\
Everytime a change has been made to these two files $make$ should be executed in the SampleAnalyzer directory in order to process these changes.\\
\\
The .lhe files (MadAnalysis cannot process .lhe.gz files so they should be unpacked using gunzip .lhe.gz) which should be considered should be wirtten down in a .txt files which is saved in the SampleAnalyzer directory. \\
\\
The actual running of MadAnalysis is done with the following command:
\begin{equation}
  ./SampleAnalyzer \; \; --analysis="Name ~ of ~ analysis" ~~~~ List.txt
\end{equation}

\subsection{Content of analysis file in MadAnalysis}
The latest analysis file which has been used with all the necessary information can be found in the following directory on the m-machines:
\begin{eqnarray*}
  AnomalousCouplings/MadAnalysis\_v112/Wtb\_PtCutInfluence/SampleAnalyzer/Analysis 
\end{eqnarray*}
The two analysis files with full detailed information are the LeptonPtCutInfluence.cpp and the JetPtCutInfluence.cpp files.
They both consist of two different functions, namely the $Execute$ and the $Finalize$ function. The first one allows to access the information of each event while the second one is entered for each file.
Therefore the particle content is reconstructed in the $Execute$ function and the histograms for all the considered files are constructed in the $Finalize$ function.
Currently these analyzer files look at 28 different kinematic variables. All the kinematic information of each of the particles present in the expected semi-leptonic $t\bar{t}$ event is created.\\
\\
In order to separate the two b-quarks in the event, the Particle Id of the leptonic top quark needs to be known. Therefore an integer $LeptonicTopPdgId$ is used and the kinematic information of the b-quarks can only be stored when this integer is different from zero.
This will normally not result in by-passing the b-quark information since the events in the .lhe files are read in in the same order as they are created by the MadGraph command. So first the top quarks are considered and only then the final state particles.\\
\\
\textit{? Isn't it safer to use the daughter information of the top quark? Because in the case of t t -> b jj b l v the lepton comes after the b-quark ... Should investigate what happens in this case ...} \textbf{TO CHECK: } I think some extra safety is incorporated and the actual loop of filling the event content is only done when the b's are reconstructed!\\
\\
In order to automatically create the 28 histograms for all the different files and distinguish the different RVR and RVL values, an automatic name-givng loop is used. The name of the content of the histogram is each of the time combined with the correct name of the RVL/RVR content of the .lhe file and the decayChannel.\\
\\
\textit{Currently the name of the considered .lhe files has to be adapted every time the files are changed. Maybe this should be made automatically read in from the .txt file to ease the processing of different configurations. If the .lhe files have a clear name where the RVL, RVR and PtCut values can be obtained from using a python script it shouldn't be too much work to save this in the .cpp file with this script. \textbf{TO DO!}}

\subsection{Analyzing the MadGraph files}\label{subsec:MadGraphFiles}
Currenlty the created model should be completely understood and the behavior of the model when the coupling coefficients are larger than $1$ should be investigated. Therefore new MadGraph files have been processed for the following configuration:
\begin{eqnarray*}
  Re(V_L) & \in & \left[  0.7, \; 1.3\right] \\
  Re(V_R) & \in & \left[ -0.3, \; 0.3\right]
\end{eqnarray*}
These files can be found in the following directory on the m-machines and contain 100 000 events.
\begin{eqnarray*}
  /user/aolbrech/AnomalousCouplings/MadGraph\_v155/MassiveLeptons/\\ MadGraph5\_v1\_5\_5/Wtb\_ttbarSemiElMinus/ResultsXSGrid003
\end{eqnarray*}

\subsection{Width of the decay}\label{subsec:DecayWidth}
%**************************************************

\section{FeynRules}
\textbf{Should be understood why the mass of the top quark is equal to 180 within the created FeynRules model!} %
%°°°°°°°°°°°°°°°°°°°°°°°°°°°°°°°°°°°°°°°°°°°°°°°°°°°°°°°°°°°°°°°°°°°°°°°°

%°°°°°°°°°°°°°°°°°°°°°°°°°°°°°°°°°°°°°°°°°°°°°°°°°°°°°°°°°°°°°°°°°°°°°°°°
%	CHAPTER: Generator level bottlenecks
\chapter{Generator level bottlenecks}

The goal is to obtain as fast as possible results on generator level in order to ensure that no bottlenecks are found when running MadWeight.
The advantage of only using generator level results is that one can be completely sure that the studied events are actual semi-leptonic ttbar events. Hence MadWeight should not have any problems calculating the weight for these kind of events and no CPU time will be spent on uncorrect events.\\
This implies that any deviation from the expected results implies a bias, or even a problem, concerning the MadWeight output.\\
\\
Once the results correspond to the expectations these preliminary results should be easily extended to reconstructed events. Finalizing the event selection then allows to fully trust the results obtained on reconstructed level and make sure that any deviation should be explained by the influence of the applied event selection.\\
These results can then be used to optimize the event selection with respect to the MadWeight output and CPU time needed.

\section{Transfer Functions}
In order to obtain results with MadWeight, the Transfer Functions which link the reconstructed energy distribution with the actual energy distributions should be calculated. In the case of generator level events this is not that important since no smearing of the energy is expected, but in order to avoid any bias from the used Transfer Functions it is adviced to use the real Transfer Functions with a smaller width.\\
\\
The method to obtain the parameters describing the energy smearing is (partly) explained in the PhD Thesis of Arnaud Pin\footnote{It should be noted that in the PhD Thesis of Arnaud, and in the current double Gaussian transfer function syntax in MadWeight, only 5 parameters are used. The narrow gaussian distribution doesn't have a normalisation parameter in front, but is normalized afterwards in the MadWeight code.} which can be found in:
\begin{eqnarray*}
 & & https://cp3.irmp.ucl.ac.be/upload/theses/phd/Pin\_Arnaud.pdf \\
& & /home/annik/Documents/Vub/PhD/ThesisSubjects/AnomalousCouplings/\\ & & PrepareGenLevelRunning\_Sep2014/TransferFunctions
\end{eqnarray*}

The method to obtain the parameters of the Transfer Functions is based on the code received from Petra and Lieselotte, which was used in the Master thesis of Lieselotte\footnote{Additional feedback was also received from Arnaud, but he was using a completely different unbinned likelihood method. Since this only arrived after a couple of weeks waiting, it was decided to continue with the binned likelihood method used by Petra and Lieselotte.}. However it should be noted that the original code only had $4$ bins due to limited statistics, and now $10$ bins are used.\\
This code is almost identical to the ROOT class FitSlicesY() but has some minor differences. Some of these have already been changed in order to match better with the ROOT class.\\
One of the most important differences between the two approaches was the treatment of the underflow and overflow bin. In the received code these two bins were respectively added to the first and last bin and hence included in the fitting range. This is not the desired behavior since the size of underflow/overflow bin can be relatively large compared to the first/last bin and significantly change the value of these bins. This would then imply that the position of the first and last bin is not located at the correct position and will, especially in the case of a limited number of bins, have a significant influence on the fit result.
Now these underflow and overflow bins are just discarded from the fit range and will have no influence on the final result.\\
\\
Another important, but useful, difference between the two methods is the number of histograms which are saved. The received code saves for each distribution which is considered the ProjectionY distribution together with the double Gaussian fit for this bin. This can not be changed in the ROOT class which only stores the distribution of the 6 parameters of the double Gaussian fit formula.\\
\\
However, even after carefully ensuring that both methods are identical the obtained results are not.
Up to now it is not clear what is the reason for the discrepancy between the two results and the only way to found out is comparing the distributions and results for a significant amount of statistics.\\
One possibility is the used fit ranges and number of bins. If the last bins are low on statistics their distribution might not agree with a double Gaussian distribution and hence result in a failed fit. Therefore the distribution for each ProjectionY bin is now closely studied for all of the considered histograms. This can be found is the following section \ref{subsec::FitRanges}.

\subsection{Creating the Transfer Function}
In this subsection the different steps which were performed in order to built the actual Transfer Functions and implemented in MadWeight will be discussed in detail. The Transfer Functions used in this analysis are assumed to be uncorrelated, as shortly discussed in the PhD thesis of Arnaud. This assumption is necessary since otherwise it would be impossible to build them ourselves. The implication of this assumption is given in Equation \ref{eq::TFAssumption} and should be checked by looking at $E$ vs $\theta$, $E$ vs $\phi$ and $\theta$ vs $\phi$ histograms.
\begin{equation} \label{eq::TFAssumption}
 W(E, \theta, \phi) = W(E) W(\theta) W(\phi)
\end{equation}

A first important remark which should be made is the choice for using $p_{T}$ dependent Transfer Functions, in stead of the general $E$-dependent Transfer Functions already implemented in MadWeight. The changes which have to be done in order to correcty implement the Transfer Function in MadWeight are discussed in \ref{subsubsec::PtDependency}.\\
\\
The following equation, Equation \ref{eq::FunctForm}, gives explicitely the functional form of both the double Gaussian fit and the $p_T$-dependent calorimeter formula. As can be seen from this equation, the calorimeter part should not be taken to literally since it is only the actual calorimeter $p_T$ dependency in the case of the $\sigma$-parameters of the two Gaussian fits.
\begin{eqnarray}
 a_3 \times e^{-\frac{(x-a_1)^2}{2*a_{2}^{2}}} + a_6 \times e^{-\frac{(x-a_4)^2}{2*a_{5}^{2}}} \nonumber \\
 a_{x,0} + a_{x,1} \times \sqrt{p_T} + a_{x,2} \times p_T \;\;\;  \textrm{( for x = 2 \& 5 )}\nonumber \\
 a_{x,0} + a_{x,1} \times p_T + a_{x,2} \times p_{T}^{2} + a_{x,3} \times p_{T}^{3} + a_{x,4} \times p_{T}^{4} \;\;\; \textrm{(for x = 1, 3, 4 \& 6)} \label{eq::FunctForm}
\end{eqnarray}

\subsubsection{Using $p_{T}$ dependency} \label{subsubsec::PtDependency}
The Transfer Function configuration files in MadWeight are flexible enough to change the kinematic variables used for the Transfer Function calculations. Hence it seems to be more relevant to utilize the transverse momentum in stead of the energy of the considered partons and jets. Especially since the LHCO file used in the MadWeight calculations has the transverse momentum as input variable. Therefore it seems much more useful and realistic to use this parameter and not the energy. This implies that the Transfer Function configuration file should be adapted to use the \textit{pt(p)} and \textit{pt(pexp)} variables and not the currently used \textit{p(0)} and \textit{pexp(0)}.\\
\\
After actually implementing the created Transfer Function files\footnote{This was tested the first time on 30 November 2014.} and trying to use this Transfer Function one issue showed up which is currently not completely solved yet. MadWeight contains hard-coded files which only allow the use of the E, THETA and PHI variable for the Transfer Function dependencies. Therefore the so-called ``block name'' PT is not accepted by the $change\_tf.py$ file. However it seems that there is no clear physical motivation for this limited choice of kinematic variables, but in order to be completely sure additional information is needed from Olivier and Pierre.\\
Once this issue is resolved the following file, containing the restricted list of kinematic variables, should be adapted\footnote{Another solution is to keep the block name equal to E but just use the transverse momentum information. However this might result in confusion when in some files the hard-coded E parameter gets written down (for example on plots).}:
\begin{eqnarray*}
 bin/internal/madweight/change\_tf.py
\end{eqnarray*}

\subsubsection{Creation of the Transfer Functions}
The Transfer Functions are created from the simulated $t\bar{t}$ sample which will be used throughout the entire analysis. A very small nTuple is created from this simulated sample containing only the TLorentzVector information of both generated and reconstructed particles. From this the necessary diagrams, such as the 2D distributions of $p_T$, $\theta$ and $\eta$ difference between the generated and reconstructed particle with respect to the generated value, are created and saved in a ROOT file:
\begin{eqnarray*}
 AnomalousCouplings/TFInformation/PlotsForTransferFunctions\_FromTree.root 
\end{eqnarray*}

This analyzer, called $TFFit.cc$, also performs the double Gaussian fit of these 2D histograms and afterwards the $E$-dependent calorimeter fit. The technicalities and specific details of these fit procedures are documented in the following class:
\begin{eqnarray*}
 AnomalousCouplings/PersonalClasses/src/TFCreation.cc \\
 AnomalousCouplings/PersonalClasses/interface/TFCreation.h
\end{eqnarray*}

The results of the two consecutive fits performed on the Y-projections of these 2D distributions are stored in another ROOT file, together with the original 2D histograms. Also the function form of the double Gaussian fit formula using the obtained fit parameters, added in order to test the robustness of the fit results outside the fitted range, can be found in this ROOT file.
\begin{eqnarray*}
 AnomalousCouplings/TFInformation/CreatedTFFromDistributions\_FromTree.root
\end{eqnarray*}

This analyzer also has the flexibility to perform the fit on the entire range or on pre-defined ranges set by the user. For this a separate function, called \textit{SetFitRange}, is created where for each histogram the fit range for each separate bin can be defined. This is extremely useful to optimize the doubleGaussian fit which has to cover both the peak and the tail of the distributions in order to correctly calculate the $6$ fit parameters.\\
Another useful aspect of this analyzer is the automatic creation of the necessary .dat Transfer Function files needed for implementation in MadWeight. This is done in the \textit{WriteTF} class and the created files are:
\begin{eqnarray*}
 AnomalousCouplings/TFInformation/TF\_user.dat \\
 AnomalousCouplings/TFInformation/transfer\_card\_user.dat
\end{eqnarray*}
This first file contains the functional form of both the $E$-dependent calorimeter fit and the functional form of how these $6$ parameters should be included in the double Gaussian formula. Also the width for the different kinematic variables is defined within this file. For the moment the method used in the Transfer Functions already implemented in MadWeight is followed, implying that the width of the Transfer Function is defined as the maximum of the $\sigma$-parameter of the two Gaussians considered in the double Gaussian fit. The only difference is that in stead of the generated kinematic information, which is the variable on the abscissa of the considered 2D histograms, the reconstructed one is used.\\
\\
The second file contains the actual values of the different fit parameters for all the considered kinematic variables and particle types. Therefore this file is an extensive list of values to which is referred in the previous $TF\_user.dat$ file. For each particle type and kinematic variable the numbering used should be unique such that the correct values are implemented in the functional forms of the used fit formulas.

\subsubsection{Importance of start values}
Since the double Gaussian fit really needs accurate information of both the peak and the tails, detailed review of the start values for each of the $6$ fit parameters significantly improves the success rate of the fitting method. However it is important to note that setting these start values clearly improves the results obtained from the fit so special care should be taken to ensure the correctness of these start values. One way is by intensively comparing the fit distributions for the different particles since in the case of light jets and b-jets some similar behavior is expected.\\
After quite a while it is possible to quickly see whether the fit distribution has the expected shape and hence whether the used start values can be considered as stable. The start values used for the different 2D histograms is given in Table \ref{table::StartValues}
\begin{table}[h!]
 \centering
 \begin{tabular}{|c|c|c|c|c|c|c|}
  \hline
  \multirow{2}{*}{2D histogram}	& \multicolumn{6}{|c|}{Used start value for fit parameter}	  	\\
				& \multicolumn{3}{|c}{First (narrow) gaussian} 		& \multicolumn{3}{c|}{Second (wide) gaussian}		\\
				&  Mean $a_1$	& Sigma $a_2$ 	& Amplitude $a_3$ 	& Mean $a_4$ 	& Sigma $a_5$ 	& Amplitude $a_6$ 	\\
  \hline
    b-jet $\Delta \phi$ 	& 0.0002	& 0.022		&	8000		& 0.0002	& 0.06		&	3000		\\
    b-jet $\Delta p_T$ 		& 10		& -12		&	20000		& 13		& 10		&	-5000		\\
    b-jet $\Delta \theta$  	& 0		& 0.013		&	6000		& 0		& 0.04		&	2000		\\
  \hline
    light jet $\Delta \phi$  	& 0		& 0.022		&	8000		& 0.0004	& 0.002		&	3000		\\
    light jet $\Delta p_T$  	& 0		& 8		&	4000		& 0		& 12		&	4000		\\
    light jet $\Delta \theta$ 	& 0		& -0.014	&	6000		& 0		& -0.05		&	2000		\\
  \hline 
    electron $\Delta \phi$ 	& 0		& 0.0012	&	1500		& 0		& 0.006		& 	600 		\\
    electron $\Delta p_T$  	& 0		& 0.9		&	1500		& 0		& -2		& 	600		\\
    electron $\Delta \theta$ 	& 0		& 0.0013	&	2500		& 0		& 0.007		& 	600		\\
  \hline
    muon $\Delta \phi$ 		& 0		& 0.0004	&	800		& 0		& 0.0026	& 	600		\\
    muon $\frac{1}{\Delta p_T}$ & 0		& 0.0003	& 	2000		& 0		& 0.0006	&	500		\\
    muon $\Delta \theta$ 	& 0		& 0.002		& 	500		& 0		& 0.0004	&	500		\\
  \hline
 \end{tabular} 
 \caption{caption ... \textbf{Need to make sure that the narrow and wide gaussian is always the same ... Otherwise are the start values not correct for the different eta-bins ...!!}\\ \textcolor{red}{\textbf{Need to look at the sigma value to decide on the narrow/wide gaussian!}} } \label{table::StartValues}
\end{table}

\subsubsection{Splitting in separate $\vert \eta \vert$ bins}
Since the kinematic variables tend do depend on the pseudorapidity $\eta$ the considered 2D histograms are created for four distinctive $\vert \eta \vert$ regions. It has been chosen to split the barrel region into three separate bins while the entire endcap region is contained within one single bin. The chosen binning is given in Table \ref{table::EtaBins} together with the percentage of events present in each of the considered $\vert \eta \vert$ bins. This clearly shows the lower statistics available in the endcap region which results in larger difficulties of properly reconstructing the fit parameters in this region. This is shortly discussed below.
\begin{table}[h!]
 \centering
 \begin{tabular}{|c|c|c|c|c|}
  \hline
			 & $\vert\eta\vert$ $\leq$ 0.375	& 0.375 $<$ $\vert\eta\vert$ $\leq$ 0.75	& 0.75 $<$ $\vert\eta\vert$ $\leq$ 1.45	& 1.45 $<$ $\vert\eta\vert$ $\leq$ 2.5	\\
  \hline
    Relative $\#$ events &  26.21 $\%$				& 23.91 $\%$					&	32.54 $\%$			& 	17.34 $\%$			\\
  \hline
 \end{tabular} 
 \caption{Different $\vert\eta\vert$ bins used for the Transfer Function creation. It is important to note that the $\eta$-values used for this binning are the reconstructed ones since generated values could be higher than the $2.5$ cut-value.} \label{table::EtaBins}
\end{table}

The analyzer mentioned above is developed in such a way that both the fit results for all events as the results for the four separate $\vert \eta \vert$ bins are stored together. Therefore all the results can always be compared in the created ROOT files and in the distinct $dat$ files.\\
\\
One important difference between the 2D-distributions containing all events and the 2D-distributions specific for one of the $\vert \eta \vert$ bins is the number of bins used. Since the statistics is significantly lower for the $\vert \eta \vert$ specific histograms, the predefined bin number is lowered with $25 \%$. This ensures a more stable tail for the distribution and still a correct reconstruction of the peak.\\
\\
However for the last $\vert \eta \vert$ bin considered, the endcap part, a slightly different method is used. Since the statistics is much lower in this part of the detector the used range had to be slightly stretched in order to ensure a nice overview of the tails. This is necessary for the correctness of the double Gaussian fit. Therefore the range of the abscissa is enlarged on both sides with $20 \%$ and the used number of bins for this axis is identical to the one used for the overall 2D-distribution.

\subsubsection{Separating narrow and wide gaussian}
\textbf{\underline{Remark:} Should check whether this is actually necessary for the MadWeight implementation. MadWeight only seems to need the general fit formula and doesn't need to know which of the two distributions is the narrow and which is the wide one.}\\
\\
In order to select the narrow and wide gaussian both the amplitude and the $\sigma$-parameter should be compared, but keeping in mind that this second one is the most important variable. However it is expected that the distribution with the narrowest distribution also has the highest peak. In order to easily compare the two distributions a stacked canvas is added to the ROOT file which shows both distributions for different $p_{T,gen}$ values in one stacked canvas. The distinction between narrow and wide gaussian distribution is currently being made by the size of the $\sigma$ variable. Hence the histogram with the narrowest distribution is plotted in red while the widest one is plotted in green.\\
Such a canvas is made for each of the considered 2D-histograms, both for the one containing all events and for the ones splitted in separate $\vert \eta\vert$ bins. They should be analyzed in detail in order to understand the correctness of the double Gaussian fit applied for the creation of the Transfer Functions. For the moment there a still a couple of 2D-histograms which don't show the expected behavior. Additional investigation of these stacked canvasses should be performed as soon as possible.

\subsubsection{Coping with low statistics in last $\vert \eta \vert$ bin}
Solution : Excluding bins! + combining bins

\subsubsection{Extrapolation method}
The implementation in MadWeight should be done in such a way that the value of the outermost bins is used for all the $p_T$ values outside the fitted region. This is necessary since the extrapolation using the obtained fit parameters doesn't result in the desired double Gaussian behavior for these $p_T$ values. For values outside the fitted region, an inverted double Gaussian distribution or a distribution with two distinct peaks occurs rather often. This can be seen in Figure \ref{fig::doubleGaussExtrap}
\begin{figure}[!h]
  \centering
  \includegraphics[width = 0.9 \textwidth]{/home/annik/GitTopTree/TopBrussels/AnomalousCouplings/TFInformation/FitDistributions/FromTree/BJet_DiffPtVsGenPt/Overview_DblGausDistribution.png}
  \caption{Extrapolation obtained using the fit parameters of the double Gaussian functional form. Values outside the fitted range show a distinctly different shape with respect to the ones actually fitted. \textit{Maybe consider to lower the number of bins and make the titles more visible (Only $p_T$ cut value is really important!)}} \label{fig::doubleGaussExtrap} 
\end{figure}

\subsubsection{Actual implemenation in MadWeight}
All the possible Transfer Functions which are implemented in MadWeight can be found in the $Source/MadWeight/transfer\_function/data$ directory. Any file can be added to this list and used within MadWeight.\\
The relevant files used for the creation of the Transfer Functions are given below. The first one is the translation of the used $TF\_user.dat$ into a MadWeight readable file which can be implemented. The second file is only relevant around line $292$ where the function used for the Transfer Function creation is explained. This is the general MadWeight constructor file where all the different MadWeight functions are defined.
\begin{eqnarray} 
 Source/MadWeight/transfer\_function/transfer\_function.f \nonumber \\
 bin/internal/madweight\_interface.py \nonumber
\end{eqnarray}
The commands which have to be executed in order to build this MadWeight readable file for the Transfer Function are given here:
\begin{eqnarray} 
 ./bin/mw\_options \nonumber \\
 define\_transfer\_fct \nonumber
\end{eqnarray}

\subsection{Obtained distributions}\label{subsec::FitRanges}
Each of the considered histograms is a 2D histogram where the abscissa represents the transverse momentum of the generator level parton and the ordinate the difference between the generator level parton and the reconstructed matched particle. This is done for the difference in transverse momentum and in $\theta$ and $\phi$ angles.\\
All the interesting histograms can be created automatically, for each of the desired $\vert \eta \vert$ bins separately, using the following ROOT analyzer:
\begin{eqnarray*}
 AnomalousCouplings/TFInformation/FitDistributions/SaveFitHistograms.C
\end{eqnarray*}

This analyzer is able to create each of the histograms separately, both as $pdf$ and $png$, and automatically saves all the histograms of one type in a large stacked canvas. This allows to quickly see the used 2D distributions for each of the particle types (b-jets, light jets, electrons and muons) and the three kinematic variables ($p_T$, $\eta$ and $\phi$). This stacked canvas is shown in Figure \ref{fig::ColorPlots}.  Also the general behavior of each of the fitted diagrams together with the overal $\chi^{2}$ distribution is created for each 2D histogram as can be seen from Figure \ref{fig::StackedHistoBJetPt}, showing this for the $p_T$ distribution of the b-jets. Finally the distribution of the $6$ double Gaussian fit parameters together with the fit result of the $E$-dependent calorimeter formula is also collected in a stacked canvas, as can be seen in Figure \ref{fig::FitParamsBJetPt}.\\
\\
\begin{figure}[!h]
  \centering
  \includegraphics[width = 0.9 \textwidth]{/home/annik/GitTopTree/TopBrussels/AnomalousCouplings/TFInformation/FitDistributions/FromTree/ColorPlots.png}
  \caption{Used 2D distributions for double Gaussian fit. \textbf{IMPROVE CAPTION!}} \label{fig::ColorPlots} 
\end{figure}
The 2D distributions for the difference in transverse momentum tend to show a slightly asymmetric behavior, as can be seen from Figure \ref{fig::ColorPlots}. This can be explained by the influence of the event selection, which has a difference effect on the generated particle than the reconstructed particle. This because a particle surviving the $p_T$ cut actually has a different $p_T$ value on generated level due to \textbf{bad resolution, detector effects (???)}. This effect is almost negligible for the $\phi$ and $\theta$ angles \textbf{(Definitely sure that this is the case ??)}.\\
\\
\begin{figure}[!h]
  \centering
  \includegraphics[width = 0.9 \textwidth]{/home/annik/GitTopTree/TopBrussels/AnomalousCouplings/TFInformation/FitDistributions/FromTree/BJet_DiffPtVsGenPt/Overview_FitDistributions.png}
  \caption{Distribution of the energy difference between the generator level parton and the corresponding light quark jet for each of the 10 considered bins and the overflow bin. All distributions were fitted with a double Gaussian function.} \label{fig::StackedHistoBJetPt}
\end{figure}

\begin{figure}[!h]
  \centering
  \includegraphics[width = 0.9 \textwidth]{/home/annik/GitTopTree/TopBrussels/AnomalousCouplings/TFInformation/FitDistributions/FromTree/BJet_DiffPtVsGenPt/Overview_FitParameters.png}
  \caption{Energy dependency of the 6 parameters of the double Gaussian fit function. The result of the double Gaussian fit for each $p_T$ bin is combined in a $p_T$ dependent histogram and then fitted with the Calorimeter energy function as explained in the PhD Thesis of Arnaud Pin.} \label{fig::FitParamsBJetPt}
\end{figure}

\newpage
\textbf{STILL TO REVIEW}
\newpage
\subsection{Control checks for Transfer Functions}

\subsubsection{Compare results with normal (single) Gaussian}

\subsubsection{Compare result with previous analyses}
This will imply to put $p_{T,reco}$ information on the abscissa in stead of the current $p_{T,gen}$ one. Also it will result in a huge change of the start values which are however necessary to perform a succesful double Gaussian fit.

\subsubsection{Compare results with predefined ROOT class}
%**************************************************

\newpage
\section{Cross Section distribution for new grid}
%**************************************************
\paragraph{\underline{Remark: Cross section comparison}\\}
Need to check how it is possible that the XS values in the XS grid are identical for both versions (first check whether this is indeed the case), but are different for the top quark mass simulation ... Table \ref{table::MGXS}.

\section{First results: Wrong log(likelihood) minimum}

The first obtained MadWeight results for the enlarged grid ($V_L$ $\in$ $[0.8,1.2]$ and $V_R$ $\in$ $[-1,1]$) using only parton-level ttbar events did not result in the expected minimum of $(V_L,V_R)$ = $(1, 0)$. This can be seen from Figure \ref{fig::Likelihood}, which shows the distribution of the log(likelihood) for each point in the considered grid.\\ \\
\begin{figure}[!h]
 \centering
 \includegraphics[width = 0.9 \textwidth]{/home/annik/Documents/Vub/PhD/ThesisSubjects/AnomalousCouplings/UnderstandLikelihoodDistr_July2014/AnomCouplings_GenEvent_NoSelect_Oct3200Events_SingleGaussUsed/Likelihood_NoXSNorm.png}
 \caption{Distribution of the log(likelihood) for each point in the grid using 3200 parton-level positive semi-muonic ttbar events. The transfer function used to smear the parton-level kinematics is the single-gaussian function standard included in MadWeight.}
 \label{fig::Likelihood}
\end{figure}
One of the possible influences on the displaced minimim of the log(likelihood) distribution could be the normalisation of the Cross Section influence. This XS normalisation ($\frac{XS}{XS^{SM}}$) should be multiplied with the likelihood value, not the log(likelihood). Hence in order to correctly take this into account the obtained log(likelihood) value for each point in the grid should be corrected using the logarithm of this XS normalisation. The distribution of the normalisation on the Cross Section can be found in Figure \ref{fig::XSandLikelihoodNorm} together with the log(likelihood) distribution after correctly taking into account this XS normalisation.\\ \\
\begin{figure}[!h]
 \includegraphics[width = 0.45 \textwidth]{/home/annik/Documents/Vub/PhD/ThesisSubjects/AnomalousCouplings/UnderstandLikelihoodDistr_July2014/AnomCouplings_GenEvent_NoSelect_Oct3200Events_SingleGaussUsed/XSNorm.png}
 \includegraphics[width = 0.45 \textwidth]{/home/annik/Documents/Vub/PhD/ThesisSubjects/AnomalousCouplings/UnderstandLikelihoodDistr_July2014/AnomCouplings_GenEvent_NoSelect_Oct3200Events_SingleGaussUsed/Likelihood_XSNorm.png}
 \caption{Distribution of the XS normalisation for positive semi-muonic ttbar events (left) and distribution of the log(likelhood) after taking into account this normalisation. As in the previous figure 3200 positive semi-muonic have been used to obtain this distribution and a single gaussian transfer function has been applied to smear the kinematics of these parton-level events.}
 \label{fig::XSandLikelihoodNorm}
\end{figure}
The formula which has been used is the following (Equation \ref{eq::ProbMW}):
\begin{eqnarray}
 P(y \vert a) = \frac{1}{\textcolor{red}{\sigma(a)}*Acc(a)} \int W(y|x,a)*Eff(x,a) \vert M(x,a) \vert^{2} T(x,a) dx \label{eq::ProbMW}\\
 \mathcal{L} = \prod P(y \vert a)
\end{eqnarray}
The normalisation which has been applied in Figure \ref{fig::XSandLikelihoodNorm} is given in Equation \ref{eq::LikelihoodNorm}:
\begin{equation}\label{eq::LikelihoodNorm}
 \mathcal{L}_{Norm} = - ln(\sum P(y \vert a)*\frac{XS}{XS^{SM}}) = -ln(\mathcal{L}) - ln(\frac{XS}{XS^{SM}}*N)
\end{equation}
\textbf{It should be checked whether this is the correct method to take into account the normalisation of the XS. Currently it has been assumed that this XS normalisation should be applied for each weight and hence is multiplied with the number of considered events. In the case that this normalisation should just be multiplied with the overall likelihood value ($\mathcal{L}$) the sum over the number of considered events drops out of the equation implying a very small influence of the XS on the log(likelihood) distribution.}\\ \\

As highlighted in the general Matrix Element Method formula (Equation \ref{eq::ProbMW}), the probability to measure the observed quantities $y$ already has a normalisation factor for the cross section. This factor is defined as the channel cross section and is calculated using Equation \ref{eq::ChannelXSMW}:\\ \\
\begin{equation}\label{eq::ChannelXSMW}
 \sigma(a) = \int_{X_i} \vert M(x,a) \vert^{2} T(x,a) dx
\end{equation}
\textbf{Hence it should be investigated in detail whether this XS normalisation should still be applied. From the above equations could be concluded that the change in cross section is actually already incorporated in the weight obtained from MadWeight. This could make sense since MadWeight has all the necessary information to calculate the cross section for each point in the considered grid. The cross section values for each point have been calculated using MadGraph and the same model as used for the MadWeight calculations. Unfortunately it is not completely clear from the MadWeight documentation whether this is actually included in the weight or not.}

\section{Correct normalisation of Matrix Element probability}
As could be seen in Equation \ref{eq::ProbMW} a term $\sigma(a)$ is included in the general Matrix Element Techniques formula. However it is not clear whether this cross section normalisation is actually performed within the MadWeight calculations or whether this normalisation should be done afterwards.\\
This question is closely related to the order of the current obtained weight values with MadWeight. Up to now no weight larger than $10^{-22}$ have been obtained, resulting in a very large log(likelihood) value. \\
\\
This small value can be caused by many different reasons for which the most plausible ones are listed here:
\begin{itemize}
 \item The normalisation of the MadWeight probability should still be done and is not performed within the Matrix Element Techniques formula.\\ \textbf{This can only be ruled out by contacting the Madweight experts and asking explicitely what is done in the Madweight calculations. Also a possible hint could be found inside the MadWeight python files (but this should only be done if the received answer is not perfectly clear).}
 \item The smallness of the weight could be caused by an error inside the created FeynRules model. \textit{Should also look for the mail where one of the MadWeight experts (Olivier/Pierre or even Celine) answered about the possible explanation for the smallness of the weight and whether this implies some wrong assumptions).}\\ \textbf{A possible way to exclude that the origin of this problem is the AnomalousCouplings FeynRules model is by comparing the results for the top mass fit when both the SM FeynRules model and the AnomalousCouplings model is used. If the weights are also this small when the SM model is used this smallness should be solved by an additional normalisation factor.}
\end{itemize}

\paragraph{Update 31/10/2014: Probability function NOT normalized (according to mail Olivier)\\}
As was expected from the smallness of the obtained MadWeight probabilities should the cross section normalisation be applied afterwards. Only in the older versions of MadWeight (based on MG) was this normalisation included automatically.

\subsection{Measurement of top quark mass using Matrix Element Method}

\paragraph{Comparing SM model with AnomalousCouplings model\\}
As a first step the Feynmann diagrams belonging to the two different models should be compared. This information can be found in the \textit{index.html} file in the following directories (and the files should be opened using firefox on mtop since this is the only m-machine with a working browser):
\begin{eqnarray*}
 \tiny{/AnomalousCouplings/MadGraph5\_aMC@NLO/madgraph5/SM\_ttbarSemiMuPlus} \\
 \tiny{/AnomalousCouplings/MadGraph5\_aMC@NLO/madgraph5/ttbarSemiMuPlus\_QED2}
\end{eqnarray*}

\paragraph{Comparing SM cross section with MassiveLeptons cross section\\}
In order to be sure that both models have the same Standard Model base, the cross sections for both models have been compared. This resulted in an unexpected outcome, namely that the obtained cross sections differ significantly depending on which MadGraph version is used to generate the considered events. A summary can be found in Table \ref{table::MGXS}.
\begin{table}[h!]
 \centering
 \begin{tabular}{|c|c|c|c|c|c|}
  \hline
  \multirow{2}{*}{Top quark mass}	&  \multicolumn{2}{|c|}{MadGraph aMC@NLO}	& \multicolumn{2}{|c|}{MadGraph v155}  	\\
					&  SM model	& MassiveLeptons model		& SM model 	& MassiveLeptons model	\\
  \hline
    153 				& $9.23$ pb	& $9.645$ pb			& $6.692$ pb	& $6.984$ pb		\\
    163					& $11.12$ pb	& $11.63$ pb			& $7.844$ pb	& $8.199$ pb		\\
    173					& $12.98$ pb	& $13.54$ pb			& $8.897$ pb	& $9.281$ pb		\\
    183					& $14.77$ pb	& $15.4$ pb			& $9.884$ pb	& $10.3$ pb		\\
    193					& $16.5$ pb	& $17.22$ pb			& $10.78$ pb	& $11.25$ pb		\\
  \hline 
 \end{tabular} 
 \caption{Cross section values for semi-muonic (+) ttbar decay obtained using two different MadGraph versions.} \label{table::MGXS}
\end{table}

From this table can be seen that there is, for both considered MadGraph versions, a small difference between the SM FeynRules model and the MassiveLeptons one. This could be caused by the different treatment of the leptons. In the SM model they are considered to be massless while in the MassiveLeptons one they are defined to have their actual mass.\\
A larger differrence occurs when both MadGraph versions are compared. From the answer received by Olivier it is not clear whether this difference is worrysome or could be explained by the LO theoretical uncertainties. Should also be investigated whether this difference is related to the NLO behavior of the newest MadGraph version. In case the MadGraph v155 version is not up to NLO a difference in cross section is definitely expected.

\paragraph{Understanding why Top Mass simulation does result in correct Likelihood minimum\\}
When using the $MadGraph5\_aMC@NLO$ MadWeight version is used to scan over the different top mass values the correct minimum is obtained directly, so without any normalisation or acceptance influences. This simulation only uses the Standard Model information and doesn't consider the AnomalousCouplings FeynRules model.
The results can be found in Table \ref{Table::MWTopMass}

\begin{table}[h!]
 \centering
 \begin{tabular}{|c|c|}
  \hline
  Top quark mass	&  MadGraph aMC@NLO MadWeight log(Likelihood)\\
  \hline
    153 		& 196 499	\\
    163			& 188 002	\\
    173			& 181 027	\\
    183			& 186 284	\\
    193			& 192 926	\\
  \hline 
 \end{tabular} 
 \caption{ caption ..} \label{table::MGXS}
\end{table}

\subsection{Influence of the acceptance term}

In order to finalize the normalisation of the weight obtained from MadWeight also the influence of the acceptance term $Acc(a)$ has to be investigated. Since no reconstructed events exist are created for the different vector coupling coefficients, the influence of the applied event selection can only be applied on generator-level. This will normally introduce a slight bias since applying the reco-level cuts on generator-level will result in more stringent cuts than actually applied on the reconstructed events due to the smaller width of the kinematic distributions. However it is the only way to get an idea of the influence of the event selection and, as long as a flat dependency is found throughout the vector couplings, an acceptable one.\\
\\
The event selection influence is investigated by looking both at the change in cross section and at the difference of different kinematic distributions. This first is achieved by simulating events with different vector coupling coefficients and analyze the change in cross section. This is done for 1-dimensional scans of both the left-handed as right-handed vector coupling coefficients while the other is set to its Standard Model expectation value. The results together with the procentual reduction in cross section by the application of the event selection is given in Table \ref{table::XSChangeAccVL} and Table \ref{table::XSChangeAccVR} for the 1D change of $V_L$ and $V_R$, respectively. All the created MadGraph files are located in the following directory:
\begin{eqnarray*}
  MadGraph5\_aMC@NLO/madgraph5/TopMassCheckQED2\_ttbarSemiMuPlus\_ML
\end{eqnarray*}

\begin{table}[h!]
 \centering
 \begin{tabular}{|c|c|c|c|}
  \hline
  \multirow{2}{*}{Top quark mass} 	&  \multicolumn{3}{|c|}{1D change of Re($V_L$)}				\\
					&  All events	& Reco $p_T$ cuts applied	& Reduction ($\%$) 	\\
  \hline
    (0.8, 0.0) 				& $3.62605$ pb	& $0.9423$ pb			& $25.99$ 		\\
    (0.9, 0.0)				& $5.81248$ pb	& $1.51$ pb			& $25.98$		\\
    (1.0, 0.0)				& $8.85979$ pb	& $2.30454$ pb			& $26.01$ 		\\
    (1.1, 0.0)				& $12.96357$ pb	& $3.37064$ pb			& $26.00$ 		\\
    (1.2, 0.0)				& $18.3674$ pb	& $4.768$ pb			& $25.96$ 		\\
  \hline 
 \end{tabular} 
 \caption{} \label{table::XSChangeAccVL}
\end{table}

\begin{table}[h!]
 \centering
 \begin{tabular}{|c|c|c|c|}
  \hline
  \multirow{2}{*}{Top quark mass} 	& \multicolumn{3}{|c|}{1D change of Re($V_R$)}  			\\
					& All events	& Reco$p_T$ cuts applied	& Reduction ($\%$) 	\\
  \hline
    (1.0, -1.0)				& $37.7415$ pb	& $11.98$ pb			& $31.74$		\\
    (1.0, -0.5)				& $14.5606$ pb	& $4.466$ pb			& $30.67$		\\
    (1.0,  0.0)				& $8.85979$ pb	& $2.661$ pb			& $30.03$		\\
    (1.0,  0.5)				& $13.1236$ pb	& $4.04$ pb			& $30.78$		\\
    (1.0,  1.0)				& $33.1415$ pb	& $10.61$ pb			& $32.01$		\\
  \hline 
 \end{tabular} 
 \caption{} \label{table::XSChangeAccVR}
\end{table}

From these Tables can be concluded that the influence of the event selection on the cross section is flat and equal throughout the entire vector coupling grid. This would be a positive result since it would imply that no additional analytical function is necessary to normalise the MadWeight output. Hence the MadWeight output should only be normalized for the Cross Section influence, but not the event selection one.\\
\\
The second method of analyzing the ifnluence of the event selection on the vector couplings is by comparing the kinematical distributions before and after the event selection is applied. Special attention is awarded to ensuring no difference in shape is observed for the different vector coupling coefficients after the event selection is applied. If this would be the case the influence of the event selection wouldn't be flat as suggested by the change in cross section. However, comparing all kinematic distributions indeed suggests that no significant change is observed when comparing the different kinematic distributions. The full list of distributions can be found in the directory listed below, but the ones for the $\cos \theta^{*}$ variable, the $p_T$ distribution for the lepton, for the b-quark originating from the hadronically decaying top quark and for the down-quark originating from the W-boson decay are given in Figure \ref{fig::KinChangeCosTheta}, Figure \ref{fig::KinChangeLeptonPt}, Figure \ref{fig::KinChangeBJetPt} and Figure \ref{fig::KinChangeDownQPt}, respectively.
\begin{eqnarray*}
 .../ThesisSubjects/AnomalousCouplings/KinematicDistributions\_AcceptanceTerm\_Dec2014 \\
\end{eqnarray*}

\begin{figure}[!h]
 \centering
 \includegraphics[width = 0.48 \textwidth]{/home/annik/Documents/Vub/PhD/ThesisSubjects/AnomalousCouplings/KinematicDistributions_AcceptanceTerm_Dec2014/ReVLStable_ReVRVarying_CosTheta.pdf}
 \includegraphics[width = 0.48 \textwidth]{/home/annik/Documents/Vub/PhD/ThesisSubjects/AnomalousCouplings/KinematicDistributions_AcceptanceTerm_Dec2014/ReVRStable_ReVLVarying_CosTheta.pdf}\\
 \includegraphics[width = 0.48 \textwidth]{/home/annik/Documents/Vub/PhD/ThesisSubjects/AnomalousCouplings/KinematicDistributions_AcceptanceTerm_Dec2014/ReVLStable_ReVRVarying_PtCuts_CosTheta.pdf}
 \includegraphics[width = 0.48 \textwidth]{/home/annik/Documents/Vub/PhD/ThesisSubjects/AnomalousCouplings/KinematicDistributions_AcceptanceTerm_Dec2014/ReVRStable_ReVLVarying_PtCuts_CosTheta.pdf}
 \caption{Distribution of $\cos \theta^{*}$ variable for both 1D-variation of the vector coupling coefficients. The top figures depict the distributions before the application of any event selection while the lower ones show the same distribution but after the reco-level $p_T$ cuts have been applied.}
 \label{fig::KinChangeCosTheta}
\end{figure}

\begin{figure}[!h]
 \centering
 \includegraphics[width = 0.48 \textwidth]{/home/annik/Documents/Vub/PhD/ThesisSubjects/AnomalousCouplings/KinematicDistributions_AcceptanceTerm_Dec2014/ReVLStable_ReVRVarying_LeptonPt.pdf}
 \includegraphics[width = 0.48 \textwidth]{/home/annik/Documents/Vub/PhD/ThesisSubjects/AnomalousCouplings/KinematicDistributions_AcceptanceTerm_Dec2014/ReVRStable_ReVLVarying_LeptonPt.pdf}\\
 \includegraphics[width = 0.48 \textwidth]{/home/annik/Documents/Vub/PhD/ThesisSubjects/AnomalousCouplings/KinematicDistributions_AcceptanceTerm_Dec2014/ReVLStable_ReVRVarying_PtCuts_LeptonPt.pdf}
 \includegraphics[width = 0.48 \textwidth]{/home/annik/Documents/Vub/PhD/ThesisSubjects/AnomalousCouplings/KinematicDistributions_AcceptanceTerm_Dec2014/ReVRStable_ReVLVarying_PtCuts_LeptonPt.pdf}
 \caption{Distribution of transverse momentum of the lepton for both 1D-variation of the vector coupling coefficients. The top figures depict the distributions before the application of any event selection while the lower ones show the same distribution but after the reco-level $p_T$ cuts have been applied.}
 \label{fig::KinChangeLeptonPt}
\end{figure}

\begin{figure}[!h]
 \centering
 \includegraphics[width = 0.48 \textwidth]{/home/annik/Documents/Vub/PhD/ThesisSubjects/AnomalousCouplings/KinematicDistributions_AcceptanceTerm_Dec2014/ReVLStable_ReVRVarying_HadrBJetPt.pdf}
 \includegraphics[width = 0.48 \textwidth]{/home/annik/Documents/Vub/PhD/ThesisSubjects/AnomalousCouplings/KinematicDistributions_AcceptanceTerm_Dec2014/ReVRStable_ReVLVarying_HadrBJetPt.pdf}\\
 \includegraphics[width = 0.48 \textwidth]{/home/annik/Documents/Vub/PhD/ThesisSubjects/AnomalousCouplings/KinematicDistributions_AcceptanceTerm_Dec2014/ReVLStable_ReVRVarying_PtCuts_HadrBJetPt.pdf}
 \includegraphics[width = 0.48 \textwidth]{/home/annik/Documents/Vub/PhD/ThesisSubjects/AnomalousCouplings/KinematicDistributions_AcceptanceTerm_Dec2014/ReVRStable_ReVLVarying_PtCuts_HadrBJetPt.pdf}
 \caption{Distribution of the transverse momentum of the b-quark originating from the hadronically decaying top quark for both 1D-variation of the vector coupling coefficients. The top figures depict the distributions before the application of any event selection while the lower ones show the same distribution but after the reco-level $p_T$ cuts have been applied.}
 \label{fig::KinChangeBJetPt}
\end{figure}

\begin{figure}[!h]
 \centering
 \includegraphics[width = 0.48 \textwidth]{/home/annik/Documents/Vub/PhD/ThesisSubjects/AnomalousCouplings/KinematicDistributions_AcceptanceTerm_Dec2014/ReVLStable_ReVRVarying_LightQuarkDownPt.pdf}
 \includegraphics[width = 0.48 \textwidth]{/home/annik/Documents/Vub/PhD/ThesisSubjects/AnomalousCouplings/KinematicDistributions_AcceptanceTerm_Dec2014/ReVRStable_ReVLVarying_LightQuarkDownPt.pdf}\\
 \includegraphics[width = 0.48 \textwidth]{/home/annik/Documents/Vub/PhD/ThesisSubjects/AnomalousCouplings/KinematicDistributions_AcceptanceTerm_Dec2014/ReVLStable_ReVRVarying_PtCuts_LightQuarkDownPt.pdf}
 \includegraphics[width = 0.48 \textwidth]{/home/annik/Documents/Vub/PhD/ThesisSubjects/AnomalousCouplings/KinematicDistributions_AcceptanceTerm_Dec2014/ReVRStable_ReVLVarying_PtCuts_LightQuarkDownPt.pdf}
 \caption{Distribution of the transverse momentum of the down-type quark originating from the W-boson for both 1D-variation of the vector coupling coefficients. The top figures depict the distributions before the application of any event selection while the lower ones show the same distribution but after the reco-level $p_T$ cuts have been applied.}
 \label{fig::KinChangeDownQPt}
\end{figure}

From the distributions given above can easily be concluded that some shape difference are visible for the 1D-variation of the right-handed vector coupling. However for the left-handed vector coupling $V_R$ no influence is visible, besides some minor statistical fluctuations, when varying the value of $V_R$. It is also clear that the influence of the $V_R$ 1D-variation is not the same for each of the kinematical distributions, and even negligible for the $p_T$ distribution of the b-jet originating from the hadronically decaying top quark. Since the shift of the distribution is different for the $p_T$ distribution of the lepton and the down-type quark of the W-boson, the result is still in agreement with the flat change in cross section. This because the additional events for one distribution are balanced out by the reduced number of events for another distribution resulting a net effect of zero and hence a flat behavior troughout the entire vector coupling coefficients grid.

\subsubsection{Understanding 1D-variation of $V_L$}
The influence of the variation of the right-handed vector coupling $V_R$ on the kinematic distributions was rather satisfactory and in some way agreeing with expactations. However the same is definetely not true for the variation of the left-handed vector coupling $V_L$. On the contrary, it can even be concluded that the obtained result was a complete surprise and needs to be understood as soon as possible in order to keep confidence in the created FeynRules model. This because a possible explanation can always be a wrong configuration of the anomalousCouplings FeynRules model with would result in a major setback for the analysis and the tight time schedule.\\
\\
Therefore new MadGraph files, and kinematic distributions, have been created with the same 1D-variation of the left-handed vector coupling $V_L$ but in stead of setting the $V_R$ value equal to its Standard Model expectation of $0$ it was set to $0.2$. The reason behind this different value for $V_R$ is the idea that when the right-handed vector coupling is excluded from the full $Wtb$ Lagrangian any change of $V_L$ only affects the $\vert V_{tb} \vert$ value and hence the cross section. This would imply that no real change of physical concepts is done since the Lagrangian is in a sense unchanged because no mixing of the different vector couplings appears.\\
The results for this 1D-variation can be found in Table \ref{table::XSChangeAccVLNot0} and in Figure \ref{fig::KinChangeNot0}, and shown unfortunately identical results to the 1D-variation of $V_L$ with $V_R$ fixed to $0$. 

\begin{table}[h!]
 \centering
 \begin{tabular}{|c|c|c|c|}
  \hline
  \multirow{2}{*}{Top quark mass} 	&  \multicolumn{3}{|c|}{1D change of Re($V_L$)}				\\
					&  All events	& Reco $p_T$ cuts applied	& Reduction ($\%$) 	\\
  \hline
    (0.8, 0.2) 				& $4.223$ pb	& $1.201$ pb			& $28.43$ 		\\
    (0.9, 0.2)				& $6.615$ pb	& $1.882$ pb			& $28.45$		\\
    (1.0, 0.2)				& $9.995$ pb	& $2.787$ pb			& $27.88$ 		\\
    (1.1, 0.2)				& $14.46$ pb	& $4.043$ pb			& $27.96$ 		\\
    (1.2, 0.2)				& $20.26$ pb	& $5.695$ pb			& $28.11$ 		\\
  \hline 
 \end{tabular} 
 \caption{} \label{table::XSChangeAccVLNot0}
\end{table}

\begin{figure}[!h]
 \centering
 \includegraphics[width=0.48\textwidth]{/home/annik/Documents/Vub/PhD/ThesisSubjects/AnomalousCouplings/KinematicDistributions_AcceptanceTerm_Dec2014/ReVRStableButNot0_ReVLVarying_CosTheta.pdf}
 \includegraphics[width=0.48\textwidth]{/home/annik/Documents/Vub/PhD/ThesisSubjects/AnomalousCouplings/KinematicDistributions_AcceptanceTerm_Dec2014/ReVRStableButNot0_ReVLVarying_PtCuts_CosTheta.pdf}\\
 \includegraphics[width=0.48\textwidth]{/home/annik/Documents/Vub/PhD/ThesisSubjects/AnomalousCouplings/KinematicDistributions_AcceptanceTerm_Dec2014/ReVRStableButNot0_ReVLVarying_LeptonPt.pdf}
 \includegraphics[width=0.48\textwidth]{/home/annik/Documents/Vub/PhD/ThesisSubjects/AnomalousCouplings/KinematicDistributions_AcceptanceTerm_Dec2014/ReVRStableButNot0_ReVLVarying_PtCuts_LeptonPt.pdf}\\
 \includegraphics[width=0.48\textwidth]{/home/annik/Documents/Vub/PhD/ThesisSubjects/AnomalousCouplings/KinematicDistributions_AcceptanceTerm_Dec2014/ReVRStableButNot0_ReVLVarying_HadrBJetPt.pdf}
 \includegraphics[width=0.48\textwidth]{/home/annik/Documents/Vub/PhD/ThesisSubjects/AnomalousCouplings/KinematicDistributions_AcceptanceTerm_Dec2014/ReVRStableButNot0_ReVLVarying_PtCuts_HadrBJetPt.pdf}\\
 \includegraphics[width=0.48\textwidth]{/home/annik/Documents/Vub/PhD/ThesisSubjects/AnomalousCouplings/KinematicDistributions_AcceptanceTerm_Dec2014/ReVRStableButNot0_ReVLVarying_LightQuarkDownPt.pdf}
 \includegraphics[width=0.48\textwidth]{/home/annik/Documents/Vub/PhD/ThesisSubjects/AnomalousCouplings/KinematicDistributions_AcceptanceTerm_Dec2014/ReVRStableButNot0_ReVLVarying_PtCuts_LightQuarkDownPt.pdf}
 \caption{}
 \label{fig::KinChangeNot0}
\end{figure}

The obtained results for the 1D-variation of $V_L$ with $V_R$ fixed to $0.2$ seems to suggest that either the left-handed vector component of the full $Wtb$ Lagrangian only influences the cross section and alters in no way the kinematic distributions of the decay particles or otherwise that this component is wrongly implemented in the created FeynRules model. \\
The following test which has been performed is looking at a larger 1D-variation of $V_L$ while still keeping the $V_R$ component equal to $0.2$. This resulted in a slightly unexpected outcome, as can be seen in Figure \ref{fig::KinChangeNot0LARGE}, since some deviation of the kinematic distribution is found for the configuration where $V_L$ is equal to $0$. However there is no difference between the four remaining configurations considered for the large $V_L$ 1D-variation.\\
This is surprising since there was actually some hope that the lack of influence on the kinematic distributions when varying the $V_L$ component could be caused by the small variation applied. This was motivated by the small influence when changing the $V_R$ component to the boundaries of the coupling coefficients grid while originally the $V_L$ coupling was contained within a small region due to its precise measurement.\\
However the result seems to suggest that there is really no influence on the kinematic distributions and that only the $V_R$ component is responsible for these distortions. Probably a similar result would have been found when a different value of $V_R$ would have been used.
\begin{figure}[!h]
 \centering
 \includegraphics[width=0.48\textwidth]{/home/annik/Documents/Vub/PhD/ThesisSubjects/AnomalousCouplings/KinematicDistributions_AcceptanceTerm_Dec2014/ReVRStableButNot0_ReVLVaryingLARGE_CosTheta.pdf}
 \includegraphics[width=0.48\textwidth]{/home/annik/Documents/Vub/PhD/ThesisSubjects/AnomalousCouplings/KinematicDistributions_AcceptanceTerm_Dec2014/ReVRStableButNot0_ReVLVaryingLARGE_LeptonPt.pdf}\\
 \includegraphics[width=0.48\textwidth]{/home/annik/Documents/Vub/PhD/ThesisSubjects/AnomalousCouplings/KinematicDistributions_AcceptanceTerm_Dec2014/ReVRStableButNot0_ReVLVaryingLARGE_HadrBJetPt.pdf}
 \includegraphics[width=0.48\textwidth]{/home/annik/Documents/Vub/PhD/ThesisSubjects/AnomalousCouplings/KinematicDistributions_AcceptanceTerm_Dec2014/ReVRStableButNot0_ReVLVaryingLARGE_LightQuarkDownPt.pdf}
 \caption{}
 \label{fig::KinChangeNot0LARGE}
\end{figure}

\subsubsection{Fixing normalisation for 1D-variation of $V_R$ only}
Since the above mentioned results seem to predict that the left-handed vector coupling actually only influences the cross section and not the kinematics, it has been decided to ensure a correct cross section normalization for the 1D-variation of $V_R$ only. This because the influence of the left-handed vector coupling \\

\paragraph{\underline{Remark:} Correct way of thinking?\\}
Isn't it just the opposite normalisation which should be corrected? The left-handed vector coupling only influences the cross section so the log(likelihood) should actually be normalized in such a way that it results in a flat distribution for the 1D-variation of $V_L$. This is which is currently still not correct because of the large difference in relative cross section for values with $V_L$ smaller than $1.0$. While I think that the 1D-variation will, just as was the case for the top quark mass, result in the correct minimization of the log(likelihood) distribution. This because both the influence of this 1D-variation on the kinematic distribution and the variation of the cross section is symmetric around the Standard Model expectation value of $0$.\\
\\
Possible suggestion could be to check the influence of another coupling constant, for instance the imaginary part of the left-handed vector coupling. The distinct difference between the left-handed vector coupling and all the other coupling parameters implemented in the FeynRules model is its Standard Model value. The left-handed vector coupling is the only parameter which is supposed to contribute to the $Wtb$ Lagrangian, and hence the only one different from $0$. Therefore it could be a possible test of the created FeynRules model to check both the variation of the cross section as the influence on the kinematic distributions when a different coupling constant is considered. If it would only be that the real part of the $V_R$ doesn't contribute to the kinematics of the decay particles while all the other ones do have a distinct effect, it could be a clear indication that the created FeynRules model is not capable of dealing with coupling constants higher than $1$.

\paragraph{Thinking $\cdots$ \\} 
However this last conclusion is not completely correct because then a possible effect would have been visible when the $V_L$ variable was varied in a larger range. In this case also the configurations $V_L$ = 0.5 and $V_L$ = 0.0 was compared to the Standard Model expectation of $1.0$ but no effect was found. So there is more behind the non-influence of the left-handed vector coupling $V_L$ than just a problem with dealing with coupling coefficients larger than $1.0$ $\cdots$.
%°°°°°°°°°°°°°°°°°°°°°°°°°°°°°°°°°°°°°°°°°°°°°°°°°°°°°°°°°°°°°°°°°°°°°°°°

%°°°°°°°°°°°°°°°°°°°°°°°°°°°°°°°°°°°°°°°°°°°°°°°°°°°°°°°°°°°°°°°°°°°°°°°°
%	CHAPTER: Transfer Functions
\chapter{Transfer Functions}
In order to obtain reliable MadWeight results, the Transfer Functions which link the reconstructed energy distribution with the actual energy distributions should be taken into account. In the case of generator-level events this is less relevant since no significant smearing of the energy is expected. However in order to avoid any bias related to the used Transfer Functions it has been opted to use the constructed Transfer Functions, possibly with reduced width.\\

In this analysis it has been chosen to use a double Gaussian Transfer Function\footnote{Some information about this kind of TF can be found in the thesis of Arnaud Pin ($cp3.irmp.ucl.ac.be/upload/theses/phd/$). The main difference is that he has chosen to use only 5 parameters and hence a slightly different normalisation factor. The current definition is based on code received from Petra and Lieselotte, which has been applied in the Master Thesis of Lieselotte.}. This type of function is succesful in describing both the Gaussian distribution of the peak of the distributions, but also correctly takes into account the tail. This is prefered to a single Gaussian which only fits the peak and just discards the tail of the distribution. However, as can be seen from some of the distributions further in the text, the tail can become rather significant depending on the amount of available statistics.
Equation \ref{eq::FunctForm} gives explicitely the functional form of the double Gaussian fit and Equations (\ref{eq::PtCaloTF})-(\ref{eq::PtOtherTF}) the $p_T$-dependent ``calorimeter'' formula. This name should not be taken to literally since it is only the actual calorimeter $p_T$-dependency for the $\sigma$-parameters ($2$ $\&$ $5$) of the two Gaussian fits.
\begin{eqnarray}
 & & \frac{1}{\sqrt{2\pi}}*\frac{1}{a_2 ~ a_3 + a_3 ~ a_5}\left( a_3 ~ \exp\left[-\frac{(x-a_1)^2}{2*a_{2}^{2}}\right] + a_6 ~ \exp \left[-\frac{(x-a_4)^2}{2*a_{5}^{2}} \right] \right) \label{eq::FunctForm} \\
 & & a_{x,0} + a_{x,1} \times \sqrt{p_T} + a_{x,2} \times p_T \;\;\;  \textrm{( for x = 2 \& 5 )} \label{eq::PtCaloTF} \\
 & & a_{x,0} + a_{x,1} \times p_T + a_{x,2} \times p_{T}^{2} + a_{x,3} \times p_{T}^{3} + a_{x,4} \times p_{T}^{4} \;\;\; \textrm{(for x = 1, 3, 4 \& 6)} \label{eq::PtOtherTF} 
\end{eqnarray}

The method used to calculate the Transfer Functions is based on the $FitSlicesY()$ $ROOT$ class, however originally some differences existed\footnote{Even even after carefully ensuring that both methods are identical the obtained results were not. Up to now it is not clear what is the reason for the discrepancy between the two results and the only way to find out is comparing the distributions and results for a significant amount of statistics.}.
The most important difference between the two approaches was the treatment of the underflow and overflow bin. In the original method these two bins were added to the first and last bin, respectively, and hence included in the fit input. This is not desired since the size of the underflow/overflow bin can be relatively large compared to the first/last bin and significantly change the bin content. This could then result in a wrong position of the first/last bin and, especially in the case of a limited number of bins, have a significant influence on the fit output.
Now these underflow and overflow bins are excluded from the fit range and will have no influence on the final result.\\
The benefit of using this method is that it is able to save much more histograms than the $ROOT$ class. It saves for each considered distribution, the ProjectionY distribution together with the double Gaussian fit for this bin. 

Detailed information is stored in the following directories:
\begin{eqnarray*}
 & & AnomalousCouplings/PrepareGenLevelRunning\_Sep2014/TransferFunctions \\
 & & m-machines ~~ directory?
\end{eqnarray*}

STILL TO REVIEW
\section{Creation of the Transfer Function}
In this section the different steps which were performed in order to built and implement the Transfer Function (TF) will be discussed in detail. The TF's used in this analysis are assumed to be uncorrelated, as shortly discussed in the PhD thesis of Arnaud. This assumption is important since otherwise it allows to built them for each variable separately, as given in Equation~\ref{eq::TFAssumption}. The correctness of this assumption can be checked by looking at $E$ vs $\theta$, $E$ vs $\phi$ and $\theta$ vs $\phi$ histograms.
\begin{equation} \label{eq::TFAssumption}
 W(E, \theta, \phi) = W(E) W(\theta) W(\phi)
\end{equation}

In this analysis it has been opted to use $p_{T}$-dependent TF in stead of the generally used $E$-dependent ones. This results in a large amount of adaptations of the MadWeight configuration files as will be briefly explained here. Since the TF configuration files in MadWeight allow the change of the considered kinematic variables it seemed more relevant to utilize the transverse momentum of the considered partons and jets in stead of the energy. This choice was especially motivated by the fact that the used $.lhco$ file was constructed to contain the transverse momentum as input variable. As a consequence the TF configuration file \textbf{file} should be adapted to use the \textit{pt(p)} and \textit{pt(pexp)} variables in stead of \textit{p(0)} and \textit{pexp(0)}.

Since MadWeight contains some hard-coded information about the kinematical information used in the calculation of the TF's, the use of $p_T$-dependent ones resulted in some difficulties. One important observation was the fact that as soon as the configuration files are adapted to be compatible with the $p_T$-dependent ones, there is no easy turning back to the original $E$-dependent TF's. This is not the desired behavior since it excludes the possibility of quickly testing any discrepancy between the created TF and the original ones. Hence it has been decided to include the relevant MadWeight directory on gitHub and develop two separate branches, one for each type of kinematic variable. This allows to easily switch between the TF's and conduct as much tests as desired. The corresponding branches on GitHub are \textbf{TF$\_$EDependent} and \textbf{TF$\_$PtDependent}.

Within these branch also all the different Event directories containing all executed tests are stored. This is used as a back-up since these results are only stored on localgrid where no back-up scheme is foreseen. The only disadvantage of this is that a clear overview of the performed changes to the configuration files is missing. \textbf{Give a short overview of what had to be changed!}

\subsection{Used analysis files}
The Transfer Functions are calculated using the simulated $t\bar{t}$ sample which will be used throughout the entire analysis. A very small nTuple is created from this simulated sample containing only the TLorentzVector information of both generated and reconstructed particles. From this the necessary diagrams, such as the 2D distributions of the $p_T$-, $\theta$- and $\eta$-difference between the generated and reconstructed particle with respect to the generated value, are created and saved in the following ROOT file (located on the m-machines \textbf{or also copied locally??}):
\begin{eqnarray*}
 AnomalousCouplings/TFInformation/PlotsForTransferFunctions\_FromTree.root 
\end{eqnarray*}

The main analyzer, called $TFFit.cc$, performs the double Gaussian fit of these 2D histograms and afterwards the $E$-dependent calorimeter fit. The technicalities and specific details of these fit procedures are documented in the $TFCreation$ class which can be found in the $PersonalClasses$ directory.

The results of the two consecutive fits performed on the Y-projections of these 2D distributions are stored in a different ROOT file, together with the original 2D histograms. Also the function form of the double Gaussian fit formula using the obtained fit parameters, added in order to test the robustness of the fit results outside the fitted range, can be found in this ROOT file.
\begin{eqnarray*}
 AnomalousCouplings/TFInformation/CreatedTFFromDistributions\_FromTree.root
\end{eqnarray*}

This analyzer also has the flexibility to perform the fit on the entire range or on pre-defined ranges set by the user. For this a separate function, called \textit{SetFitRange}, is created where for each histogram the fit range for each separate bin can be defined. This is extremely useful to optimize the doubleGaussian fit which has to cover both the peak and the tail of the distributions in order to correctly calculate the $6$ fit parameters.\\
Another useful aspect of this analyzer is the automatic creation of the necessary $.dat$ Transfer Function files needed for implementation in MadWeight. This is done in the \textit{WriteTF} class and the created files are\footnote{Currently two different files are created, one for separate $\eta$ bins and one for all of the events.}:
\begin{eqnarray*}
 AnomalousCouplings/TFInformation/TF\_user.dat \\
 AnomalousCouplings/TFInformation/transfer\_card\_user.dat
\end{eqnarray*}
This first file contains the functional form of both the $E$-dependent calorimeter fit and the functional form of how these $6$ parameters should be included in the double Gaussian formula. Also the width for the different kinematic variables is defined within this file. For the moment the method used in the Transfer Functions already implemented in MadWeight is followed, implying that the width of the Transfer Function is defined as the maximum of the $\sigma$-parameter of the two Gaussians considered in the double Gaussian fit. The only difference is that in stead of the generated kinematic information, which is the variable on the abscissa of the considered 2D histograms, the reconstructed one is used.\\
\\
The second file contains the actual values of the different fit parameters for all the considered kinematic variables and particle types. Therefore this file is an extensive list of values to which is referred in the previous $TF\_user.dat$ file. For each particle type and kinematic variable the numbering used should be unique such that the correct values are implemented in the functional forms of the used fit formulas.

\paragraph{Separating narrow and wide gaussian\\}
In order to differentiate between the narrow and wide gaussian both the amplitude and the $\sigma$-parameter should be compared. However the most important parameter is the latter one, $\sigma$, since this represents the width of the corresponding Gaussian distribution. Nevertheless it is expected that the distribution with the narrowest distribution also has the highest peak.\\
In order to easily compare the two distributions a stacked canvas is added to the ROOT file which shows both distributions for different $p_{T,gen}$ values together. The distinction between narrow and wide gaussian distribution is currently being made by the size of the $\sigma$ variable. Hence the histogram with the narrowest distribution is plotted in red while the widest one is plotted in green.
Such a canvas is made for each of the considered 2D-histograms. They should be analyzed in detail in order to understand the correctness of the double Gaussian fit applied for the creation of the Transfer Functions. For the moment there a still a couple of 2D-histograms which don't show the expected behavior. \textbf{Additional investigation of these stacked canvasses should be performed as soon as possible.}\\

\textbf{\underline{Remark:}} Should check whether this splitting in narrow and wide gaussian is actually necessary for the MadWeight implementation. MadWeight only seems to need the general fit formula and doesn't need to know which of the two distributions is the narrow and which is the wide one.

\paragraph{Importance of start values\\} Since the double Gaussian fit really needs accurate information of both the peak and the tails, detailed review of the start values for each of the $6$ fit parameters significantly improves the success rate of the fitting method. Hence special care should be awarded to ensure the correctness of these start values by comparing the fit distributions for the different particles.% since in the case of light jets and b-jets some similar behavior is expected.\\
After quite a while it is possible to quickly see whether the fit distribution has the expected shape and whether the used start values can be considered as stable. The start values used for the different 2D histograms is given in Table \ref{table::StartValues}.
\begin{table}[h!]
 \centering
 \begin{tabular}{|c|c|c|c|c|c|c|}
  \hline
  \multirow{2}{*}{2D histogram}	& \multicolumn{6}{|c|}{Used start value for fit parameter}	  	\\
				& \multicolumn{3}{|c}{First (narrow) gaussian} 		& \multicolumn{3}{c|}{Second (wide) gaussian}		\\
				&  Mean $a_1$	& Sigma $a_2$ 	& Amplitude $a_3$ 	& Mean $a_4$ 	& Sigma $a_5$ 	& Amplitude $a_6$ 	\\
  \hline
    b-jet $\Delta \phi$ 	& 0.0002	& 0.022		&	8000		& 0.0002	& 0.06		&	3000		\\
    b-jet $\Delta p_T$ 		& 10		& -12		&	20000		& 13		& 10		&	-5000		\\
    b-jet $\Delta \theta$  	& 0		& 0.013		&	6000		& 0		& 0.04		&	2000		\\
  \hline
    light jet $\Delta \phi$  	& 0		& 0.022		&	8000		& 0.0004	& 0.002		&	3000		\\
    light jet $\Delta p_T$  	& 0		& 8		&	4000		& 0		& 12		&	4000		\\
    light jet $\Delta \theta$ 	& 0		& -0.014	&	6000		& 0		& -0.05		&	2000		\\
  \hline 
    electron $\Delta \phi$ 	& 0		& 0.0012	&	1500		& 0		& 0.006		& 	600 		\\
    electron $\Delta p_T$  	& 0		& 0.9		&	1500		& 0		& -2		& 	600		\\
    electron $\Delta \theta$ 	& 0		& 0.0013	&	2500		& 0		& 0.007		& 	600		\\
  \hline
    muon $\Delta \phi$ 		& 0		& 0.0004	&	800		& 0		& 0.0026	& 	600		\\
    muon $\frac{1}{\Delta p_T}$ & 0		& 0.0003	& 	2000		& 0		& 0.0006	&	500		\\
    muon $\Delta \theta$ 	& 0		& 0.002		& 	500		& 0		& 0.0004	&	500		\\
  \hline
 \end{tabular} 
 \caption{caption ... \textbf{Need to make sure that the narrow and wide gaussian is always the same ... Otherwise are the start values not correct for the different eta-bins ...!!} } \label{table::StartValues}
\end{table}

\subsection{Applied $\vert \eta \vert$ binning}
Since the kinematic variables tend do depend on the pseudorapidity $\eta$ the considered 2D histograms are created for four distinctive $\vert \eta \vert$ regions. It has been chosen to split the barrel region into three separate bins while the entire endcap region is contained within one single bin. The chosen binning is given in Table \ref{table::EtaBins} together with the percentage of events present in each of the considered $\vert \eta \vert$ bins. This clearly shows the lower statistics available in the endcap region which results in larger difficulties of properly reconstructing the fit parameters in this region. This is shortly discussed below.
\begin{table}[h!]
 \centering
 \begin{tabular}{|c|c|c|c|c|}
  \hline
			 & $\vert\eta\vert$ $\leq$ 0.375	& 0.375 $<$ $\vert\eta\vert$ $\leq$ 0.75	& 0.75 $<$ $\vert\eta\vert$ $\leq$ 1.45	& 1.45 $<$ $\vert\eta\vert$ $\leq$ 2.5	\\
  \hline
    Relative $\#$ events &  26.21 $\%$				& 23.91 $\%$					&	32.54 $\%$			& 	17.34 $\%$			\\
  \hline
 \end{tabular} 
 \caption{Different $\vert\eta\vert$ bins used for the Transfer Function creation. It is important to note that the $\eta$-values used for this binning are the reconstructed ones since generated values could be higher than the $2.5$ cut-value.} \label{table::EtaBins}
\end{table}

The analyzer mentioned above is developed in such a way that both the fit results for all events as the results for the four separate $\vert \eta \vert$ bins are stored together. Therefore all the results can always be compared in the created ROOT files and in the distinct $.dat$ files.

One important difference between the 2D-distributions containing all events and the 2D-distributions specific for one of the $\vert \eta \vert$ bins is the number of bins used. Since the statistics is significantly lower for the $\vert \eta \vert$ specific histograms, the predefined bin number is lowered with $25 \%$. This ensures a more stable tail for the distribution and still a correct reconstruction of the peak.
However for the last $\vert \eta \vert$ bin considered, the endcap part, a slightly different method is used. Since the statistics is much lower in this part of the detector the used range had to be slightly stretched in order to ensure a nice overview of the tails. This is necessary for the correctness of the double Gaussian fit. Therefore the range of the abscissa is enlarged on both sides with $20 \%$ and the used number of bins for this axis is identical to the one used for the overall 2D-distribution.

\paragraph{Coping with low statistics in last $\vert \eta \vert$ bin\\}
Solution : Excluding bins! + combining bins\\
\textit{Explanations!!}

\subsection{Implemenation in MadWeight -- NEED TO UPDATE!}
All the possible Transfer Functions which are implemented in MadWeight can be found in the $Source/MadWeight/transfer\_function/data$ directory. Any file can be added to this list and used within MadWeight.\\
The relevant files used for the creation of the Transfer Functions are given below. The first one is the translation of the used $TF\_user.dat$ into a MadWeight readable file which can be implemented. The second file is only relevant around line $292$ where the function used for the Transfer Function creation is explained. This is the general MadWeight constructor file where all the different MadWeight functions are defined.
\begin{eqnarray} 
 Source/MadWeight/transfer\_function/transfer\_function.f \nonumber \\
 bin/internal/madweight\_interface.py \nonumber
\end{eqnarray}

\textit{What about $call\_tf.f$ file which has to be changed every time a new TF is initialized!}

The implementation in MadWeight should be done in such a way that the value of the outermost bins is used for all the $p_T$ values outside the fitted region. This is necessary since the extrapolation using the obtained fit parameters doesn't result in the desired double Gaussian behavior for these $p_T$ values. For values outside the fitted region, an inverted double Gaussian distribution or a distribution with two distinct peaks occurs rather often. This can be seen in Figure \ref{fig::doubleGaussExtrap}
\begin{figure}[!h]
  \centering
  \includegraphics[width = 0.9 \textwidth]{/home/annik/GitTopTree/TopBrussels/AnomalousCouplings/TFInformation/FitDistributions/FromTree/BJet_DiffPtVsGenPt/Overview_DblGausDistribution.png}
  \caption{Extrapolation obtained using the fit parameters of the double Gaussian functional form. Values outside the fitted range show a distinctly different shape with respect to the ones actually fitted. \textit{Maybe consider to lower the number of bins and make the titles more visible (Only $p_T$ cut value is really important!)}} \label{fig::doubleGaussExtrap} 
\end{figure}

\section{Obtained distributions}\label{subsec::FitRanges}
Each of the considered histograms is a 2D histogram where the abscissa represents the transverse momentum of the generator level parton and the ordinate the difference between the generator level parton and the reconstructed matched particle. This is done for the difference in transverse momentum and in $\theta$ and $\phi$ angles.\\
All the interesting histograms can be created automatically, for each of the desired $\vert \eta \vert$ bins separately, using the following ROOT analyzer:
\begin{eqnarray*}
 AnomalousCouplings/TFInformation/FitDistributions/SaveFitHistograms.C
\end{eqnarray*}

This analyzer is able to create each of the histograms separately, both as $.pdf$ and $.png$\footnote{$.png$ files are less interesting since with the package $graphicx$ $.pdf$ figures can be included in $LaTeX$.}, and automatically saves all the histograms of one type in a large stacked canvas. This allows to quickly see the used 2D distributions for each of the particle types (b-jets, light jets, electrons and muons) and the three kinematic variables ($p_T$, $\eta$ and $\phi$). This stacked canvas is shown in Figure \ref{fig::ColorPlots}.  Also the general behavior of each of the fitted diagrams together with the overal $\chi^{2}$ distribution is created for each 2D histogram as can be seen from Figure \ref{fig::StackedHistoBJetPt}, showing this for the $p_T$ distribution of the b-jets. Finally the distribution of the $6$ double Gaussian fit parameters together with the fit result of the $E$-dependent calorimeter formula is also collected in a stacked canvas, as can be seen in Figure \ref{fig::FitParamsBJetPt}.
\begin{figure}[!h]
  \centering
  \includegraphics[width = 0.9 \textwidth]{/home/annik/GitTopTree/TopBrussels/AnomalousCouplings/TFInformation/FitDistributions/FromTree/ColorPlots.png}
  \caption{Used 2D distributions for double Gaussian fit. \textbf{IMPROVE CAPTION!}} \label{fig::ColorPlots} 
\end{figure}
The 2D distributions for the difference in transverse momentum tend to show a slightly asymmetric behavior, as can be seen from Figure \ref{fig::ColorPlots}. This can be explained by the influence of the event selection, which has a difference effect on the generated particle than the reconstructed particle. This because a particle surviving the $p_T$ cut actually has a different $p_T$ value on generated level due to \textbf{bad resolution, detector effects (???)}. This effect is almost negligible for the $\phi$ and $\theta$ angles \textbf{(Definitely sure that this is the case ??)}.
\begin{figure}[!h]
  \centering
  \includegraphics[width = 0.9 \textwidth]{/home/annik/GitTopTree/TopBrussels/AnomalousCouplings/TFInformation/FitDistributions/FromTree/BJet_DiffPtVsGenPt/Overview_FitDistributions.png}
  \caption{Distribution of the energy difference between the generator level parton and the corresponding light quark jet for each of the 10 considered bins and the overflow bin. All distributions were fitted with a double Gaussian function.} \label{fig::StackedHistoBJetPt}
\end{figure}

\begin{figure}[!h]
  \centering
  \includegraphics[width = 0.9 \textwidth]{/home/annik/GitTopTree/TopBrussels/AnomalousCouplings/TFInformation/FitDistributions/FromTree/BJet_DiffPtVsGenPt/Overview_FitParameters.png}
  \caption{Energy dependency of the 6 parameters of the double Gaussian fit function. The result of the double Gaussian fit for each $p_T$ bin is combined in a $p_T$ dependent histogram and then fitted with the Calorimeter energy function as explained in the PhD Thesis of Arnaud Pin.} \label{fig::FitParamsBJetPt}
\end{figure}

\section{Control checks for Transfer Functions}

\paragraph{Compare results with normal (single) Gaussian\\}
\textit{Still to do ...}

\paragraph{Compare result with previous analyses\\}
This will imply to put $p_{T,reco}$ information on the abscissa in stead of the current $p_{T,gen}$ one. Also it will result in a huge change of the start values which are however necessary to perform a succesful double Gaussian fit.

\paragraph{Compare results with predefined ROOT class\\}
\textit{Still to do ...}
%**************************************************

%°°°°°°°°°°°°°°°°°°°°°°°°°°°°°°°°°°°°°°°°°°°°°°°°°°°°°°°°°°°°°°°°°°°°°°°°

%°°°°°°°°°°°°°°°°°°°°°°°°°°°°°°°°°°°°°°°°°°°°°°°°°°°°°°°°°°°°°°°°°°°°°°°°
%	CHAPTER: Preliminary Results
\chapter{Preliminary Results}

The goal is to obtain as fast as possible results on generator level in order to ensure that no bottlenecks are found when running MadWeight.
The advantage of only using generator level results is that one can be completely sure that the studied events are properly semi-leptonic ttbar events. Hence MadWeight should not have any problems calculating the weight for these kind of events and no CPU time will be spent on uncorrect events.\\
This implies that any deviation from the expected results implies a bias, or even a problem, concerning the MadWeight output.\\
\\
Once the results correspond to the expectations these preliminary results should be easily extended to reconstructed events. Finalizing the event selection then allows to fully trust the results obtained on reconstructed level and make sure that any deviation should be explained by the influence of the applied event selection.\\
These results can then be used to optimize the event selection with respect to the MadWeight output and CPU time needed.

\section{Results on generator level}
\textit{Need to check the influence of the starting values now that the number of bins have been changed!}

\subsection{Transfer Functions}
In order to obtain results with MadWeight, the Transfer Functions which link the reconstructed energy distribution with the actual energy distributions should be calculated. In the case of generator level events this is not that important since no smearing of the energy is expected, but in order to avoid any bias from the used Transfer Functions it is adviced to use the real Transfer Functions with a slightly changed width.\\
\\
The method to obtain the parameters describing the energy smearing is (partly) explained in the PhD Thesis of Arnaud Pin\footnote{It should be noted that in the PhD Thesis of Arnaud, and in the current double Gaussian transfer function syntax in MadWeight, only 5 parameters are used. The narrow gaussian distribution doesn't have a normalisation parameter in front, but is normalized afterwards in the MadWeight code.} which can be found in:
\begin{eqnarray*}
 & & https://cp3.irmp.ucl.ac.be/upload/theses/phd/Pin\_Arnaud.pdf \\
& & /home/annik/Documents/Vub/PhD/ThesisSubjects/AnomalousCouplings/\\ & & PrepareGenLevelRunning\_Sep2014/TransferFunctions
\end{eqnarray*}

\textit{For the moment (3 October 2014) I'm still waiting on feedback of Arnaud concerning the actual code he used to construct these Transfer Functions and the control checks and histograms he has built for ensuring the correctness of the built Transfer Functions.}\\
Up to now the method to obtain the parameters of the Transfer Functions is based on the code received from Petra and Lieselotte, which was used in the Master thesis of Lieselotte.
This code is actually based on the ROOT class FitSlicesY() but has some small differences. Some of these have already been changed in order to match with the ROOT class.\\
One of the most important differences between the two approaches was the treatment of the underflow and overflow bin. In the personal code these two bins were respectively added to the first and last bin and hence included in the fitting range. This is not the desired behavior since the size of underflow/overflow bin can be relatively large compared to the first/last bin and significantly change the value of these bins. This would then imply that the position of the first and last bin is not located at the correct position and will, especially in the case of a limited number of bins, have a significant influence on the fit result.
Now these underflow and overflow bins are just discarded from the fit range and will have no influence on the final result.\\
\\
Another important, but useful, difference between the two methods is the number of histograms which are saved. The personal code saves for each distribution which is considered the ProjectionY distribution together with the double Gaussian fit for this bin. This can not be changed in the ROOT class which only stores the distribution of the 6 parameters of the double Gaussian fit formula.\\
\\
However, even after carefully ensuring that both methods are identical the obtained results are not.
Up to now it is not clear what is the reason for the discrepancy between the two results and the only way to found out is comparing the distributions and results for a significant amount of statistics.\\
One possibility is the used fit ranges and number of bins. If the last bins are low on statistics their distribution might not agree with a double Gaussian distribution and hence result in a failed fit. Therefore the distribution for each ProjectionY bin is now closely studied for all of the considered histograms. This can be found is the following section \ref{subsubsec::FitRanges}.

\subsubsection{Comparing ProjectionY distributions}\label{subsubsec::FitRanges}
Each of the considered histograms is a 2D histogram where the x-axis represents the energy of the generator level parton and the y-axis the difference between the generator level parton and the reconstructed matched particle. This is done for the energy distribution and the $\theta$ and $\phi$ angles.

\paragraph{\underline{Remark:} Using transverse momentum in stead of energy\\}
The Transfer Function configuration files in MadWeight are flexible enough to change the kinematic variables used for the Transfer Function calculations. Hence it seems to be more relevant to utilize the transverse momentum in stead of the energy of the considered partons and jets. Especially since the LHCO file used in the MadWeight calculations has the transverse momentum as input variable. Therefore it seems much more useful and realistic to use this parameter and not the energy.\\
This implies that the Transfer Function configuration file should be adapted to use the \textit{pt(p)} and \textit{pt(pexp)} variables and not the currently used \textit{p(0)} and \textit{pexp(0)}.

\paragraph{Energy difference between parton and light jets\\}
\begin{figure}[!h]
  \centering
  \includegraphics[width = 0.4 \textwidth]{/home/annik/Documents/Vub/PhD/ThesisSubjects/AnomalousCouplings/PrepareGenLevelRunning_Sep2014/TransferFunctions/Epart_vs_Enonbjet.png}
  \includegraphics[width = 0.4 \textwidth]{/home/annik/Documents/Vub/PhD/ThesisSubjects/AnomalousCouplings/PrepareGenLevelRunning_Sep2014/TransferFunctions/Epart_vs_DiffEpartEnonbjet.png}
  \caption{Energy of the generator parton versus the reconstructed jets for the two light quarks in the semi-muonic ttbar event topology (left) and their respective difference with respect to the generator level energy (right).}
\end{figure}
The first bin in the left-hand histogram is asymetric because of the applied event selection cuts on the reconstructed level. These event selection cuts will not have the same influence on the generator level partons explaining the slight asymetric behavior.
\newpage

\begin{figure}[!h]
  \centering
  \includegraphics[width = 0.3 \textwidth]{/home/annik/Documents/Vub/PhD/ThesisSubjects/AnomalousCouplings/PrepareGenLevelRunning_Sep2014/TransferFunctions/DiffEpartEnonbjet/Epart_vs_DiffEpartEnonbjet_Bin1.png}
  \includegraphics[width = 0.3 \textwidth]{/home/annik/Documents/Vub/PhD/ThesisSubjects/AnomalousCouplings/PrepareGenLevelRunning_Sep2014/TransferFunctions/DiffEpartEnonbjet/Epart_vs_DiffEpartEnonbjet_Bin2.png}
  \includegraphics[width = 0.3 \textwidth]{/home/annik/Documents/Vub/PhD/ThesisSubjects/AnomalousCouplings/PrepareGenLevelRunning_Sep2014/TransferFunctions/DiffEpartEnonbjet/Epart_vs_DiffEpartEnonbjet_Bin3.png}
  \includegraphics[width = 0.3 \textwidth]{/home/annik/Documents/Vub/PhD/ThesisSubjects/AnomalousCouplings/PrepareGenLevelRunning_Sep2014/TransferFunctions/DiffEpartEnonbjet/Epart_vs_DiffEpartEnonbjet_Bin4.png}
  \includegraphics[width = 0.3 \textwidth]{/home/annik/Documents/Vub/PhD/ThesisSubjects/AnomalousCouplings/PrepareGenLevelRunning_Sep2014/TransferFunctions/DiffEpartEnonbjet/Epart_vs_DiffEpartEnonbjet_Bin5.png}
  \includegraphics[width = 0.3 \textwidth]{/home/annik/Documents/Vub/PhD/ThesisSubjects/AnomalousCouplings/PrepareGenLevelRunning_Sep2014/TransferFunctions/DiffEpartEnonbjet/Epart_vs_DiffEpartEnonbjet_Bin6.png}
  \includegraphics[width = 0.3 \textwidth]{/home/annik/Documents/Vub/PhD/ThesisSubjects/AnomalousCouplings/PrepareGenLevelRunning_Sep2014/TransferFunctions/DiffEpartEnonbjet/Epart_vs_DiffEpartEnonbjet_Bin7.png}
  \includegraphics[width = 0.3 \textwidth]{/home/annik/Documents/Vub/PhD/ThesisSubjects/AnomalousCouplings/PrepareGenLevelRunning_Sep2014/TransferFunctions/DiffEpartEnonbjet/Epart_vs_DiffEpartEnonbjet_Bin8.png}
  \includegraphics[width = 0.3 \textwidth]{/home/annik/Documents/Vub/PhD/ThesisSubjects/AnomalousCouplings/PrepareGenLevelRunning_Sep2014/TransferFunctions/DiffEpartEnonbjet/Epart_vs_DiffEpartEnonbjet_Bin9.png}
  \includegraphics[width = 0.3 \textwidth]{/home/annik/Documents/Vub/PhD/ThesisSubjects/AnomalousCouplings/PrepareGenLevelRunning_Sep2014/TransferFunctions/DiffEpartEnonbjet/Epart_vs_DiffEpartEnonbjet_Bin10.png}
  \includegraphics[width = 0.3 \textwidth]{/home/annik/Documents/Vub/PhD/ThesisSubjects/AnomalousCouplings/PrepareGenLevelRunning_Sep2014/TransferFunctions/DiffEpartEnonbjet/Epart_vs_DiffEpartEnonbjet_Bin11.png}
  \includegraphics[width = 0.3 \textwidth]{/home/annik/Documents/Vub/PhD/ThesisSubjects/AnomalousCouplings/PrepareGenLevelRunning_Sep2014/TransferFunctions/DiffEpartEnonbjet/Epart_vs_DiffEpartEnonbjet_Chi2Log.png}
  \caption{Distribution of the energy difference between the generator level parton and the corresponding light quark jet for each of the 10 considered bins and the overflow bin. All distributions were fitted with a double Gaussian function and besides bin 8, 9 and 10 returned with status ``converged''. Bin 9 had status ``failed'' while the remaining two bins had status ``ok''.\\ The final histogram (bottom right) shows the $\chi^{2}$ distribution for this fit for each bin. }
\end{figure}

\begin{figure}[!h]
  \centering
  \includegraphics[width = 0.32 \textwidth]{/home/annik/Documents/Vub/PhD/ThesisSubjects/AnomalousCouplings/PrepareGenLevelRunning_Sep2014/TransferFunctions/DiffEpartEnonbjet/Epart_vs_DiffEpartEnonbjet_Fita1.png}
  \includegraphics[width = 0.32 \textwidth]{/home/annik/Documents/Vub/PhD/ThesisSubjects/AnomalousCouplings/PrepareGenLevelRunning_Sep2014/TransferFunctions/DiffEpartEnonbjet/Epart_vs_DiffEpartEnonbjet_Fita2.png}
  \includegraphics[width = 0.32 \textwidth]{/home/annik/Documents/Vub/PhD/ThesisSubjects/AnomalousCouplings/PrepareGenLevelRunning_Sep2014/TransferFunctions/DiffEpartEnonbjet/Epart_vs_DiffEpartEnonbjet_Fita3.png}
  \includegraphics[width = 0.32 \textwidth]{/home/annik/Documents/Vub/PhD/ThesisSubjects/AnomalousCouplings/PrepareGenLevelRunning_Sep2014/TransferFunctions/DiffEpartEnonbjet/Epart_vs_DiffEpartEnonbjet_Fita4.png}
  \includegraphics[width = 0.32 \textwidth]{/home/annik/Documents/Vub/PhD/ThesisSubjects/AnomalousCouplings/PrepareGenLevelRunning_Sep2014/TransferFunctions/DiffEpartEnonbjet/Epart_vs_DiffEpartEnonbjet_Fita5.png}
  \includegraphics[width = 0.32 \textwidth]{/home/annik/Documents/Vub/PhD/ThesisSubjects/AnomalousCouplings/PrepareGenLevelRunning_Sep2014/TransferFunctions/DiffEpartEnonbjet/Epart_vs_DiffEpartEnonbjet_Fita6.png}
  \includegraphics[width = 0.32 \textwidth]{/home/annik/Documents/Vub/PhD/ThesisSubjects/AnomalousCouplings/PrepareGenLevelRunning_Sep2014/TransferFunctions/DiffEpartEnonbjet/Epart_vs_DiffEpartEnonbjet_Fita4Zoom.png}
  \includegraphics[width = 0.32 \textwidth]{/home/annik/Documents/Vub/PhD/ThesisSubjects/AnomalousCouplings/PrepareGenLevelRunning_Sep2014/TransferFunctions/DiffEpartEnonbjet/Epart_vs_DiffEpartEnonbjet_Fita5Zoom.png}
  \includegraphics[width = 0.32 \textwidth]{/home/annik/Documents/Vub/PhD/ThesisSubjects/AnomalousCouplings/PrepareGenLevelRunning_Sep2014/TransferFunctions/DiffEpartEnonbjet/Epart_vs_DiffEpartEnonbjet_Fita6Zoom.png}
  \caption{Energy dependency of the 6 parameters of the double Gaussian fit function. The two lowest rows show twice the same result, but with a different y-axis range. The double Gaussian fit parameters are combined in a histogram and then fitted with the Calorimeter energy function as explained in the PhD Thesis of Arnaud Pin.}
\end{figure}

From these fit results can be concluded that the double Gaussian distribution is well recovered for lower energy values. As can be seen from the very first histogram, the largest bulk of events have an energy value lower than 100 GeV. The low statistics in the higher energy range can partly explain the failing fitting performance. The double Gaussian function form is used in order to properly reconstruct the narrow peak together with the wider Gaussian distribution representing the tail of the energy distribution. However in the high parton energy range, this first narrow distribution dissapeared due to low statistics in this region. Hence fitting these distributions will indeed result in an unsuccesful fit status.\\
\\
A possible solution which can be considered is adding some of these histograms together and in this way enhancing the statistics in these combined bins. From the collection of distributions per bin could be concluded that the ``ok'' and ``failed'' fits are the ones with only about 2$\%$ of the total number of events. Hence it could be possible to adapt the code in such a way that in these cases the bin is combined with the following ones until a percentage higher than 3$\%$ is obtained.\\
Another solution consists of reducing the range of the fitted histograms and only fit the distributions with $E_{parton}$ $<$ 150GeV. However before continuing with this option it should be completely understood whether MadWeight is able to extrapolate in a correct way to higher energy values.
%**************************************************

\newpage
\subsection{Cross Section distribution for new grid}
%**************************************************

\subsection{First results: Wrong log(likelihood) minimum}

The first obtained MadWeight results for the enlarged grid ($V_L$ $\in$ $[0.8,1.2]$ and $V_R$ $\in$ $[-1,1]$) using only parton-level ttbar events did not result in the expected minimum of $(V_L,V_R)$ = $(1, 0)$. This can be seen from Figure \ref{fig::Likelihood}, which shows the distribution of the log(likelihood) for each point in the considered grid.\\ \\
\begin{figure}[!h]
 \centering
 \includegraphics[width = 0.9 \textwidth]{/home/annik/Documents/Vub/PhD/ThesisSubjects/AnomalousCouplings/UnderstandLikelihoodDistr_July2014/AnomCouplings_GenEvent_NoSelect_Oct3200Events_SingleGaussUsed/Likelihood_NoXSNorm.png}
 \caption{Distribution of the log(likelihood) for each point in the grid using 3200 parton-level positive semi-muonic ttbar events. The transfer function used to smear the parton-level kinematics is the single-gaussian function standard included in MadWeight.}
 \label{fig::Likelihood}
\end{figure}
One of the possible influences on the displaced minimim of the log(likelihood) distribution could be the normalisation of the Cross Section influence. This XS normalisation ($\frac{XS}{XS^{SM}}$) should be multiplied with the likelihood value, not the log(likelihood). Hence in order to correctly take this into account the obtained log(likelihood) value for each point in the grid should be corrected using the logarithm of this XS normalisation. The distribution of the normalisation on the Cross Section can be found in Figure \ref{fig::XSandLikelihoodNorm} together with the log(likelihood) distribution after correctly taking into account this XS normalisation.\\ \\
\begin{figure}[!h]
 \includegraphics[width = 0.45 \textwidth]{/home/annik/Documents/Vub/PhD/ThesisSubjects/AnomalousCouplings/UnderstandLikelihoodDistr_July2014/AnomCouplings_GenEvent_NoSelect_Oct3200Events_SingleGaussUsed/XSNorm.png}
 \includegraphics[width = 0.45 \textwidth]{/home/annik/Documents/Vub/PhD/ThesisSubjects/AnomalousCouplings/UnderstandLikelihoodDistr_July2014/AnomCouplings_GenEvent_NoSelect_Oct3200Events_SingleGaussUsed/Likelihood_XSNorm.png}
 \caption{Distribution of the XS normalisation for positive semi-muonic ttbar events (left) and distribution of the log(likelhood) after taking into account this normalisation. As in the previous figure 3200 positive semi-muonic have been used to obtain this distribution and a single gaussian transfer function has been applied to smear the kinematics of these parton-level events.}
 \label{fig::XSandLikelihoodNorm}
\end{figure}
The formula which has been used is the following (Equation \ref{eq::ProbMW}):
\begin{eqnarray}
 P(y \vert a) = \frac{1}{\textcolor{red}{\sigma(a)}*Acc(a)} \int W(y|x,a)*Eff(x,a) \vert M(x,a) \vert^{2} T(x,a) dx \label{eq::ProbMW}\\
 \mathcal{L} = \prod P(y \vert a)
\end{eqnarray}
The normalisation which has been applied in Figure \ref{fig::XSandLikelihoodNorm} is given in Equation \ref{eq::LikelihoodNorm}:
\begin{equation}\label{eq::LikelihoodNorm}
 \mathcal{L}_{Norm} = - ln(\sum P(y \vert a)*\frac{XS}{XS^{SM}}) = -ln(\mathcal{L}) - ln(\frac{XS}{XS^{SM}}*N)
\end{equation}
\textbf{It should be checked whether this is the correct method to take into account the normalisation of the XS. Currently it has been assumed that this XS normalisation should be applied for each weight and hence is multiplied with the number of considered events. In the case that this normalisation should just be multiplied with the overall likelihood value ($\mathcal{L}$) the sum over the number of considered events drops out of the equation implying a very small influence of the XS on the log(likelihood) distribution.}\\ \\

As highlighted in the general Matrix Element Method formula (Equation \ref{eq::ProbMW}), the probability to measure the observed quantities $y$ already has a normalisation factor for the cross section. This factor is defined as the channel cross section and is calculated using Equation \ref{eq::ChannelXSMW}:\\ \\
\begin{equation}\label{eq::ChannelXSMW}
 \sigma(a) = \int_{X_i} \vert M(x,a) \vert^{2} T(x,a) dx
\end{equation}
\textbf{Hence it should be investigated in detail whether this XS normalisation should still be applied. From the above equations could be concluded that the change in cross section is actually already incorporated in the weight obtained from MadWeight. This could make sense since MadWeight has all the necessary information to calculate the cross section for each point in the considered grid. The cross section values for each point have been calculated using MadGraph and the same model as used for the MadWeight calculations. Unfortunately it is not completely clear from the MadWeight documentation whether this is actually included in the weight or not.}

\subsection{Correct normalisation of Matrix Element probability}
As could be seen in Equation \ref{eq::ProbMW} a term $\sigma(a)$ is included in the general Matrix Element Techniques formula. However it is not clear whether this cross section normalisation is actually performed within the MadWeight calculations or whether this normalisation should be done afterwards.\\
This question is closely related to the order of the current obtained weight values with MadWeight. Up to now no weight larger than $10^{-22}$ have been obtained, resulting in a very large log(likelihood) value. \\
\\
This small value can be caused by many different reasons for which the most plausible ones are listed here:
\begin{itemize}
 \item The normalisation of the MadWeight probability should still be done and is not performed within the Matrix Element Techniques formula.\\ \textbf{This can only be ruled out by contacting the Madweight experts and asking explicitely what is done in the Madweight calculations. Also a possible hint could be found inside the MadWeight python files (but this should only be done if the received answer is not perfectly clear).}
 \item The smallness of the weight could be caused by an error inside the created FeynRules model. \textit{Should also look for the mail where one of the MadWeight experts (Olivier/Pierre or even Celine) answered about the possible explanation for the smallness of the weight and whether this implies some wrong assumptions).}\\ \textbf{A possible way to exclude that the origin of this problem is the AnomalousCouplings FeynRules model is by comparing the results for the top mass fit when both the SM FeynRules model and the AnomalousCouplings model is used. If the weights are also this small when the SM model is used this smallness should be solved by an additional normalisation factor.}
\end{itemize}

\paragraph{Update 31/10/2014: Probability function NOT normalized (according to mail Olivier)\\}
As was expected from the smallness of the obtained MadWeight probabilities should the cross section normalisation be applied afterwards. Only in the older versions of MadWeight (based on MG) was this normalisation included automatically.

\subsubsection{Measurement of top quark mass using Matrix Element Method}

\paragraph{Comparing SM model with AnomalousCouplings model\\}
As a first step the Feynmann diagrams belonging to the two different models should be compared. This information can be found in the \textit{index.html} file in the following directories (and the files should be opened using firefox on mtop since this is the only m-machine with a working browser):
\begin{eqnarray*}
 \tiny{/AnomalousCouplings/MadGraph5\_aMC@NLO/madgraph5/SM\_ttbarSemiMuPlus} \\
 \tiny{/AnomalousCouplings/MadGraph5\_aMC@NLO/madgraph5/ttbarSemiMuPlus\_QED2}
\end{eqnarray*}

\paragraph{Comparing SM cross section with MassiveLeptons cross section\\}
In order to be sure that both models have the same Standard Model base, the cross sections for both models have been compared. This resulted in an unexpected outcome, namely that the obtained cross sections differ significantly depending on which MadGraph version is used to generate the considered events. A summary can be found in Table \ref{table::MGXS}.
\begin{table}[h!]
 \centering
 \begin{tabular}{|c|c|c|c|c|c|}
  \hline
  \multirow{2}{*}{Top quark mass}	&  \multicolumn{2}{|c|}{MadGraph aMC@NLO}	& \multicolumn{2}{|c|}{MadGraph v155}  	\\
					&  SM model	& MassiveLeptons model		& SM model 	& MassiveLeptons model	\\
  \hline
    153 				& $9.23$ pb	& $9.645$ pb			& $6.692$ pb	& $6.984$ pb		\\
    163					& $11.12$ pb	& $11.63$ pb			& $7.844$ pb	& $8.199$ pb		\\
    173					& $12.98$ pb	& $13.54$ pb			& $8.897$ pb	& $9.281$ pb		\\
    183					& $14.77$ pb	& $15.4$ pb			& $9.884$ pb	& $10.3$ pb		\\
    193					& $16.5$ pb	& $17.22$ pb			& $10.78$ pb	& $11.25$ pb		\\
  \hline 
 \end{tabular} 
 \caption{Cross section values for semi-muonic (+) ttbar decay obtained using two different MadGraph versions.} \label{table::MGXS}
\end{table}

From this table can be seen that there is, for both considered MadGraph versions, a small difference between the SM FeynRules model and the MassiveLeptons one. This could be caused by the different treatment of the leptons. In the SM model they are considered to be massless while in the MassiveLeptons one they are defined to have their actual mass.\\
A larger differrence occurs when both MadGraph versions are compared. From the answer received by Olivier it is not clear whether this difference is worrysome or could be explained by the LO theoretical uncertainties. Should also be investigated whether this difference is related to the NLO behavior of the newest MadGraph version. In case the MadGraph v155 version is not up to NLO a difference in cross section is definitely expected.

%**************************************************

\section{Results on reconstructed level}
%**************************************************

\section{Background influence}

%°°°°°°°°°°°°°°°°°°°°°°°°°°°°°°°°°°°°°°°°°°°°°°°°°°°°°°°°°°°°°°°°°°°°°°°°

%°°°°°°°°°°°°°°°°°°°°°°°°°°°°°°°°°°°°°°°°°°°°°°°°°°°°°°°°°°°°°°°°°°°°°°°°
%	CHAPTER: Understanding Results Obtained With MadWeight
\chapter{Understanding MadWeight Results}

\section{Comparing the two used MadWeight versions}
Since it was found that the latest MadWeight version (aMC@NLO) resulted in many events with weight equal to $0$ it was decided to compare this version with the previous one (mc$\_$perm). It is expected that both versions result in similar weights when identical events are considered, otherwise the used version of MadWeight would have a too large influence on the analysis result.\\
Therefore two events which could succesfully run in the newest MadWeight version were also calculated using the old version. The obtained weights and their uncertainty can be compared in the following table and all the relevant information can be found in:
\begin{eqnarray*}
 /home/annik/Documents/Vub/PhD/ThesisSubjects/AnomalousCouplings\\ /CompareMWVersions\_May2014
\end{eqnarray*}

\begin{table}[h!]
 \centering
 \begin{tabular}{|c|c|c|c|c|c|}
  \hline
  \multirow{2}{*}{Event Number} &  \multirow{2}{*}{$V_L$ value}	& \multicolumn{2}{|c|}{aMC@NLO version}		& \multicolumn{2}{|c|}{mc$\_$perm version}  	\\
				&  				& Weight		& Uncertainty		& Weight 		& Uncertainty  		\\
  \hline
  \multirow{3}{*}{1} 		& 1.5 				& $9.76 10^{-28}$	& $4.07 10^{-30}$	& $1.44 10^{-26}$	& $4.19 10^{-29}$	\\
				& 1.0				& $1.92 10^{-28}$	& $8.04 10^{-31}$	& $2.84 10^{-27}$	& $8.28 10^{-30}$	\\
				& 0.5				& $1.20 10^{-29}$	& $5.03 10^{-32}$	& $1.78 10^{-28}$	& $5.17 10^{-31}$	\\
  \hline 
  \multirow{3}{*}{2}	 	& 1.5 				& $1.86 10^{-23}$	& $1.15 10^{-25}$	& $1.77 10^{-24}$	& $1.23 10^{-26}$	\\
				& 1.0				& $3.65 10^{-24}$	& $2.25 10^{-26}$	& $3.57 10^{-25}$	& $2.75 10^{-27}$	\\
				& 0.5				& $2.27 10^{-25}$	& $1.47 10^{-27}$	& $2.23 10^{-26}$	& $1.72 10^{-28}$	\\
  \hline 
 \end{tabular} 
 \caption{Weight obtained from MadWeight for two specific ttbar semi-muonic (+) events. For these events the $V_R$ was fixed to its Standard Model expectation value, which is $0$, while the $V_L$ value was varied.} 
\end{table}

Comparing these values clearly shows that there is a significant difference between the two MadWeight versions which were considered in this analysis. With some effort a general difference of a facto 10 can be identified between the two versions, with a higher weight value for the older mc$\_$perm MadWeight version.\\
However when showing the relative differences between the different weights it can be seen that the behavior of these two MadWeight versions is actually very similar. Therefore the histograms below give firstly the actual weight value and secondly the weight value normalised to the weight corresponding to the coupling parameter $V_L$ = 0.
\begin{figure}[!h]
\includegraphics[width = 0.48 \textwidth]{/home/annik/Documents/Vub/PhD/ThesisSubjects/AnomalousCouplings/CompareMWVersions_May2014/FirstEvtCanvas.png}
\includegraphics[width = 0.48 \textwidth]{/home/annik/Documents/Vub/PhD/ThesisSubjects/AnomalousCouplings/CompareMWVersions_May2014/SecondEvtCanvas.png}\\
\includegraphics[width = 0.48 \textwidth]{/home/annik/Documents/Vub/PhD/ThesisSubjects/AnomalousCouplings/CompareMWVersions_May2014/FirstEvtCanvas_Rel.png}
\includegraphics[width = 0.48 \textwidth]{/home/annik/Documents/Vub/PhD/ThesisSubjects/AnomalousCouplings/CompareMWVersions_May2014/SecondEvtCanvas_Rel.png}
\caption{Distribution of the weights obtained from MadWeight for the two considered MadWeight versions (aMC@NLO and mc$\_$perm) for two specific ttbar semi-muonic (+) events.}
\end{figure}

The above histograms clearly indicate that however the actual values of the weight significantly differ, the normalized results are almost identical.\\
\textbf{This implies that the results obtained with MadWeight should always be normalized with respect to another MadWeight result} (obtained using the same MadWeight version of course). 
%**************************************************

\section{Individual weight distribution for condsidered gridpoints}
The first Likelihood distribution for the considered gridpoints gave rise to an unexpected result. In order to understand whether this strange behavior can be explained by a couple of events with a bad weight, the indiviudual weight distribution for a couple random events was studied. If these distributions indeed result in the correct behavior the events which influence the overal Likelihood distribution can be studied individually.\\
All the relevant information can be found in the following directory:
\begin{eqnarray*}
 /home/annik/Documents/Vub/PhD/ThesisSubjects/AnomalousCouplings/ \\ UnderstandLikelihoodDistr\_July2014
\end{eqnarray*}
And the creation of the MadWeight weights together with the python scripts making the corresponding histograms can be found in:
\begin{eqnarray*}
 /localgrid/aolbrech/madweight/ttbarSemiMuPlus\_QED2/Events
\end{eqnarray*}
The relevant python scripts are the following:
\begin{itemize}
 \item \textbf{CalculateLikelihood.py} which simultaneously creates the histograms for the XS distribution, the raw and normalized Likelihood distribution, and if required the individual weight distributions for the selected events. All these histograms are saved in the Histos.root file.
 \item \textbf{RemoveZeroWeightEvents.py} which removes the events with weight equal to zero from the list and saves the non-zero events in a new .out file. On the other hand the events which failed the MadWeight computation are saved on a new .lhco file together with one succesful control event and are send again through MadWeight for a new weight calculation. \textcolor{red}{This should be updated since the new computation of MadWeight also results in weights equal to zero for these failing events. So should be investigated what is different about these events ...}
\end{itemize}

The first three histograms show the obtained Likelihood distribution for the different gridpoints which were considered. From these can be concluded that the behavior of the Likelihood normalized with the corresponding cross section divided by the Standard Model cross section is dominated by the distribution of the normalized cross section values. This means that any small deviations of the raw Likelihood values gets washed out by the multiplication with the normalized cross section values.
\begin{figure}[!h]
\includegraphics[width = 0.32 \textwidth]{/home/annik/Documents/Vub/PhD/ThesisSubjects/AnomalousCouplings/UnderstandLikelihoodDistr_July2014/NormalizedLikelihoodForSMXS.png}
\includegraphics[width = 0.32 \textwidth]{/home/annik/Documents/Vub/PhD/ThesisSubjects/AnomalousCouplings/UnderstandLikelihoodDistr_July2014/NormalizedXS.png}
\includegraphics[width = 0.32 \textwidth]{/home/annik/Documents/Vub/PhD/ThesisSubjects/AnomalousCouplings/UnderstandLikelihoodDistr_July2014/LikelihoodForSMXS.png}
\caption{...}
\end{figure}

The individual weight distribution for some random events can be found here.
\begin{figure}[!h]
\includegraphics[width = 0.32 \textwidth]{/home/annik/Documents/Vub/PhD/ThesisSubjects/AnomalousCouplings/UnderstandLikelihoodDistr_July2014/WeightIndividualEventNr10.png}
\includegraphics[width = 0.32 \textwidth]{/home/annik/Documents/Vub/PhD/ThesisSubjects/AnomalousCouplings/UnderstandLikelihoodDistr_July2014/WeightIndividualEventNr45.png}
\includegraphics[width = 0.32 \textwidth]{/home/annik/Documents/Vub/PhD/ThesisSubjects/AnomalousCouplings/UnderstandLikelihoodDistr_July2014/WeightIndividualEventNr120.png}
\caption{...\textbf{Script should be ran again, and names for the axis should be added for these individual weight histograms!}}
\end{figure}

%°°°°°°°°°°°°°°°°°°°°°°°°°°°°°°°°°°°°°°°°°°°°°°°°°°°°°°°°°°°°°°°°°°°°°°°°

%°°°°°°°°°°°°°°°°°°°°°°°°°°°°°°°°°°°°°°°°°°°°°°°°°°°°°°°°°°°°°°°°°°°°°°°°
%	CHAPTER: Analyzing created FeynRules model
\chapter{Analyzing FeynRules model}

\section{Normalized coupling parameters}
In order to investigate the actual influence of the value of the coupling parameters on the kinematics of the event, the considered coupling parameters should be normalized to unitarity before any hard conclusions can be made.
Therefore the configurations which should be investigated in large detail are the ones for which the width of the decay remains unchanged. This is explained in detail in one of the previous sections (\ref{subsec:DecayWidth} on page \pageref{subsec:DecayWidth}).
%**************************************************

\section{Understanding parameters larger than 1}
Before starting to look at the ttbar Monte Carlo and reweighting the events with MadWeight, the created model in FeynRules should be completely understood.
Special care goes out to the behavior of the kinematic distributions for values of the coupling parameters larger than 1. Since the Standard Model expectation puts the real part of the left-handed vector coupling $V_L$ almost equal to 1, simulation should be available around this Standard Model expectation value.
Therefore the created model should be able to cope with coupling parameters larger than 1. \\
\\
For this reason .lhco files were generated with MadGraph with the following configuration, as mentioned in \ref{subsec:MadGraphFiles}:
\begin{eqnarray*}
  Re(V_L) & \in & \left[  0.7, \; 1.3\right] \\
  Re(V_R) & \in & \left[ -0.3, \; 0.3\right]
\end{eqnarray*}
For these generated events the main kinematic distributions have been investigated.
No clear difference between the behavior below and above 1 has been found.

\newpage
\subsection{Performed checks}
\subsubsection{Cross section change}
\begin{table}[h!] 
 \begin{tabular}{c|c c c c c c c} 
  RVR &  -0.30  &  -0.20  &  -0.10  &  0.00  &  0.10  &  0.20  &   0.30  \\  
  RVL & & & & & & & \\ 
  \hline 
  0.70  & 0.3775 &  0.3120  & 0.2724  & 0.3275  & 0.2632  & 0.2910 &  0.3436  \\ 
  0.80  & 0.5964 &  0.5097  & 0.4595  & 0.4385  & 0.4444  & 0.4816 &  0.5471  \\ 
  0.90  & 0.9011 &  0.7950  & 0.7308  & 0.7026  & 0.7103  & 0.3229 &  0.8356  \\ 
  1.00  & 1.3187 &  1.1874  & 1.1085  & 1.0711  & 1.0823  & 1.1317 &  1.2263  \\ 
  1.10  & 1.8700 &  1.7116  & 1.6154  & 1.5669  & 1.5763  & 1.6335 &  1.7522  \\ 
  1.20  & 2.5858 &  2.3996  & 2.2789  & 2.2200  & 2.2263  & 2.2983 &  2.4322  \\ 
  1.30  & 3.4896 &  3.2711  & 3.1278  & 3.0626  & 3.0655  & 3.1506 &  3.2983  \\ 
 \end{tabular} 
 \caption{Cross sections for the different RVR-RVL couplings normalized to the SemiElMinus Standard Model Cross section (8.261 pb)} 
\end{table}
From this table can be seen that the cross section increases when the real component of $V_L$ gets larger. The value of the right-handed vector coupling has only a minor influence on the cross section.

\subsubsection{Relative increase visible in XS, but not in kinematic distributions}
Since the observed model cannot represent physics at values larger than 1, one option is to look at specific fixed values of the real part of the left-handed and right-handed vector couplings. 
A proportional change in both of these coupling parameters should change the cross section values, but the kinematic should remain unchanged. Therefore the following configurations will be investigated:
\begin{eqnarray*}
  Re(V_L) = 0.5 ~~~ \& ~~~ Re(V_R) = 0.5 \rightarrow ~2.07115~ pb\\
  Re(V_L) = 1.0 ~~~ \& ~~~ Re(V_R) = 1.0 \rightarrow ~33.1479~ pb\\
  Re(V_L) = 2.0 ~~~ \& ~~~ Re(V_R) = 2.0 \rightarrow ~530.027~ pb
\end{eqnarray*}
From the above numbers is clear that the Cross section becomes very large when the two coupling parameters increase.
This can be understood quite easily since the second option allows much more decay options since the top quarks can decay both through the left-handed and the right-handed vector coupling side of the interaction vertex. The width of this configuration is not equal to the width of the Standard Model expectation and hence does not correspond to an actual physical solution. It is merely seen as a test of the model since the kinematics of the interaction should not differ.\\
\\
Looking at these plots clearly indicates that the kinematics doesn't change at all.\\
Hence the created FeynRules model is able to deal in a correct way with these coupling parameters larger than 1.\\
\\
These MadGraph files have been created and can be found in:
\begin{eqnarray*}
  /user/aolbrech/AnomalousCouplings/MadGraph\_v155/MassiveLeptons/\\ MadGraph5\_v1\_5\_5/Wtb\_ttbarSemiElMinus/RelativeChange
\end{eqnarray*}

The distributions shown in this subsection are for fixed Jet Pt Cut value, set to 0. Also no Pt cut on the lepton was applied. Both coupling parameters have been changed proportionally.
\begin{center}
\includegraphics[width = 0.32 \textwidth]{../April2014/KinematicPlots_RelativeChange/StackCanvas_CosTheta_JetPt0.png}
\includegraphics[width = 0.32 \textwidth]{../April2014/KinematicPlots_RelativeChange/StackCanvas_EventContent_JetPt0.png}
\includegraphics[width = 0.32 \textwidth]{../April2014/KinematicPlots_RelativeChange/StackCanvas_HadronicBMass_JetPt0.png}
\includegraphics[width = 0.32 \textwidth]{../April2014/KinematicPlots_RelativeChange/StackCanvas_HadronicBPt_JetPt0.png}
\includegraphics[width = 0.32 \textwidth]{../April2014/KinematicPlots_RelativeChange/StackCanvas_HadronicBR_JetPt0.png}
\includegraphics[width = 0.32 \textwidth]{../April2014/KinematicPlots_RelativeChange/StackCanvas_HadronicBTheta_JetPt0.png}
\includegraphics[width = 0.32 \textwidth]{../April2014/KinematicPlots_RelativeChange/StackCanvas_LeptonicBMass_JetPt0.png}
\includegraphics[width = 0.32 \textwidth]{../April2014/KinematicPlots_RelativeChange/StackCanvas_LeptonicBPt_JetPt0.png}
\includegraphics[width = 0.32 \textwidth]{../April2014/KinematicPlots_RelativeChange/StackCanvas_LeptonicBR_JetPt0.png}
\includegraphics[width = 0.32 \textwidth]{../April2014/KinematicPlots_RelativeChange/StackCanvas_LeptonicBTheta_JetPt0.png}
\includegraphics[width = 0.32 \textwidth]{../April2014/KinematicPlots_RelativeChange/StackCanvas_LeptonId_JetPt0.png}
\includegraphics[width = 0.32 \textwidth]{../April2014/KinematicPlots_RelativeChange/StackCanvas_LeptonMass_JetPt0.png}
\includegraphics[width = 0.32 \textwidth]{../April2014/KinematicPlots_RelativeChange/StackCanvas_LeptonPt_JetPt0.png}
\includegraphics[width = 0.32 \textwidth]{../April2014/KinematicPlots_RelativeChange/StackCanvas_LeptonR_JetPt0.png}
\includegraphics[width = 0.32 \textwidth]{../April2014/KinematicPlots_RelativeChange/StackCanvas_LeptonTheta_JetPt0.png}
\includegraphics[width = 0.32 \textwidth]{../April2014/KinematicPlots_RelativeChange/StackCanvas_LightAntiQuarkId_JetPt0.png}
\includegraphics[width = 0.32 \textwidth]{../April2014/KinematicPlots_RelativeChange/StackCanvas_LightAntiQuarkMass_JetPt0.png}
\includegraphics[width = 0.32 \textwidth]{../April2014/KinematicPlots_RelativeChange/StackCanvas_LightAntiQuarkPt_JetPt0.png}
\includegraphics[width = 0.32 \textwidth]{../April2014/KinematicPlots_RelativeChange/StackCanvas_LightAntiQuarkR_JetPt0.png}
\includegraphics[width = 0.32 \textwidth]{../April2014/KinematicPlots_RelativeChange/StackCanvas_LightAntiQuarkTheta_JetPt0.png}
\includegraphics[width = 0.32 \textwidth]{../April2014/KinematicPlots_RelativeChange/StackCanvas_LightQuarkId_JetPt0.png}
\includegraphics[width = 0.32 \textwidth]{../April2014/KinematicPlots_RelativeChange/StackCanvas_LightQuarkMass_JetPt0.png}
\includegraphics[width = 0.32 \textwidth]{../April2014/KinematicPlots_RelativeChange/StackCanvas_LightQuarkPt_JetPt0.png}
\includegraphics[width = 0.32 \textwidth]{../April2014/KinematicPlots_RelativeChange/StackCanvas_LightQuarkR_JetPt0.png}
\includegraphics[width = 0.32 \textwidth]{../April2014/KinematicPlots_RelativeChange/StackCanvas_LightQuarkTheta_JetPt0.png}
\includegraphics[width = 0.32 \textwidth]{../April2014/KinematicPlots_RelativeChange/StackCanvas_TopMass_JetPt0.png}
\includegraphics[width = 0.32 \textwidth]{../April2014/KinematicPlots_RelativeChange/StackCanvas_TopProductionId_JetPt0.png}
\includegraphics[width = 0.32 \textwidth]{../April2014/KinematicPlots_RelativeChange/StackCanvas_WBosonMass_JetPt0.png}
\end{center}

\subsubsection{Model plots for fixed Pt Cut}
The distributions shown in this subsection are for fixed Jet Pt Cut value, set to 0. Also no Pt cut on the lepton was applied. In this case the real part of the left-handed vector coupling has been varied between $0.7$ and $1.3$ in steps of $0.1$ while the right-handed parameter has been fixed to its Standard Model value ($0.0$).
\begin{center}
\includegraphics[width = 0.32 \textwidth]{../April2014/KinematicPlots_CoeffLargerThan1/StackCanvas_CosTheta_JetPt0.png}
\includegraphics[width = 0.32 \textwidth]{../April2014/KinematicPlots_CoeffLargerThan1/StackCanvas_EventContent_JetPt0.png}
\includegraphics[width = 0.32 \textwidth]{../April2014/KinematicPlots_CoeffLargerThan1/StackCanvas_HadronicBMass_JetPt0.png}
\includegraphics[width = 0.32 \textwidth]{../April2014/KinematicPlots_CoeffLargerThan1/StackCanvas_HadronicBPt_JetPt0.png}
\includegraphics[width = 0.32 \textwidth]{../April2014/KinematicPlots_CoeffLargerThan1/StackCanvas_HadronicBR_JetPt0.png}
\includegraphics[width = 0.32 \textwidth]{../April2014/KinematicPlots_CoeffLargerThan1/StackCanvas_HadronicBTheta_JetPt0.png}
\includegraphics[width = 0.32 \textwidth]{../April2014/KinematicPlots_CoeffLargerThan1/StackCanvas_LeptonicBMass_JetPt0.png}
\includegraphics[width = 0.32 \textwidth]{../April2014/KinematicPlots_CoeffLargerThan1/StackCanvas_LeptonicBPt_JetPt0.png}
\includegraphics[width = 0.32 \textwidth]{../April2014/KinematicPlots_CoeffLargerThan1/StackCanvas_LeptonicBR_JetPt0.png}
\includegraphics[width = 0.32 \textwidth]{../April2014/KinematicPlots_CoeffLargerThan1/StackCanvas_LeptonicBTheta_JetPt0.png}
\includegraphics[width = 0.32 \textwidth]{../April2014/KinematicPlots_CoeffLargerThan1/StackCanvas_LeptonId_JetPt0.png}
\includegraphics[width = 0.32 \textwidth]{../April2014/KinematicPlots_CoeffLargerThan1/StackCanvas_LeptonMass_JetPt0.png}
\includegraphics[width = 0.32 \textwidth]{../April2014/KinematicPlots_CoeffLargerThan1/StackCanvas_LeptonPt_JetPt0.png}
\includegraphics[width = 0.32 \textwidth]{../April2014/KinematicPlots_CoeffLargerThan1/StackCanvas_LeptonR_JetPt0.png}
\includegraphics[width = 0.32 \textwidth]{../April2014/KinematicPlots_CoeffLargerThan1/StackCanvas_LeptonTheta_JetPt0.png}
\includegraphics[width = 0.32 \textwidth]{../April2014/KinematicPlots_CoeffLargerThan1/StackCanvas_LightAntiQuarkId_JetPt0.png}
\includegraphics[width = 0.32 \textwidth]{../April2014/KinematicPlots_CoeffLargerThan1/StackCanvas_LightAntiQuarkMass_JetPt0.png}
\includegraphics[width = 0.32 \textwidth]{../April2014/KinematicPlots_CoeffLargerThan1/StackCanvas_LightAntiQuarkPt_JetPt0.png}
\includegraphics[width = 0.32 \textwidth]{../April2014/KinematicPlots_CoeffLargerThan1/StackCanvas_LightAntiQuarkR_JetPt0.png}
\includegraphics[width = 0.32 \textwidth]{../April2014/KinematicPlots_CoeffLargerThan1/StackCanvas_LightAntiQuarkTheta_JetPt0.png}
\includegraphics[width = 0.32 \textwidth]{../April2014/KinematicPlots_CoeffLargerThan1/StackCanvas_LightQuarkId_JetPt0.png}
\includegraphics[width = 0.32 \textwidth]{../April2014/KinematicPlots_CoeffLargerThan1/StackCanvas_LightQuarkMass_JetPt0.png}
\includegraphics[width = 0.32 \textwidth]{../April2014/KinematicPlots_CoeffLargerThan1/StackCanvas_LightQuarkPt_JetPt0.png}
\includegraphics[width = 0.32 \textwidth]{../April2014/KinematicPlots_CoeffLargerThan1/StackCanvas_LightQuarkR_JetPt0.png}
\includegraphics[width = 0.32 \textwidth]{../April2014/KinematicPlots_CoeffLargerThan1/StackCanvas_LightQuarkTheta_JetPt0.png}
\includegraphics[width = 0.32 \textwidth]{../April2014/KinematicPlots_CoeffLargerThan1/StackCanvas_TopMass_JetPt0.png}
\includegraphics[width = 0.32 \textwidth]{../April2014/KinematicPlots_CoeffLargerThan1/StackCanvas_TopProductionId_JetPt0.png}
\includegraphics[width = 0.32 \textwidth]{../April2014/KinematicPlots_CoeffLargerThan1/StackCanvas_WBosonMass_JetPt0.png}
\end{center}

\subsubsection{Model plots for varying Pt Cut}
Script adapted, but error obtained when running the python script ...\\
Worked when everything was copied to the TestDir ...\\
Maybe the use of nohup gives the problem ...

%°°°°°°°°°°°°°°°°°°°°°°°°°°°°°°°°°°°°°°°°°°°°°°°°°°°°°°°°°°°°°°°°°°°°°°°°

%°°°°°°°°°°°°°°°°°°°°°°°°°°°°°°°°°°°°°°°°°°°°°°°°°°°°°°°°°°°°°°°°°°°°°°°°
%	CHAPTER: MadGraph/MadWeight issues
\chapter{MadGraph/MadWeight issues}
\section{Discussion with Olivier Mattelaer (8-9/01/2015)}

During this discussion moment with Olivier Mattelaer many interesting subjects have been discussed, the most important of them are summarized below. A detailed overview with all suggestions and recommendations is also created, but not all of them have been added in this document. Many of these items ended up in the to-do list added further in the section.

\subsection{Transfer Functions}

The suggested way to investigate the influence of the created Transfer Functions is by first running the sample with a pure delta function for parton-level events. Afterwards the kinematics of the parton-level events of this sample should be smeared with the created Transfer Functions and then the sample should be ran again, this time with the new Transfer Functions. This ensures that in both cases the Transfer Function which is applied corresponds to the kinematics of the considered events. Any difference between the two results indicates a bias introduced by the Transfer Function.\\

An important remark was the missing normalisation of the double Gaussian Transfer Function. Currently the necessary normalisation factors have not been added and should be done as soon as possible. This because a non-normalized Transfer Function could result in wrong weights and introduce a significant bias to the obtained results.\\

A final important comment about the correct implementation of personal Transfer Functions is the behaviour outside the considered fit range. The strange distributions obtained for $p_T$ values outside the considered fit range did not appear in the results of Olivier (and Arnaud) because they had a different fitting algorithm which explicitely forced the Minuit fitter to search only for a local minimum where the relative normalisation of both Gaussians was positive\footnote{This is actually an allowed and physically motivated restriction which can be applied since it simply corresponds to requiring a probability to always stay positive.}.\\
Since probabilities should always remain positive, the negative distributions should be excluded from the allowed options. This can be done by either adapting the used fitting algorithm to only consider positive probabilities or by simply changing the fit formula to use $max(0,a_{3})$ in stead of $a_{3}$. The second solution is the most straightforward to implement, but has as disadvantage that any event with very high or low $p_T$ will have weight $0$ and hence be thrown away.

\subsection{Likelihood normalization}

The correct formula for the $\ln(\mathcal{L})$ is given in Equation (\ref{eq::LLNorm}):
\begin{equation} \label{eq::LLNorm}
 - \sum_{i}^{N} \ln (\frac{1}{\sigma Acc} weight_{MW}) = N \ln \sigma + N \ln Acc - \sum_{i}^{N} \ln weight_{MW}
\end{equation}

The above given normalisation formula has already been applied succesfully on 1D-variation of the $V_R$ parameter while keeping the $V_{L}$ value fixed to its Standard Model expectation value. However similar success has not been obtained for the variation of the $V_L$ component. It is currently been investigated whether this is caused by considering a too restricted grid or whether it is caused by a lack of sensitivity of the $V_L$ variable. Both possibilities are likely because the observed variation between the different $V_L$ configurations is comparable to the uncertainty introduced by the XS normalisation part in the formula.\\

This directly points the largest danger of using such a normalisation based on the XS of the considered configuration, namely the strong dependence on the uncertainty of the calculated cross section. This is illustrated in Equation (\ref{eq::LLNormUnc}):
\begin{equation} \label{eq::LLNormUnc}
 unc \sim N \frac{\Delta \sigma}{\sigma} + \sum \frac{\Delta weight_{MW}}{weight_{MW}}
\end{equation}

The first term in this Equation scales as $N$ while the second term only scales as $\sqrt{N}$ meaning that the first term quickly starts to dominate the total uncertainty. Therefore it is important to calculate these MadGraph cross sections for a large number of events in order to reduce the uncertainty as much as possible. The size of the relative uncertainty is also closely related to the number of events which will be calculated by MadWeight and hence a good balance should be found\footnote{Important to note is that the $N$ in the uncertainty formula should not be the same as the number of events used to calculate the MadGraph cross section. This $N$ variable is really the number of events which are submitted to MadWeight in order to calculate the corresponding weight for each specific configuration.}.

\subsection{Cluster optimization}

When discussing with Olivier about the restrictions of the IIHE cluster he mentioned the possibility to set a maximum number of jobs which can be submitted simultaneously to the cluster. This means that this maximum can be set to $2000$ in order to be in agreement with the maximum number of allowed jobs at the IIHE cluster. There exists even the option to explicitely set the number of remaining jobs which should be reached before a new bunch of jobs should be submitted. Hence this allows to wait until the number of jobs running is reduced to about $1500$ before a new set of jobs is submitted in order to reach again the maximum number of $2000$.\\
This can all be changed in the $cluster.py$ file which has a class ``PBSCluster'' at about line $1016$.\\
\\
The huge benefit of the above mentioned cluster optimization is that it allows to keep using the ``collect'' option of MadWeight without the necessity of combining multiple separate runs of each $2000$ jobs. Hence one should just wait until all the events have been submitted and run the final collect step of the MadWeight setup.

\subsection{Optimizations and bug fix}

During this two-day discussion moment a couple of small optimizations of the used MadWeight configuration have been discovered.\\
At first it was clear that the used MadWeight version was considered to be an old version for Olivier. Since rather recently a stable version is kept with multiple updates and necessary bug-fixes. The installation command for this newer version is the following and is currently installed as a new directory $NewestMW\_amcnlo$ on $/localgrid$.
\begin{equation}
  bzr ~~ branch ~~ lp: mg5amcnlo \nonumber
\end{equation}

Secondly the issue of negative weights when using a $\delta$ Transfer Function was resolved. This was caused by a discrepancy between the integral implemented in MadWeight and the one developed by Olivier. The reason for a second integral is in order to deal with multiple Transfer Functions during one single submission. The bug-fix was directly created by Olivier and is currently implemented in all directories on $/localgrid$.\\

A final but rather important remark was the fact that the developed model allowed CKM suppressed W-boson decays. This did not influence the final result since the probability for such decays was extremely small but significantly enlarged the CPU time since MadWeight considers each possible decay and does the calculation of the probability for all of them. Hence excluding these types of decays should improve the CPU running time with possible a factor $4$.\\
The restrictions on the model can easily be added in MadGraph as explained in the launchpad FAQ, and are currently implemented in the newest $/localgrid$ directory.
%°°°°°°°°°°°°°°°°°°°°°°°°°°°°°°°°°°°°°°°°°°°°°°°°°°°°°°°°°°°°°°°°°°°°°°°°

%°°°°°°°°°°°°°°°°°°°°°°°°°°°°°°°°°°°°°°°°°°°°°°°°°°°°°°°°°°°°°°°°°°°°°°°°
%	CHAPTER: Event Selection
\chapter{Event Selection}

\section{Choice of b-tag requirements}

\subsection{Signal vs background comparison}
Since the complete event should be reconstructed as accurately as possible the 'signal' is represented by the case that all four particles are matched correctly while the 'background' events are the events for which at least one particle is matched wrongly.
Events which are not matched using the JetPartonMatching algorithm (currently ptOrderedMinDist with dR of 0.3 is used) are not included in either of these two variables and are shown separately in the tables found below.\\
First the different b-tag options are compared for the four possible combinations which are still allowed, namely the interchange of both the two light quarks and the two b-jets.

 \begin{table}[!h] 
 \begin{tabular}{c|c|c|c|c|c} 
 \textbf{Option} (no $\chi^{2}$ $m_{lb}$) & all 4 correct   & $\geq$ 1 wrong       & $\frac{s}{\sqrt{b}}$ & $\frac{s}{b}$ & non-matched \\ \hline 
2 L b-tags,                & 55967 & 50416 & 249.257 & 1.1101 & 225217 \\ 
2 M b-tags,              & 49983 & 32012 & 279.361 & 1.56138 & 146633 \\ 
2 M b-tags, light L-veto & 39661 & 27751 & 238.081 & 1.42917 & 118389 \\ 
2 T b-tags,              & 31444 & 16061 & 248.114 & 1.95779 & 78062 \\ 
2 T b-tags, light M-veto & 29160 & 15585 & 233.579 & 1.87103 & 73570 \\ 
2 T b-tags, light L-veto & 23159 & 14093 & 195.082 & 1.6433 & 59997 \\ 
 \end{tabular} 
 \caption{Overview of correct and wrong reconstructed events for the different b-tags without the use of a $\chi^{2}$ $m_{lb}$ - $m_{qqb}$ method} 
 \end{table} 
 
 \begin{table}[!h] 
 \begin{tabular}{c|c|c|c|c|c} 
 \textbf{Option} (no $\chi^{2}$ $m_{lb}$) & 2 b's good      & $\geq$ 1 b wrong     & $\frac{s}{\sqrt{b}}$ & $\frac{s}{b}$ & non-matched \\ \hline 
2 L b-tags,                & 78896 & 27487 & 475.873 & 2.8703 & 225217 \\ 
2 M b-tags,              & 70073 & 11922 & 641.765 & 5.87762 & 146633 \\ 
2 M b-tags, light L-veto & 58778 & 8634 & 632.57 & 6.80774 & 118389 \\ 
2 T b-tags,              & 43804 & 3701 & 720.036 & 11.8357 & 78062 \\ 
2 T b-tags, light M-veto & 41617 & 3128 & 744.11 & 13.3047 & 73570 \\ 
2 T b-tags, light L-veto & 34908 & 2344 & 721.018 & 14.8925 & 59997 \\ 
 \end{tabular} 
 \caption{Overview of correct and wrong reconstructed b-jets for the different b-tags without the use of a $\chi^{2}$ $m_{lb}$ - $m_{qqb}$ method} 
 \end{table} 
 
 \begin{table}[!h] 
 \begin{tabular}{c|c|c|c|c|c} 
 \textbf{Option} (no $\chi^{2}$ $m_{lb}$) & 2 light good    & $\geq$ 1 light wrong & $\frac{s}{\sqrt{b}}$ & $\frac{s}{b}$ & non-matched \\ \hline 
2 L b-tags,                & 58707 & 47676 & 268.869 & 1.23137 & 225217 \\ 
2 M b-tags,              & 50676 & 31319 & 286.351 & 1.61806 & 146633 \\ 
2 M b-tags, light L-veto & 40361 & 27051 & 245.398 & 1.49203 & 118389 \\ 
2 T b-tags,              & 31690 & 15815 & 251.993 & 2.00379 & 78062 \\ 
2 T b-tags, light M-veto & 29466 & 15279 & 238.382 & 1.92853 & 73570 \\ 
2 T b-tags, light L-veto & 23456 & 13796 & 199.7 & 1.7002 & 59997 \\ 
 \end{tabular} 
 \caption{Overview of correct and wrong reconstructed light jets for the different b-tags without the use of a $\chi^{2}$ $m_{lb}$ - $m_{qqb}$ method} 
 \end{table} 
 
Above tables show a clear improvement for the 2 Tight b-tag case since suddenly the $\frac{s}{b}$ value goes up to almost 2. An additional benefit of the 2 Tight bt-tag case is that it as well will take care of a large part of the process backgrounds. Hence the motivation for selecting this b-tag option.\\
Comparing the normal 2 Tight b-tag case, where the light jets are defined as not being Tight b-tagged jets, against the two possibilities using a light-jet veto indicates no motivation to go for the light-veto option.\\
\\
However the second table, showing only the reconstruction efficiency of the b-jets, shows an improvement when using a light-jet veto.\\
This means that, even if the number of selected events gets lower, the percentage of correct events does improve when asking for a light-jet veto since it ensures that mistagged b-jet events doesn't by mistake get identified as light jets. So events with a b-jet with a too low CSV discriminant now don't get selected anymore because these so-called light-jets don't survive the veto cut.\\
\\
But, as confirmed by the last table, the efficiency of the light-jet reconstruction shows no distinct improvement. So for some reason the light-jets which are chosen with this veto method are not by definition the actual light-jets.
%**************************************************

\section{Use of $m_{lb}$ $\chi^{2}$ method for selecting the correct b-jets}
 \begin{table}[!h] 
 \begin{tabular}{c|c|c|c|c|c} 
 \textbf{Option} (with $\chi^{2}$ $m_{lb}$) & all 4 correct      & $\geq$ 1 wrong  & $\frac{s}{\sqrt{b}}$ & $\frac{s}{b}$ & non-matched \\ \hline 
2 L b-tags,                & 51055 & 55328 & 217.053 & 0.92277 & 514406 \\ 
2 M b-tags,              & 45664 & 36331 & 239.572 & 1.25689 & 538794 \\ 
2 M b-tags, light L-veto & 36271 & 31141 & 205.539 & 1.16473 & 553377 \\ 
2 T b-tags,              & 28580 & 18925 & 207.752 & 1.51017 & 573284 \\ 
2 T b-tags, light M-veto & 26513 & 18232 & 196.355 & 1.4542 & 576044 \\ 
2 T b-tags, light L-veto & 21073 & 16179 & 165.673 & 1.30249 & 583537 \\ 
 \end{tabular} 
 \caption{Overview of correct and wrong reconstructed events for the different b-tags when a $\chi^{2}$ $m_{lb}$ - $m_{qqb}$ method is applied} 
 \end{table}
 
 \begin{table}[!h] 
 \begin{tabular}{c|c|c|c|c|c} 
 \textbf{Option} (with $\chi^{2}$ $m_{lb}$) & Correct b's & Wrong b's & $\frac{s}{\sqrt{b}}$ & $\frac{s}{b}$ & Correct option exists \\ \hline 
2 L b-tags,                & 66882 & 12014 & 610.19 & 5.56701 & 78896 \\ 
2 M b-tags,              & 59520 & 10553 & 579.395 & 5.6401 & 70073 \\ 
2 M b-tags, light L-veto & 49556 & 9222 & 516.04 & 5.37367 & 58778 \\ 
2 T b-tags,              & 37013 & 6791 & 449.146 & 5.4503 & 43804 \\ 
2 T b-tags, light M-veto & 35015 & 6602 & 430.94 & 5.3037 & 41617 \\ 
2 T b-tags, light L-veto & 29139 & 5769 & 383.64 & 5.05096 & 34908 \\ 
 \end{tabular} 
 \caption{Overview of the number of times the correct b-jet combination is chosen when using a $\chi^{2}$ $m_{lb}$ - $m_{qqb}$ method} 
 \end{table}
 
The two tables in this subsection require a different interpretation. The first one is actually a combined test of the $\chi^{2}$ $m_{lb}$ - $m_{qqb}$ method and the optimal b-tag choice while the second one is merely a performance check of the mlb method. \\
This second table had to be added since the first table can't be directly compared against the tables in the previous subsection since currently only one b-jet combination is left while in the previous case an iteration between the different b-jets was allowed. So the numbers will be lower by definition when the $\chi^{2}$ $m_{lb}$ - $m_{qqb}$ method is applied.\\
Therefore the second table is relevant in order to select whether some clear gain can be obtained when applying this method since it represents the number of times the correct b-jet combination. If this percentage is higher than 50$\%$, which is the case, an improvement is obtained compared to an iteration between the two possible combinations.\\
\\
The first table indicates again that no real difference is found between the different b-tag options, but that as soon as 2 Tight b-tags are applied the $\frac{s}{b}$ improves slightly. Also the second table shows no difference in efficiency between the different b-tag options, but shows however that the use of this $\chi^{2}$ $m_{lb}$ - $m_{qqb}$ method significantly enhances the correct choice of the b-jet combination. In about 84$\%$ of the cases the correct b-jet combination is chosen. 
%**************************************************

\section{Histograms for event selection choice}
\begin{center}

\begin{figure}[!h]
 \includegraphics[width = 0.32 \textwidth]{/home/annik/Documents/Vub/PhD/ThesisSubjects/AnomalousCouplings/EventSelectionChoice_June2014/StCosTheta_BeforeEvtSel.png}
\includegraphics[width = 0.32 \textwidth]{/home/annik/Documents/Vub/PhD/ThesisSubjects/AnomalousCouplings/EventSelectionChoice_June2014/StCosThetaNoBTag.png}
\includegraphics[width = 0.32 \textwidth]{/home/annik/Documents/Vub/PhD/ThesisSubjects/AnomalousCouplings/EventSelectionChoice_June2014/StCosThetaLCSV.png}
\includegraphics[width = 0.32 \textwidth]{/home/annik/Documents/Vub/PhD/ThesisSubjects/AnomalousCouplings/EventSelectionChoice_June2014/StCosThetaMCSV.png}
\includegraphics[width = 0.32 \textwidth]{/home/annik/Documents/Vub/PhD/ThesisSubjects/AnomalousCouplings/EventSelectionChoice_June2014/StCosThetaTCSV.png}
\caption{The $\cos \theta^{*}$ distribution for the different b-tag options (all of them imply double b-tag), which are not really influenced by the application of a b-tag. The only relevant distortion is caused by the event selection which is applied.}
\end{figure}

\begin{figure}[!h]
 \includegraphics[width = 0.32 \textwidth]{/home/annik/Documents/Vub/PhD/ThesisSubjects/AnomalousCouplings/EventSelectionChoice_June2014/CorrectBHadrCSVDiscr.png}
\includegraphics[width = 0.32 \textwidth]{/home/annik/Documents/Vub/PhD/ThesisSubjects/AnomalousCouplings/EventSelectionChoice_June2014/CorrectBLeptCSVDiscr.png}
\includegraphics[width = 0.32 \textwidth]{/home/annik/Documents/Vub/PhD/ThesisSubjects/AnomalousCouplings/EventSelectionChoice_June2014/CorrectQuark1CSVDiscr.png}
\includegraphics[width = 0.32 \textwidth]{/home/annik/Documents/Vub/PhD/ThesisSubjects/AnomalousCouplings/EventSelectionChoice_June2014/CorrectQuark2CSVDiscr.png}
\caption{Distribution of the CSV discriminant for the different correctly matched quark-jet pairs. The value $-2$ is used to represent a non-matched jet. As expected the b-jets have a large peak at $1$, so the Tight b-tag of 0.898 will not take away to many correct b-jets. The problem in the matching is clearly represented in distribution of the second quark which is only reconstructed in less than half of the cases.}
\end{figure}

\begin{figure}[!h]
\includegraphics[width = 0.32 \textwidth]{/home/annik/Documents/Vub/PhD/ThesisSubjects/AnomalousCouplings/EventSelectionChoice_June2014/CSVDiscrLCSVLightJets.png}
\caption{ Distribution of the CSV discriminant of the selected light jets (all of them). \textbf{Add same histograms for Medium and Tight option, this will show how many of the light jets actually have a large CSV discriminant (maybe only focus on the two/three leading light jets) }}
\end{figure}

\begin{figure}[!h]
\includegraphics[width = 0.32 \textwidth]{/home/annik/Documents/Vub/PhD/ThesisSubjects/AnomalousCouplings/EventSelectionChoice_June2014/JetTypeLCSVLightJets.png}
\caption{Jet type of the of the light jets (all of them). The value $25$ means that the found light jet couldn't be matched to a Parton witht he JetPartonMatching method. \textbf{Same for M and T ... ?} }
\end{figure}

\begin{figure}[!h]
\includegraphics[width = 0.32 \textwidth]{/home/annik/Documents/Vub/PhD/ThesisSubjects/AnomalousCouplings/EventSelectionChoice_June2014/JetTypeLCSV.png}
\includegraphics[width = 0.32 \textwidth]{/home/annik/Documents/Vub/PhD/ThesisSubjects/AnomalousCouplings/EventSelectionChoice_June2014/JetTypeMCSV.png}
\includegraphics[width = 0.32 \textwidth]{/home/annik/Documents/Vub/PhD/ThesisSubjects/AnomalousCouplings/EventSelectionChoice_June2014/JetTypeTCSV.png}
\caption{Jet type of the b-tagged jets (all of them) with the same convention for the non-matched jets. }
\end{figure}

\begin{figure}[!h]
\includegraphics[width = 0.32 \textwidth]{/home/annik/Documents/Vub/PhD/ThesisSubjects/AnomalousCouplings/EventSelectionChoice_June2014/MlbMass.png}
\includegraphics[width = 0.32 \textwidth]{/home/annik/Documents/Vub/PhD/ThesisSubjects/AnomalousCouplings/EventSelectionChoice_June2014/MqqbMass.png}
\includegraphics[width = 0.32 \textwidth]{/home/annik/Documents/Vub/PhD/ThesisSubjects/AnomalousCouplings/EventSelectionChoice_June2014/1July2014/MlbMqqbCorrectAll.png}
\caption{The first two histograms show the mass distribution for the correctly matched and reconstructed particles. A gaussian fit is applied in order to obtain both the $m_{lb}$ and $m_{qqb}$ mass and sigma. The last histogram shows the 2D behavior of these distributions. }
\end{figure}
\end{center}
%**************************************************

\newpage
\section{Considering 2 or 3 light jets}
In order to try to improve the signal efficiency it can be considered to add a third light jet to the particles which have to be taken into account for the jet selection. Adding this third jet will however result in 4 additional combinations which have to be considered so this method will only benefit when the $\chi^{2}$ $m_{lb}$ - $m_{qqb}$ method is applied. Since MadWeight uses so much CPU sending the $6$ possible combinations to MadWeight will not be beneficial.
\subsection{Event selection numbers comparison}
 \begin{table}[!h] 
 \begin{tabular}{c|c|c|c|c} 
\textbf{Option} (no $\chi^{2}$ $m_{lb}$) & chosen jets are correct ($\%$)       & $\frac{s}{b}$ & 3rd jet is correct ($\%$) \\ \hline 
 5 jet case, 2 T b-tags              & 76.2852 & 3.21677 & 70.244\\ 
 4 jet case, 2 T b-tags              & 66.9091 & 2.02198 & X \\ 
 \end{tabular} 
\caption{Overview of correct and wrong reconstructed events for the different b-tags without the use of a $\chi^{2}$ $m_{lb}$ - $m_{qqb}$ method} 
 \end{table} 
 
 \begin{table}[!h] 
 \begin{tabular}{c|c|c|c|c} 
\textbf{Option} (no $\chi^{2}$ $m_{lb}$) & 2 b's chosen correctly ($\%$)        & $\frac{s}{b}$ & 3rd jet is correct ($\%$) \\ \hline 
 5 jet case, 2 T b-tags              & 90.5014 & 9.52784 & 64.5863\\ 
 4 jet case, 2 T b-tags              & 92.3745 & 12.1138 & X \\ 
 \end{tabular} 
\caption{Overview of correct and wrong reconstructed b-jets for the different b-tags without the use of a $\chi^{2}$ $m_{lb}$ - $m_{qqb}$ method} 
 \end{table} 
 
 \begin{table}[!h] 
 \begin{tabular}{c|c|c|c|c} 
\textbf{Option} (no $\chi^{2}$ $m_{lb}$) & chosen light jets are correct ($\%$) & $\frac{s}{b}$ & 3rd jet is correct ($\%$) \\ \hline 
 5 jet case, 2 T b-tags              & 79.2892 & 3.8284 & 70.2241\\ 
 4 jet case, 2 T b-tags              & 67.4722 & 2.0743 & X \\ 
 \end{tabular} 
\caption{Overview of correct and wrong reconstructed light jets for the different b-tags without the use of a $\chi^{2}$ $m_{lb}$ - $m_{qqb}$ method} 
 \end{table} 

These three tables look at either a pure 5 jet case or a pure 4 jet case and comparing the numbers in each table with eachother is probably not extremely relevant. This because in the pure 5 jets case the matching requirement is loosened to two out of the three chosen light jets correctly matching with the partons. So in $1$ out of $3$ possibilities the so-called correct event will not be correct resulting in too high numbers for this case.\\
\\
The first column gives the percentage how often the chosen jets are indeed the correct partons, hence in the 5-jet case the number of times the 5 possible jets match with the 4 correct partons. In the 4-jet case it implies that the four chosen jets are matched correctly with the 4 partons. The second column gives a similar value since the signal is defined as the number of times the matching was done correctly for the four partons while the background stands for the events where one of the matching is not succesful. The third column checks how often the third jet is one of the two correct light jets and compares it against the number in the first column. So it represents the number of times adding the third jet results in an improvement of the event reconstruction. \\
\\
The numbers which are relevant in these tables are exactly these two last columns. These numbers show that in about $70 \%$ of the cases the third jet is actually one of the correct quarks. Therefore it can be decided from these numbers that in quite a lot of events, an improvement can be obtained when this third light jet is considered as well.

\subsection{Mlb-algorithm numbers comparison}
  \begin{table}[!h] 
 \begin{tabular}{c|c|c|c|c} 
\multirow{2}{*}{\textbf{Option} (with $\chi^{2}$ $m_{lb}$)} & 4 chosen jets & $\frac{s}{b}$ & 3rd jet is one of the & 3rd jet is chosen \\ & are correct ($\%$)    & 	             & 2 correct light jets ($\%$) &  and correct ($\%$)	  \\ \hline 
 5 jet case,      2 T b-tags              & 73.2413 & 2.73711 & 21.3059 & 89.2513 \\ 
 4 jet case,      2 T b-tags              & 76.9258 & 3.33384 & 0 & -nan \\ 
 Pure 5 jet case, 2 T b-tags              & 65.6683 & 1.91276 & 74.3984 & 89.2513 \\ 
 \end{tabular} 
 \caption{Overview of correct and wrong reconstructed events for the different b-tags when a $\chi^{2}$ $m_{lb}$ - $m_{qqb}$ method is applied} 
 \end{table} 
 
 \begin{table}[!h] 
 \begin{tabular}{c|c|c|c|c} 
 \textbf{Option} (with $\chi^{2}$ $m_{lb}$) & \% b's correct   & $\frac{s}{b}$ &  &  \\ \hline 
 5 jet case,      2 T b-tags              & 90.3229 & 9.33364 & 0 & 0 \\ 
 4 jet case,      2 T b-tags              & 90.7656 & 9.82911 & 0 & 0 \\ 
 Pure 5 jet case, 2 T b-tags              & 89.9057 & 8.90659 & 0 & 0 \\ 
 \end{tabular} 
 \caption{Overview of the number of times the correct b-jet combination is chosen when using a $\chi^{2}$ $m_{lb}$ - $m_{qqb}$ method} 
 \end{table} 
When the $m_{lb}$ method is applied, the two first columns in the given tables represent similar quantities with the only difference that now the 4 jets which are actually chosen by the $\chi^{2}$ $m_{lb}$ - $m_{qqb}$ method are considered. 
In this case the third column represents the number of times the third jet is chosen when the two light jets are matched correctly, implying that the third jet is one of the correct ones and that the second light jet is also correctly matched. The final column only looks at the third jet and puts no requirement on the matching of the second light jet. So it compares the number of times one of the chosen jets is the third jet and in how many cases this chosen third jet is one of the correct partons.

\subsection{Compare efficiencies for 3$^{rd}$ jet with 1$^{st}$ and 2$^{nd}$}
The obtained efficiency numbers for the 3$^{rd}$ jet seemed to be rather high so to exclude any possible double-counting mistakes the percentages for the 1$^{st}$ and 2$^{nd}$ jet where also calculated and compared. Since the considered percentage represents the number of times the 3$^{rd}$ jet corresponds to one of the actual light quarks should the sum of the three percentages not become any larger than 200$\%$.\\
The percentages were calculated both before and after the application of the $\chi^{2}$ algorithm and the results can be found in Tables \ref{table::FirstSecondThirdJetPerc} and \ref{table::FirstSecondThirdJetPercMlb}. The first table shows the numbers before the application of the $\chi^{2}$ algorithm implying that for correctly matched events 2 of the 3 light jets are correctly matched. The second table gives the results after the $\chi^2$ minimization method.

\begin{table}[h!]
 \centering
 \begin{tabular}{c|c|c}
                 & Number of events & Percentage ($\%$) \\
  \hline
  Matched events & 2455 & \\
  \hline
  First jet      & 1596 & 65   \\
  Second jet     & 1740 & 70.9 \\
  Third jet      & 1574 & 64.1 \\
  \hline
  Total          & 4910 & 200
 \end{tabular}
 \caption{Number of times the first, second or third jet corresponds to one of the two correct light jets before the application of the $\chi^{2}$ method.}\label{table::FirstSecondThirdJetPerc}
\end{table}

\begin{table}[h!]
 \centering
 \begin{tabular}{c|c|c}
                 & Number of events & Percentage ($\%$) \\
  \hline
  Matched events & 241 & \\
  \hline
  First jet      & 162 & 67.2   \\
  Second jet     & 178 & 73.8 \\
  Third jet      & 142 & 58.9 \\
  \hline
  Total          & 842 & 200
 \end{tabular}
 \caption{Number of times the first, second or third jet corresponds to one of the two correct light jets after the application of the $\chi^{2}$ method.}\label{table::FirstSecondThirdJetPercMlb}
\end{table}

From these tables can be concluded that the obtained percentage of about $70\%$ for the 3$^{rd}$ jet is actually correct and that the result can be trusted. It also implies that in the 5-jet case (meaning that there actually is a third light jet) the three jets have a rather similar probability of being the correct jet.

\subsection{Considering separate categories}
In order to be certain whether the 3$^{rd}$ light jet should be considered the considered events have been divided into two categories. The first only consists of events with exactly 2 light jets, hence 2 or more b-tagged jets\footnote{The number of events with a third b-tagged jet is extremely small and will probably not really influence the efficiency as can be seen from the plots shown in \ref{subsec::MSPlotsNBTaggedJets}.} and 2 light jets, while the second category allows for more light jets. In case of multiple b-tagged jets only the two highest $p_T$ jets are considered.\\
In the second category each event is treated in two separate ways. First the event is seen as a pure 4-jet event implying that only the two leading light jets are kept while for the second approach the third light jet is also included in the $\chi^{2}$ algorithm resulting in 6 possible solutions.\\ \\
For these three cases the matching and $\chi^{2}$ minimization efficiency have been compared in order to ensure that the most efficient event selection will be used. The results can be found in Table \ref{table::LightJetCategories} and indicates that including the third light jet doesn't result in a large gain of efficiency. On the contrary, including events with more than 4 jets but discarding the third light jet results in a significant decrease of efficiency.

\begin{table}[!h]
 \centering
 \begin{tabular}{c|c|c}
                         & N(2 light jets) & N(2+ light jets)  \\
  \hline
  $\#$ events            & 9328            & 6436 \\
  $\#$ matched events    & 4018            & 3319 \\
  $\#$ good combi chosen & 3273            & 788 -- 1604
 \end{tabular}
 \caption{Number of events, number of matched events and number of events for which the correct jet combination is chosen using the $\chi^{2}$ $m_{lb} - m_{qqb}$ algorithm for the two considered categories. The first number in the right-hand bottom corner respresents the number of good combinations chosen when the event is treated as a 4-jet event while the second number is for the treatment of a 5-jet event. An event is considered as matching if the 4 jets corresponding to the generator event are included in the collection of selected jets.}\label{table::LightJetCategories}
\end{table}

The obtained results seem rather suprising since it implies that only asking the 4 leading jets results in the worst efficiencies. \\
However it should be noted that for this configuration the 2 Tight b-tag constraint might influence this result. It could be possible that for looser b-tag requirements the mis-identification of the two b-jets results in a worse ``combination choice'' efficiency.\\

\begin{table}[!h]
 \centering
 \begin{tabular}{c|c|c|c}
                         & Only 4-jet events & All, but treated as 4-jet & All, but treated as 5-jet  \\
  \hline
  $\%$ matched events    & 43.07                   & 46.5                               & 46.5  \\
  $\%$ good combi        & 81.4                    & 55.3                               & 66.5
 \end{tabular}
 \caption{Percentages for matching the reconstructed event with the generated and for selecting the good combination using the $\chi^{2}$ $m_{lb} - m_{qqb}$ algorithm. }\label{table::JetCategoryPercentages}
\end{table}

\textit{Is it possible to understand these results using the percentages for the first, second and third jet being correct .. ? \\ When treating a 5-jet event as a 4-jet event the correct jet would have been the third, discarded, jet in $\frac{1}{3}$ of the cases which explains the significant reduction. However the difference between treating the event as a 4-jet or a 5-jet event doesn't result in a large difference. It seems that in much of the N(2+ light jet) cases the correct jet is actually still one of the following jets and is not included in the three leading light jets.}

\subsubsection{Histograms for number of selected, b-tagged and light jets}\label{subsec::MSPlotsNBTaggedJets}
\begin{figure}[!h]
\includegraphics[width = 0.32 \textwidth]{/home/annik/Documents/Vub/PhD/ThesisSubjects/AnomalousCouplings/EventSelectionChoice_June2014/LightJetCategories_Sept2014/CanvasStack_nSelectedJets_BeforeBTag_mu.png}
\includegraphics[width = 0.32 \textwidth]{/home/annik/Documents/Vub/PhD/ThesisSubjects/AnomalousCouplings/EventSelectionChoice_June2014/LightJetCategories_Sept2014/CanvasStack_nBTaggedJets_BeforeBTag_mu.png}
\includegraphics[width = 0.32 \textwidth]{/home/annik/Documents/Vub/PhD/ThesisSubjects/AnomalousCouplings/EventSelectionChoice_June2014/LightJetCategories_Sept2014/CanvasStack_nLightJets_BeforeBTag_mu.png}
\caption{Number of selected, b-tagged and light jets, respectively, before requiring two Tight b-tags. These distributions are for the muon channel only.}
\end{figure}

\begin{figure}[!h]
\includegraphics[width = 0.32 \textwidth]{/home/annik/Documents/Vub/PhD/ThesisSubjects/AnomalousCouplings/EventSelectionChoice_June2014/LightJetCategories_Sept2014/CanvasStack_nSelectedJets_AfterBTag_mu.png}
\includegraphics[width = 0.32 \textwidth]{/home/annik/Documents/Vub/PhD/ThesisSubjects/AnomalousCouplings/EventSelectionChoice_June2014/LightJetCategories_Sept2014/CanvasStack_nBTaggedJets_AfterBTag_mu.png}
\includegraphics[width = 0.32 \textwidth]{/home/annik/Documents/Vub/PhD/ThesisSubjects/AnomalousCouplings/EventSelectionChoice_June2014/LightJetCategories_Sept2014/CanvasStack_nLightJets_AfterBTag_mu.png}
\caption{Number of selected, b-tagged and light jets, respectively, after requiring two Tight b-tags. These distributions are for the muon channel only.}
\end{figure}

%**************************************************
 
\section{Studying optimal cut on $\chi^{2}$ value}
The tables shown here indicate that the application of a cut on the $\chi^{2}$ value of the $m_{lb}$ - $m_{qqb}$ method doesn't change the rates of good and wrong chosen events. Therefore it will only reduce the number of selected events, and hence reduce the needed CPU time, but still keep an as pure sample as obtained without any cut.\\
Since there is no real difference visible between setting the cut value to $3$ or $5$ it is advisable to use the cut value of $3$ to reduce the number of selected events.
 
\subsection{$\chi^{2}$ required to be smaller than $5$}
  \begin{table}[!h] 
 \begin{tabular}{c|c|c|c|c} 
\multirow{2}{*}{\textbf{Option} (with $\chi^{2}$ $m_{lb}$)} & 4 chosen jets & $\frac{s}{b}$ & 3rd jet is one of the & 3rd jet is chosen \\ & are correct ($\%$)    & 	             & 2 correct light jets ($\%$) &  and correct ($\%$)	  \\ \hline 
 5 jet case,      2 T b-tags              & 73.1485 & 2.72419 & 21.5302 & 89.0292 \\ 
 4 jet case,      2 T b-tags              & 76.89 & 3.32712 & 0 & -nan \\ 
 Pure 5 jet case, 2 T b-tags              & 65.5254 & 1.90068 & 74.5263 & 89.0292 \\ 
 \end{tabular} 
 \caption{Overview of correct and wrong reconstructed events for the different b-tags when a $\chi^{2}$ $m_{lb}$ - $m_{qqb}$ method is applied} 
 \end{table} 
 
 \begin{table}[!h] 
 \begin{tabular}{c|c|c|c|c} 
 \textbf{Option} (with $\chi^{2}$ $m_{lb}$) & \% b's correct   & $\frac{s}{b}$ &  &  \\ \hline 
 5 jet case,      2 T b-tags              & 90.3451 & 9.35743 & 0 & 0 \\ 
 4 jet case,      2 T b-tags              & 90.8573 & 9.93767 & 0 & 0 \\ 
 Pure 5 jet case, 2 T b-tags              & 89.8465 & 8.84884 & 0 & 0 \\ 
 \end{tabular} 
 \caption{Overview of the number of times the correct b-jet combination is chosen when using a $\chi^{2}$ $m_{lb}$ - $m_{qqb}$ method} 
 \end{table} 
 
 \subsection{$\chi^{2}$ required to be smaller than $3$}
 \begin{table}[!h] 
 \begin{tabular}{c|c|c|c|c} 
\multirow{2}{*}{\textbf{Option} (with $\chi^{2}$ $m_{lb}$)} & 4 chosen jets & $\frac{s}{b}$ & 3rd jet is one of the & 3rd jet is chosen \\ & are correct ($\%$)    & 	             & 2 correct light jets ($\%$) &  and correct ($\%$)	  \\ \hline 
 5 jet case,      2 T b-tags              & 72.8939 & 2.68921 & 21.7785 & 88.6241 \\ 
 4 jet case,      2 T b-tags              & 77.0226 & 3.3521 & 0 & -nan \\ 
 Pure 5 jet case, 2 T b-tags              & 64.7571 & 1.83745 & 74.192 & 88.6241 \\ 
 \end{tabular} 
 \caption{Overview of correct and wrong reconstructed events for the different b-tags when a $\chi^{2}$ $m_{lb}$ - $m_{qqb}$ method is applied} 
 \end{table} 
 
 \begin{table}[!h] 
 \begin{tabular}{c|c|c|c|c} 
 \textbf{Option} (with $\chi^{2}$ $m_{lb}$) & \% b's correct   & $\frac{s}{b}$ &  &  \\ \hline 
 5 jet case,      2 T b-tags              & 90.1586 & 9.16114 & 0 & 0 \\ 
 4 jet case,      2 T b-tags              & 90.8239 & 9.89789 & 0 & 0 \\ 
 Pure 5 jet case, 2 T b-tags              & 89.4472 & 8.47619 & 0 & 0 \\ 
 \end{tabular} 
 \caption{Overview of the number of times the correct b-jet combination is chosen when using a $\chi^{2}$ $m_{lb}$ - $m_{qqb}$ method} 
 \end{table} 
 
\subsection{$\chi^{2}$ required to be smaller than $1$}
 \begin{table}[!h] 
 \begin{tabular}{c|c|c|c|c} 
\multirow{2}{*}{\textbf{Option} (with $\chi^{2}$ $m_{lb}$)} & 4 chosen jets & $\frac{s}{b}$ & 3rd jet is one of the & 3rd jet is chosen \\ & are correct ($\%$)    & 	             & 2 correct light jets ($\%$) &  and correct ($\%$)	  \\ \hline 
 5 jet case,      2 T b-tags              & 72.4364 & 2.62798 & 21.6013 & 87.5833 \\ 
 4 jet case,      2 T b-tags              & 77.3908 & 3.42298 & 0 & -nan \\ 
 Pure 5 jet case, 2 T b-tags              & 62.9538 & 1.69934 & 73.7615 & 87.5833 \\ 
 \end{tabular} 
 \caption{Overview of correct and wrong reconstructed events for the different b-tags when a $\chi^{2}$ $m_{lb}$ - $m_{qqb}$ method is applied} 
 \end{table} 
 
 \begin{table}[!h] 
 \begin{tabular}{c|c|c|c|c} 
 \textbf{Option} (with $\chi^{2}$ $m_{lb}$) & \% b's correct   & $\frac{s}{b}$ &  &  \\ \hline 
 5 jet case,      2 T b-tags              & 90.233 & 9.23861 & 0 & 0 \\ 
 4 jet case,      2 T b-tags              & 90.9792 & 10.0854 & 0 & 0 \\ 
 Pure 5 jet case, 2 T b-tags              & 89.5385 & 8.55882 & 0 & 0 \\ 
 \end{tabular} 
 \caption{Overview of the number of times the correct b-jet combination is chosen when using a $\chi^{2}$ $m_{lb}$ - $m_{qqb}$ method} 
 \end{table} 
%**************************************************

\section{Influence of using $p_T$ cuts suggested by the TOP reference selection Twiki}

The TOP Reference Selection Twiki, and the different subgroup Twikis, suggest to use different event selection requirements than used before. The values suggested can be found in the table below, together with the values which were used for producing the tables and figures given above.\\
In this section the different results will be compared for these two event selections in order to ensure that the influence of this event selection can be ignored. If this is not the case, all the above tables have to be replaced and updated (which will however happen on a larger time-scale since these will be the correct values which will be used further in this analysis). The main goal is to quickly check whether the conclusion obtained from the above figures and tables still remain valid.
\begin{table}[!h]
 \centering
 \begin{tabular}{c|c|c}
                      & Old values & TOP RefSel values \\
   \hline
   selected jets      &   40 GeV   &       30 GeV      \\
   selected muons     &   25 GeV   &       26 GeV      \\
   selected electrons &   32 GeV   &       30 GeV      \\
   veto muons         &   10 GeV   &       10 GeV      \\
   veto electrons     &   10 GeV   &       20 GeV      
 \end{tabular}
\end{table}

In the following table the number of selected events after the different event selection cuts which are applied in this analysis can be found. The left-handed columns contain the information when the old $p_T$ cuts are applied while the right-handed columns gives the number of selected events when the recommendations of the Top Reference Selection Twiki are followed.\\
\\
From the comparison of the two columns can be seen that lowering the $p_T$ cut on the selected jets significantly improves the percentage of selected events, especially because of the improved selection efficiency for the third and fourth jet. This can easily be understood from the $p_T$ distribution histogram for the different leading jets which can be found below.
These distributions show that in the case of the fourth jet, moving the $p_T$ cut from $30$ GeV to $40$ GeV cuts away the largest percentage of this fourth jet since its distribution is peaked at a lower $p_T$ value. The influence is much lower for the first and second jet because their distrubition is maximal around $100$ GeV so changing this $p_T$ cut only affects the left side of the tail of the distribution.\\
\begin{table}
\caption{Event selection table before (left) and after (right) the pT cuts were updated to the ones suggested by the TOP reference twiki, for Semi-elec channel $t\bar{t}+jets$. ($19600.8~pb^{-1}$ of int. lumi.)}
\centering
\begin{tabular}{|c|c|c|c|c|}
\hline
				& \multicolumn{2}{|c|}{Old $p_T$ cuts}	& \multicolumn{2}{|c|}{New $p_T$ cuts}	\\
\hline
preselected			& 1.91516e+07	& 		 	& 1.91516e+07	&	 		\\

trigged				& 3.52179e+06	&	18.4 $\%$	& 3.52179e+06	&	18.4 $\%$	\\

Good PV				& 3.52179e+06	&	100 $\%$	& 3.52179e+06	&	100 $\%$	\\

1 selected electron		& 2.76014e+06	&	78.4 $\%$	& 2.85992e+06	&	\textcolor{red}{81.2 $\%$}	\\

Veto muon			& 2.75335e+06	&	99.8 $\%$	& 2.8529e+06	&	99.8 $\%$	\\

Veto 2nd electron from Z-decay	& 2.7483e+06	&	99.8 $\%$	& 2.84766e+06	&	99.8 $\%$	\\

Conversion veto			& 2.7483e+06	&	100 $\%$	& 2.84766e+06	&	100 $\%$	\\

$\geq$ 1 jets			& 2.72998e+06	&	99.3 $\%$	& 2.84406e+06	&	\textcolor{red}{99.9 $\%$}	\\

$\geq$ 2 jets			& 2.50625e+06	&	91.8 $\%$	& 2.77442e+06	&	\textcolor{red}{97.6 $\%$}	\\

$\geq$ 3 jets			& 1.74245e+06	&	69.5 $\%$	& 2.34978e+06	&	\textcolor{red}{84.7 $\%$}	\\

$\geq$ 4 jets			& 0.753281e+06	&	43.2 $\%$	& 1.38732e+06	&	\textcolor{red}{59.0 $\%$}	\\

\hline
\end{tabular}
\end{table}

\begin{figure}[!h]
\includegraphics[width = 0.45 \textwidth]{/home/annik/Documents/Vub/PhD/ThesisSubjects/AnomalousCouplings/EventSelectionChoice_June2014/UpdatedPtCuts_30July2014/PtDistributionFirstJet_NewPtCuts_El.png}
\includegraphics[width = 0.45 \textwidth]{/home/annik/Documents/Vub/PhD/ThesisSubjects/AnomalousCouplings/EventSelectionChoice_June2014/UpdatedPtCuts_30July2014/PtDistributionSecondJet_NewPtCuts_El.png}\\
\includegraphics[width = 0.45 \textwidth]{/home/annik/Documents/Vub/PhD/ThesisSubjects/AnomalousCouplings/EventSelectionChoice_June2014/UpdatedPtCuts_30July2014/PtDistributionThirdJet_NewPtCuts_El.png}
\includegraphics[width = 0.45 \textwidth]{/home/annik/Documents/Vub/PhD/ThesisSubjects/AnomalousCouplings/EventSelectionChoice_June2014/UpdatedPtCuts_30July2014/PtDistributionFourthJet_NewPtCuts_El.png}
\caption{$P_T$ distributions for the first, second, third and fourth jet respectively. All four histograms are for the semi-electronic decay channel. (Information about the Cross Section and number of events should be ignored for the moment. Correct Cross Section value is not yet used ...) }
\end{figure}

\newpage
The same table can be created for the semi-muonic decay channel which shows a similar result.\\
\begin{table}
\caption{Event selection table before (left) and after (right) the pT cuts were updated to the ones suggested by the TOP reference twiki, for Semi-muon channel $t\bar{t}+jets$. ($19600.8~pb^{-1}$ of int. lumi.)}
\centering
\begin{tabular}{|c|c|c|c|c|}
\hline
				& \multicolumn{2}{|c|}{Old $p_T$ cuts}	& \multicolumn{2}{|c|}{New $p_T$ cuts}	\\
\hline
preselected			& 1.91516e+07	& 		 	& 1.91516e+07	&	 		\\

trigged				& 4.01265e+06	&	20.9 $\%$	& 4.01265e+06	&	20.9 $\%$	\\

Good PV				& 4.01265e+06	&	 100 $\%$	& 4.01265e+06	&	100 $\%$	\\

1 selected muon			& 3.46903e+06	&	 86.4 $\%$	& 3.40822e+06	&	\textcolor{red}{84.9 $\%$}	\\

Veto 2nd muon			& 3.46369e+06	&	 99.8 $\%$	& 3.40291e+06	&	99.8 $\%$	\\

Veto electron			& 3.45244e+06	&	 99.7 $\%$	& 3.39184e+06	&	99.7$\%$	\\

$\geq$ 1 jets			& 3.4281e+06	&	 99.3 $\%$	& 3.38698e+06	&	\textcolor{red}{99.9 $\%$}	\\

$\geq$ 2 jets			& 3.14456e+06	&	 91.7 $\%$	& 3.30209e+06	&	\textcolor{red}{97.5 $\%$}	\\

$\geq$ 3 jets			& 2.19456e+06	&	 69.8 $\%$	& 2.80066e+06	&	\textcolor{red}{84.8 $\%$}	\\

$\geq$ 4 jets			& 944572	&	 43.0 $\%$	& 1.65345e+06	&	\textcolor{red}{59.0 $\%$}	\\
\hline
\end{tabular}
\end{table}

\newpage
\subsection{Influence on choice of b-tag option and use of $\chi^{2}$ $m_{lb}$ - $m_{qqb}$ method}
With these new values considered for the event selection requirements, the obtained percentages and $\frac{s}{b}$ values should be compared again.
The values wich are obtained using these new $p_T$ cuts can be found in the following tables.\\
The first three tables show the results before the use of a $\chi^{2}$ $m_{lb}$ - $m_{qqb}$ method while the last two tables give the obtained numbers when this $\chi^{2}$ $m_{lb}$ - $m_{qqb}$ method is applied.\\
\\
Analyzing the numbers in this tables clearly indicates that the obtained results for the old $p_T$ cuts correspond to the values obtained earlier, which implies that the obtained values can easily be compared against each other and stil represent the same variables. The small differences which are however visible can be easily explained by statistical deviations.\\
Comparing the results obtained using the new $p_T$ cut values with the ones using the old cut values clearly indicates that the old values resulted in slightly better selection efficiency and $\frac{s}{b}$ value. Both for the selection efficiency of the b-quark jets and the light jets a higher percentage is found when using the old $p_T$ cut values. \\
A positive remark corresponding to the new $p_T$ cut values is that the behavior of the different considered b-tag options is similar. The 2 Tight b-tag case without any veto on the light jets results in the highest selection efficiency and $\frac{b}{s}$ value. Hence the choice of the used b-tag option does not need to be updated.\\
\\
Even the use of a $\chi^{2}$ $m_{lb}$ - $m_{qqb}$ method doesn't improve the selection efficiency and $\frac{s}{b}$ values when changing to the newer $p_T$ cut values. However compared to the previous case when no $\chi^{2}$ $m_{lb}$ - $m_{qqb}$ method is applied the difference between the two $p_T$ cut options becomes slightly less significant (from 18 $\%$ to 13 $\%$ difference). Still no improvement can be found impying that lowering the $p_T$ cut on the jets only increases the number of selected events but doesn't insures selection more good events, on the contrary the selection efficiencies decreases even.
\begin{landscape}
\begin{table}[!h] 
 \begin{tabular}{c|c|c|c|c|c|c} 
& \textbf{Option} (no $\chi^{2}$ $m_{lb}$) & all 4 correct & $\geq$ 1 wrong & correct ($\%$)       & $\frac{s}{b}$ & non-matched \\ \hline 
\multirow{6}{*}{\textbf{New $p_T$ cuts}} 
& L b-tags                & 31797 & 36132 & 46.8092 & 0.880023 & 103988  \\ 
&2 M b-tags               & 28380 & 21485 & 56.9137 & 1.32092  & 63717  \\ 
&2 M b-tags, light L-veto & 21578 & 18821 & 53.4122 & 1.14649  & 50549  \\ 
&2 T b-tags               & 17858 & 10684 & 62.5674 & 1.67147  & 33551  \\ 
&2 T b-tags, light M-veto & 16574 & 10496 & 61.2264 & 1.57908  & 31778  \\ 
&2 T b-tags, light L-veto & 12664 & 9552  & 57.004  & 1.3258   & 25440  \\
\hline
\multirow{6}{*}{\textbf{Old $p_T$ cuts}} 
& 2 L b-tags              & 15285 & 13577 & 52.9589 & 1.1258 & 61307  \\ 
& 2 M b-tags              & 13480 & 8590 & 61.0784 & 1.56927 & 39800  \\ 
& 2 M b-tags, light L-veto & 10718 & 7461 & 58.9581 & 1.43654 & 32048 \\ 
& 2 T b-tags              & 8555 & 4231 & 66.9091 & 2.02198 & 21234  \\ 
& 2 T b-tags, light M-veto & 7915 & 4097 & 65.8924 & 1.9319 & 19946  \\ 
& 2 T b-tags, light L-veto & 6327 & 3701 & 63.0933 & 1.70954 & 16156  \\ 
 \end{tabular} 
\caption{Overview of correct and wrong reconstructed events for the different b-tags without the use of a $\chi^{2}$ $m_{lb}$ - $m_{qqb}$ method} 
 \end{table} 
 
 \begin{table}[!h] 
 \begin{tabular}{c|c|c|c|c|c|c} 
&\textbf{Option} (no $\chi^{2}$ $m_{lb}$) & 2 b's correct & $\geq$ 1 b wrong & b's correct ($\%$) & $\frac{s}{b}$ & non-matched \\ \hline 
\multirow{6}{*}{\textbf{New $p_T$ cuts}} 
& 2 L b-tags              & 47926 & 20003 & 70.5531 & 2.39594 & 103988  \\ 
& 2 M b-tags              & 42045 & 7820 & 84.3177 & 5.3766 & 63717  \\ 
& 2 M b-tags, light L-veto & 34622 & 5777 & 85.7001 & 5.99308 & 50549  \\ 
& 2 T b-tags              & 26117 & 2425 & 91.5037 & 10.7699 & 33551  \\ 
& 2 T b-tags, light M-veto & 24974 & 2096 & 92.2571 & 11.9151 & 31778  \\ 
& 2 T b-tags, light L-veto & 20585 & 1631 & 92.6584 & 12.6211 & 25440  \\ 
\hline
\multirow{6}{*}{\textbf{Old $p_T$ cuts}} 
& 2 L b-tags              & 21456 & 7406 & 74.34 & 2.89711 & 61307  \\ 
& 2 M b-tags              & 18863 & 3207 & 85.469 & 5.88182 & 39800  \\ 
& 2 M b-tags, light L-veto & 15859 & 2320 & 87.238 & 6.83578 & 32048  \\ 
& 2 T b-tags              & 11811 & 975 & 92.3745 & 12.1138 & 21234  \\ 
& 2 T b-tags, light M-veto & 11203 & 809 & 93.2651 & 13.848 & 19946  \\ 
& 2 T b-tags, light L-veto & 9424 & 604 & 93.9769 & 15.6026 & 16156  \\ 
 \end{tabular} 
\caption{Overview of correct and wrong reconstructed b-jets for the different b-tags without the use of a $\chi^{2}$ $m_{lb}$ - $m_{qqb}$ method} 
 \end{table} 
 
\begin{table}[!h] 
\begin{tabular}{c|c|c|c|c|c|c} 
&\textbf{Option} (no $\chi^{2}$ $m_{lb}$) & 2 light good  & $\geq$ 1 light wrong & light correct ($\%$) & $\frac{s}{b}$ & non-matched \\ \hline 
\multirow{6}{*}{\textbf{New $p_T$ cuts}} 
& 2 L b-tags              & 33731 & 34198 & 49.6563 & 0.986344 & 103988  \\ 
& 2 M b-tags              & 28893 & 20972 & 57.9424 & 1.37769 & 63717  \\ 
& 2 M b-tags, light L-veto & 22118 & 18281 & 54.7489 & 1.20989 & 50549  \\ 
& 2 T b-tags              & 18030 & 10512 & 63.1701 & 1.71518 & 33551  \\ 
& 2 T b-tags, light M-veto & 16788 & 10282 & 62.017 & 1.63276 & 31778  \\ 
& 2 T b-tags, light L-veto & 12868 & 9348 & 57.9222 & 1.37655 & 25440  \\ 
\hline
\multirow{6}{*}{\textbf{Old $p_T$ cuts}} 
& 2 L b-tags              & 16015 & 12847 & 55.4882 & 1.24659 & 61307  \\ 
& 2 M b-tags              & 13688 & 8382 & 62.0208 & 1.63302 & 39800  \\ 
& 2 M b-tags, light L-veto & 10938 & 7241 & 60.1683 & 1.51056 & 32048  \\ 
& 2 T b-tags              & 8627 & 4159 & 67.4722 & 2.0743 & 21234  \\ 
& 2 T b-tags, light M-veto & 8005 & 4007 & 66.6417 & 1.99775 & 19946  \\ 
& 2 T b-tags, light L-veto & 6409 & 3619 & 63.911 & 1.77093 & 16156  \\ 
 \end{tabular} 
\caption{Overview of correct and wrong reconstructed light jets for the different b-tags without the use of a $\chi^{2}$ $m_{lb}$ - $m_{qqb}$ method} 
 \end{table}

 \begin{table}[!h] 
 \begin{tabular}{c|c|c|c|c|c|c} 
&\textbf{Option} (with $\chi^{2}$ $m_{lb}$) & all 4 correct & $\geq$ 1 wrong & 4 chosen jets correct ($\%$) & $\frac{s}{b}$ & non-matched \\ \hline 
\multirow{6}{*}{\textbf{New $p_T$ cuts}}
& 2 L b-tags              & 24611 & 43318 & 77.4004 & 0.568147 & 103988 \\ 
& 2 M b-tags              & 21813 & 28052 & 76.8605 & 0.777592 & 63717 \\ 
& 2 M b-tags, light L-veto & 16731 & 23668 & 77.5373 & 0.706904 & 50549 \\ 
& 2 T b-tags              & 13681 & 14861 & 76.6099 & 0.920598 & 33551 \\ 
& 2 T b-tags, light M-veto & 12674 & 14396 & 76.4692 & 0.880383 & 31778 \\ 
& 2 T b-tags, light L-veto & 9757 & 12459 & 77.0452 & 0.783129 & 25440 \\ 
\hline
\multirow{6}{*}{\textbf{Old $p_T$ cuts}} 
& 2 L b-tags              & 11893 & 16969 & 77.8083 & 0.700866 & 61307 \\ 
& 2 M b-tags              & 10422 & 11648 & 77.3145 & 0.894746 & 39800 \\ 
& 2 M b-tags, light L-veto & 8369 & 9810 & 78.0836 & 0.853109 & 32048 \\ 
& 2 T b-tags              & 6581 & 6205 & 76.9258 & 1.0606 & 21234 \\ 
& 2 T b-tags, light M-veto & 6080 & 5932 & 76.8162 & 1.02495 & 19946 \\ 
& 2 T b-tags, light L-veto & 4903 & 5125 & 77.4933 & 0.956683 & 16156 \\ 
 \end{tabular} 
 \caption{Overview of correct and wrong reconstructed events for the different b-tags when a $\chi^{2}$ $m_{lb}$ - $m_{qqb}$ method is applied} 
 \end{table} 
 
 \begin{table}[!h] 
 \begin{tabular}{c|c|c|c|c|c|c} 
& \textbf{Option} (with $\chi^{2}$ $m_{lb}$) & Correct b's & Wrong b's & \% b's correct   & $\frac{s}{b}$ & Correct option exists \\ \hline 
\multirow{6}{*}{\textbf{New $p_T$ cuts}}
& 2 L b-tags              & 28190 & 3607 & 88.6562 & 7.81536 & 31797 \\ 
& 2 M b-tags              & 25143 & 3237 & 88.5941 & 7.76738 & 28380 \\ 
& 2 M b-tags, light L-veto & 19183 & 2395 & 88.9007 & 8.0096 & 21578 \\ 
& 2 T b-tags              & 15736 & 2122 & 88.1174 & 7.41565 & 17858 \\ 
& 2 T b-tags, light M-veto & 14586 & 1988 & 88.0053 & 7.33702 & 16574 \\ 
& 2 T b-tags, light L-veto & 11178 & 1486 & 88.266 & 7.52221 & 12664 \\ 
\hline
\multirow{6}{*}{\textbf{Old $p_T$ cuts}} 
& 2 L b-tags              & 13935 & 1350 & 91.1678 & 10.3222 & 15285 \\ 
& 2 M b-tags              & 12306 & 1174 & 91.2908 & 10.4821 & 13480 \\ 
& 2 M b-tags, light L-veto & 9808 & 910 & 91.5096 & 10.778 & 10718 \\ 
& 2 T b-tags              & 7765 & 790 & 90.7656 & 9.82911 & 8555 \\ 
& 2 T b-tags, light M-veto & 7172 & 743 & 90.6128 & 9.65276 & 7915 \\ 
& 2 T b-tags, light L-veto & 5743 & 584 & 90.7697 & 9.8339 & 6327 \\ 
 \end{tabular} 
 \caption{Overview of the number of times the correct b-jet combination is chosen when using a $\chi^{2}$ $m_{lb}$ - $m_{qqb}$ method} 
 \end{table} 

\end{landscape}

%°°°°°°°°°°°°°°°°°°°°°°°°°°°°°°°°°°°°°°°°°°°°°°°°°°°°°°°°°°°°°°°°°°°°°°°°

%°°°°°°°°°°°°°°°°°°°°°°°°°°°°°°°°°°°°°°°°°°°°°°°°°°°°°°°°°°°°°°°°°°°°°°°°
%	CHAPTER: Event Corrections and Reconstruction
\chapter{Event corrections and reconstruction}
%
\section{Which event corrections should be applied?}

\subsection{Trigger choice}
Two possible triggers exist, namely the so-called SingleLepton triggers or the CrossTriggers. The former kind only uses the information of the lepton while the latter one combines the lepton information together with the jets present in the event. These second kind of triggers were developed since it was expected that the $p_T$ cut which had to be applied on this SingleLepton trigger would become too high to be physically relevant. However it has been found that the applied $p_T$ cut doesn't differ that much between these two triggers, especially in the muon case the difference is almost negligible.\\
\\
Since the scale factors which should be applied are much more complex in the case of the CrossTriggers preference is given to these SingleLepton triggers.\\
Minor disadvantage of these kind of triggers is that no information can be found on the TOP Twiki page (\url{https://twiki.cern.ch/twiki/bin/viewauth/CMS/TopTrigger}) and currently no other Twiki page has been found with similar information ... Trigger contacts of TOP group prefer the use of the CrossTriggers since a lot of time and effort has been put into the development of these triggers. Probably the reason why no clear documentation is found on the Top Twiki page.\\
But James is using the same triggers in his analysis so all these triggers, and the different versions, can be found in the following two analyzer files:
\begin{itemize}
 \item \url{https://github.com/TopBrussels/FourTops/blob/master/FourTop_EventSelection.cc} which contains information about the muon event selection and used triggers.
 \item \url{https://github.com/TopBrussels/FourTops/blob/master/FourTop_EventSelection_El.cc} which contains the same information but about the electron case.
\end{itemize}

\subsection{Lumi- or PileUp Reweighting}
The choice of the root file which should be used for the LumiReweighting is related to the Monte Carlo samples which will be used. If the Summer 12 MC samples, which describe the 8 TeV data, are used the \textit{S10} file should be used.\\
LumiReweighting is actually taking into account the influence of PileUp but can't be called this way when publishing results since a systematic influence of PileUp is no physical variable. However the luminosity of MinimumBias events is. This systematic influence can be calculated by comparing the influence of the up and down part.

\subsection{Lepton Scale Factors}
Lepton scale factors should be used and can be found in the code of James.\\
In the muon case the scale factors correspond to the entire dataset (ABCD) since different scale factors have been obtained for the different runs. Therefore caution should be applied when only running over a limited range of data in order to avoid to run only over the A part. However the influence shouldn't be too large ...\\
\\
In the electron case the scale factors have been hard coded and do not vary for different parts of the data sample. \textbf{Therefore it should be checked whether the hard-coded numbers which are used in James code are still the correct ones which should be used. Hence a twiki with this information should be found ...}

\subsection{Jet Energy Correction factors}
Information available in the analyzer of James is up-to-date and can be copied.


\subsection{Jet corrections (on the fly ...)}
In James analyzer code Jet corrections are applied on the fly which is no problems, since even if the considered sample has these corrections already applied, the method of incorporating these jet corrections uses the gen information. So this avoids that these jet corrections would be applied twice ...
%**************************************************

\section{Choice of Monte Carlo samples}
Same samples as James can be used, which are the latest Jan22 rereco samples. A full overview of these samples can be found on:
\url{https://docs.google.com/spreadsheet/ccc?key=0Apc0aJdnaVjSdFVaLVU2dlk4RDZHcjlaakE3NWIxTUE&usp=sharing_eil#gid=0}

%°°°°°°°°°°°°°°°°°°°°°°°°°°°°°°°°°°°°°°°°°°°°°°°°°°°°°°°°°°°°°°°°°°°°°°°°

%°°°°°°°°°°°°°°°°°°°°°°°°°°°°°°°°°°°°°°°°°°°°°°°°°°°°°°°°°°°°°°°°°°°°°°°°
%	CHAPTER: Theory link with partial widths
\chapter{Theory link with partial widths}


\section{Partial width of top-quark decay}

The partial width of the top-quark decay can be expressed in terms of the anomalous couplings in the $Wtb$ interaction as represented in the following equations.\\
The first and less extensive one describes the longitudinal decay.

\begin{eqnarray}
  \Gamma_{0} & = & \frac{g^{2} \vert \vec{q} \vert}{32 \pi} \left\lbrace \frac{m_{t}^{2}}{m_{W}^{2}} \left[ \vert V_{L} \vert^{2} + \vert V_{R} \vert^{2} \right] (1 - x_{W}^{2} - 2x_{b}^{2} -x_{W}^{2} x_{b}^{2} + x_{b}^{4}) - 4x_{b}Re V_{L}V_{R}^{*} \right. \nonumber \\
             &   & + \left[ \vert g_{L} \vert^{2} + \vert g_{R} \vert^{2} \right] (1 - x_{W}^{2} + x_{b}^{2}) - 4 x_{b} Re g_{L} g_{R}^{*} \nonumber \\
             &   & - 2 \frac{m_{t}}{m_{W}} Re \left[ V_{L}g_{R}^{*} + V_{R}g_{L}^{*} \right] (1- x_{W}^{2} - x_{b}^{2}) \nonumber \\
             &   & \left. +2 \frac{m_t}{m_W} x_b Re \left[ V_{L}g_{L}^{*} + V_{R}g_{R}^{*} \right] (1+ x_{W}^{2} - x_{b}^{2}) \right\rbrace
\end{eqnarray}

with:
\begin{subequations} \label{eq::Simplified}
 \begin{align} 
  x_W & = \frac{m_{W}}{m_{t}} \\
  x_b & = \frac{m_{b}}{m_{t}} \\
  \vert \vec{q} \vert & = \frac{1}{2 m_{t}} \sqrt{m_{t}^{4} + m_{W}^{4} + m_{b}^{4} - 2m_{t}^{2} m_{W}^{2} - 2m_{t}^{2} m_{b}^{2} - 2m_{W}^{2}m_{b}^{2}}  
 \end{align}
\end{subequations}

A similar equation can also be formulated for the left- and right-handed top-quark decay, which only differ partially with a minus sign. The right-handed part corresponds to the plus-sign option while the left-handed contribution contains the minus-sign.
\begin{eqnarray}
 \Gamma_{R,L} & = & \frac{g^{2} \vert \vec{q} \vert}{32 \pi} \left\lbrace \frac{m_{t}^{2}}{m_{W}^{2}} \left[ \vert V_{L} \vert^{2} + \vert V_{R} \vert^{2} \right] (1 - x_{W}^{2} + x_{b}^{2}) - 4x_{b}Re V_{L}V_{R}^{*} \right. \nonumber \\
            &   & + \frac{m_{t}^{2}}{m_{W}^{2}} \left[ \vert g_{L} \vert^{2} + \vert g_{R} \vert^{2} \right] (1 - x_{W}^{2} - 2x_{b}^{2} -x_{W}^{2} x_{b}^{2} + x_{b}^{4}) - 4x_{b}Re g_{L}g_{R}^{*} \nonumber \\
            &   & - 2 \frac{m_t}{m_W} Re \left[ V_{L}g_{R}^{*} + V_{R}g_{L}^{*} \right] (1- x_{W}^{2} - x_{b}^{2}) \nonumber \\
            &   &  \left. + 2 \frac{m_t}{m_W} x_b Re \left[ V_{L}g_{L}^{*} + V_{R}g_{R}^{*} \right] (1+ x_{W}^{2} - x_{b}^{2}) \right\rbrace \nonumber \\
           &   & \pm \frac{g^{2}}{64 \pi} \frac{m_{t}^{3}}{m_{W}^{2}} \left\lbrace -x_{W}^{2} \left[ \vert V_{L} \vert^{2} - \vert V_{R} \vert^{2} \right] + \left[ \vert g_{L} \vert^{2} - \vert g_{R} \vert^{2} \right] (1-x_{b}^{2}) \right. \nonumber \\
            &   &  \left. + 2 x_{W} Re \left[ V_{L}g_{R}^{*} + V_{R}g_{L}^{*} \right] + 2x_{W} x_{b} Re \left[ V_{L}g_{L}^{*} + V_{R}g_{R}^{*} \right] \right\rbrace \nonumber \\
            &   & \times (1-2x_{W}^{2} - 2x_{b}^{2} + x_{W}^{4} - 2x_{W}^{2} x_{b}^{2} + x_{b}^{4})
\end{eqnarray}

In order to transform these partial width formulas into helicity fractions, also the total width of the top quark decay is needed. This because each helicity fraction is defined as the corresponding partial width divided by the total width.

\begin{eqnarray}
 \Gamma & = & \frac{g^{2} \vert \vec{q} \vert}{32 \pi} \frac{m_{t}^{2}}{m_{W}^{2}} \left\lbrace \left[ \vert V_{L} \vert^{2} + \vert V_{R} \vert^{2} \right] (1 + x_{W}^{2} - 2x_{b}^{2} - 2 x_{W}^{4} + x_{W}^{2} x_{b}^{2} + x_{b}^{4}) - 4x_{b}Re V_{L}V_{R}^{*} \right. \nonumber \\
        &   & -12 x_{W}^{2} x_{b} Re V_{L} V_{R}^{*} + 2 \left[ \vert g_{L} \vert^{2} + \vert g_{R} \vert^{2} \right] \left( 1 - \frac{x_{W}^{2}}{2} - 2 x_{b}^{2} - \frac{x_{X}^{4}}{2} - \frac{x_{W}^{2} x_{b}^{2}}{2} +x_{b}^{4} \right) \nonumber \\
        &   & -12 x_{W}^{2} x_{b} Re g_{L} g_{R}^{*} - 6 x_{W} Re \left[ V_{L}g_{R}^{*} + V_{R}g_{L}^{*} \right] (1 - x_{W}^{2} - x_{b}^{2}) \nonumber \\
        &   & \left. + 6 x_{W} x_{b} Re \left[ V_{L}g_{L}^{*} + V_{R}g_{R}^{*} \right] (1 + x_{W}^{2} - x_{b}^{2}) \right\rbrace
\end{eqnarray}

\section{Simplification in limit-cases}
\subsection{Only 1 coupling non-zero}
If we consider the case where only the $\VL$ coupling parameter is non-zero, the above definitions get reduced to the following formulas.

\begin{eqnarray}
 \Gamma_{0} & = & \frac{g^{2} \vert \vec{q}}{32 \pi} \frac{m_{t}^{2}}{m_{W}^{2}} \vert V_{L} \vert^{2} (1 - x_{W}^{2} - 2x_{b}^{2} -x_{W}^{2} x_{b}^{2} + x_{b}^{4}) \\
 \Gamma_{R,L} & = & \frac{g^{2} \vert \vec{q}}{32 \pi}  \vert V_{L} \vert^{2} (1 - x_{W}^{2} + x_{b}^{2}) \pm \frac{g^{2}}{64 \pi} \frac{m_{t}^{3}}{m_{W}^{2}} \left\lbrace -x_{W}^{2}  \vert V_{L} \vert^{2} (1-x_{b}^{2}) \right\rbrace \\
 \Gamma & = & \frac{g^{2} \vert \vec{q}}{32 \pi} \frac{m_{t}^{2}}{m_{W}^{2}} \vert V_{L} \vert^{2} (1 + x_{W}^{2} - 2x_{b}^{2} - 2 x_{W}^{4} + x_{W}^{2} x_{b}^{2})
\end{eqnarray}

So this implies that the helicity fractions can be defined as follows:

\begin{eqnarray}
 F_{0}   & = \frac{\Gamma_{0}}{\Gamma} =   & \frac{(1 - x_{W}^{2} - 2x_{b}^{2} -x_{W}^{2} x_{b}^{2} + x_{b}^{4})}{(1 + x_{W}^{2} - 2x_{b}^{2} - 2 x_{W}^{4} + x_{W}^{2} x_{b}^{2})} \label{eq::F0MasslessB} \\
 F_{R,L} & = \frac{\Gamma_{R,L}}{\Gamma} = & \frac{m_{W}^{2}}{m_{t}^{2}} \frac{(1 - x_{W}^{2} + x_{b}^{2})}{(1 + x_{W}^{2} - 2x_{b}^{2} - 2 x_{W}^{4} + x_{W}^{2} x_{b}^{2})} \nonumber \\
         &                                 & \pm \frac{m_{t}}{2 \vert \vec{q} \vert} \frac{ -x_{W}^{2} (1-x_{b}^{2})}{(1 + x_{W}^{2} - 2x_{b}^{2} - 2 x_{W}^{4} + x_{W}^{2} x_{b}^{2})} \label{eq::FRLMasslessB}
\end{eqnarray}

From the above two equations can easily be concluded that in cases where only one of the Wtb-coupling coefficients is non-zero, there will be no influence visible on the helicity fractions. This because no interference between different coupling coefficients occurs which allows the cancellation of this single coupling coefficient. This behavior is indeed retrieved in the different $\csTh$ distributions which have been studied in detail. The ones with only one coupling constant active can be found in Figure \ref{fig::CosThetaOneCoupling}.\\

\begin{figure}[h!]
 \centering
 \includegraphics[width = 0.45 \textwidth]{Afbeeldingen/Chapter_LinkWithTopWidth/CosThetaResults/RVLvsIVL/RVLIVL_CosTheta_IVLFixedTo00.pdf}    %RVL varied
 \includegraphics[width = 0.45 \textwidth]{Afbeeldingen/Chapter_LinkWithTopWidth/CosThetaResults/RVLvsRVR/RVLRVR_CosTheta_RVLFixedTo00.pdf}\\  %RVR varied
 \includegraphics[width = 0.45 \textwidth]{Afbeeldingen/Chapter_LinkWithTopWidth/CosThetaResults/IVLvsIVR/IVLIVR_CosTheta_IVLFixedTo00.pdf}    %IVR varied
 \includegraphics[width = 0.45 \textwidth]{Afbeeldingen/Chapter_LinkWithTopWidth/CosThetaResults/IVLvsIVR/IVLIVR_CosTheta_IVRFixedTo00.pdf}\\  %IVL varied
 \includegraphics[width = 0.45 \textwidth]{Afbeeldingen/Chapter_LinkWithTopWidth/CosThetaResults/RgRvsRgL/RgLRgR_CosTheta_RgLFixedTo00.pdf}    %RgR varied
 \includegraphics[width = 0.45 \textwidth]{Afbeeldingen/Chapter_LinkWithTopWidth/CosThetaResults/RgRvsRgL/RgLRgR_CosTheta_RgRFixedTo00.pdf}    %RgL varied
 \caption{Distributions of $\csTh$ for configurations where only one of the Wtb-coupling coefficients is non-zero while the other one is varied between $-1$ and $1$. The two upper distributions show the case when only the real part of the two vector couplings is non-zero while the two middle ones represent the case when only their imaginary part is non-zero. Finally the two lower distributions give the $\csTh$ distribution for the real part of the tensor couplings. The same conclusion holds for all distributions shown here, namely that no shape difference occurs when only one of the couplings is non-zero as was indicated by the formulas given above.}
 \label{fig::CosThetaOneCoupling}
\end{figure}

It is important to note that the same is also true when only the real and imaginary part of one coupling are varied. In those cases there is also no interference between different coupling constants which is necessary in order to introduce a different between the partial widths and the total width. For example when considering only the left-handed vector coupling, the only term which remains in the width definitions is $\vert V_{L} \vert^{2}$ $=$ $Re(V_{L})^{2} + Im(V_{L})^{2}$. However since it does not matter whether this term consists of both the real and the imaginary part or only the real part, it will cancel out when dividing the partial width part by the total width. This behavior is indeed retrieved in the $\csTh$ distributions, as can be seen in Figure \ref{fig::CosThetaOneFullCoupling}.\\

\begin{figure}[h!]
 \centering
 \includegraphics[width = 0.45 \textwidth]{Afbeeldingen/Chapter_LinkWithTopWidth/CosThetaResults/RVLvsIVL/RVLIVL_CosTheta_IVLFixedTo05.pdf}
 \includegraphics[width = 0.45 \textwidth]{Afbeeldingen/Chapter_LinkWithTopWidth/CosThetaResults/RVLvsIVL/RVLIVL_CosTheta_RVLFixedTo05.pdf}\\
 \includegraphics[width = 0.45 \textwidth]{Afbeeldingen/Chapter_LinkWithTopWidth/CosThetaResults/RVRvsIVR/RVRIVR_CosTheta_IVRFixedTo05.pdf}
 \includegraphics[width = 0.45 \textwidth]{Afbeeldingen/Chapter_LinkWithTopWidth/CosThetaResults/RVRvsIVR/RVRIVR_CosTheta_RVRFixedTo05.pdf}
 \caption{}
 \label{fig::CosThetaOneFullCoupling}
\end{figure}

\subsection{Massless b-limit}

Within the Wtb interaction the mass of the bottom quark is almost negligible compared to the massive W-boson and the top quark. Hence it is rather standard to use a so-called massless b-limit when considering the Wtb interaction, an approach which also explains the suppression of the right-handed helicity fraction of the W-boson.\\
When applying this assumption, the partial and total width formulas can be simplified significantly as will be shown in the following equations since the terms containing $x_{b}$ can be neglected. For simplicity only the real parts of the vector couplings are considered, but the same is true for other combinations.

\begin{eqnarray}
 \Gamma_{0}   & = & \frac{g^{2} \vert \vec{q}}{32 \pi} \frac{m_{t}^{2}}{m_{W}^{2}} \left[ \vert V_{L} \vert^{2} + \vert V_{R} \vert^{2} \right] (1 - x_{W}^{2} ) \\
 \Gamma_{R,L} & = & \frac{g^{2} \vert \vec{q}}{32 \pi} \frac{m_{t}^{2}}{m_{W}^{2}} \left[ \vert V_{L} \vert^{2} + \vert V_{R} \vert^{2} \right] (1 - x_{W}^{2})  \nonumber \\
              &   & \pm \frac{g^{2}}{64 \pi} \frac{m_{t}^{3}}{m_{W}^{2}} \left\lbrace -x_{W}^{2} \left[ \vert V_{L} \vert^{2} - \vert V_{R} \vert^{2}  \right] \right\rbrace (1-2x_{W}^{2} + x_{W}^{4})\\
 \Gamma       & = & \frac{g^{2} \vert \vec{q}}{32 \pi} \frac{m_{t}^{2}}{m_{W}^{2}} \left[ \vert V_{L} \vert^{2} + \vert V_{R} \vert^{2} \right] (1 + x_{W}^{2} - 2 x_{W}^{4})
\end{eqnarray}

From this can be seen that the longitudinal helicity fraction is not influenced at all when working in the massless b-limit. The right-handed and left-handed helicity fractions, on the other hand, have an opposite coupling-coefficient dependent part. This is again clearly visible in the studied $\csTh$ distributions, which all behave simiar around $\csTh$ = $0$. This is shown in Figure \ref{fig::CosThetaRVLRVR}.

\begin{figure}[h!]
 \centering
 \includegraphics[width = 0.45 \textwidth]{Afbeeldingen/Chapter_LinkWithTopWidth/CosThetaResults/RVLvsRVR/RVLRVR_CosTheta_RVLFixedTo02.pdf}
 \includegraphics[width = 0.45 \textwidth]{Afbeeldingen/Chapter_LinkWithTopWidth/CosThetaResults/RVLvsRVR/RVLRVR_CosTheta_RVRFixedTo02.pdf}
 \caption{...}
 \label{fig::CosThetaRVLRVR}
\end{figure}

\subsection{Only 1 coupling non-zero within the massless b-limit}

The equations defined when considering only 1 non-zero coupling, Equations (\ref{eq::F0MasslessB}) and (\ref{eq::FRLMasslessB}), can be simplified even further when assuming the massless b-limit.

\begin{eqnarray}
 F_{0}   & = & \frac{(1 - x_{W}^{2} - 2x_{b}^{2} -x_{W}^{2} x_{b}^{2} + x_{b}^{4})}{(1 + x_{W}^{2} - 2x_{b}^{2} - 2 x_{W}^{4} + x_{W}^{2} x_{b}^{2})} \\
 F_{R,L} & = & \frac{m_{W}^{2}}{m_{t}^{2}} \frac{(1 - x_{W}^{2} + x_{b}^{2})}{(1 + x_{W}^{2} - 2x_{b}^{2} - 2 x_{W}^{4} + x_{W}^{2} x_{b}^{2})} \nonumber \\
         &                                 & \pm \frac{m_{t}}{2 \vert \vec{q} \vert} \frac{ -x_{W}^{2} (1-x_{b}^{2})}{(1 + x_{W}^{2} - 2x_{b}^{2} - 2 x_{W}^{4} + x_{W}^{2} x_{b}^{2})} \label{eq::FRLMasslessB}
\end{eqnarray}

For the simplified case where only 1 of the couplings is varied the dependency on the top quark mass can be calculated explicitely by using Equations (\ref{eq::Simplified}). Applying these definitions changes Equations (\ref{eq::F0MasslessB}) and (\ref{eq::FRLMasslessB}) as follows\footnote{This because within this massless b-limit $\vert \vec{q} \vert$ can be simplified to $\frac{m_{t}^{2} - m_{W}^{2}}{2m_{t}}$.}:

\begin{subequations}
 \begin{align*}
  F_{0} & = \frac{1 - x_{W}^{2} - 2x_{b}^{2} -x_{W}^{2} x_{b}^{2} + x_{b}^{4}}{1 + x_{W}^{2} - 2x_{b}^{2} - 2 x_{W}^{4} + x_{W}^{2} x_{b}^{2}} \\
        & = \frac{1 - \frac{m_{W}^{2}}{m_{t}^{2}} - 2 \frac{m_{b}^{2}}{m_{t}^{2}} - \frac{m_{W}^{2}}{m_{t}^{2}} \frac{m_{b}^{2}}{m_{t}^{2}} + \frac{m_{b}^{4}}{m_{t}^{4}}}{1 + \frac{m_{W}^{2}}{m_{t}^{2}} - 2 \frac{m_{b}^{2}}{m_{t}^{2}} - 2 \frac{m_{W}^{4}}{m_{t}^{2}} + \frac{m_{W}^{2}}{m_{t}^{2}} \frac{m_{b}^{2}}{m_{t}^{2}}} \\
        & = \frac{m_{t}^{4}}{m_{t}^{4}} \frac{m_{t}^{4} - m_{W}^{2} m_{t}^{2} - 2m_{b}^{2} m_{t}^{2} - m_{W}^{2} m_{b}^{2} + m_{b}^{4}}{1 + m_{W}^{2} m_{t}^{2} - 2 m_{b}^{2} m_{t}^{2} - 2 m_{W}^{4} + m_{W}^{2} m_{b}^{2} + m_{b}^{4}} \\
        & \textcolor{green}{\approx_{m_{b} = 0}} \frac{m_{t}^{4} - m_{W}^{2} m_{t}^{2}}{m_{t}^{4} + m_{W}^{2} m_{t}^{2} - 2m_{W}^{4}}
 \end{align*}
\end{subequations}

\begin{subequations}
 \begin{align*}
  F_{R,L} = & ~\frac{m_{W}^{2}}{m_{t}^{2}} \frac{1 - x_{W}^{2} + x_{b}^{2}}{1 + x_{W}^{2} - 2x_{b}^{2} - 2 x_{W}^{4} + x_{W}^{2} x_{b}^{2}} \pm \frac{m_{t}}{2 \vert \vec{q} \vert} \frac{ -x_{W}^{2} (1-2x_{W}^{2} - 2x_{b}^{2} + x_{W}^{4} - 2x_{W}^{2} x_{b}^{2} + x_{b}^{4})}{1 + x_{W}^{2} - 2x_{b}^{2} - 2 x_{W}^{4} + x_{W}^{2} x_{b}^{2}} \\
          = & ~\frac{m_{t}^{4}}{m_{t}^{4}} \left\lbrace \frac{m_{W}^{2}}{m_{t}^{2}}  \frac{1 - m_{W}^{2}m_{t}^{2} + m_{b}^{2}m_{t}^{2}}{1 + m_{W}^{2}m_{t}^{2} - 2m_{b}^{2}m_{t}^{2} - 2 m_{W}^{4} + m_{W}^{2} m_{b}^{2}} \right. \\
            & ~\left. \pm \frac{m_{t}}{2 \vert \vec{q} \vert} \frac{ -m_{W}^{2}}{m_{t}^{2}} \frac{m_{t}^{4} - 2m_{W}^{2}m_{t}^{2} - 2m_{b}^{2}m_{t}^{2} + m_{W}^{4} - 2m_{W}^{2} m_{b}^{2} + m_{b}^{4}}{m_{t}^{4} + m_{W}^{2}m_{t}^{2} - 2m_{b}^{2}m_{t}^{2} - 2 m_{W}^{4} + m_{W}^{2} m_{b}^{2}} \right\rbrace \\          
          \textcolor{green}{\approx} & ~\frac{m_{W}^{2}}{m_{t}^{2}}  \frac{m_{t}^{4} - m_{W}^{2}m_{t}^{2}}{1 + m_{W}^{2}m_{t}^{2} - 2 m_{W}^{4}} \pm \frac{m_{t}}{2 \vert \vec{q} \vert} \frac{ -m_{W}^{2}}{m_{t}^{2}} \frac{m_{t}^{4} - 2m_{W}^{2}m_{t}^{2} + m_{W}^{4}}{m_{t}^{4} + m_{W}^{2}m_{t}^{2} - 2 m_{W}^{4}} \\
          \textcolor{green}{\approx} & ~\frac{m_{W}^{2}}{m_{t}^{2}}  \frac{m_{t}^{2}(m_{t}^{2} - m_{W}^{2})}{m_{t}^{4} + m_{W}^{2}m_{t}^{2} - 2 m_{W}^{4}} \pm \frac{2m_{t}^{2}}{2 (m_{t}^{2} - m_{W}^{2})} \frac{ -m_{W}^{2}}{m_{t}^{2}} \frac{(m_{t}^{2} - m_{W}^{2})^{2}}{m_{t}^{4} + m_{W}^{2}m_{t}^{2} - 2 m_{W}^{4}} \\
          \textcolor{green}{\approx} & ~m_{W}^{2} \left( \frac{m_{t}^{2} - m_{W}^{2}}{m_{t}^{4} + m_{W}^{2}m_{t}^{2} - 2 m_{W}^{4}} \right) \mp m_{W}^{2} \left( \frac{m_{t}^{2} - m_{W}^{2}}{m_{t}^{4} + m_{W}^{2}m_{t}^{2} - 2 m_{W}^{4}} \right) \\
          & = \left\{ \begin{array}{l} 
            ~ 0 \qquad \qquad \qquad \quad \quad ~~~ \textrm{for right-handed helicity fraction} \\
            ~2m_{W}^{2} \frac{m_{t}^{2} - m_{W}^{2}}{m_{t}^{4} + m_{W}^{2}m_{t}^{2} - 2 m_{W}^{4}} \quad \textrm{for left-handed helicity fraction}
          \end{array} \right. 
 \end{align*}
\end{subequations}

Hence when creating the $\csTh$ distribution for different top-quark masses it is expected to see a clear shape diffence for the left-handed helicity fraction. As postulated by the Standard Model, no right-handed contribution is expected and from the above equation is clear that this contribution does not depend on the considered top-quark mass. Again this can be seen on the $\csTh$ distributions when varying the top quark mass between $153$ GeV and $193$ GeV in steps of $10$ GeV, as shown in Figure \ref{fig::CosThetaTopMass}.

\begin{figure}[!h]
 \centering
 \includegraphics[width = 0.85 \textwidth]{Afbeeldingen/Chapter_LinkWithTopWidth/CosThetaResults/RVLvsRVR/RVLRVR_CosTheta_CosTheta_TopMassVaried.pdf}
 \caption{Distribution of $\csTh$ when varying the top quark mass.}
 \label{fig::CosThetaTopMass}
\end{figure}

\section{Understanding symmetric behavior}

While studying all the different $\csTh$ distributions for all the considered configurations, there was one peculiar observation. Every studied $\csTh$ distribution clearly showed a symmetric relationship between the negative and positive part of each coupling coefficient. Again this can be explained using the partial top quark width definitions given above since with these equations it is possible to track down the term which depends on the sign of the coefficient.\\
The only terms which need to know the sign of the considered coefficient are the mixing terms such as $Re V_{L} V_{R}^{*}$ for example. However most of these mixing terms, and all of the most straightforward ones, are scaled with a factor $x_{b}$ implying that they are negligible in the massless b-limit. So mixing the vector and tensor couplings should give less symmetric $\csTh$ distributions than the current studied mixings between vector and tensor couplings separately.\\
Figure \ref{fig::CosThetaHeavyB} clearly shows that the symmetric behavior is caused by low mass of the bottom quark. Both distributions represent an identical configuration, with the only difference the bottom-quark mass.

\begin{figure}[h!]
 \centering
 \includegraphics[width = 0.45 \textwidth]{Afbeeldingen/Chapter_LinkWithTopWidth/CosThetaResults/RVLvsRVR/RVLRVR_CosTheta_NormalBMass_RVRChange_RVLSM.pdf}
 \includegraphics[width = 0.45 \textwidth]{Afbeeldingen/Chapter_LinkWithTopWidth/CosThetaResults/RVLvsRVR/RVLRVR_CosTheta_HeavyBMass_RVRChange_RVLSM.pdf}
 \caption{Influence of bottom-quark mass}
 \label{fig::CosThetaHeavyB}
\end{figure}
%°°°°°°°°°°°°°°°°°°°°°°°°°°°°°°°°°°°°°°°°°°°°°°°°°°°°°°°°°°°°°°°°°°°°°°°°

%°°°°°°°°°°°°°°°°°°°°°°°°°°°°°°°°°°°°°°°°°°°°°°°°°°°°°°°°°°°°°°°°°°°°°°°°
%	CHAPTER: Likelihood optimization
\chapter{Likelihood optimization}
First results of the \NegLL distribution of the right-handed vector coefficient, $\VR$, for reco-level events indicated that the expected shape, a minimum around $\VR$ = 0, is not retrieved. However this behavior is recovered for generator-level events. Therefore this can be seen as a clear influence of the event selection and further investigation of the origin of this deviation might possibly result in an improved likelihood distribution. Hence effort has been put in investigating whether a specific cut on the likelihood distribution can result in the desired distribution.\\
Since the reco-level events are simulated using the Standard Model constraints, namely $\VR$ = 0, this value should be recovered using the MadWeight output in order to exclude any bias caused by the event selection.

\section{Comparison between correct, wrong and un-matched jet combinations}
As a first step the chosen \ttbar jet combination has been divided in distinct categories based on the jet-parton matching output: correctly matched, wrongly matched and un-matched jet combinations. Since the wrongly-matched can be considered as a kind of background sample while the correctly matched correspond to clear signal events, their comparison can result in a possible hint for an optimal cut in order to reduce the contribution of background events. The number of events in each of these categories is given in the following table:
\begin{table}[!h]
 \centering
 \caption{Grouping of the different jet-matching types for 10 000 ttbar semi-muonic (+) events.}
 \begin{tabular}{c|c|c}
  Correctly matched 	& Wrongly matched  	& Unmatched  	\\
  \hline
  13 608 		& 15 345 		& 34 176    	\\
  21.56 $\%$ 		& 24.31 $\%$		& 54.14 $\%$
 \end{tabular}
\end{table}

The top mass distributions, leptonically and hadronically decaying top, for each of the categories can be seen in Figure \ref{fig::MTRecoDistr}.
\begin{figure}[h!t]
 \centering
 \includegraphics[width = 0.49 \textwidth]{/home/annik/Documents/Vub/PhD/ThesisSubjects/AnomalousCouplings/April2015_LikelihoodCuts/TopMassInfluence/CorrectReco_TopMassHadr.pdf}
 \includegraphics[width = 0.49 \textwidth]{/home/annik/Documents/Vub/PhD/ThesisSubjects/AnomalousCouplings/April2015_LikelihoodCuts/TopMassInfluence/CorrectReco_TopMassLept.pdf}
 \includegraphics[width = 0.49 \textwidth]{/home/annik/Documents/Vub/PhD/ThesisSubjects/AnomalousCouplings/April2015_LikelihoodCuts/TopMassInfluence/WrongReco_TopMassHadr.pdf}
 \includegraphics[width = 0.49 \textwidth]{/home/annik/Documents/Vub/PhD/ThesisSubjects/AnomalousCouplings/April2015_LikelihoodCuts/TopMassInfluence/WrongReco_TopMassLept.pdf}
 \includegraphics[width = 0.49 \textwidth]{/home/annik/Documents/Vub/PhD/ThesisSubjects/AnomalousCouplings/April2015_LikelihoodCuts/TopMassInfluence/UnmatchedReco_TopMassHadr.pdf}
 \includegraphics[width = 0.49 \textwidth]{/home/annik/Documents/Vub/PhD/ThesisSubjects/AnomalousCouplings/April2015_LikelihoodCuts/TopMassInfluence/UnmatchedReco_TopMassLept.pdf}
 \caption{Distributions for the hadronically (left) and leptonically (right) decaying top quark mass for the correctly matched, wrongly matched and unmatched jet combinations, respectively.}
 \label{fig::MTRecoDistr}
\end{figure}

The obtained mass distributions show mostly the expected behavior, indicating that the leptonically decaying top quark is less dependent of the correctness of the chosen jet combinations. The hadronically decyaing top quark on the other hand is significantly influenced by the chosen jet combination as can be seen by the large difference in tail for the correctly and wrongly matched jet combinations. \\

The mass distribution of the hadronically decaying top quark for un-matched reco-level events show a rather unexpected behavior on the other hand ...\\
\textit{I expected that this distribution would more be like a combination of the correctly and wrongly matched ones since in quite a lot of cases the correct parton-level jet combination is not available in the list of matched jet combinations. These should then be recovered when looking at the entire list of unmatched jet combinations. Of course quite often the wrong combination will be chosen, but still it seems rather strange that the tail of the unmatched jet combinations is significantly higher and more energetic than the one corresponding to the wrongly matched jet combinations ...}

\subsection{Inefficiency of MadWeight depends on category-type}
When using the different categories for for Matrix Element calculations, the number of events succesfully calculated depends quite heavily on the category considered. Althought it is important to mention that quite often this efficiency can vary when resubmitting the same configuration which could hint towards an influence of the cluster used for running the calculations.\\
The number of remaining events which have been used for the measurements discussed further in this Chapter are given in Table \ref{table::MWEff}.

\begin{table}[h!t]
 \caption{Number of events for each of the four considered categories succesfully calculated by MadWeight. The number of failing events, for which a weight equal to $0.0$ has been returned or for which one of the considered configurations is missing, is especially significant for the category of unmatched reco-level events.}
 \label{table::MWEff}
 \begin{tabular}{c|c|c|c|c}
  \multirow{2}{*}{Category}	& Generator-level 	& \multicolumn{3}{|c}{Reco-level events} 			\\
				& events 		& Correctly matched	& Wrongly matched 	& Unmatched 	\\
  \hline
  Succesfull events 		& 10000 		& 9982 			& 9085			& 7538 		
 \end{tabular}
\end{table}

\section{Measurement of top-quark mass using Matrix Element Method}
In order to check the influence of the event selection, first the measurement of the top-quark mass has been performed using MadWeight.
This event selection influence can then be understood by comparing the obtained measurement of the top quark mass on generator level with the one on reco-level.
The results of the generator-level measurement can be found in Figure \ref{fig::MTGenLL} and Table \ref{table::MTGenFit} while Figure \ref{fig:MTGenDistr}\\
For this type of events no acceptance normalisation can be applied since no event selection is applied on these generator-level events. Hence the only normalisation which can be applied is the cross-section normalisation.

\begin{figure}[h!t]
 \centering
 \includegraphics[width = 0.95 \textwidth]{/home/annik/Documents/Vub/PhD/ThesisSubjects/AnomalousCouplings/April2015_LikelihoodCuts/TopMassInfluence/LLTopMass_Gen.pdf}
 \caption{\NegLL distribution for $10 000$ generator-level \ttbar semi-mu (+) events. The minimum of the distribution corresponds to the correct minimum used for simulating these events, namely 172.5.}
 \label{fig::MTGenLL}
\end{figure}

\begin{figure}[h!t]
 \centering
 \includegraphics[width = 0.45 \textwidth]{/home/annik/Documents/Vub/PhD/ThesisSubjects/AnomalousCouplings/April2015_LikelihoodCuts/TopMassInfluence/Gen_TopMassHadr.pdf}
 \includegraphics[width = 0.45 \textwidth]{/home/annik/Documents/Vub/PhD/ThesisSubjects/AnomalousCouplings/April2015_LikelihoodCuts/TopMassInfluence/Gen_TopMassLept.pdf}
 \caption{Distributions for the hadronically (left) and leptonically (right) decaying top quark for generator-level events.}\label{fig::MTGenDistr}
\end{figure}

\begin{table}[h!t]
 \centering 
 \caption{Fit parameters of 2nd degree polynomial ($a_{0} + a_{1}*x + a_{2}*x^{2}$) and corresponding minimum for Gen events.} \label{table::MTGenFit} 
 \begin{tabular}{c|c|c|c|c} 
  & $a_{0}$ & $a_{1}$ & $a_{2}$ & $m_{top}$ \\ 
  \hline 
  no normalisation & 4390547.54588 & -44426.0021316 & 127.642247948 & 174.199999984 \\ 
  XS normalisation & 4357397.72421 & -43885.9177784 & 126.428128287 & 173.400000012 
 \end{tabular} 
\end{table} 

Similar results have been calculated for the three considered categories of reco-level events, and they are summarised in the Figures and Tables given below. The order of both the figures and the tables is the same: first correctly matched jet combinations, then wrongly matched ones and finally the unmatched jet combinations.\\
The \NegLL distribution for the correctly matched jet combinations, which correspond the most with generator-level events, is the only one which follows the distribution obtained for the generator-level events. The two other distributions show a significant deviation of the position of the minimum indicating an important bias introduced by the applied event selection.

\begin{figure}[h!]
 \centering
 \includegraphics[width = 0.75 \textwidth]{/home/annik/Documents/Vub/PhD/ThesisSubjects/AnomalousCouplings/April2015_LikelihoodCuts/TopMassInfluence/LLTopMass_CorrectReco.pdf}
 \includegraphics[width = 0.75 \textwidth]{/home/annik/Documents/Vub/PhD/ThesisSubjects/AnomalousCouplings/April2015_LikelihoodCuts/TopMassInfluence/LLTopMass_WrongReco.pdf}
 \includegraphics[width = 0.75 \textwidth]{/home/annik/Documents/Vub/PhD/ThesisSubjects/AnomalousCouplings/April2015_LikelihoodCuts/TopMassInfluence/LLTopMass_UnmatchedReco.pdf}
 \caption{\NegLL distributions for $10000$ reco-level \ttbar semi-mu (+) events, respectively correctly matched, wrongly matched and unmatched jet combinations. The position of the minimum for the correctly matched jet combinations still corresponds with the value used for simulating these events, while the wrongly matched or unmatched jet combinations significantly distort the agreement with the expected minimum position.}
 \label{fig::MTRecoLL}
\end{figure}

\begin{table}[h!]
 \caption{Fit parameters of 2nd degree polynomial ($a_{0} + a_{1}*x + a_{2}*x^{2}$) and corresponding minimum for CorrectReco events.} \label{table::MTRecoCFit}
 \begin{tabular}{c|c|c|c|c} 
  \centering 
  & $a_{0}$ & $a_{1}$ & $a_{2}$ & $m_{top}$ \\ 
  \hline 
  no normalisation & 2231296.75883 & -19063.0863774 & 54.4843673637 & 175.099999986 \\ 
  XS normalisation & 2198831.67235 & -18531.2718603 & 53.2936154243 & 173.839999993 \\ 
  Acc normalisation & 2127835.59494 & -18006.1769325 & 52.1206777931 & 172.580000005  
 \end{tabular} 
\end{table} 


\begin{table}[h!]
 \caption{Fit parameters of 2nd degree polynomial ($a_{0} + a_{1}*x + a_{2}*x^{2}$) and corresponding minimum for WrongReco events.} \label{table::MTRecoWFit}
 \begin{tabular}{c|c|c|c|c} 
  \centering 
  & $a_{0}$ & $a_{1}$ & $a_{2}$ & $m_{top}$ \\ 
  \hline 
  no normalisation & 1430120.89175 & -9693.4402167 & 26.7893587541 & 180.999999991 \\ 
  XS normalisation & 1400565.78862 & -9209.34336566 & 25.7054394299 & 179.000000001 \\ 
  Acc normalisation & 1335938.33746 & -8731.32874723 & 24.6376683024 & 177.000000004
 \end{tabular} 
\end{table} 

\begin{table}[h!]
 \caption{Fit parameters of 2nd degree polynomial ($a_{0} + a_{1}*x + a_{2}*x^{2}$) and corresponding minimum for UnmatchedReco events.} \label{table::MTRecoUFit}
 \begin{tabular}{c|c|c|c|c} 
  \centering 
  & $a_{0}$ & $a_{1}$ & $a_{2}$ & $m_{top}$ \\ 
  \hline 
  no normalisation & 1246891.66214 & -8364.64777064 & 22.579725034 & 185.399999984 \\ 
  XS normalisation & 1222399.17029 & -7963.32843204 & 21.6813673937 & 183.79999998 \\ 
  Acc normalisation & 1168822.11268 & -7567.24056311 & 20.7969519086 & 181.800000013
 \end{tabular} 
\end{table} 

Comparing the different $m_{top}$ measurements for each of the categories and for the different normalisations applied clearly shows that applying both the cross-section normalisation and the acceptance normalisation significantly improves the measurement of the top-quark mass and brings it closer to the expected value.\\
\textit{Current results are given without uncertainties.}

\subsection{Improvement of top-quark mass measurement by applying cuts on \NegLL}
In order to reduce the influence of the event selection on the top-quark mass measurement using a Matrix Element Method, MadWeight, the effect of applying a cut on the obtained \NegLL has been studied. For this only the events for which the \NegLL has a negative second derivative, and hence behaves as a parabola with a minimum in the range of interest, have been used to perform the top-quark mass measurement.\\

The results of this study are summarized in Tables \ref{table::LLCutEff} and \ref{table::LLCutMT}, first the efficiency of this cut has been given by showing the percentage of remaining events after applying this cut on the different categories. The second table shows the obtained top-quark mass measurement after requiring one or both of the second derivatives to be positive.\\
For the calculation of this second derivative 5 different points have been studied with the middle point the expected SM value. Hence a distinction can be made whether the second derivative of the inner three points, the second derivative of the two outer ones with the middle point or both of the two should be positive. This distinction, and their respective influence, can be retrieved in Table \ref{table::LLCutMT} where the categories have been named \textit{Inner}, \textit{Outer} and \textit{Both}, respectively.

\newcolumntype{C}{>{\centering\arraybackslash}p{6.8em}} %Defined such that each of the boxes with numbers has the same width!!
\begin{table}[h!t]
 \caption{Percentage of remaining events for the four considered categories and three possible second derivative requirements. The numbers given here have been found by applying the above-mentioned cut on the \NegLL obtained by running MadWeight on $10 000$ \ttbar semi-mu (+) events. The number of successfully calculated events by MadWeight have been given before in Table \ref{table::MWEff}.}
 \label{table::LLCutEff}
 \begin{tabular}{c|C|C|C}
					& \multicolumn{3}{c}{Events remaining after requiring $2^{nd}$ derivative $>$ $0$}  	\\
					& \textit{Inner} ($\%$) 	& \textit{Outer} ($\%$) 	& \textit{Both}	($\%$)	\\
  \hline
  Generator level 			& 89.87				& 94.75				& 88.91			\\
  Reco-level, correctly matched 	& 84.32 			& 75.22 			& 71.27 		\\
  Reco-level, wrongly matched 		& 73.31 			& 67.58 			& 59.87 		\\
  Reco-level, unmatched 		& 70.60 			& 66.40 			& 57.38 		
 \end{tabular}
\end{table}

\begin{table}[h!t]
 \caption{Measured top-quark mass for the four considered categories and the three possible second derivative requirements compared to the mass measured originally and documented in Tables \ref{table::MTGenFit} - \ref{table::MTRecoUFit}.}
 \label{table::LLCutMT}
 \begin{tabular}{c|c||c|c|c}
					& \multirow{2}{*}{Original $m_{top}$} 	& \multicolumn{3}{c}{$m_{top}$ after requiring $2^{nd}$ derivative $>$ $0$}  		\\
					& 					& \textit{Inner} (GeV) 	& \textit{Outer} (GeV) 	& \textit{Both}	(GeV)	\\
  \hline
  Generator level 			& 173.40 				& 172.73 		& 172.77		& 172.72		\\
  Reco-level, correctly matched 	& 172.58 				& 172.69		& 172.92		& 172.86		\\
  Reco-level, wrongly matched 		& 177.00 				& 173.24		& 173.07		& 173.08		\\
  Reco-level, unmatched 		& 181.80 				& 173.55		& 173.26		& 173.27		
 \end{tabular}
\end{table}

Table \ref{table::LLCutMT} clearly indicates the large improvement which can be gained when applying a requirement on the sign of the second derivative of the \NegLL. 
The difference between the three cut options is almost negligible, but the influence of applying a restriction on the sign of the second derivative significantly approaches the measured top-quark mass to the one used for the simulation.\\
However since the efficiency of the three different cut options, shown in Table \ref{table::LLCutEff}, clearly differs a lot the optimal cut is on the inner second derivative. Hence using the mass-points $172$, $173$ and $174$ GeV.\\

However this conclusion only holds in the case of the top-quark mass measurement and will have to be revised for the measurement of the anomalous couplings. The main message which should be kept from this study is the fact that a clear and important improvement can be obtained when applying a cut on the \NegLL distribution. Using the case of the top-quark mass measurement, which was already quite decent without applying any cut, indicated that the considered method can be trusted and behaves as expected.\\

\textit{\textbf{\textcolor{blue}{Only point which should still be considered (and which is probably more imporant for RVR measurement) is how the outer points are distributed with respect to the inner five points which are used for the fit ... This to have an idea whether the cut requirement results in weird-shaped events (and in the case of RVR now a reversed Mexican hat shape for example ...)}}}

\section{Measurement of right-handed vector coupling, $\VR$, using Matrix Element Method}
Now that the method and the influence of the event selection has been, partially, understood by first studying the measurement of the top-quark mass, the Matrix Element Technique can be applied on the right-handed vector coupling.\\
Again the measurement has been done in a similar way and has been repeated for generator-level events and for correctly, wrongly and unmatched jet combinations of reco-level events.

%°°°°°°°°°°°°°°°°°°°°°°°°°°°°°°°°°°°°°°°°°°°°°°°°°°°°°°°°°°°°°°°°°°°°°°°°

%°°°°°°°°°°°°°°°°°°°°°°°°°°°°°°°°°°°°°°°°°°°°°°°°°°°°°°°°°°°°°°°°°°°°°°°°
%	CHAPTER: Likelihood event selection
\chapter{Likelihood event selection}
\section{Issue: Change of VR not found in likelihood shape}

\underline{First test configuration:} \\
MadGraph sample created with $Re(V_{R})$ = + 0.08 \\
$\Rightarrow$ Should result in a minimum around +0.08 when looking at summed likelihood!
\\
\\
\textbf{\underline{But obtained result is not as expected ...}}\\
Distributions given below are the total -ln(likelihood) when no cuts are applied, when only $\chi^{2}$-cut is applied and when the $\chi^{2}$-cut is combined with requiring the slope of the polynomial fit to be positive.\\
\\
For each of the cases this result is obtained by first fitting all the 21 points with a 2nd degree polynomial, removing the 7 points which are the farthest away from this fit distribution and then fitting again the remaining 14 points with another 2nd degree polynomial.\\
The overal -ln(likelihood) distribution is obtained by summing all the separate -ln(L) fit distributions.\\
\\
The cuts which are applied always use the $\chi^{2}$ or the slope of this second fit.\\
\\
\underline{Remark: Low statistics ..}\\
For the moment the statistics is still rather low, because my MadWeight calculation for the full 10000 events crashed yesterday.\\
They are currently still running and should be finished later today.
\newpage

\begin{figure}[h!t]
 \centering
 \includegraphics[width = 0.71 \textwidth]{/home/annik/Documents/Vub/PhD/ThesisSubjects/AnomalousCouplings/May2015_LikelihoodEvtSel/SecondPolAcc_Summed_NoCuts_MGSampleWithRVR008.pdf}
 \includegraphics[width = 0.71 \textwidth]{/home/annik/Documents/Vub/PhD/ThesisSubjects/AnomalousCouplings/May2015_LikelihoodEvtSel/SecondPolAcc_Summed_ChiSqCut_MGSampleWithRVR008.pdf}
 \includegraphics[width = 0.71 \textwidth]{/home/annik/Documents/Vub/PhD/ThesisSubjects/AnomalousCouplings/May2015_LikelihoodEvtSel/SecondPolAcc_Summed_ChiSqCutPosSlope_MGSampleWithRVR008.pdf}
\end{figure}

So in order to make sure whether this unexpected position of the minimum was caused by the created ROOT macros I decided to repeat this test also for the top mass. If the position of the minimum would correspond with the top mass used for creating the MadGraph sample, the VR-issue above is not caused by the used ROOT macros ...\\
\\
As expected for the top mass all results look as expected. However since the top mass can be measured much more precisely the influence of the cuts on the -ln(likelihood) are less significant. The -ln(L) distribution without any cut applied already results in the correct position of the minimum ..\\
\\
Here only 5 points have been used (mainly for speeding up MadWeight CPU time) so only 1 point is removed before the second fit is applied.\\
\\
The first three distributions correspond to the MadGraph sample created with $m_{top}$ = 172 GeV while the second one was created with $m_{top}$ = 174 GeV.\\
\\
\textbf{\underline{Search for VR-issue}}\\
Now that the control check using the top mass MadGraph samples indicated that the created ROOT macros and the consecutive fitting procedure does work as should be, I will continue to investigate why this is not the case for the anomalous couplings.\\
Maybe the use of more statistics can help but I found it rather strange that with reduced statistics the position of the minimum is nicely positioned around $0$. Maybe the change of VR-component was not translated into the produced MadGraph sample, this I will check by creating some MadAnalysis plots which easily compares the kinematic information of different MadGraph samples.

\begin{figure}[h!t]
 \centering
 \includegraphics[width = 0.71 \textwidth]{/home/annik/Documents/Vub/PhD/ThesisSubjects/AnomalousCouplings/May2015_LikelihoodEvtSel/SecondPolAcc_Summed_NoCuts_MGSampleWith172GeV.pdf}
 \includegraphics[width = 0.71 \textwidth]{/home/annik/Documents/Vub/PhD/ThesisSubjects/AnomalousCouplings/May2015_LikelihoodEvtSel/SecondPolAcc_Summed_ChiSqCut_MGSampleWith172GeV.pdf}
 \includegraphics[width = 0.71 \textwidth]{/home/annik/Documents/Vub/PhD/ThesisSubjects/AnomalousCouplings/May2015_LikelihoodEvtSel/SecondPolAcc_Summed_ChiSqCutPosSlope_MGSampleWith172GeV.pdf}
\end{figure}

\begin{figure}[h!t]
 \centering
 \includegraphics[width = 0.71 \textwidth]{/home/annik/Documents/Vub/PhD/ThesisSubjects/AnomalousCouplings/May2015_LikelihoodEvtSel/SecondPolAcc_Summed_NoCuts_MGSampleWith174GeV.pdf}
 \includegraphics[width = 0.71 \textwidth]{/home/annik/Documents/Vub/PhD/ThesisSubjects/AnomalousCouplings/May2015_LikelihoodEvtSel/SecondPolAcc_Summed_ChiSqCut_MGSampleWith174GeV.pdf}
 \includegraphics[width = 0.71 \textwidth]{/home/annik/Documents/Vub/PhD/ThesisSubjects/AnomalousCouplings/May2015_LikelihoodEvtSel/SecondPolAcc_Summed_ChiSqCutPosSlope_MGSampleWith174GeV.pdf}
\end{figure}

%°°°°°°°°°°°°°°°°°°°°°°°°°°°°°°°°°°°°°°°°°°°°°°°°°°°°°°°°°°°°°°°°°°°°°°°°

%°°°°°°°°°°°°°°°°°°°°°°°°°°°°°°°°°°°°°°°°°°°°°°°°°°°°°°°°°°°°°°°°°°°°°°°°
%	CHAPTER: Results after applying double fit
\chapter{Obtained result after applying double fit}
In order to overcome the influence of the few configuration points which destroy the shape of the likelihood distribution, it had been decided to apply a double-fit procedure in order to extract the measured anomalous coupling coefficient.
This double fit procedure will be similar for both the $\VR$ and $\gR$ coefficient, however the symmetry in $\VR^{2}$ implies that for this coefficient a $4^{th}$ order polynomial should be applied while for $\gR$ a $2^{nd}$ degree polynomial is sufficient.
\\
\\
The double-fit procedure is the following, defined on an event-by-event basis, and follows the following structure:
\begin{enumerate}
 \item The first fit is applied onto all the considered configuration points which have been calculated by MadWeight.
 \item For each configuration point which has been used in the first fit, the deviation between the fit distribution and the actual MadWeight measurement is calculated and compared against all other configuration points. Depending on the original number of configuration points considered, a specific number of points with the highest fit deviation is excluded as input for the following fit.
 \item Once these points have been identified a second, completely identical for the rest, fit is applied onto the remaining configuration points. This second fit is then used as the final fit from which the $\VR$ and $\gR$ coefficients will be extracted.
\end{enumerate}
The main benefit of such a double-fit approach is that the dependence with respect to a couple of deviating configuration points can easily be avoided by simply excluding them from the final fit.
This will definitely help to make sure that the shape of the final fit corresponds the best with the actual likelihood shape and ensures that the distribution used for the coefficient measurement is not influenced by statistical fluctuations or erroneous MadWeight calculations.
It is therefore important to exclude this type of configuration points since it is not advisable that the extracted measurements are sensitive to possible phase-space issues occuring during the MadWeight calculation step.
\\
\\
Another important advantage of applying a double-fit on an event-by-event basis is that the likelihood distribution for each event is replaced by a smooth $\scd$ or $\fth$ order polynomial. The fact that a double-fit method is used even ensures that the $\chisq$ value and other characteristics make sense because the $x$ worst points have been removed from the fitted range. Hence it is possible, in the case of the $\scd$ order polynomial, to put a constraint on the slope of the likelihood shape by using the corresponding fit. Also cutting on the $\chisq$ value can help to select events for which a lot of events deviate from the proposed polynomial shape and not just contain two or three completely ``crazy'' events.
\\
The obtained improvement of the $\chisq$ variable between the first and second fit is shown in Figure~\ref{fig::ChiSqImprovement}. This corresponds to a simulated generator-level sample created by MadGraph using a different anomalous couplings coefficient than originally included in the Standard Model. Hence as expected, even the $\chisq$ distribution of the first fit is rather smooth and exhibits a nice peak around very low values. Still a clear improvement is visible when applying the second fit on a limited range.
\begin{figure}[h!t]
 \centering
 \includegraphics[width = 0.8 \textwidth]{/home/annik/Documents/Vub/PhD/ThesisSubjects/AnomalousCouplings/June2015_DoubleFitResults/ChiSqImprovementBetweenFits.pdf}
 \caption{Obtained $\chisq$ distribution when applying the first fit (blue) and when applying the second fit on the reduced number of configuration points (red).}
 \label{fig::ChiSqImprovement}
\end{figure}

In order to select the optimal $\chisq$ constraint that should be applied a comparison of the $\chisq$ distribution for generator-level, signal reco and background reco events. Studying the overall $\loglik$ distribution for different $\chisq$ cuts indicated that cutting too tight might result in an altered shape in a rather negative way. So it seems that restricting the event selection too much actually has the opposite effect on the overall $\loglik$ distribution, probably because the power of MadWeight lies in the fact that the integrated phase space gets flattened out by considering multiple events. Comparing the individual MadWeight weight distributions on an event-by-event level clearly indicates that quite a lot of variation can be found in the shape, even in the case of the well-measured $\mT$. But again here the overall $\loglik$ shows a nice minimum almost at the expected position indicating again that MadWeight output should not be taken too serious for one single event.\\
A couple of these type of event comparisons between the original MadWeight calculations and the second polynomial fit are given in Figure \ref{fig::SplitCanvas}.
\\
\begin{figure}[h!t]
 \centering
 \includegraphics[width = 0.9 \textwidth]{/home/annik/Documents/Vub/PhD/ThesisSubjects/AnomalousCouplings/June2015_DoubleFitResults/SplitCanvasLLXS_Nr47.pdf}
 \caption{Overview of the MadWeight output for $25$ different, but consecutive, events fitted with the second $fth$ order polynomial. The used sample is a MadGraph sample created with $\VR$ $=$ $0.3$.}
 \label{fig::SplitCanvas}
\end{figure}

For the moment three different $\chisq$ constraints have been studied, which should possibly be good enough to be applied for both gen-level events as reco-level events.
In the following table, Table~\ref{table::ChiSqCuts}, the number of selected events for the different $\chisq$-cuts considered are listed for as much as possible generator- and reco-level samples.
\\
\begin{table}[h!t]
 \centering
 \caption{Number of selected events after applying the different $\chisq$-cuts considered. } 
 \label{table::ChiSqCuts}
 \begin{tabular}{c|c|c|c}
  Sample 									& $\chisq$ $<$ 0.001 	& $\chisq$ $<$ 0.0005 	& $\chisq$ $<$ 0.0002 	\\
  \hline
  MG with $\VR$ $=$ 0.3 in $\left[-0.5, -0.3, -0.2, ..., 0.3, 0.5 \right]$ 	& 99.91 $\%$		& 99.78 $\%$		& 99.02 $\%$		\\
  MG with $\VR$ $=$ -0.1 in $\left[-0.3, -0.275, ..., 0.3 \right]$ 		& 94.74 $\%$		& 90.30 $\%$		& 80.89 $\%$		\\
  MG with $\VR$ $=$ -0.08 in $\left[-0.1, -0.09, ..., 0.1 \right]$ 		& 96.99 $\%$		& 93.85 $\%$		& 87.26 $\%$		\\
  MG SM in $\left[-0.5, -0.3, -0.2, ..., 0.3, 0.5 \right]$ 			& 99.82 $\%$		& 99.62 $\%$		& 98.89 $\%$		\\
  Gen in $\left[-0.5, -0.3, -0.2, ..., 0.3, 0.5 \right]$ 			& 97.35 $\%$		& 95.82 $\%$		& 92.63 $\%$		\\
  Correct reco in $\left[-0.5, -0.3, -0.2, ..., 0.3, 0.5 \right]$ 	 	& 93.57 $\%$		& 90.58 $\%$		& 85.75 $\%$		\\
  Wrong reco in $\left[-0.5, -0.3, -0.2, ..., 0.3, 0.5 \right]$			& 77.91 $\%$		& 72.55 $\%$		& 64.95 $\%$		\\
  \hline
 \end{tabular}
\end{table}

The improvement obtained when applying the $\chisq$ cuts on the final $\loglik$ distributions can be understood from the shape comparisons given in Figure~\ref{fig::ChiSqOnLL}, which show the original $\loglik$ distribution and the distribution of the events remaining after applying the specific $\chisq$ cuts.\\
Strangly enough can be concluded from these distributions that applying the $\chisq$ cuts especially improves the bias in the case of the reco-level events. The middle-left histogram shows the influence on the $\loglik$ in the case of generator-level events and for these type of events the $\chisq$ cuts almost do not alter the position of the minimum. Maybe this observation can be explained by the much lower $\chisq$ values of the fit in the generator-level case such that requiring $\chisq$ $<$ $0.0005$ does not have the same influence on the shape. In order to be sure about this, maybe an additional $\chisq$ cut of, for example $0.00005$ can be applied in order to ensure that some influence can be seen on the overall $\loglik$ shape.\\
\textbf{Update: } The middle-right histogram now contains almost the same distributions, but the $\chisq$-cut of $0.0005$ has been replaced by $0.00005$ in order to double-check whether applying a $\chisq$ cut also influences or improves the generator-level distributions. The obtained result is rather positive, since the blue distribution in this case nicely corresponds to a minimum around $\VR$ $=$ $0.0$ as suggested by the Standard Model! The corresponding percentages for this $\chisq$ cut is given in the Table~\ref{table::ChiSqCutTight}.
\begin{table}[h!t]
 \centering
 \caption{Number of selected events after applying the different $\chisq$-cuts considered. Here the tighter $\chisq$-cuts have been applied which are only useful for generator-level events.} 
 \label{table::ChiSqCutTight}
 \begin{tabular}{c|c|c|c}
  Sample 							& $\chisq$ $<$ 0.001 	& $\chisq$ $<$ 0.0002 	& $\chisq$ $<$ 0.00005 	\\
  \hline
  Gen in $\left[-0.5, -0.3, -0.2, ..., 0.3, 0.5 \right]$ 	& 97.35 $\%$		& 92.63 $\%$		& 85.66 $\%$		\\
  MG SM in $\left[-0.5, -0.3, -0.2, ..., 0.3, 0.5 \right]$ 	& 99.82 $\%$		& 98.89 $\%$		& 95.77 $\%$		
 \end{tabular}
\end{table}

\begin{figure}[h!t]
 \centering
 \includegraphics[width = 0.49 \textwidth]{/home/annik/Documents/Vub/PhD/ThesisSubjects/AnomalousCouplings/June2015_DoubleFitResults/LLComparison_ChiSqCutsMGSample_RVR_10000Evts.pdf}
 \includegraphics[width = 0.49 \textwidth]{/home/annik/Documents/Vub/PhD/ThesisSubjects/AnomalousCouplings/June2015_DoubleFitResults/LLComparison_ChiSqCutsMGSampleNew_RVR_10000Evts.pdf}\\
 \includegraphics[width = 0.49 \textwidth]{/home/annik/Documents/Vub/PhD/ThesisSubjects/AnomalousCouplings/June2015_DoubleFitResults/LLComparison_ChiSqCutsGen_RVR_10000Evts.pdf}
 \includegraphics[width = 0.49 \textwidth]{/home/annik/Documents/Vub/PhD/ThesisSubjects/AnomalousCouplings/June2015_DoubleFitResults/LLComparison_ChiSqCutsGen_RVR_10000Evts_TighterCut.pdf}\\
 \includegraphics[width = 0.49 \textwidth]{/home/annik/Documents/Vub/PhD/ThesisSubjects/AnomalousCouplings/June2015_DoubleFitResults/LLComparison_ChiSqCutsCorrectReco_RVR_9995Evts.pdf}
 \includegraphics[width = 0.49 \textwidth]{/home/annik/Documents/Vub/PhD/ThesisSubjects/AnomalousCouplings/June2015_DoubleFitResults/LLComparison_ChiSqCutsWrongReco_RVR_9376Evts.pdf}
 \caption{Influence of the different $\chisq$ cuts on the overall $\loglik$ distribution.}
 \label{fig::ChiSqOnLL}
\end{figure}

A second strange, and even slightly worrysome, observation concerns the two upper distributions. These contain the likelihood distributions obtained for two MadGraph samples, both created separately using the Standard Model configuration. Hence they should definitely correspond to a minimum around $\VR$ $=$ $0.0$ since they are not even influenced by event selection or reconstruction effects ... In order to be completely sure that the first obtained result was created with a correct MG sample, a second one was created but as can be seen by comparing the right and left figure, they are clearly identical. Both of them correspond to a minimum at around $\VR$ $=$ $-0.2$. How this is possible, especially when the generator-level distribution does correspond with the predicted Standard Model value, is not really understood. Even the tighter $\chisq$ cut which is applied in the right distribution does not solve the issue of the positioning of the minimum.\\
As a sort of consistency check which will be performed is comparing these two distributions with the result obtained when looking at a MG sample created with $\VR$ $=$ $0.05$. 

\section{Double-check of method using $gR$ coefficient}
In order to compare the correctness of the method, the same results can also be studied for the $\gR$ coefficient. In the best case looking at this anomalous couplings coefficient can give a possible explanation of the incorrect position of the minimum when looking at the Standard Model MG sample. However, then should still be understood why this issue arises for the $\VR$ coefficient.
Otherwise the obtained results and distributions all have similar properties such that the double-fit method is completely double-checked and hence reliable to be used for analysing actual data events.
\\

One of the most important differences between measuring the $\VR$ coefficient and the $\gR$ coefficient is the fit function which has to be applied. As said before the $\gR$ coefficient is not supposed to be symmetric around $0$ implying that a quadratic function is sufficient for correctly measuring the anomalous coupling coefficient. Another important difference between the two coefficients is the sensitivity at low values, which is much higher in the $\gR$ case than in the $\VR$ case. This should normally imply that the $\gR$ coefficient can be studied in a narrower range than the $\VR$ one without introducing statistical fluctuations.\\
The reason for this different sensitivity follows directly from the theoretical equations and is depicted in Figure~\ref{fig::CoefSensitivity}, which contains the shape influence on the $\csTh$ distribution when considering the same range.\\
The range which will be used for the two anomalous coupling coefficient can easily be motivated by looking at these distributions, see Table~\ref{table::CoefRange}. However from the variation of the $\csTh$ distribution can be expected that even in the more narrow range used for the $\gR$ coefficient, the measurements will still be more sensitive since the variations here are still larger than in the range for the $\VR$ coefficient. But it is not wise to consider an even wider range for the $\VR$ measurement since the uncertainty on the likelihood distribution is so small, in general less than $0.1$ fluctuation on $\VR$ is retrieved, that the $5\sigma$ interval will not be visible anymore. Hence it can be concluded that the range given in the table can be somewhat seen as the minimum range for which fluctuations of the $\VR$ coefficient should be detectable. Considering a smaller range will result in a shape dominated by statistical fluctuations since the kinematics cannot be differentiated between the different coefficients calculated.

\begin{table}[h!t]
 \centering
 \caption{Range which will be used for the two anomalous coupling coefficient considered here.}
 \label{table::CoefRange}
 \begin{tabular}{c|c}
  coefficient 	& Range 								\\
  \hline
  $\VR$ 	& $\left[-0.5, -0.3, -0.2, -0.1, 0.0, 0.1, 0.2, 0.3, 0.5 \right]$ 	\\
  $\gR$ 	& $\left[-0.2, -0.15, -0.1, -0.05, 0.0, 0.05, 0.1, 0.15, 0.2 \right]$
 \end{tabular}
\end{table}

\begin{figure}[h!t]
 \centering
 \includegraphics[width = 0.49 \textwidth]{/home/annik/Documents/Vub/PhD/ThesisSubjects/AnomalousCouplings/May2015_LikelihoodEvtSel/CosThetaVariation/CosThetaChange_RVRScan_FewerPoints.pdf}
 \includegraphics[width = 0.4 \textwidth]{/home/annik/Documents/Vub/PhD/ThesisSubjects/AnomalousCouplings/May2015_LikelihoodEvtSel/CosThetaVariation/CosThetaVariation_RVRVariation_RgRNarrowRange.pdf}\\
 \includegraphics[width = 0.49 \textwidth]{/home/annik/Documents/Vub/PhD/ThesisSubjects/AnomalousCouplings/May2015_LikelihoodEvtSel/CosThetaVariation/RgRStudy/CosThetaVariation_RgRVariation.pdf}
 \includegraphics[width = 0.49 \textwidth]{/home/annik/Documents/Vub/PhD/ThesisSubjects/AnomalousCouplings/May2015_LikelihoodEvtSel/CosThetaVariation/RgRStudy/CosThetaVariation_RgRVariationNarrow.pdf}
 \caption{Stronger dependence of the $\csTh$ distribution on the $\gR$ coefficient than on the $\VR$ one. Therefore the $\gR$ coefficient will be measured in a more narrow range than the one used for the $\VR$ measurement.}
 \label{fig::CoefSensitivity}
\end{figure}

For the moment it seems that there is an issue for the $\gR$ measurement when the acceptance normalisation is applied. As can be seen from Figure~\ref{fig::ChiSqgR}, the $\chisq$ distribution is completely wrong when going from the XS-normalisation to the acceptance-normalisation distribution. Looking at the individual distributions for both normalisation cases also shows that something strange is happening. Strangely enough the overall $\loglik$ distribution obtained from the double-fit procedure still seems pretty decent when no $\chisq$-cut is applied, which seems to suggest that the overall distribution should not be used without looking at the $\chisq$ distribution to ensure that only good fits are considered. 
\\
\begin{figure}[h!t]
 \centering
 \includegraphics[width = 0.6 \textwidth]{/home/annik/Documents/Vub/PhD/ThesisSubjects/AnomalousCouplings/June2015_DoubleFitResults/gRResults/ChiSqDistribution_XSandACCNorm.pdf}
 \includegraphics[width = 0.7 \textwidth]{/home/annik/Documents/Vub/PhD/ThesisSubjects/AnomalousCouplings/June2015_DoubleFitResults/gRResults/SplitCanvas_gRAccNorm.pdf}
 \includegraphics[width = 0.6 \textwidth]{/home/annik/Documents/Vub/PhD/ThesisSubjects/AnomalousCouplings/June2015_DoubleFitResults/gRResults/SecondPolAcc_Summed.pdf}
 \caption{Strange behavior for $\chisq$ distribution when the acceptance normalisation is applied.}
 \label{fig::ChiSqgR}
\end{figure}


%In order to be able to compare the obtained results of the $\gR$ case with the $\VR$ coefficient, at first the percentage of events surviving the $\chisq$ cuts are given in Table~\ref{table::ChiSqCutgR}. For each of the samples considered, the range used is the one given in Table~\ref{table::CoefRange}.

%\begin{table}[h!t]
% \centering
% \caption{Number of selected events after applying the different $\chisq$-cuts considered. } 
% \label{table::ChiSqCutgR}
% \begin{tabular}{c|c|c|c}
%  Sample 	& $\chisq$ $<$ 0.001 	& $\chisq$ $<$ 0.0005 	& $\chisq$ $<$ 0.0002 	& $\chisq$ $<$ 0.00005 	\\
%  \hline
%  MG SM  	&  $\%$		&  $\%$		&  $\%$		& 			\\
%  Gen  		&  $\%$		&  $\%$		&  $\%$		& 			\\
%  Correct reco 	&  $\%$		&  $\%$		&  $\%$		& 			\\
%  Wrong reco 	&  $\%$		&  $\%$		&  $\%$		& 			
%  \hline
% \end{tabular}
%\end{table}

\newpage
\section{Bias-dependence on used $\VR$ value MG sample}

As a next check the measured $\VR$ should be compared with the $\VR$ coefficient used for generating the MadGraph sample. Therefore multiple MadWeight calculations should be done for each of the configurations considered in the range. It would be advisable that the bias found between the coefficient used for generating the sample and the coefficient retrieved from the measurement is independent of the value used for generating the sample. This kind of behavior would be more difficult to understand since it implies that the bias is not perfectly linear and influenced by some other effects.

\begin{table}[h!t]
 \centering
 \begin{tabular}{c|c|c}
  $\VR$ coefficient 	& Present? 	& Measurement 	\\
  \hline
  -0.5 			& Yes 		& 		\\
  -0.3 			& Yes 		& 		\\
  -0.2 			& Yes 		& 		\\
  -0.1 			& Yes 		& 		\\
  0.0 			& Yes 		& 		\\
  0.1 			& Yes 		& 		\\
  0.2 			& No 		& 		\\
  0.3 			& Yes 		& 		\\
  0.5 			& No 		& 		\\
 \end{tabular}
\end{table}

\section{Influence of tighter $\chisq$ cuts}







%°°°°°°°°°°°°°°°°°°°°°°°°°°°°°°°°°°°°°°°°°°°°°°°°°°°°°°°°°°°°°°°°°°°°°°°°

%°°°°°°°°°°°°°°°°°°°°°°°°°°°°°°°°°°°°°°°°°°°°°°°°°°°°°°°°°°°°°°°°°°°°°°°°
%	CHAPTER: Comparison of SM behaviour
\chapter{Comparison of SM behaviour}
From analyzing the result obtained after applying the double-fit procedure, it was clear that a significant discrepancy occurs between the generator-level result using the CMS Monte Carlo samples and the self-simulated MG samples.
This was clearly visible in Figure~\ref{fig::ChiSqOnLL} which shows the expected location of the minimum for the CMS generator-level samples but a completely wrong distribution for the MG sample created with $\VR$ $=$ $0.0$. Hence from this can be concluded that the created FeynRules model does not correspond with the actual ``Standard Model'' of MadGraph and more investigation in the matter is necessary.
\\

In order to be sure that the issue is caused by the created FeynRules model an additional test has been performed. For this a personal MG sample has been created using the ``Standard Model'' model existing in MadGraph, and also used within CMS, which was processed in the same way by MadWeight and the analysis scripts.
This sample resulted in almost exactly the same distributions as was obtained using the sample created with the personally created AnomCoup model, hence a minimum differing from the expected position. The distributions for this sample are given in Figure~\ref{fig::MGSampleLL}.
\begin{figure}[h!t]
 \centering
 \includegraphics[width = 0.49 \textwidth]{/home/annik/Documents/Vub/PhD/ThesisSubjects/AnomalousCouplings/June2015_SMComparisonMGModels/SecondPolAdd_Summed_SMMadGraphModel.pdf}
 \includegraphics[width = 0.49 \textwidth]{/home/annik/Documents/Vub/PhD/ThesisSubjects/AnomalousCouplings/June2015_SMComparisonMGModels/LLComparison_ChiSqCutsMGSample_RVR_10000Evts.pdf}
 \caption{$\loglik$ distribution obtained when the MadGraph sample is created using the SM FeynRules model}
 \label{fig::MGSampleLL}
\end{figure}

However the right-handed figure shows that the application of the $\chisq$ cut in a very tight way, removing about $15\%$ of the events, repositions the overall likelihood distribution to have a minimum at the correct position. This seems to suggest that in the CMS sample the few event selection cuts which have been applied probably influence the $\chisq$ distribution of the fit.
The percentages of event kept after applying the tighter cuts can be found in Table~\ref{table::tightChiSqMG}.
\begin{table}[h!t]
 \centering
 \caption{Number of selected events after applying the different $\chisq$-cuts considered.} 
 \label{table::tightChiSqMG}
 \begin{tabular}{c|c|c|c}
  Sample 								& $\chisq$ $<$ 0.0002 	& $\chisq$ $<$ 0.00001 	& $\chisq$ $<$ 0.000005 	\\
  \hline
  MG SM-model in $\left[-0.5, -0.3, -0.2, ..., 0.3, 0.5 \right]$ 	& 98.69 $\%$		& 84.46 $\%$		& 76.28 $\%$		
 \end{tabular}
\end{table}

So as can be understood from the obtained results, there seems to be no real difference between the two different FeynRules models. This is again summarized in Figure~\ref{fig::FRModelComp} where some of the more important kinematic properties are shown together for the two models.
\begin{figure}[h!t]
 \centering
 \includegraphics[width = 0.49 \textwidth]{/home/annik/Documents/Vub/PhD/ThesisSubjects/AnomalousCouplings/June2015_SMComparisonMGModels/SMComparison_CosThetaDist.pdf}
 \includegraphics[width = 0.49 \textwidth]{/home/annik/Documents/Vub/PhD/ThesisSubjects/AnomalousCouplings/June2015_SMComparisonMGModels/SMComparison_LeptMassDist.pdf}
 \includegraphics[width = 0.49 \textwidth]{/home/annik/Documents/Vub/PhD/ThesisSubjects/AnomalousCouplings/June2015_SMComparisonMGModels/SMComparison_TopQuarkMassDist.pdf}
 \includegraphics[width = 0.49 \textwidth]{/home/annik/Documents/Vub/PhD/ThesisSubjects/AnomalousCouplings/June2015_SMComparisonMGModels/SMComparison_bQuarkPtDist.pdf} 
 \caption{Selected distributions from the comparison between the SM FeynRules model and the AnomCoup one.}
 \label{fig::FRModelComp}
\end{figure}

Hence it seems that the difference originates from the light pre-selection which is applied for the CMS Monte Carlo $t\bar{t}$ sample and not for the personally created MadGraph samples. Hence it should be investigated whether the influence of the event selection actually has a positive effect on the shape of the $\loglik$ distribution, and not a negative one as was thought in the beginning.
\\
%But first, a final closure is given in Figure~\ref{fig::LLBiasTightCut} containing the difference between the expected minimum and the calculated minimum for MadGraph samples created with different $\VR$-values.
%\\

However the application of such a tight $\chisq$-cut clearly improved the obtained $\loglik$ shape for the Standard Model case, it seems to worsen the result for different $\VR$ values.
The influence of the $\chisq$ cut on the $\loglik$ shape for two $\VR$ values which were actually good for less tight cuts is given in Figure~\ref{fig::BadLLTight}.
\begin{figure}[h!t]
 \centering
 \includegraphics[width = 0.49 \textwidth]{/home/annik/Documents/Vub/PhD/ThesisSubjects/AnomalousCouplings/June2015_BiasTest/InfluenceTighterCuts/LLComparison_ChiSqCutsMGSample_RVR_10000Evts_CreatedWithNeg05.pdf}
 \includegraphics[width = 0.49 \textwidth]{/home/annik/Documents/Vub/PhD/ThesisSubjects/AnomalousCouplings/June2015_BiasTest/InfluenceTighterCuts/LLComparison_ChiSqCutsMGSample_RVR_10000Evts_CreatedWithPos03.pdf}
 \caption{$\loglik$ distribution for two MadGraph samples which actually had a good agreement with the expected shape when looser $\chisq$ cuts are applied, but become completely distorted in the case of the very tight cuts. The left one corresponds to a MG sample created with $\VR$ $=$ $-0.5$ and the right one with $\VR$ $=$ $0.3$.}
 \label{fig::BadLLTight}
\end{figure}

So from this small study can be decided that the problem which occurs for the MadGraph generator-level events is not caused by a wrong ``Standard Model'' implementation but can (hopefully) be explained by the minor event selection which is applied for the CMS generator-level sample.
Hence the next study will discuss the influence of the $\pT$-cuts  on the obtained $\loglik$ distribution.
%°°°°°°°°°°°°°°°°°°°°°°°°°°°°°°°°°°°°°°°°°°°°°°°°°°°°°°°°°°°°°°°°°°°°°°°°

%°°°°°°°°°°°°°°°°°°°°°°°°°°°°°°°°°°°°°°°°°°°°°°°°°°°°°°°°°°°°°°°°°°°°°°°°
%	CHAPTER: Influence of generator-level event selection
\chapter{Influence of generator-level event selection}
A quick way to see the influence of the event selection, and avoid the submission of all samples to MadWeight, is to apply the event selection requirements after the MadWeight calculation has been performed. This implies that the event selection constraints can be applied in a similar way as the $\chisq$-cut by simply using the kinematic information stored in the $.lhco$ file for each individual event. The main difference with the $\chisq$-cut is that the event selection should be applied prior to the double-fit procedure since this takes place in an event-by-event basis. So during this step the event selection constraints should give as additional information a ``keep'' or ``hold'' flag such that the fitting procedure is only done for the good events. The constraints which will be applied should contain the general $\pT$ ones but also the $\Delta R$ ones should be added for example.
\\
However at first a simplified check will be performed where only a requirement on the $\pT$ of the different particles present in the event will be applied.
\\

The influence of the event selection (limited to $\pT$-cut and $\chisq$ requirement) is summarised in Figures~\ref{fig::PtCutInflPos05} - \ref{fig::PtCutInflNeg05}.
These figures each contain three different $\loglik$ distributions where the left one is the result when no $\pT$-cut is applied, the middle when $\pT$ $>$ 15 $\GeV$ is asked and the right one requires $\pT$ $>$ 30 $\GeV$.

\begin{figure}[h!t]
 \centering
 \includegraphics[width = 0.32 \textwidth]{/home/annik/Documents/Vub/PhD/ThesisSubjects/AnomalousCouplings/June2015_PtCutInfluence/Pos05/LLComparison_ChiSqCutsRVR_MGSamplePos05_SingleGausTF_10000Evts_WideRange_10000Evts.pdf}
 \includegraphics[width = 0.32 \textwidth]{/home/annik/Documents/Vub/PhD/ThesisSubjects/AnomalousCouplings/June2015_PtCutInfluence/Pos05/LLComparison_ChiSqCutsRVR_MGSamplePos05_SingleGausTF_10000Evts_WideRange_NoLowPtEvts_Cut15_7572Evts.pdf}
 \includegraphics[width = 0.32 \textwidth]{/home/annik/Documents/Vub/PhD/ThesisSubjects/AnomalousCouplings/June2015_PtCutInfluence/Pos05/LLComparison_ChiSqCutsRVR_MGSamplePos05_SingleGausTF_10000Evts_WideRange_NoLowPtEvts_Cut30_2689Evts.pdf}
 \caption{Influence of the $\pT$-cut on the $\loglik$ distribution for increasing $\pT$-cut value (0 $\GeV$, 15 $\GeV$ and 30 $\GeV$) for a MG sample created with $\VR$ = 0.5.}
 \label{fig::PtCutInflPos05}
\end{figure}

\begin{figure}[h!t]
 \centering
 \includegraphics[width = 0.32 \textwidth]{/home/annik/Documents/Vub/PhD/ThesisSubjects/AnomalousCouplings/June2015_PtCutInfluence/Pos03/LLComparison_ChiSqCutsRVR_MGSamplePos03_SingleGausTF_10000Evts_WideRange_10000Evts.pdf}
 \includegraphics[width = 0.32 \textwidth]{/home/annik/Documents/Vub/PhD/ThesisSubjects/AnomalousCouplings/June2015_PtCutInfluence/Pos03/LLComparison_ChiSqCutsRVR_MGSamplePos03_SingleGausTF_10000Evts_WideRange_NoLowPtEvts_Cut15_7539Evts.pdf}
 \includegraphics[width = 0.32 \textwidth]{/home/annik/Documents/Vub/PhD/ThesisSubjects/AnomalousCouplings/June2015_PtCutInfluence/Pos03/LLComparison_ChiSqCutsRVR_MGSamplePos03_SingleGausTF_10000Evts_WideRange_NoLowPtEvts_Cut30_2563Evts.pdf}
 \caption{Influence of the $\pT$-cut on the $\loglik$ distribution for increasing $\pT$-cut value ($0$ $\GeV$, $15$ $\GeV$ and $30$ $\GeV$) for a MG sample created with $\VR$ $=$ $0.3$.}
 \label{fig::PtCutInflPos03}
\end{figure}

\begin{figure}[h!t]
 \centering
 \includegraphics[width = 0.32 \textwidth]{/home/annik/Documents/Vub/PhD/ThesisSubjects/AnomalousCouplings/June2015_PtCutInfluence/Pos02/LLComparison_ChiSqCutsRVR_MGSamplePos02_SingleGausTF_10000Evts_WideRange_10000Evts.pdf}
 \includegraphics[width = 0.32 \textwidth]{/home/annik/Documents/Vub/PhD/ThesisSubjects/AnomalousCouplings/June2015_PtCutInfluence/Pos02/LLComparison_ChiSqCutsRVR_MGSamplePos02_SingleGausTF_10000Evts_WideRange_NoLowPtEvts_Cut15_7405Evts.pdf}
 \includegraphics[width = 0.32 \textwidth]{/home/annik/Documents/Vub/PhD/ThesisSubjects/AnomalousCouplings/June2015_PtCutInfluence/Pos02/LLComparison_ChiSqCutsRVR_MGSamplePos02_SingleGausTF_10000Evts_WideRange_NoLowPtEvts_Cut30_2526Evts.pdf}
 \caption{Influence of the $\pT$-cut on the $\loglik$ distribution for increasing $\pT$-cut value (0 $\GeV$, 15 $\GeV$ and 30 $\GeV$) for a MG sample created with $\VR$ = 0.2.}
 \label{fig::PtCutInflPos02}
\end{figure}

\begin{figure}[h!t]
 \centering
 \includegraphics[width = 0.32 \textwidth]{/home/annik/Documents/Vub/PhD/ThesisSubjects/AnomalousCouplings/June2015_PtCutInfluence/Pos01/LLComparison_ChiSqCutsRVR_MGSamplePos01_SingleGausTF_10000Evts_WideRange_10000Evts.pdf}
 \includegraphics[width = 0.32 \textwidth]{/home/annik/Documents/Vub/PhD/ThesisSubjects/AnomalousCouplings/June2015_PtCutInfluence/Pos01/LLComparison_ChiSqCutsRVR_MGSamplePos01_SingleGausTF_10000Evts_WideRange_NoLowPtEvts_Cut15_7512Evts.pdf}
 \includegraphics[width = 0.32 \textwidth]{/home/annik/Documents/Vub/PhD/ThesisSubjects/AnomalousCouplings/June2015_PtCutInfluence/Pos01/LLComparison_ChiSqCutsRVR_MGSamplePos01_SingleGausTF_10000Evts_WideRange_NoLowPtEvts_Cut30_2664Evts.pdf}
 \caption{Influence of the $\pT$-cut on the $\loglik$ distribution for increasing $\pT$-cut value (0 $\GeV$, 15 $\GeV$ and 30 $\GeV$) for a MG sample created with $\VR$ = 0.1.}
 \label{fig::PtCutInflPos01}
\end{figure}

\begin{figure}[h!t]
 \centering
 \includegraphics[width = 0.32 \textwidth]{/home/annik/Documents/Vub/PhD/ThesisSubjects/AnomalousCouplings/June2015_PtCutInfluence/SM/LLComparison_ChiSqCutsRVR_MGSampleSMNew_SingleGausTF_10000Evts_WideRange_10000Evts.pdf}
 \includegraphics[width = 0.32 \textwidth]{/home/annik/Documents/Vub/PhD/ThesisSubjects/AnomalousCouplings/June2015_PtCutInfluence/SM/LLComparison_ChiSqCutsRVR_MGSampleSMNew_SingleGausTF_10000Evts_WideRange_NoLowPtEvts_Cut15_7425Evts.pdf}
 \includegraphics[width = 0.32 \textwidth]{/home/annik/Documents/Vub/PhD/ThesisSubjects/AnomalousCouplings/June2015_PtCutInfluence/SM/LLComparison_ChiSqCutsRVR_MGSampleSMNew_SingleGausTF_10000Evts_WideRange_NoLowPtEvts_Cut30_2542Evts.pdf}
 \caption{Influence of the $\pT$-cut on the $\loglik$ distribution for increasing $\pT$-cut value (0 $\GeV$, 15 $\GeV$ and 30 $\GeV$) for a MG sample created with $\VR$ = 0.0.}
 \label{fig::PtCutInflSM}
\end{figure}

\begin{figure}[h!t]
 \centering
 \includegraphics[width = 0.32 \textwidth]{/home/annik/Documents/Vub/PhD/ThesisSubjects/AnomalousCouplings/June2015_PtCutInfluence/Neg01/LLComparison_ChiSqCutsRVR_MGSampleNeg01_SingleGausTF_10000Evts_WideRange_10000Evts.pdf}
 \includegraphics[width = 0.32 \textwidth]{/home/annik/Documents/Vub/PhD/ThesisSubjects/AnomalousCouplings/June2015_PtCutInfluence/Neg01/LLComparison_ChiSqCutsRVR_MGSampleNeg01_SingleGausTF_10000Evts_WideRange_NoLowPtEvts_Cut15_7461Evts.pdf}
 \includegraphics[width = 0.32 \textwidth]{/home/annik/Documents/Vub/PhD/ThesisSubjects/AnomalousCouplings/June2015_PtCutInfluence/Neg01/LLComparison_ChiSqCutsRVR_MGSampleNeg01_SingleGausTF_10000Evts_WideRange_NoLowPtEvts_Cut30_2526Evts.pdf}
 \caption{Influence of the $\pT$-cut on the $\loglik$ distribution for increasing $\pT$-cut value (0 $\GeV$, 15 $\GeV$ and 30 $\GeV$) for a MG sample created with $\VR$ = -0.1.}
 \label{fig::PtCutInflNeg01}
\end{figure}

\begin{figure}[h!t]
 \centering
 \includegraphics[width = 0.32 \textwidth]{/home/annik/Documents/Vub/PhD/ThesisSubjects/AnomalousCouplings/June2015_PtCutInfluence/Neg02/LLComparison_ChiSqCutsRVR_MGSampleNeg02_SingleGausTF_10000Evts_WideRange_10000Evts.pdf}
 \includegraphics[width = 0.32 \textwidth]{/home/annik/Documents/Vub/PhD/ThesisSubjects/AnomalousCouplings/June2015_PtCutInfluence/Neg02/LLComparison_ChiSqCutsRVR_MGSampleNeg02_SingleGausTF_10000Evts_WideRange_NoLowPtEvts_Cut15_7506Evts.pdf}
 \includegraphics[width = 0.32 \textwidth]{/home/annik/Documents/Vub/PhD/ThesisSubjects/AnomalousCouplings/June2015_PtCutInfluence/Neg02/LLComparison_ChiSqCutsRVR_MGSampleNeg02_SingleGausTF_10000Evts_WideRange_NoLowPtEvts_Cut30_2548Evts.pdf}
 \caption{Influence of the $\pT$-cut on the $\loglik$ distribution for increasing $\pT$-cut value (0 $\GeV$, 15 $\GeV$ and 30 $\GeV$) for a MG sample created with $\VR$ = -0.2.}
 \label{fig::PtCutInflNeg02}
\end{figure}

\begin{figure}[h!t]
 \centering
 \includegraphics[width = 0.32 \textwidth]{/home/annik/Documents/Vub/PhD/ThesisSubjects/AnomalousCouplings/June2015_PtCutInfluence/Neg03/LLComparison_ChiSqCutsRVR_MGSampleNeg03_SingleGausTF_10000Evts_WideRange_10000Evts.pdf}
 \includegraphics[width = 0.32 \textwidth]{/home/annik/Documents/Vub/PhD/ThesisSubjects/AnomalousCouplings/June2015_PtCutInfluence/Neg03/LLComparison_ChiSqCutsRVR_MGSampleNeg03_SingleGausTF_10000Evts_WideRange_NoLowPtEvts_Cut15_7519Evts.pdf}
 \includegraphics[width = 0.32 \textwidth]{/home/annik/Documents/Vub/PhD/ThesisSubjects/AnomalousCouplings/June2015_PtCutInfluence/Neg03/LLComparison_ChiSqCutsRVR_MGSampleNeg03_SingleGausTF_10000Evts_WideRange_NoLowPtEvts_Cut30_2625Evts.pdf}
 \caption{Influence of the $\pT$-cut on the $\loglik$ distribution for increasing $\pT$-cut value (0 $\GeV$, 15 $\GeV$ and 30 $\GeV$) for a MG sample created with $\VR$ = -0.3.}
 \label{fig::PtCutInflNeg03}
\end{figure}

\begin{figure}[h!t]
 \centering
 \includegraphics[width = 0.32 \textwidth]{/home/annik/Documents/Vub/PhD/ThesisSubjects/AnomalousCouplings/June2015_PtCutInfluence/Neg05/LLComparison_ChiSqCutsRVR_MGSampleNeg05_SingleGausTF_10000Evts_WideRange_10000Evts.pdf}
 \includegraphics[width = 0.32 \textwidth]{/home/annik/Documents/Vub/PhD/ThesisSubjects/AnomalousCouplings/June2015_PtCutInfluence/Neg05/LLComparison_ChiSqCutsRVR_MGSampleNeg05_SingleGausTF_10000Evts_WideRange_NoLowPtEvts_Cut15_7488Evts.pdf}
 \includegraphics[width = 0.32 \textwidth]{/home/annik/Documents/Vub/PhD/ThesisSubjects/AnomalousCouplings/June2015_PtCutInfluence/Neg05/LLComparison_ChiSqCutsRVR_MGSampleNeg05_SingleGausTF_10000Evts_WideRange_NoLowPtEvts_Cut30_2752Evts.pdf}
 \caption{Influence of the $\pT$-cut on the $\loglik$ distribution for increasing $\pT$-cut value (0 $\GeV$, 15 $\GeV$ and 30 $\GeV$) for a MG sample created with $\VR$ = -0.5.}
 \label{fig::PtCutInflNeg05}
\end{figure}

From the $\loglik$ distributions given for each of the $\VR$ configurations can easily be concluded that requiring the $\pT$ value of the particles in the event (excluding the neutrino) does not have the desired effect. On the contrary it seems that only looking at events with high $\pT$ broadens the $\fth$-order polynomial and actually converts the overall distribution into one with a maximum shape. Since the considered sample concerns generator-level event, it should be understood how the correct minimum-position can be recovered. This different options considered will be discussed in detail in the following sections.

\section{Reweighting MadWeight probability using $\csTh$}
As a first possible solution to the $\loglik$ distortion caused by the event selection, it was studied how the $\csTh$ normalisation could fix the issue. To apply this reweighting the $\csTh$ distribution is calculated for each of the different $\VR$ configurations considered in the studied range and for each of the $\pT$-cuts which could be applied. Then for each of the 35 bins considered in the $\csTh$ distribution the bin content of the distribution prior to the cut is divided by the one after the cut.
\begin{equation}
 weight_{i} = f(\csTh) = \frac{g_{i,All}}{g_{i,Cut}}
\end{equation}
where $i$ runs over all bins of the distribution.\\
Hence this results in a weight for each event which is then multiplied with the MadWeight probability corresponding to this event (for each of the considered $\VR$ configurations). The effect of this normalisation or reweighting on the $\loglik$ distributions is shown in Figure~\ref{fig::CosThetaNorm} for some $\VR$ configurations.
\begin{figure}[h!t]
 \centering
 %Pos05
 \includegraphics[width = 0.24 \textwidth]{/home/annik/Documents/Vub/PhD/ThesisSubjects/AnomalousCouplings/June2015_PtCutInfluence/Pos05/LLComparison_ChiSqCutsRVR_MGSamplePos05_SingleGausTF_10000Evts_WideRange_NoLowPtEvts_Cut15_7572Evts.pdf}
 \includegraphics[width = 0.24 \textwidth]{/home/annik/Documents/Vub/PhD/ThesisSubjects/AnomalousCouplings/June2015_PtCutInfluence/Pos05/LLComparison_ChiSqCutsRVR_MGSamplePos05_SingleGausTF_10000Evts_WideRange_NoLowPtEvts_Cut30_2689Evts.pdf}
 \includegraphics[width = 0.24 \textwidth]{/home/annik/Documents/Vub/PhD/ThesisSubjects/AnomalousCouplings/June2015_PtCutInfluence/Pos05/LLComparison_ChiSqCutsRVR_MGSamplePos05_SingleGausTF_10000Evts_WideRange_NoLowPtEvts_Cut15_ApplyCosThetaReweighting_7572Evts.pdf}
 \includegraphics[width = 0.24 \textwidth]{/home/annik/Documents/Vub/PhD/ThesisSubjects/AnomalousCouplings/June2015_PtCutInfluence/Pos05/LLComparison_ChiSqCutsRVR_MGSamplePos05_SingleGausTF_10000Evts_WideRange_NoLowPtEvts_Cut30_ApplyCosThetaReweighting_2689Evts.pdf}
 %Pos02
 \includegraphics[width = 0.24 \textwidth]{/home/annik/Documents/Vub/PhD/ThesisSubjects/AnomalousCouplings/June2015_PtCutInfluence/Pos02/LLComparison_ChiSqCutsRVR_MGSamplePos02_SingleGausTF_10000Evts_WideRange_NoLowPtEvts_Cut15_7405Evts.pdf}
 \includegraphics[width = 0.24 \textwidth]{/home/annik/Documents/Vub/PhD/ThesisSubjects/AnomalousCouplings/June2015_PtCutInfluence/Pos02/LLComparison_ChiSqCutsRVR_MGSamplePos02_SingleGausTF_10000Evts_WideRange_NoLowPtEvts_Cut30_2526Evts.pdf}
 \includegraphics[width = 0.24 \textwidth]{/home/annik/Documents/Vub/PhD/ThesisSubjects/AnomalousCouplings/June2015_PtCutInfluence/Pos02/LLComparison_ChiSqCutsRVR_MGSamplePos02_SingleGausTF_10000Evts_WideRange_NoLowPtEvts_Cut15_ApplyCosThetaReweighting_7405Evts.pdf}
 \includegraphics[width = 0.24 \textwidth]{/home/annik/Documents/Vub/PhD/ThesisSubjects/AnomalousCouplings/June2015_PtCutInfluence/Pos02/LLComparison_ChiSqCutsRVR_MGSamplePos02_SingleGausTF_10000Evts_WideRange_NoLowPtEvts_Cut30_ApplyCosThetaReweighting_2526Evts.pdf}
 %Standard Model
 \includegraphics[width = 0.24 \textwidth]{/home/annik/Documents/Vub/PhD/ThesisSubjects/AnomalousCouplings/June2015_PtCutInfluence/SM/LLComparison_ChiSqCutsRVR_MGSampleSMNew_SingleGausTF_10000Evts_WideRange_NoLowPtEvts_Cut15_7425Evts.pdf}
 \includegraphics[width = 0.24 \textwidth]{/home/annik/Documents/Vub/PhD/ThesisSubjects/AnomalousCouplings/June2015_PtCutInfluence/SM/LLComparison_ChiSqCutsRVR_MGSampleSMNew_SingleGausTF_10000Evts_WideRange_NoLowPtEvts_Cut30_2542Evts.pdf}
 \includegraphics[width = 0.24 \textwidth]{/home/annik/Documents/Vub/PhD/ThesisSubjects/AnomalousCouplings/June2015_PtCutInfluence/SM/LLComparison_ChiSqCutsRVR_MGSampleSMNew_SingleGausTF_10000Evts_WideRange_NoLowPtEvts_Cut15_ApplyCosThetaReweighting_7425Evts.pdf}
 \includegraphics[width = 0.24 \textwidth]{/home/annik/Documents/Vub/PhD/ThesisSubjects/AnomalousCouplings/June2015_PtCutInfluence/SM/LLComparison_ChiSqCutsRVR_MGSampleSMNew_SingleGausTF_10000Evts_WideRange_NoLowPtEvts_Cut30_ApplyCosThetaReweighting_2542Evts.pdf}
 %Neg02
 \includegraphics[width = 0.24 \textwidth]{/home/annik/Documents/Vub/PhD/ThesisSubjects/AnomalousCouplings/June2015_PtCutInfluence/Neg02/LLComparison_ChiSqCutsRVR_MGSampleNeg02_SingleGausTF_10000Evts_WideRange_NoLowPtEvts_Cut15_7506Evts.pdf}
 \includegraphics[width = 0.24 \textwidth]{/home/annik/Documents/Vub/PhD/ThesisSubjects/AnomalousCouplings/June2015_PtCutInfluence/Neg02/LLComparison_ChiSqCutsRVR_MGSampleNeg02_SingleGausTF_10000Evts_WideRange_NoLowPtEvts_Cut30_2548Evts.pdf}
 \includegraphics[width = 0.24 \textwidth]{/home/annik/Documents/Vub/PhD/ThesisSubjects/AnomalousCouplings/June2015_PtCutInfluence/Neg02/LLComparison_ChiSqCutsRVR_MGSampleNeg02_SingleGausTF_10000Evts_WideRange_NoLowPtEvts_Cut15_ApplyCosThetaReweighting_7506Evts.pdf}
 \includegraphics[width = 0.24 \textwidth]{/home/annik/Documents/Vub/PhD/ThesisSubjects/AnomalousCouplings/June2015_PtCutInfluence/Neg02/LLComparison_ChiSqCutsRVR_MGSampleNeg02_SingleGausTF_10000Evts_WideRange_NoLowPtEvts_Cut30_ApplyCosThetaReweighting_2548Evts.pdf}
 %Neg05
 \includegraphics[width = 0.24 \textwidth]{/home/annik/Documents/Vub/PhD/ThesisSubjects/AnomalousCouplings/June2015_PtCutInfluence/Neg05/LLComparison_ChiSqCutsRVR_MGSampleNeg05_SingleGausTF_10000Evts_WideRange_NoLowPtEvts_Cut15_7488Evts.pdf}
 \includegraphics[width = 0.24 \textwidth]{/home/annik/Documents/Vub/PhD/ThesisSubjects/AnomalousCouplings/June2015_PtCutInfluence/Neg05/LLComparison_ChiSqCutsRVR_MGSampleNeg05_SingleGausTF_10000Evts_WideRange_NoLowPtEvts_Cut30_2752Evts.pdf}
 \includegraphics[width = 0.24 \textwidth]{/home/annik/Documents/Vub/PhD/ThesisSubjects/AnomalousCouplings/June2015_PtCutInfluence/Neg05/LLComparison_ChiSqCutsRVR_MGSampleNeg05_SingleGausTF_10000Evts_WideRange_NoLowPtEvts_Cut15_ApplyCosThetaReweighting_7488Evts.pdf}
 \includegraphics[width = 0.24 \textwidth]{/home/annik/Documents/Vub/PhD/ThesisSubjects/AnomalousCouplings/June2015_PtCutInfluence/Neg05/LLComparison_ChiSqCutsRVR_MGSampleNeg05_SingleGausTF_10000Evts_WideRange_NoLowPtEvts_Cut30_ApplyCosThetaReweighting_2752Evts.pdf}
 \caption{Influence of combining the $\csTh$ reweighting with the MadWeight probability on an event-by-event basis. From left to right the different plots always show the same distributions: first the 15 $\GeV$ case and 30 $\GeV$ case without this normalisation and then both with the reweighting. From top to bottom the different $\VR$ configurations which have been shown are $-0.5$, $-0.2$, $0.0$, $0.2$ and $0.5$.}
 \label{fig::CosThetaNorm}
 \end{figure}

Figure~\ref{fig::CosThetaNorm} is useful for comparing the influence of this $\csTh$ reweighting in a direct way since the original distributions can be found next, but in order to have some more detail and larger size they are also all listed in Figures~\ref{fig::CosThetaPos05} - \ref{fig::CosThetaNeg05}.

\begin{figure}[h!t]
 \centering
 \includegraphics[width = 0.32 \textwidth]{/home/annik/Documents/Vub/PhD/ThesisSubjects/AnomalousCouplings/June2015_PtCutInfluence/Pos05/LLComparison_ChiSqCutsRVR_MGSamplePos05_SingleGausTF_10000Evts_WideRange_10000Evts.pdf}
 \includegraphics[width = 0.32 \textwidth]{/home/annik/Documents/Vub/PhD/ThesisSubjects/AnomalousCouplings/June2015_PtCutInfluence/Pos05/LLComparison_ChiSqCutsRVR_MGSamplePos05_SingleGausTF_10000Evts_WideRange_NoLowPtEvts_Cut15_ApplyCosThetaReweighting_7572Evts.pdf}
 \includegraphics[width = 0.32 \textwidth]{/home/annik/Documents/Vub/PhD/ThesisSubjects/AnomalousCouplings/June2015_PtCutInfluence/Pos05/LLComparison_ChiSqCutsRVR_MGSamplePos05_SingleGausTF_10000Evts_WideRange_NoLowPtEvts_Cut30_ApplyCosThetaReweighting_2689Evts.pdf}
 \caption{Influence of the $\pT$-cut on the $\loglik$ distribution for increasing $\pT$-cut value (0 $\GeV$, 15 $\GeV$ and 30 $\GeV$) for a MG sample created with $\VR$ = 0.5 with $\csTh$ reweighting applied.}
 \label{fig::CosThetaPos05}
\end{figure}

\begin{figure}[h!t]
 \centering
 \includegraphics[width = 0.32 \textwidth]{/home/annik/Documents/Vub/PhD/ThesisSubjects/AnomalousCouplings/June2015_PtCutInfluence/Pos03/LLComparison_ChiSqCutsRVR_MGSamplePos03_SingleGausTF_10000Evts_WideRange_10000Evts.pdf}
 \includegraphics[width = 0.32 \textwidth]{/home/annik/Documents/Vub/PhD/ThesisSubjects/AnomalousCouplings/June2015_PtCutInfluence/Pos03/LLComparison_ChiSqCutsRVR_MGSamplePos03_SingleGausTF_10000Evts_WideRange_NoLowPtEvts_Cut15_ApplyCosThetaReweighting_7539Evts.pdf}
 \includegraphics[width = 0.32 \textwidth]{/home/annik/Documents/Vub/PhD/ThesisSubjects/AnomalousCouplings/June2015_PtCutInfluence/Pos03/LLComparison_ChiSqCutsRVR_MGSamplePos03_SingleGausTF_10000Evts_WideRange_NoLowPtEvts_Cut30_ApplyCosThetaReweighting_2563Evts.pdf}
 \caption{Influence of the $\pT$-cut on the $\loglik$ distribution for increasing $\pT$-cut value ($0$ $\GeV$, $15$ $\GeV$ and $30$ $\GeV$) for a MG sample created with $\VR$ $=$ $0.3$ with $\csTh$ reweighting applied.}
 \label{fig::CosThetaPos03}
\end{figure}

\begin{figure}[h!t]
 \centering
 \includegraphics[width = 0.32 \textwidth]{/home/annik/Documents/Vub/PhD/ThesisSubjects/AnomalousCouplings/June2015_PtCutInfluence/Pos02/LLComparison_ChiSqCutsRVR_MGSamplePos02_SingleGausTF_10000Evts_WideRange_10000Evts.pdf}
 \includegraphics[width = 0.32 \textwidth]{/home/annik/Documents/Vub/PhD/ThesisSubjects/AnomalousCouplings/June2015_PtCutInfluence/Pos02/LLComparison_ChiSqCutsRVR_MGSamplePos02_SingleGausTF_10000Evts_WideRange_NoLowPtEvts_Cut15_ApplyCosThetaReweighting_7405Evts.pdf}
 \includegraphics[width = 0.32 \textwidth]{/home/annik/Documents/Vub/PhD/ThesisSubjects/AnomalousCouplings/June2015_PtCutInfluence/Pos02/LLComparison_ChiSqCutsRVR_MGSamplePos02_SingleGausTF_10000Evts_WideRange_NoLowPtEvts_Cut30_ApplyCosThetaReweighting_2526Evts.pdf}
 \caption{Influence of the $\pT$-cut on the $\loglik$ distribution for increasing $\pT$-cut value (0 $\GeV$, 15 $\GeV$ and 30 $\GeV$) for a MG sample created with $\VR$ = 0.2 with $\csTh$ reweighting applied.}
 \label{fig::CosThetaPos02}
\end{figure}

\begin{figure}[h!t]
 \centering
 \includegraphics[width = 0.32 \textwidth]{/home/annik/Documents/Vub/PhD/ThesisSubjects/AnomalousCouplings/June2015_PtCutInfluence/Pos01/LLComparison_ChiSqCutsRVR_MGSamplePos01_SingleGausTF_10000Evts_WideRange_10000Evts.pdf}
 \includegraphics[width = 0.32 \textwidth]{/home/annik/Documents/Vub/PhD/ThesisSubjects/AnomalousCouplings/June2015_PtCutInfluence/Pos01/LLComparison_ChiSqCutsRVR_MGSamplePos01_SingleGausTF_10000Evts_WideRange_NoLowPtEvts_Cut15_ApplyCosThetaReweighting_7512Evts.pdf}
 \includegraphics[width = 0.32 \textwidth]{/home/annik/Documents/Vub/PhD/ThesisSubjects/AnomalousCouplings/June2015_PtCutInfluence/Pos01/LLComparison_ChiSqCutsRVR_MGSamplePos01_SingleGausTF_10000Evts_WideRange_NoLowPtEvts_Cut30_ApplyCosThetaReweighting_2664Evts.pdf}
 \caption{Influence of the $\pT$-cut on the $\loglik$ distribution for increasing $\pT$-cut value (0 $\GeV$, 15 $\GeV$ and 30 $\GeV$) for a MG sample created with $\VR$ = 0.1 with $\csTh$ reweighting applied.}
 \label{fig::CosThetaPos01}
\end{figure}

\begin{figure}[h!t]
 \centering
 \includegraphics[width = 0.32 \textwidth]{/home/annik/Documents/Vub/PhD/ThesisSubjects/AnomalousCouplings/June2015_PtCutInfluence/SM/LLComparison_ChiSqCutsRVR_MGSampleSMNew_SingleGausTF_10000Evts_WideRange_10000Evts.pdf}
 \includegraphics[width = 0.32 \textwidth]{/home/annik/Documents/Vub/PhD/ThesisSubjects/AnomalousCouplings/June2015_PtCutInfluence/SM/LLComparison_ChiSqCutsRVR_MGSampleSMNew_SingleGausTF_10000Evts_WideRange_NoLowPtEvts_Cut15_ApplyCosThetaReweighting_7425Evts.pdf}
 \includegraphics[width = 0.32 \textwidth]{/home/annik/Documents/Vub/PhD/ThesisSubjects/AnomalousCouplings/June2015_PtCutInfluence/SM/LLComparison_ChiSqCutsRVR_MGSampleSMNew_SingleGausTF_10000Evts_WideRange_NoLowPtEvts_Cut30_ApplyCosThetaReweighting_2542Evts.pdf}
 \caption{Influence of the $\pT$-cut on the $\loglik$ distribution for increasing $\pT$-cut value (0 $\GeV$, 15 $\GeV$ and 30 $\GeV$) for a MG sample created with $\VR$ = 0.0 with $\csTh$ reweighting applied.}
 \label{fig::CosThetaSM}
\end{figure}

\begin{figure}[h!t]
 \centering
 \includegraphics[width = 0.32 \textwidth]{/home/annik/Documents/Vub/PhD/ThesisSubjects/AnomalousCouplings/June2015_PtCutInfluence/Neg01/LLComparison_ChiSqCutsRVR_MGSampleNeg01_SingleGausTF_10000Evts_WideRange_10000Evts.pdf}
 \includegraphics[width = 0.32 \textwidth]{/home/annik/Documents/Vub/PhD/ThesisSubjects/AnomalousCouplings/June2015_PtCutInfluence/Neg01/LLComparison_ChiSqCutsRVR_MGSampleNeg01_SingleGausTF_10000Evts_WideRange_NoLowPtEvts_Cut15_ApplyCosThetaReweighting_7461Evts.pdf}
 \includegraphics[width = 0.32 \textwidth]{/home/annik/Documents/Vub/PhD/ThesisSubjects/AnomalousCouplings/June2015_PtCutInfluence/Neg01/LLComparison_ChiSqCutsRVR_MGSampleNeg01_SingleGausTF_10000Evts_WideRange_NoLowPtEvts_Cut30_ApplyCosThetaReweighting_2526Evts.pdf}
 \caption{Influence of the $\pT$-cut on the $\loglik$ distribution for increasing $\pT$-cut value (0 $\GeV$, 15 $\GeV$ and 30 $\GeV$) for a MG sample created with $\VR$ = -0.1 with $\csTh$ reweighting applied.}
 \label{fig::CosThetaNeg01}
\end{figure}

\begin{figure}[h!t]
 \centering
 \includegraphics[width = 0.32 \textwidth]{/home/annik/Documents/Vub/PhD/ThesisSubjects/AnomalousCouplings/June2015_PtCutInfluence/Neg02/LLComparison_ChiSqCutsRVR_MGSampleNeg02_SingleGausTF_10000Evts_WideRange_10000Evts.pdf}
 \includegraphics[width = 0.32 \textwidth]{/home/annik/Documents/Vub/PhD/ThesisSubjects/AnomalousCouplings/June2015_PtCutInfluence/Neg02/LLComparison_ChiSqCutsRVR_MGSampleNeg02_SingleGausTF_10000Evts_WideRange_NoLowPtEvts_Cut15_ApplyCosThetaReweighting_7506Evts.pdf}
 \includegraphics[width = 0.32 \textwidth]{/home/annik/Documents/Vub/PhD/ThesisSubjects/AnomalousCouplings/June2015_PtCutInfluence/Neg02/LLComparison_ChiSqCutsRVR_MGSampleNeg02_SingleGausTF_10000Evts_WideRange_NoLowPtEvts_Cut30_ApplyCosThetaReweighting_2548Evts.pdf}
 \caption{Influence of the $\pT$-cut on the $\loglik$ distribution for increasing $\pT$-cut value (0 $\GeV$, 15 $\GeV$ and 30 $\GeV$) for a MG sample created with $\VR$ = -0.2 with $\csTh$ reweighting applied.}
 \label{fig::CosThetaNeg02}
\end{figure}

\begin{figure}[h!t]
 \centering
 \includegraphics[width = 0.32 \textwidth]{/home/annik/Documents/Vub/PhD/ThesisSubjects/AnomalousCouplings/June2015_PtCutInfluence/Neg03/LLComparison_ChiSqCutsRVR_MGSampleNeg03_SingleGausTF_10000Evts_WideRange_10000Evts.pdf}
 \includegraphics[width = 0.32 \textwidth]{/home/annik/Documents/Vub/PhD/ThesisSubjects/AnomalousCouplings/June2015_PtCutInfluence/Neg03/LLComparison_ChiSqCutsRVR_MGSampleNeg03_SingleGausTF_10000Evts_WideRange_NoLowPtEvts_Cut15_ApplyCosThetaReweighting_7519Evts.pdf}
 \includegraphics[width = 0.32 \textwidth]{/home/annik/Documents/Vub/PhD/ThesisSubjects/AnomalousCouplings/June2015_PtCutInfluence/Neg03/LLComparison_ChiSqCutsRVR_MGSampleNeg03_SingleGausTF_10000Evts_WideRange_NoLowPtEvts_Cut30_ApplyCosThetaReweighting_2625Evts.pdf}
 \caption{Influence of the $\pT$-cut on the $\loglik$ distribution for increasing $\pT$-cut value (0 $\GeV$, 15 $\GeV$ and 30 $\GeV$) for a MG sample created with $\VR$ = -0.3 with $\csTh$ reweighting applied.}
 \label{fig::CosThetaNeg03}
\end{figure}

\begin{figure}[h!t]
 \centering
 \includegraphics[width = 0.32 \textwidth]{/home/annik/Documents/Vub/PhD/ThesisSubjects/AnomalousCouplings/June2015_PtCutInfluence/Neg05/LLComparison_ChiSqCutsRVR_MGSampleNeg05_SingleGausTF_10000Evts_WideRange_10000Evts.pdf}
 \includegraphics[width = 0.32 \textwidth]{/home/annik/Documents/Vub/PhD/ThesisSubjects/AnomalousCouplings/June2015_PtCutInfluence/Neg05/LLComparison_ChiSqCutsRVR_MGSampleNeg05_SingleGausTF_10000Evts_WideRange_NoLowPtEvts_Cut15_ApplyCosThetaReweighting_7488Evts.pdf}
 \includegraphics[width = 0.32 \textwidth]{/home/annik/Documents/Vub/PhD/ThesisSubjects/AnomalousCouplings/June2015_PtCutInfluence/Neg05/LLComparison_ChiSqCutsRVR_MGSampleNeg05_SingleGausTF_10000Evts_WideRange_NoLowPtEvts_Cut30_ApplyCosThetaReweighting_2752Evts.pdf}
 \caption{Influence of the $\pT$-cut on the $\loglik$ distribution for increasing $\pT$-cut value (0 $\GeV$, 15 $\GeV$ and 30 $\GeV$) for a MG sample created with $\VR$ = -0.5 with $\csTh$ reweighting applied.}
 \label{fig::CosThetaNeg05}
\end{figure}
 
The $\loglik$ distributions obtained when the $\csTh$ reweighting is applied shows a significant improvement compared to the original distributions. In most of the cases the desired minimum position can almost be retrieved, however this is somewhat undone by the high $\pT$-cut of 30 $\GeV$. It seems that the best agreement is found for lower $\pT$ cuts such as 15 $\GeV$, which is probably also closer to the preselection cleaning applied in the CMS generator-level sample. The possible bias introduced by the more realistic event selection constraints is also visible for the reco-level $t\bar{t}$ sample which was already studied.

\section{Difference between $\csTh$ reweighting and acceptance normalisation}
However after that this $\csTh$ reweighting has been applied it was clear that it plays almost the role of the acceptance normalisation since it gives a higher weight to events which have a higher probability to be cut away by the applied event selection based on their $\csTh$ value. Therefore in order to know whether the issue shown in the beginning cannot just be resolved by the acceptance normalisation, which actually should be applied since not all $\VR$ values react the same on the applied $\pT$ constraints. 
\\

Just as was the case for visualizing the influence of the $\csTh$ reweighting discussed in the previous section, first a selection of $\loglik$ distributions will be showed allowing a comparison between the $\csTh$ reweighting (Fig~\ref{fig::AccNorm}) and the acceptance normalisation and afterwards the full list of distributions will be given for each $\VR$ configuration (Figs~\ref{fig::AccNormPos05} - \ref{fig::AccNormNeg05}).

\begin{figure}[h!t]
 \centering
 %Pos05
 \includegraphics[width = 0.24 \textwidth]{/home/annik/Documents/Vub/PhD/ThesisSubjects/AnomalousCouplings/June2015_PtCutInfluence/Pos05/LLComparison_ChiSqCutsRVR_MGSamplePos05_SingleGausTF_10000Evts_WideRange_NoLowPtEvts_Cut15_ApplyCosThetaReweighting_7572Evts.pdf}
 \includegraphics[width = 0.24 \textwidth]{/home/annik/Documents/Vub/PhD/ThesisSubjects/AnomalousCouplings/June2015_PtCutInfluence/Pos05/LLComparison_ChiSqCutsRVR_MGSamplePos05_SingleGausTF_10000Evts_WideRange_NoLowPtEvts_Cut30_ApplyCosThetaReweighting_2689Evts.pdf}
 \includegraphics[width = 0.24 \textwidth]{/home/annik/Documents/Vub/PhD/ThesisSubjects/AnomalousCouplings/June2015_PtCutInfluence/Pos05/LLComparison_ChiSqCutsRVR_MGSamplePos05_SingleGausTF_10000Evts_WideRange_NoLowPtEvts_Cut15_AccNormApplied_7572Evts.pdf}
 \includegraphics[width = 0.24 \textwidth]{/home/annik/Documents/Vub/PhD/ThesisSubjects/AnomalousCouplings/June2015_PtCutInfluence/Pos05/LLComparison_ChiSqCutsRVR_MGSamplePos05_SingleGausTF_10000Evts_WideRange_NoLowPtEvts_Cut30_AccNormApplied_2689Evts.pdf}
 %Pos02
 \includegraphics[width = 0.24 \textwidth]{/home/annik/Documents/Vub/PhD/ThesisSubjects/AnomalousCouplings/June2015_PtCutInfluence/Pos02/LLComparison_ChiSqCutsRVR_MGSamplePos02_SingleGausTF_10000Evts_WideRange_NoLowPtEvts_Cut15_ApplyCosThetaReweighting_7405Evts.pdf}
 \includegraphics[width = 0.24 \textwidth]{/home/annik/Documents/Vub/PhD/ThesisSubjects/AnomalousCouplings/June2015_PtCutInfluence/Pos02/LLComparison_ChiSqCutsRVR_MGSamplePos02_SingleGausTF_10000Evts_WideRange_NoLowPtEvts_Cut30_ApplyCosThetaReweighting_2526Evts.pdf}
 \includegraphics[width = 0.24 \textwidth]{/home/annik/Documents/Vub/PhD/ThesisSubjects/AnomalousCouplings/June2015_PtCutInfluence/Pos02/LLComparison_ChiSqCutsRVR_MGSamplePos02_SingleGausTF_10000Evts_WideRange_NoLowPtEvts_Cut15_AccNormApplied_7405Evts.pdf}
 \includegraphics[width = 0.24 \textwidth]{/home/annik/Documents/Vub/PhD/ThesisSubjects/AnomalousCouplings/June2015_PtCutInfluence/Pos02/LLComparison_ChiSqCutsRVR_MGSamplePos02_SingleGausTF_10000Evts_WideRange_NoLowPtEvts_Cut30_AccNormApplied_2526Evts.pdf}
 %Standard Model
 \includegraphics[width = 0.24 \textwidth]{/home/annik/Documents/Vub/PhD/ThesisSubjects/AnomalousCouplings/June2015_PtCutInfluence/SM/LLComparison_ChiSqCutsRVR_MGSampleSMNew_SingleGausTF_10000Evts_WideRange_NoLowPtEvts_Cut15_ApplyCosThetaReweighting_7425Evts.pdf}
 \includegraphics[width = 0.24 \textwidth]{/home/annik/Documents/Vub/PhD/ThesisSubjects/AnomalousCouplings/June2015_PtCutInfluence/SM/LLComparison_ChiSqCutsRVR_MGSampleSMNew_SingleGausTF_10000Evts_WideRange_NoLowPtEvts_Cut30_ApplyCosThetaReweighting_2542Evts.pdf}
 \includegraphics[width = 0.24 \textwidth]{/home/annik/Documents/Vub/PhD/ThesisSubjects/AnomalousCouplings/June2015_PtCutInfluence/SM/LLComparison_ChiSqCutsRVR_MGSampleSMNew_SingleGausTF_10000Evts_WideRange_NoLowPtEvts_Cut15_AccNormApplied_7425Evts.pdf}
 \includegraphics[width = 0.24 \textwidth]{/home/annik/Documents/Vub/PhD/ThesisSubjects/AnomalousCouplings/June2015_PtCutInfluence/SM/LLComparison_ChiSqCutsRVR_MGSampleSMNew_SingleGausTF_10000Evts_WideRange_NoLowPtEvts_Cut30_AccNormApplied_2542Evts.pdf}
 %Neg02
 \includegraphics[width = 0.24 \textwidth]{/home/annik/Documents/Vub/PhD/ThesisSubjects/AnomalousCouplings/June2015_PtCutInfluence/Neg02/LLComparison_ChiSqCutsRVR_MGSampleNeg02_SingleGausTF_10000Evts_WideRange_NoLowPtEvts_Cut15_ApplyCosThetaReweighting_7506Evts.pdf}
 \includegraphics[width = 0.24 \textwidth]{/home/annik/Documents/Vub/PhD/ThesisSubjects/AnomalousCouplings/June2015_PtCutInfluence/Neg02/LLComparison_ChiSqCutsRVR_MGSampleNeg02_SingleGausTF_10000Evts_WideRange_NoLowPtEvts_Cut30_ApplyCosThetaReweighting_2548Evts.pdf}
 \includegraphics[width = 0.24 \textwidth]{/home/annik/Documents/Vub/PhD/ThesisSubjects/AnomalousCouplings/June2015_PtCutInfluence/Neg02/LLComparison_ChiSqCutsRVR_MGSampleNeg02_SingleGausTF_10000Evts_WideRange_NoLowPtEvts_Cut15_AccNormApplied_7506Evts.pdf}
 \includegraphics[width = 0.24 \textwidth]{/home/annik/Documents/Vub/PhD/ThesisSubjects/AnomalousCouplings/June2015_PtCutInfluence/Neg02/LLComparison_ChiSqCutsRVR_MGSampleNeg02_SingleGausTF_10000Evts_WideRange_NoLowPtEvts_Cut30_AccNormApplied_2548Evts.pdf}
 %Neg05
 \includegraphics[width = 0.24 \textwidth]{/home/annik/Documents/Vub/PhD/ThesisSubjects/AnomalousCouplings/June2015_PtCutInfluence/Neg05/LLComparison_ChiSqCutsRVR_MGSampleNeg05_SingleGausTF_10000Evts_WideRange_NoLowPtEvts_Cut15_ApplyCosThetaReweighting_7488Evts.pdf}
 \includegraphics[width = 0.24 \textwidth]{/home/annik/Documents/Vub/PhD/ThesisSubjects/AnomalousCouplings/June2015_PtCutInfluence/Neg05/LLComparison_ChiSqCutsRVR_MGSampleNeg05_SingleGausTF_10000Evts_WideRange_NoLowPtEvts_Cut30_ApplyCosThetaReweighting_2752Evts.pdf}
 \includegraphics[width = 0.24 \textwidth]{/home/annik/Documents/Vub/PhD/ThesisSubjects/AnomalousCouplings/June2015_PtCutInfluence/Neg05/LLComparison_ChiSqCutsRVR_MGSampleNeg05_SingleGausTF_10000Evts_WideRange_NoLowPtEvts_Cut15_AccNormApplied_7488Evts.pdf}
 \includegraphics[width = 0.24 \textwidth]{/home/annik/Documents/Vub/PhD/ThesisSubjects/AnomalousCouplings/June2015_PtCutInfluence/Neg05/LLComparison_ChiSqCutsRVR_MGSampleNeg05_SingleGausTF_10000Evts_WideRange_NoLowPtEvts_Cut30_AccNormApplied_2752Evts.pdf}
 \caption{Influence of combining the acceptance normalisation with the MadWeight probability on an event-by-event basis. From left to right the different plots always show the same distributions: first the 15 $\GeV$ case and 30 $\GeV$ case with the $\csTh$ reweighting and then both with the acceptance normalisation. From top to bottom the different $\VR$ configurations which have been shown are $-0.5$, $-0.2$, $0.0$, $0.2$ and $0.5$.}
 \label{fig::AccNorm}
 \end{figure}

The full list of $\loglik$ distributions is given with the original distribution on the left as was the case in the previous Figures.

\begin{figure}[h!t]
 \centering
 \includegraphics[width = 0.32 \textwidth]{/home/annik/Documents/Vub/PhD/ThesisSubjects/AnomalousCouplings/June2015_PtCutInfluence/Pos05/LLComparison_ChiSqCutsRVR_MGSamplePos05_SingleGausTF_10000Evts_WideRange_10000Evts.pdf}
 \includegraphics[width = 0.32 \textwidth]{/home/annik/Documents/Vub/PhD/ThesisSubjects/AnomalousCouplings/June2015_PtCutInfluence/Pos05/LLComparison_ChiSqCutsRVR_MGSamplePos05_SingleGausTF_10000Evts_WideRange_NoLowPtEvts_Cut15_AccNormApplied_7572Evts.pdf}
 \includegraphics[width = 0.32 \textwidth]{/home/annik/Documents/Vub/PhD/ThesisSubjects/AnomalousCouplings/June2015_PtCutInfluence/Pos05/LLComparison_ChiSqCutsRVR_MGSamplePos05_SingleGausTF_10000Evts_WideRange_NoLowPtEvts_Cut30_AccNormApplied_2689Evts.pdf}
 \caption{Influence of the $\pT$-cut on the $\loglik$ distribution for increasing $\pT$-cut value (0 $\GeV$, 15 $\GeV$ and 30 $\GeV$) for a MG sample created with $\VR$ = 0.5 with acceptance normalisation applied.}
 \label{fig::AccNormPos05}
\end{figure}

\begin{figure}[h!t]
 \centering
 \includegraphics[width = 0.32 \textwidth]{/home/annik/Documents/Vub/PhD/ThesisSubjects/AnomalousCouplings/June2015_PtCutInfluence/Pos03/LLComparison_ChiSqCutsRVR_MGSamplePos03_SingleGausTF_10000Evts_WideRange_10000Evts.pdf}
 \includegraphics[width = 0.32 \textwidth]{/home/annik/Documents/Vub/PhD/ThesisSubjects/AnomalousCouplings/June2015_PtCutInfluence/Pos03/LLComparison_ChiSqCutsRVR_MGSamplePos03_SingleGausTF_10000Evts_WideRange_NoLowPtEvts_Cut15_AccNormApplied_7539Evts.pdf}
 \includegraphics[width = 0.32 \textwidth]{/home/annik/Documents/Vub/PhD/ThesisSubjects/AnomalousCouplings/June2015_PtCutInfluence/Pos03/LLComparison_ChiSqCutsRVR_MGSamplePos03_SingleGausTF_10000Evts_WideRange_NoLowPtEvts_Cut30_AccNormApplied_2563Evts.pdf}
 \caption{Influence of the $\pT$-cut on the $\loglik$ distribution for increasing $\pT$-cut value ($0$ $\GeV$, $15$ $\GeV$ and $30$ $\GeV$) for a MG sample created with $\VR$ $=$ $0.3$ with acceptance normalisation applied.}
 \label{fig::AccNormPos03}
\end{figure}

\begin{figure}[h!t]
 \centering
 \includegraphics[width = 0.32 \textwidth]{/home/annik/Documents/Vub/PhD/ThesisSubjects/AnomalousCouplings/June2015_PtCutInfluence/Pos02/LLComparison_ChiSqCutsRVR_MGSamplePos02_SingleGausTF_10000Evts_WideRange_10000Evts.pdf}
 \includegraphics[width = 0.32 \textwidth]{/home/annik/Documents/Vub/PhD/ThesisSubjects/AnomalousCouplings/June2015_PtCutInfluence/Pos02/LLComparison_ChiSqCutsRVR_MGSamplePos02_SingleGausTF_10000Evts_WideRange_NoLowPtEvts_Cut15_AccNormApplied_7405Evts.pdf}
 \includegraphics[width = 0.32 \textwidth]{/home/annik/Documents/Vub/PhD/ThesisSubjects/AnomalousCouplings/June2015_PtCutInfluence/Pos02/LLComparison_ChiSqCutsRVR_MGSamplePos02_SingleGausTF_10000Evts_WideRange_NoLowPtEvts_Cut30_AccNormApplied_2526Evts.pdf}
 \caption{Influence of the $\pT$-cut on the $\loglik$ distribution for increasing $\pT$-cut value (0 $\GeV$, 15 $\GeV$ and 30 $\GeV$) for a MG sample created with $\VR$ = 0.2 with acceptance normalisation applied.}
 \label{fig::AccNormPos02}
\end{figure}

\begin{figure}[h!t]
 \centering
 \includegraphics[width = 0.32 \textwidth]{/home/annik/Documents/Vub/PhD/ThesisSubjects/AnomalousCouplings/June2015_PtCutInfluence/Pos01/LLComparison_ChiSqCutsRVR_MGSamplePos01_SingleGausTF_10000Evts_WideRange_10000Evts.pdf}
 \includegraphics[width = 0.32 \textwidth]{/home/annik/Documents/Vub/PhD/ThesisSubjects/AnomalousCouplings/June2015_PtCutInfluence/Pos01/LLComparison_ChiSqCutsRVR_MGSamplePos01_SingleGausTF_10000Evts_WideRange_NoLowPtEvts_Cut15_AccNormApplied_7512Evts.pdf}
 \includegraphics[width = 0.32 \textwidth]{/home/annik/Documents/Vub/PhD/ThesisSubjects/AnomalousCouplings/June2015_PtCutInfluence/Pos01/LLComparison_ChiSqCutsRVR_MGSamplePos01_SingleGausTF_10000Evts_WideRange_NoLowPtEvts_Cut30_AccNormApplied_2664Evts.pdf}
 \caption{Influence of the $\pT$-cut on the $\loglik$ distribution for increasing $\pT$-cut value (0 $\GeV$, 15 $\GeV$ and 30 $\GeV$) for a MG sample created with $\VR$ = 0.1 with acceptance normalisation applied.}
 \label{fig::AccNormPos01}
\end{figure}

\begin{figure}[h!t]
 \centering
 \includegraphics[width = 0.32 \textwidth]{/home/annik/Documents/Vub/PhD/ThesisSubjects/AnomalousCouplings/June2015_PtCutInfluence/SM/LLComparison_ChiSqCutsRVR_MGSampleSMNew_SingleGausTF_10000Evts_WideRange_10000Evts.pdf}
 \includegraphics[width = 0.32 \textwidth]{/home/annik/Documents/Vub/PhD/ThesisSubjects/AnomalousCouplings/June2015_PtCutInfluence/SM/LLComparison_ChiSqCutsRVR_MGSampleSMNew_SingleGausTF_10000Evts_WideRange_NoLowPtEvts_Cut15_AccNormApplied_7425Evts.pdf}
 \includegraphics[width = 0.32 \textwidth]{/home/annik/Documents/Vub/PhD/ThesisSubjects/AnomalousCouplings/June2015_PtCutInfluence/SM/LLComparison_ChiSqCutsRVR_MGSampleSMNew_SingleGausTF_10000Evts_WideRange_NoLowPtEvts_Cut30_AccNormApplied_2542Evts.pdf}
 \caption{Influence of the $\pT$-cut on the $\loglik$ distribution for increasing $\pT$-cut value (0 $\GeV$, 15 $\GeV$ and 30 $\GeV$) for a MG sample created with $\VR$ = 0.0 with acceptance normalisation applied.}
 \label{fig::AccNormSM}
\end{figure}

\begin{figure}[h!t]
 \centering
 \includegraphics[width = 0.32 \textwidth]{/home/annik/Documents/Vub/PhD/ThesisSubjects/AnomalousCouplings/June2015_PtCutInfluence/Neg01/LLComparison_ChiSqCutsRVR_MGSampleNeg01_SingleGausTF_10000Evts_WideRange_10000Evts.pdf}
 \includegraphics[width = 0.32 \textwidth]{/home/annik/Documents/Vub/PhD/ThesisSubjects/AnomalousCouplings/June2015_PtCutInfluence/Neg01/LLComparison_ChiSqCutsRVR_MGSampleNeg01_SingleGausTF_10000Evts_WideRange_NoLowPtEvts_Cut15_AccNormApplied_7461Evts.pdf}
 \includegraphics[width = 0.32 \textwidth]{/home/annik/Documents/Vub/PhD/ThesisSubjects/AnomalousCouplings/June2015_PtCutInfluence/Neg01/LLComparison_ChiSqCutsRVR_MGSampleNeg01_SingleGausTF_10000Evts_WideRange_NoLowPtEvts_Cut30_AccNormApplied_2526Evts.pdf}
 \caption{Influence of the $\pT$-cut on the $\loglik$ distribution for increasing $\pT$-cut value (0 $\GeV$, 15 $\GeV$ and 30 $\GeV$) for a MG sample created with $\VR$ = -0.1 with acceptance normalisation applied.}
 \label{fig::AccNormNeg01}
\end{figure}

\begin{figure}[h!t]
 \centering
 \includegraphics[width = 0.32 \textwidth]{/home/annik/Documents/Vub/PhD/ThesisSubjects/AnomalousCouplings/June2015_PtCutInfluence/Neg02/LLComparison_ChiSqCutsRVR_MGSampleNeg02_SingleGausTF_10000Evts_WideRange_10000Evts.pdf}
 \includegraphics[width = 0.32 \textwidth]{/home/annik/Documents/Vub/PhD/ThesisSubjects/AnomalousCouplings/June2015_PtCutInfluence/Neg02/LLComparison_ChiSqCutsRVR_MGSampleNeg02_SingleGausTF_10000Evts_WideRange_NoLowPtEvts_Cut15_AccNormApplied_7506Evts.pdf}
 \includegraphics[width = 0.32 \textwidth]{/home/annik/Documents/Vub/PhD/ThesisSubjects/AnomalousCouplings/June2015_PtCutInfluence/Neg02/LLComparison_ChiSqCutsRVR_MGSampleNeg02_SingleGausTF_10000Evts_WideRange_NoLowPtEvts_Cut30_AccNormApplied_2548Evts.pdf}
 \caption{Influence of the $\pT$-cut on the $\loglik$ distribution for increasing $\pT$-cut value (0 $\GeV$, 15 $\GeV$ and 30 $\GeV$) for a MG sample created with $\VR$ = -0.2 with acceptance normalisation applied.}
 \label{fig::AccNormNeg02}
\end{figure}

\begin{figure}[h!t]
 \centering
 \includegraphics[width = 0.32 \textwidth]{/home/annik/Documents/Vub/PhD/ThesisSubjects/AnomalousCouplings/June2015_PtCutInfluence/Neg03/LLComparison_ChiSqCutsRVR_MGSampleNeg03_SingleGausTF_10000Evts_WideRange_10000Evts.pdf}
 \includegraphics[width = 0.32 \textwidth]{/home/annik/Documents/Vub/PhD/ThesisSubjects/AnomalousCouplings/June2015_PtCutInfluence/Neg03/LLComparison_ChiSqCutsRVR_MGSampleNeg03_SingleGausTF_10000Evts_WideRange_NoLowPtEvts_Cut15_AccNormApplied_7519Evts.pdf}
 \includegraphics[width = 0.32 \textwidth]{/home/annik/Documents/Vub/PhD/ThesisSubjects/AnomalousCouplings/June2015_PtCutInfluence/Neg03/LLComparison_ChiSqCutsRVR_MGSampleNeg03_SingleGausTF_10000Evts_WideRange_NoLowPtEvts_Cut30_AccNormApplied_2625Evts.pdf}
 \caption{Influence of the $\pT$-cut on the $\loglik$ distribution for increasing $\pT$-cut value (0 $\GeV$, 15 $\GeV$ and 30 $\GeV$) for a MG sample created with $\VR$ = -0.3 with acceptance normalisation applied.}
 \label{fig::AccNormNeg03}
\end{figure}

\begin{figure}[h!t]
 \centering
 \includegraphics[width = 0.32 \textwidth]{/home/annik/Documents/Vub/PhD/ThesisSubjects/AnomalousCouplings/June2015_PtCutInfluence/Neg05/LLComparison_ChiSqCutsRVR_MGSampleNeg05_SingleGausTF_10000Evts_WideRange_10000Evts.pdf}
 \includegraphics[width = 0.32 \textwidth]{/home/annik/Documents/Vub/PhD/ThesisSubjects/AnomalousCouplings/June2015_PtCutInfluence/Neg05/LLComparison_ChiSqCutsRVR_MGSampleNeg05_SingleGausTF_10000Evts_WideRange_NoLowPtEvts_Cut15_AccNormApplied_7488Evts.pdf}
 \includegraphics[width = 0.32 \textwidth]{/home/annik/Documents/Vub/PhD/ThesisSubjects/AnomalousCouplings/June2015_PtCutInfluence/Neg05/LLComparison_ChiSqCutsRVR_MGSampleNeg05_SingleGausTF_10000Evts_WideRange_NoLowPtEvts_Cut30_AccNormApplied_2752Evts.pdf}
 \caption{Influence of the $\pT$-cut on the $\loglik$ distribution for increasing $\pT$-cut value (0 $\GeV$, 15 $\GeV$ and 30 $\GeV$) for a MG sample created with $\VR$ = -0.5 with acceptance normalisation applied.}
 \label{fig::AccNormNeg05}
\end{figure}
 
\section{Applying both $\csTh$ and acceptance normalisation?}

As a final test, motivated by the fact that the results obtained from the $\csTh$ reweighting and the acceptance normalisation show some differences, is to see whether both corrections can be applied together.
\\
However on first sight it seems that this would actually result in overcorrecting the type of events which are more likely to be affected by the applied event selection constraints. But not very certain whether there exists a control check which can be performed in order to make sure that it is actually doing the right thing ...

\begin{figure}[h!t]
 \centering
 \includegraphics[width = 0.32 \textwidth]{/home/annik/Documents/Vub/PhD/ThesisSubjects/AnomalousCouplings/June2015_PtCutInfluence/Pos05/LLComparison_ChiSqCutsRVR_MGSamplePos05_SingleGausTF_10000Evts_WideRange_10000Evts.pdf}
 \includegraphics[width = 0.32 \textwidth]{/home/annik/Documents/Vub/PhD/ThesisSubjects/AnomalousCouplings/June2015_PtCutInfluence/Pos05/LLComparison_ChiSqCutsRVR_MGSamplePos05_SingleGausTF_10000Evts_WideRange_NoLowPtEvts_Cut15_ApplyCosThetaReweighting_AccNormApplied_7572Evts.pdf}
 \includegraphics[width = 0.32 \textwidth]{/home/annik/Documents/Vub/PhD/ThesisSubjects/AnomalousCouplings/June2015_PtCutInfluence/Pos05/LLComparison_ChiSqCutsRVR_MGSamplePos05_SingleGausTF_10000Evts_WideRange_NoLowPtEvts_Cut30_ApplyCosThetaReweighting_AccNormApplied_2689Evts.pdf}
 \caption{Influence of the $\pT$-cut on the $\loglik$ distribution for increasing $\pT$-cut value (0 $\GeV$, 15 $\GeV$ and 30 $\GeV$) for a MG sample created with $\VR$ = 0.5 with acceptance normalisation and $\csTh$ reweighting applied.}
 \label{fig::AccNormCosPos05}
\end{figure}

\begin{figure}[h!t]
 \centering
 \includegraphics[width = 0.32 \textwidth]{/home/annik/Documents/Vub/PhD/ThesisSubjects/AnomalousCouplings/June2015_PtCutInfluence/Pos03/LLComparison_ChiSqCutsRVR_MGSamplePos03_SingleGausTF_10000Evts_WideRange_10000Evts.pdf}
 \includegraphics[width = 0.32 \textwidth]{/home/annik/Documents/Vub/PhD/ThesisSubjects/AnomalousCouplings/June2015_PtCutInfluence/Pos03/LLComparison_ChiSqCutsRVR_MGSamplePos03_SingleGausTF_10000Evts_WideRange_NoLowPtEvts_Cut15_ApplyCosThetaReweighting_AccNormApplied_7539Evts.pdf}
 \includegraphics[width = 0.32 \textwidth]{/home/annik/Documents/Vub/PhD/ThesisSubjects/AnomalousCouplings/June2015_PtCutInfluence/Pos03/LLComparison_ChiSqCutsRVR_MGSamplePos03_SingleGausTF_10000Evts_WideRange_NoLowPtEvts_Cut30_ApplyCosThetaReweighting_AccNormApplied_2563Evts.pdf}
 \caption{Influence of the $\pT$-cut on the $\loglik$ distribution for increasing $\pT$-cut value ($0$ $\GeV$, $15$ $\GeV$ and $30$ $\GeV$) for a MG sample created with $\VR$ $=$ $0.3$ with acceptance normalisation and $\csTh$ reweighting applied.}
 \label{fig::AccNormCosPos03}
\end{figure}

\begin{figure}[h!t]
 \centering
 \includegraphics[width = 0.32 \textwidth]{/home/annik/Documents/Vub/PhD/ThesisSubjects/AnomalousCouplings/June2015_PtCutInfluence/Pos02/LLComparison_ChiSqCutsRVR_MGSamplePos02_SingleGausTF_10000Evts_WideRange_10000Evts.pdf}
 \includegraphics[width = 0.32 \textwidth]{/home/annik/Documents/Vub/PhD/ThesisSubjects/AnomalousCouplings/June2015_PtCutInfluence/Pos02/LLComparison_ChiSqCutsRVR_MGSamplePos02_SingleGausTF_10000Evts_WideRange_NoLowPtEvts_Cut15_ApplyCosThetaReweighting_AccNormApplied_7405Evts.pdf}
 \includegraphics[width = 0.32 \textwidth]{/home/annik/Documents/Vub/PhD/ThesisSubjects/AnomalousCouplings/June2015_PtCutInfluence/Pos02/LLComparison_ChiSqCutsRVR_MGSamplePos02_SingleGausTF_10000Evts_WideRange_NoLowPtEvts_Cut30_ApplyCosThetaReweighting_AccNormApplied_2526Evts.pdf}
 \caption{Influence of the $\pT$-cut on the $\loglik$ distribution for increasing $\pT$-cut value (0 $\GeV$, 15 $\GeV$ and 30 $\GeV$) for a MG sample created with $\VR$ = 0.2 with acceptance normalisation and $\csTh$ reweighting applied.}
 \label{fig::AccNormCosPos02}
\end{figure}

\begin{figure}[h!t]
 \centering
 \includegraphics[width = 0.32 \textwidth]{/home/annik/Documents/Vub/PhD/ThesisSubjects/AnomalousCouplings/June2015_PtCutInfluence/Pos01/LLComparison_ChiSqCutsRVR_MGSamplePos01_SingleGausTF_10000Evts_WideRange_10000Evts.pdf}
 \includegraphics[width = 0.32 \textwidth]{/home/annik/Documents/Vub/PhD/ThesisSubjects/AnomalousCouplings/June2015_PtCutInfluence/Pos01/LLComparison_ChiSqCutsRVR_MGSamplePos01_SingleGausTF_10000Evts_WideRange_NoLowPtEvts_Cut15_ApplyCosThetaReweighting_AccNormApplied_7512Evts.pdf}
 \includegraphics[width = 0.32 \textwidth]{/home/annik/Documents/Vub/PhD/ThesisSubjects/AnomalousCouplings/June2015_PtCutInfluence/Pos01/LLComparison_ChiSqCutsRVR_MGSamplePos01_SingleGausTF_10000Evts_WideRange_NoLowPtEvts_Cut30_ApplyCosThetaReweighting_AccNormApplied_2664Evts.pdf}
 \caption{Influence of the $\pT$-cut on the $\loglik$ distribution for increasing $\pT$-cut value (0 $\GeV$, 15 $\GeV$ and 30 $\GeV$) for a MG sample created with $\VR$ = 0.1 with acceptance normalisation and $\csTh$ reweighting applied.}
 \label{fig::AccNormCosPos01}
\end{figure}

\begin{figure}[h!t]
 \centering
 \includegraphics[width = 0.32 \textwidth]{/home/annik/Documents/Vub/PhD/ThesisSubjects/AnomalousCouplings/June2015_PtCutInfluence/SM/LLComparison_ChiSqCutsRVR_MGSampleSMNew_SingleGausTF_10000Evts_WideRange_10000Evts.pdf}
 \includegraphics[width = 0.32 \textwidth]{/home/annik/Documents/Vub/PhD/ThesisSubjects/AnomalousCouplings/June2015_PtCutInfluence/SM/LLComparison_ChiSqCutsRVR_MGSampleSMNew_SingleGausTF_10000Evts_WideRange_NoLowPtEvts_Cut15_ApplyCosThetaReweighting_AccNormApplied_7425Evts.pdf}
 \includegraphics[width = 0.32 \textwidth]{/home/annik/Documents/Vub/PhD/ThesisSubjects/AnomalousCouplings/June2015_PtCutInfluence/SM/LLComparison_ChiSqCutsRVR_MGSampleSMNew_SingleGausTF_10000Evts_WideRange_NoLowPtEvts_Cut30_ApplyCosThetaReweighting_AccNormApplied_2542Evts.pdf}
 \caption{Influence of the $\pT$-cut on the $\loglik$ distribution for increasing $\pT$-cut value (0 $\GeV$, 15 $\GeV$ and 30 $\GeV$) for a MG sample created with $\VR$ = 0.0 with acceptance normalisation and $\csTh$ reweighting applied.}
 \label{fig::AccNormCosSM}
\end{figure}

\begin{figure}[h!t]
 \centering
 \includegraphics[width = 0.32 \textwidth]{/home/annik/Documents/Vub/PhD/ThesisSubjects/AnomalousCouplings/June2015_PtCutInfluence/Neg01/LLComparison_ChiSqCutsRVR_MGSampleNeg01_SingleGausTF_10000Evts_WideRange_10000Evts.pdf}
 \includegraphics[width = 0.32 \textwidth]{/home/annik/Documents/Vub/PhD/ThesisSubjects/AnomalousCouplings/June2015_PtCutInfluence/Neg01/LLComparison_ChiSqCutsRVR_MGSampleNeg01_SingleGausTF_10000Evts_WideRange_NoLowPtEvts_Cut15_ApplyCosThetaReweighting_AccNormApplied_7461Evts.pdf}
 \includegraphics[width = 0.32 \textwidth]{/home/annik/Documents/Vub/PhD/ThesisSubjects/AnomalousCouplings/June2015_PtCutInfluence/Neg01/LLComparison_ChiSqCutsRVR_MGSampleNeg01_SingleGausTF_10000Evts_WideRange_NoLowPtEvts_Cut30_ApplyCosThetaReweighting_AccNormApplied_2526Evts.pdf}
 \caption{Influence of the $\pT$-cut on the $\loglik$ distribution for increasing $\pT$-cut value (0 $\GeV$, 15 $\GeV$ and 30 $\GeV$) for a MG sample created with $\VR$ = -0.1 with acceptance normalisation and $\csTh$ reweighting applied.}
 \label{fig::AccNormCosNeg01}
\end{figure}

\begin{figure}[h!t]
 \centering
 \includegraphics[width = 0.32 \textwidth]{/home/annik/Documents/Vub/PhD/ThesisSubjects/AnomalousCouplings/June2015_PtCutInfluence/Neg02/LLComparison_ChiSqCutsRVR_MGSampleNeg02_SingleGausTF_10000Evts_WideRange_10000Evts.pdf}
 \includegraphics[width = 0.32 \textwidth]{/home/annik/Documents/Vub/PhD/ThesisSubjects/AnomalousCouplings/June2015_PtCutInfluence/Neg02/LLComparison_ChiSqCutsRVR_MGSampleNeg02_SingleGausTF_10000Evts_WideRange_NoLowPtEvts_Cut15_ApplyCosThetaReweighting_AccNormApplied_7506Evts.pdf}
 \includegraphics[width = 0.32 \textwidth]{/home/annik/Documents/Vub/PhD/ThesisSubjects/AnomalousCouplings/June2015_PtCutInfluence/Neg02/LLComparison_ChiSqCutsRVR_MGSampleNeg02_SingleGausTF_10000Evts_WideRange_NoLowPtEvts_Cut30_ApplyCosThetaReweighting_AccNormApplied_2548Evts.pdf}
 \caption{Influence of the $\pT$-cut on the $\loglik$ distribution for increasing $\pT$-cut value (0 $\GeV$, 15 $\GeV$ and 30 $\GeV$) for a MG sample created with $\VR$ = -0.2 with acceptance normalisation and $\csTh$ reweighting applied.}
 \label{fig::AccNormCosNeg02}
\end{figure}

\begin{figure}[h!t]
 \centering
 \includegraphics[width = 0.32 \textwidth]{/home/annik/Documents/Vub/PhD/ThesisSubjects/AnomalousCouplings/June2015_PtCutInfluence/Neg03/LLComparison_ChiSqCutsRVR_MGSampleNeg03_SingleGausTF_10000Evts_WideRange_10000Evts.pdf}
 \includegraphics[width = 0.32 \textwidth]{/home/annik/Documents/Vub/PhD/ThesisSubjects/AnomalousCouplings/June2015_PtCutInfluence/Neg03/LLComparison_ChiSqCutsRVR_MGSampleNeg03_SingleGausTF_10000Evts_WideRange_NoLowPtEvts_Cut15_ApplyCosThetaReweighting_AccNormApplied_7519Evts.pdf}
 \includegraphics[width = 0.32 \textwidth]{/home/annik/Documents/Vub/PhD/ThesisSubjects/AnomalousCouplings/June2015_PtCutInfluence/Neg03/LLComparison_ChiSqCutsRVR_MGSampleNeg03_SingleGausTF_10000Evts_WideRange_NoLowPtEvts_Cut30_ApplyCosThetaReweighting_AccNormApplied_2625Evts.pdf}
 \caption{Influence of the $\pT$-cut on the $\loglik$ distribution for increasing $\pT$-cut value (0 $\GeV$, 15 $\GeV$ and 30 $\GeV$) for a MG sample created with $\VR$ = -0.3 with acceptance normalisation and $\csTh$ reweighting applied.}
 \label{fig::AccNormCosNeg03}
\end{figure}

\begin{figure}[h!t]
 \centering
 \includegraphics[width = 0.32 \textwidth]{/home/annik/Documents/Vub/PhD/ThesisSubjects/AnomalousCouplings/June2015_PtCutInfluence/Neg05/LLComparison_ChiSqCutsRVR_MGSampleNeg05_SingleGausTF_10000Evts_WideRange_10000Evts.pdf}
 \includegraphics[width = 0.32 \textwidth]{/home/annik/Documents/Vub/PhD/ThesisSubjects/AnomalousCouplings/June2015_PtCutInfluence/Neg05/LLComparison_ChiSqCutsRVR_MGSampleNeg05_SingleGausTF_10000Evts_WideRange_NoLowPtEvts_Cut15_ApplyCosThetaReweighting_AccNormApplied_7488Evts.pdf}
 \includegraphics[width = 0.32 \textwidth]{/home/annik/Documents/Vub/PhD/ThesisSubjects/AnomalousCouplings/June2015_PtCutInfluence/Neg05/LLComparison_ChiSqCutsRVR_MGSampleNeg05_SingleGausTF_10000Evts_WideRange_NoLowPtEvts_Cut30_ApplyCosThetaReweighting_AccNormApplied_2752Evts.pdf}
 \caption{Influence of the $\pT$-cut on the $\loglik$ distribution for increasing $\pT$-cut value (0 $\GeV$, 15 $\GeV$ and 30 $\GeV$) for a MG sample created with $\VR$ = -0.5 with acceptance normalisation and $\csTh$ reweighting applied.}
 \label{fig::AccNormCosNeg05}
\end{figure}

\section{Adapting the implementation used}
In the previous sections the $\csTh$ reweighting has not been applied in a correct way. For these distributions the calculated weight was only multiplied with the MadWeight probability and the cross-section normalisation was not taken into account.
Since this weight has to be applied on an event-by-event basis it should be multiplied with the term $\frac{P_{MW}}{\sigma}$ \textbf{before} the logarithm is taken.
Hence the way the reweighting is currently implemented is the following:

\begin{equation}
 -\ln \mathcal{L} = - \ln P_{MW} * w_{i,\csTh} + \ln \sigma_{XS/Acc} * w_{i,\csTh}
\end{equation}

And this is updated to:
\begin{equation}
 -\ln \mathcal{L} = - \ln (P_{MW} * w_{i,\csTh}) + \ln \sigma_{XS/Acc}
\end{equation}

The only results which differ with this altered definition are the one in the following cases:
\begin{itemize}
 \item $\csTh$ reweighting applied but no acceptance normalisation applied
 \item Both $\csTh$ reweighting and acceptance normalisation applied
\end{itemize}
For these two the corresponding histograms will be listed again and should be compared with the previously obtained ones:

\begin{figure}[h!t]
 \centering
 \includegraphics[width = 0.32 \textwidth]{/home/annik/Documents/Vub/PhD/ThesisSubjects/AnomalousCouplings/June2015_PtCutInfluence/Pos05/LLComparison_ChiSqCutsRVR_MGSamplePos05_SingleGausTF_10000Evts_WideRange_10000Evts.pdf}
 \includegraphics[width = 0.32 \textwidth]{/home/annik/Documents/Vub/PhD/ThesisSubjects/AnomalousCouplings/June2015_PtCutInfluence/Pos05/LLComparison_ChiSqCutsRVR_MGSamplePos05_SingleGausTF_10000Evts_WideRange_NoLowPtEvts_Cut15_CosThetaMultipliedWithWeight_7572Evts.pdf}
 \includegraphics[width = 0.32 \textwidth]{/home/annik/Documents/Vub/PhD/ThesisSubjects/AnomalousCouplings/June2015_PtCutInfluence/Pos05/LLComparison_ChiSqCutsRVR_MGSamplePos05_SingleGausTF_10000Evts_WideRange_NoLowPtEvts_Cut30_CosThetaMultipliedWithWeight_2689Evts.pdf}
 \caption{Influence of the $\pT$-cut on the $\loglik$ distribution for increasing $\pT$-cut value (0 $\GeV$, 15 $\GeV$ and 30 $\GeV$) for a MG sample created with $\VR$ = 0.5 with $\csTh$ reweighting applied (updated implementation).}
 \label{fig::CosThetaPos05Update}
\end{figure}

\begin{figure}[h!t]
 \centering
 \includegraphics[width = 0.32 \textwidth]{/home/annik/Documents/Vub/PhD/ThesisSubjects/AnomalousCouplings/June2015_PtCutInfluence/Pos03/LLComparison_ChiSqCutsRVR_MGSamplePos03_SingleGausTF_10000Evts_WideRange_10000Evts.pdf}
 \includegraphics[width = 0.32 \textwidth]{/home/annik/Documents/Vub/PhD/ThesisSubjects/AnomalousCouplings/June2015_PtCutInfluence/Pos03/LLComparison_ChiSqCutsRVR_MGSamplePos03_SingleGausTF_10000Evts_WideRange_NoLowPtEvts_Cut15_CosThetaMultipliedWithWeight_7539Evts.pdf}
 \includegraphics[width = 0.32 \textwidth]{/home/annik/Documents/Vub/PhD/ThesisSubjects/AnomalousCouplings/June2015_PtCutInfluence/Pos03/LLComparison_ChiSqCutsRVR_MGSamplePos03_SingleGausTF_10000Evts_WideRange_NoLowPtEvts_Cut30_CosThetaMultipliedWithWeight_2563Evts.pdf}
 \caption{Influence of the $\pT$-cut on the $\loglik$ distribution for increasing $\pT$-cut value ($0$ $\GeV$, $15$ $\GeV$ and $30$ $\GeV$) for a MG sample created with $\VR$ $=$ $0.3$ with $\csTh$ reweighting applied (updated implementation).}
 \label{fig::CosThetaPos03Update}
\end{figure}

\begin{figure}[h!t]
 \centering
 \includegraphics[width = 0.32 \textwidth]{/home/annik/Documents/Vub/PhD/ThesisSubjects/AnomalousCouplings/June2015_PtCutInfluence/Pos02/LLComparison_ChiSqCutsRVR_MGSamplePos02_SingleGausTF_10000Evts_WideRange_10000Evts.pdf}
 \includegraphics[width = 0.32 \textwidth]{/home/annik/Documents/Vub/PhD/ThesisSubjects/AnomalousCouplings/June2015_PtCutInfluence/Pos02/LLComparison_ChiSqCutsRVR_MGSamplePos02_SingleGausTF_10000Evts_WideRange_NoLowPtEvts_Cut15_CosThetaMultipliedWithWeight_7405Evts.pdf}
 \includegraphics[width = 0.32 \textwidth]{/home/annik/Documents/Vub/PhD/ThesisSubjects/AnomalousCouplings/June2015_PtCutInfluence/Pos02/LLComparison_ChiSqCutsRVR_MGSamplePos02_SingleGausTF_10000Evts_WideRange_NoLowPtEvts_Cut30_CosThetaMultipliedWithWeight_2526Evts.pdf}
 \caption{Influence of the $\pT$-cut on the $\loglik$ distribution for increasing $\pT$-cut value (0 $\GeV$, 15 $\GeV$ and 30 $\GeV$) for a MG sample created with $\VR$ = 0.2 with $\csTh$ reweighting applied (updated implementation).}
 \label{fig::CosThetaPos02Update}
\end{figure}

\begin{figure}[h!t]
 \centering
 \includegraphics[width = 0.32 \textwidth]{/home/annik/Documents/Vub/PhD/ThesisSubjects/AnomalousCouplings/June2015_PtCutInfluence/Pos01/LLComparison_ChiSqCutsRVR_MGSamplePos01_SingleGausTF_10000Evts_WideRange_10000Evts.pdf}
 \includegraphics[width = 0.32 \textwidth]{/home/annik/Documents/Vub/PhD/ThesisSubjects/AnomalousCouplings/June2015_PtCutInfluence/Pos01/LLComparison_ChiSqCutsRVR_MGSamplePos01_SingleGausTF_10000Evts_WideRange_NoLowPtEvts_Cut15_CosThetaMultipliedWithWeight_7512Evts.pdf}
 \includegraphics[width = 0.32 \textwidth]{/home/annik/Documents/Vub/PhD/ThesisSubjects/AnomalousCouplings/June2015_PtCutInfluence/Pos01/LLComparison_ChiSqCutsRVR_MGSamplePos01_SingleGausTF_10000Evts_WideRange_NoLowPtEvts_Cut30_CosThetaMultipliedWithWeight_2664Evts.pdf}
 \caption{Influence of the $\pT$-cut on the $\loglik$ distribution for increasing $\pT$-cut value (0 $\GeV$, 15 $\GeV$ and 30 $\GeV$) for a MG sample created with $\VR$ = 0.1 with $\csTh$ reweighting applied (updated implementation).}
 \label{fig::CosThetaPos01Update}
\end{figure}

\begin{figure}[h!t]
 \centering
 \includegraphics[width = 0.32 \textwidth]{/home/annik/Documents/Vub/PhD/ThesisSubjects/AnomalousCouplings/June2015_PtCutInfluence/SM/LLComparison_ChiSqCutsRVR_MGSampleSMNew_SingleGausTF_10000Evts_WideRange_10000Evts.pdf}
 \includegraphics[width = 0.32 \textwidth]{/home/annik/Documents/Vub/PhD/ThesisSubjects/AnomalousCouplings/June2015_PtCutInfluence/SM/LLComparison_ChiSqCutsRVR_MGSampleSMNew_SingleGausTF_10000Evts_WideRange_NoLowPtEvts_Cut15_CosThetaMultipliedWithWeight_7425Evts.pdf}
 \includegraphics[width = 0.32 \textwidth]{/home/annik/Documents/Vub/PhD/ThesisSubjects/AnomalousCouplings/June2015_PtCutInfluence/SM/LLComparison_ChiSqCutsRVR_MGSampleSMNew_SingleGausTF_10000Evts_WideRange_NoLowPtEvts_Cut30_CosThetaMultipliedWithWeight_2542Evts.pdf}
 \caption{Influence of the $\pT$-cut on the $\loglik$ distribution for increasing $\pT$-cut value (0 $\GeV$, 15 $\GeV$ and 30 $\GeV$) for a MG sample created with $\VR$ = 0.0 with $\csTh$ reweighting applied (updated implementation).}
 \label{fig::CosThetaSMUpdate}
\end{figure}

\begin{figure}[h!t]
 \centering
 \includegraphics[width = 0.32 \textwidth]{/home/annik/Documents/Vub/PhD/ThesisSubjects/AnomalousCouplings/June2015_PtCutInfluence/Neg01/LLComparison_ChiSqCutsRVR_MGSampleNeg01_SingleGausTF_10000Evts_WideRange_10000Evts.pdf}
 \includegraphics[width = 0.32 \textwidth]{/home/annik/Documents/Vub/PhD/ThesisSubjects/AnomalousCouplings/June2015_PtCutInfluence/Neg01/LLComparison_ChiSqCutsRVR_MGSampleNeg01_SingleGausTF_10000Evts_WideRange_NoLowPtEvts_Cut15_CosThetaMultipliedWithWeight_7461Evts.pdf}
 \includegraphics[width = 0.32 \textwidth]{/home/annik/Documents/Vub/PhD/ThesisSubjects/AnomalousCouplings/June2015_PtCutInfluence/Neg01/LLComparison_ChiSqCutsRVR_MGSampleNeg01_SingleGausTF_10000Evts_WideRange_NoLowPtEvts_Cut30_CosThetaMultipliedWithWeight_2526Evts.pdf}
 \caption{Influence of the $\pT$-cut on the $\loglik$ distribution for increasing $\pT$-cut value (0 $\GeV$, 15 $\GeV$ and 30 $\GeV$) for a MG sample created with $\VR$ = -0.1 with $\csTh$ reweighting applied (updated implementation).}
 \label{fig::CosThetaNeg01Update}
\end{figure}

\begin{figure}[h!t]
 \centering
 \includegraphics[width = 0.32 \textwidth]{/home/annik/Documents/Vub/PhD/ThesisSubjects/AnomalousCouplings/June2015_PtCutInfluence/Neg02/LLComparison_ChiSqCutsRVR_MGSampleNeg02_SingleGausTF_10000Evts_WideRange_10000Evts.pdf}
 \includegraphics[width = 0.32 \textwidth]{/home/annik/Documents/Vub/PhD/ThesisSubjects/AnomalousCouplings/June2015_PtCutInfluence/Neg02/LLComparison_ChiSqCutsRVR_MGSampleNeg02_SingleGausTF_10000Evts_WideRange_NoLowPtEvts_Cut15_CosThetaMultipliedWithWeight_7506Evts.pdf}
 \includegraphics[width = 0.32 \textwidth]{/home/annik/Documents/Vub/PhD/ThesisSubjects/AnomalousCouplings/June2015_PtCutInfluence/Neg02/LLComparison_ChiSqCutsRVR_MGSampleNeg02_SingleGausTF_10000Evts_WideRange_NoLowPtEvts_Cut30_CosThetaMultipliedWithWeight_2548Evts.pdf}
 \caption{Influence of the $\pT$-cut on the $\loglik$ distribution for increasing $\pT$-cut value (0 $\GeV$, 15 $\GeV$ and 30 $\GeV$) for a MG sample created with $\VR$ = -0.2 with $\csTh$ reweighting applied (updated implementation).}
 \label{fig::CosThetaNeg02Update}
\end{figure}

\begin{figure}[h!t]
 \centering
 \includegraphics[width = 0.32 \textwidth]{/home/annik/Documents/Vub/PhD/ThesisSubjects/AnomalousCouplings/June2015_PtCutInfluence/Neg03/LLComparison_ChiSqCutsRVR_MGSampleNeg03_SingleGausTF_10000Evts_WideRange_10000Evts.pdf}
 \includegraphics[width = 0.32 \textwidth]{/home/annik/Documents/Vub/PhD/ThesisSubjects/AnomalousCouplings/June2015_PtCutInfluence/Neg03/LLComparison_ChiSqCutsRVR_MGSampleNeg03_SingleGausTF_10000Evts_WideRange_NoLowPtEvts_Cut15_CosThetaMultipliedWithWeight_7519Evts.pdf}
 \includegraphics[width = 0.32 \textwidth]{/home/annik/Documents/Vub/PhD/ThesisSubjects/AnomalousCouplings/June2015_PtCutInfluence/Neg03/LLComparison_ChiSqCutsRVR_MGSampleNeg03_SingleGausTF_10000Evts_WideRange_NoLowPtEvts_Cut30_CosThetaMultipliedWithWeight_2625Evts.pdf}
 \caption{Influence of the $\pT$-cut on the $\loglik$ distribution for increasing $\pT$-cut value (0 $\GeV$, 15 $\GeV$ and 30 $\GeV$) for a MG sample created with $\VR$ = -0.3 with $\csTh$ reweighting applied (updated implementation).}
 \label{fig::CosThetaNeg03Update}
\end{figure}

\begin{figure}[h!t]
 \centering
 \includegraphics[width = 0.32 \textwidth]{/home/annik/Documents/Vub/PhD/ThesisSubjects/AnomalousCouplings/June2015_PtCutInfluence/Neg05/LLComparison_ChiSqCutsRVR_MGSampleNeg05_SingleGausTF_10000Evts_WideRange_10000Evts.pdf}
 \includegraphics[width = 0.32 \textwidth]{/home/annik/Documents/Vub/PhD/ThesisSubjects/AnomalousCouplings/June2015_PtCutInfluence/Neg05/LLComparison_ChiSqCutsRVR_MGSampleNeg05_SingleGausTF_10000Evts_WideRange_NoLowPtEvts_Cut15_CosThetaMultipliedWithWeight_7488Evts.pdf}
 \includegraphics[width = 0.32 \textwidth]{/home/annik/Documents/Vub/PhD/ThesisSubjects/AnomalousCouplings/June2015_PtCutInfluence/Neg05/LLComparison_ChiSqCutsRVR_MGSampleNeg05_SingleGausTF_10000Evts_WideRange_NoLowPtEvts_Cut30_CosThetaMultipliedWithWeight_2752Evts.pdf}
 \caption{Influence of the $\pT$-cut on the $\loglik$ distribution for increasing $\pT$-cut value (0 $\GeV$, 15 $\GeV$ and 30 $\GeV$) for a MG sample created with $\VR$ = -0.5 with $\csTh$ reweighting applied (updated implementation).}
 \label{fig::CosThetaNeg05Update}
\end{figure}
 
A direct comparison between the two definitions used can be found in Figure~\ref{fig::CosThInflBothDefs}.
\begin{figure}[h!t]
 \centering
 %Pos05
 \includegraphics[width = 0.24 \textwidth]{/home/annik/Documents/Vub/PhD/ThesisSubjects/AnomalousCouplings/June2015_PtCutInfluence/Pos05/LLComparison_ChiSqCutsRVR_MGSamplePos05_SingleGausTF_10000Evts_WideRange_NoLowPtEvts_Cut15_ApplyCosThetaReweighting_7572Evts.pdf}
 \includegraphics[width = 0.24 \textwidth]{/home/annik/Documents/Vub/PhD/ThesisSubjects/AnomalousCouplings/June2015_PtCutInfluence/Pos05/LLComparison_ChiSqCutsRVR_MGSamplePos05_SingleGausTF_10000Evts_WideRange_NoLowPtEvts_Cut30_ApplyCosThetaReweighting_2689Evts.pdf}
 \includegraphics[width = 0.24 \textwidth]{/home/annik/Documents/Vub/PhD/ThesisSubjects/AnomalousCouplings/June2015_PtCutInfluence/Pos05/LLComparison_ChiSqCutsRVR_MGSamplePos05_SingleGausTF_10000Evts_WideRange_NoLowPtEvts_Cut15_CosThetaMultipliedWithWeight_7572Evts.pdf}
 \includegraphics[width = 0.24 \textwidth]{/home/annik/Documents/Vub/PhD/ThesisSubjects/AnomalousCouplings/June2015_PtCutInfluence/Pos05/LLComparison_ChiSqCutsRVR_MGSamplePos05_SingleGausTF_10000Evts_WideRange_NoLowPtEvts_Cut30_CosThetaMultipliedWithWeight_2689Evts.pdf}
 %Pos02
 \includegraphics[width = 0.24 \textwidth]{/home/annik/Documents/Vub/PhD/ThesisSubjects/AnomalousCouplings/June2015_PtCutInfluence/Pos02/LLComparison_ChiSqCutsRVR_MGSamplePos02_SingleGausTF_10000Evts_WideRange_NoLowPtEvts_Cut15_ApplyCosThetaReweighting_7405Evts.pdf}
 \includegraphics[width = 0.24 \textwidth]{/home/annik/Documents/Vub/PhD/ThesisSubjects/AnomalousCouplings/June2015_PtCutInfluence/Pos02/LLComparison_ChiSqCutsRVR_MGSamplePos02_SingleGausTF_10000Evts_WideRange_NoLowPtEvts_Cut30_ApplyCosThetaReweighting_2526Evts.pdf}
 \includegraphics[width = 0.24 \textwidth]{/home/annik/Documents/Vub/PhD/ThesisSubjects/AnomalousCouplings/June2015_PtCutInfluence/Pos02/LLComparison_ChiSqCutsRVR_MGSamplePos02_SingleGausTF_10000Evts_WideRange_NoLowPtEvts_Cut15_CosThetaMultipliedWithWeight_7405Evts.pdf}
 \includegraphics[width = 0.24 \textwidth]{/home/annik/Documents/Vub/PhD/ThesisSubjects/AnomalousCouplings/June2015_PtCutInfluence/Pos02/LLComparison_ChiSqCutsRVR_MGSamplePos02_SingleGausTF_10000Evts_WideRange_NoLowPtEvts_Cut30_CosThetaMultipliedWithWeight_2526Evts.pdf}
 %Standard Model
 \includegraphics[width = 0.24 \textwidth]{/home/annik/Documents/Vub/PhD/ThesisSubjects/AnomalousCouplings/June2015_PtCutInfluence/SM/LLComparison_ChiSqCutsRVR_MGSampleSMNew_SingleGausTF_10000Evts_WideRange_NoLowPtEvts_Cut15_ApplyCosThetaReweighting_7425Evts.pdf}
 \includegraphics[width = 0.24 \textwidth]{/home/annik/Documents/Vub/PhD/ThesisSubjects/AnomalousCouplings/June2015_PtCutInfluence/SM/LLComparison_ChiSqCutsRVR_MGSampleSMNew_SingleGausTF_10000Evts_WideRange_NoLowPtEvts_Cut30_ApplyCosThetaReweighting_2542Evts.pdf}
 \includegraphics[width = 0.24 \textwidth]{/home/annik/Documents/Vub/PhD/ThesisSubjects/AnomalousCouplings/June2015_PtCutInfluence/SM/LLComparison_ChiSqCutsRVR_MGSampleSMNew_SingleGausTF_10000Evts_WideRange_NoLowPtEvts_Cut15_CosThetaMultipliedWithWeight_7425Evts.pdf}
 \includegraphics[width = 0.24 \textwidth]{/home/annik/Documents/Vub/PhD/ThesisSubjects/AnomalousCouplings/June2015_PtCutInfluence/SM/LLComparison_ChiSqCutsRVR_MGSampleSMNew_SingleGausTF_10000Evts_WideRange_NoLowPtEvts_Cut30_CosThetaMultipliedWithWeight_2542Evts.pdf}
 %Neg02
 \includegraphics[width = 0.24 \textwidth]{/home/annik/Documents/Vub/PhD/ThesisSubjects/AnomalousCouplings/June2015_PtCutInfluence/Neg02/LLComparison_ChiSqCutsRVR_MGSampleNeg02_SingleGausTF_10000Evts_WideRange_NoLowPtEvts_Cut15_ApplyCosThetaReweighting_7506Evts.pdf}
 \includegraphics[width = 0.24 \textwidth]{/home/annik/Documents/Vub/PhD/ThesisSubjects/AnomalousCouplings/June2015_PtCutInfluence/Neg02/LLComparison_ChiSqCutsRVR_MGSampleNeg02_SingleGausTF_10000Evts_WideRange_NoLowPtEvts_Cut30_ApplyCosThetaReweighting_2548Evts.pdf}
 \includegraphics[width = 0.24 \textwidth]{/home/annik/Documents/Vub/PhD/ThesisSubjects/AnomalousCouplings/June2015_PtCutInfluence/Neg02/LLComparison_ChiSqCutsRVR_MGSampleNeg02_SingleGausTF_10000Evts_WideRange_NoLowPtEvts_Cut15_CosThetaMultipliedWithWeight_7506Evts.pdf}
 \includegraphics[width = 0.24 \textwidth]{/home/annik/Documents/Vub/PhD/ThesisSubjects/AnomalousCouplings/June2015_PtCutInfluence/Neg02/LLComparison_ChiSqCutsRVR_MGSampleNeg02_SingleGausTF_10000Evts_WideRange_NoLowPtEvts_Cut30_CosThetaMultipliedWithWeight_2548Evts.pdf}
 %Neg05
 \includegraphics[width = 0.24 \textwidth]{/home/annik/Documents/Vub/PhD/ThesisSubjects/AnomalousCouplings/June2015_PtCutInfluence/Neg05/LLComparison_ChiSqCutsRVR_MGSampleNeg05_SingleGausTF_10000Evts_WideRange_NoLowPtEvts_Cut15_ApplyCosThetaReweighting_7488Evts.pdf}
 \includegraphics[width = 0.24 \textwidth]{/home/annik/Documents/Vub/PhD/ThesisSubjects/AnomalousCouplings/June2015_PtCutInfluence/Neg05/LLComparison_ChiSqCutsRVR_MGSampleNeg05_SingleGausTF_10000Evts_WideRange_NoLowPtEvts_Cut30_ApplyCosThetaReweighting_2752Evts.pdf}
 \includegraphics[width = 0.24 \textwidth]{/home/annik/Documents/Vub/PhD/ThesisSubjects/AnomalousCouplings/June2015_PtCutInfluence/Neg05/LLComparison_ChiSqCutsRVR_MGSampleNeg05_SingleGausTF_10000Evts_WideRange_NoLowPtEvts_Cut15_CosThetaMultipliedWithWeight_7488Evts.pdf}
 \includegraphics[width = 0.24 \textwidth]{/home/annik/Documents/Vub/PhD/ThesisSubjects/AnomalousCouplings/June2015_PtCutInfluence/Neg05/LLComparison_ChiSqCutsRVR_MGSampleNeg05_SingleGausTF_10000Evts_WideRange_NoLowPtEvts_Cut30_CosThetaMultipliedWithWeight_2752Evts.pdf}
 \caption{Influence of combining the $\csTh$ reweighting with the MadWeight probability on an event-by-event basis. From left to right the different plots always show the same distributions: first the 15 $\GeV$ case and 30 $\GeV$ case with the first $\csTh$ reweighting implemented and then with the updated definition. From top to bottom the different $\VR$ configurations which have been shown are $-0.5$, $-0.2$, $0.0$, $0.2$ and $0.5$.}
 \label{fig::CosThInflBothDefs}
 \end{figure}

Now the same will be shown when both the $\csTh$ reweighting and the acceptance normalisation have been applied. This is given in Figures~\ref{fig::AccNormCosPos05Update} to \ref{fig::AccNormCosNeg05Update}. Afterwards the comparison between the two considered $\csTh$ reweighting implementations is given in Figure~\ref{fig::CosThAccNormBothDefs}.

\begin{figure}[h!t]
 \centering
 \includegraphics[width = 0.32 \textwidth]{/home/annik/Documents/Vub/PhD/ThesisSubjects/AnomalousCouplings/June2015_PtCutInfluence/Pos05/LLComparison_ChiSqCutsRVR_MGSamplePos05_SingleGausTF_10000Evts_WideRange_10000Evts.pdf}
 \includegraphics[width = 0.32 \textwidth]{/home/annik/Documents/Vub/PhD/ThesisSubjects/AnomalousCouplings/June2015_PtCutInfluence/Pos05/LLComparison_ChiSqCutsRVR_MGSamplePos05_SingleGausTF_10000Evts_WideRange_NoLowPtEvts_Cut15_CosThetaMultipliedWithWeight_AccNormApplied_7572Evts.pdf}
 \includegraphics[width = 0.32 \textwidth]{/home/annik/Documents/Vub/PhD/ThesisSubjects/AnomalousCouplings/June2015_PtCutInfluence/Pos05/LLComparison_ChiSqCutsRVR_MGSamplePos05_SingleGausTF_10000Evts_WideRange_NoLowPtEvts_Cut30_CosThetaMultipliedWithWeight_AccNormApplied_2689Evts.pdf}
 \caption{Influence of the $\pT$-cut on the $\loglik$ distribution for increasing $\pT$-cut value (0 $\GeV$, 15 $\GeV$ and 30 $\GeV$) for a MG sample created with $\VR$ = 0.5 with $\csTh$ reweighting and acceptance normalisation applied (updated implementation).}
 \label{fig::AccNormCosPos05Update}
\end{figure}

\begin{figure}[h!t]
 \centering
 \includegraphics[width = 0.32 \textwidth]{/home/annik/Documents/Vub/PhD/ThesisSubjects/AnomalousCouplings/June2015_PtCutInfluence/Pos03/LLComparison_ChiSqCutsRVR_MGSamplePos03_SingleGausTF_10000Evts_WideRange_10000Evts.pdf}
 \includegraphics[width = 0.32 \textwidth]{/home/annik/Documents/Vub/PhD/ThesisSubjects/AnomalousCouplings/June2015_PtCutInfluence/Pos03/LLComparison_ChiSqCutsRVR_MGSamplePos03_SingleGausTF_10000Evts_WideRange_NoLowPtEvts_Cut15_CosThetaMultipliedWithWeight_AccNormApplied_7539Evts.pdf}
 \includegraphics[width = 0.32 \textwidth]{/home/annik/Documents/Vub/PhD/ThesisSubjects/AnomalousCouplings/June2015_PtCutInfluence/Pos03/LLComparison_ChiSqCutsRVR_MGSamplePos03_SingleGausTF_10000Evts_WideRange_NoLowPtEvts_Cut30_CosThetaMultipliedWithWeight_AccNormApplied_2563Evts.pdf}
 \caption{Influence of the $\pT$-cut on the $\loglik$ distribution for increasing $\pT$-cut value ($0$ $\GeV$, $15$ $\GeV$ and $30$ $\GeV$) for a MG sample created with $\VR$ $=$ $0.3$ with $\csTh$ reweighting and acceptance normalisation applied (updated implementation).}
 \label{fig::AccNormCosPos03Update}
\end{figure}

\begin{figure}[h!t]
 \centering
 \includegraphics[width = 0.32 \textwidth]{/home/annik/Documents/Vub/PhD/ThesisSubjects/AnomalousCouplings/June2015_PtCutInfluence/Pos02/LLComparison_ChiSqCutsRVR_MGSamplePos02_SingleGausTF_10000Evts_WideRange_10000Evts.pdf}
 \includegraphics[width = 0.32 \textwidth]{/home/annik/Documents/Vub/PhD/ThesisSubjects/AnomalousCouplings/June2015_PtCutInfluence/Pos02/LLComparison_ChiSqCutsRVR_MGSamplePos02_SingleGausTF_10000Evts_WideRange_NoLowPtEvts_Cut15_CosThetaMultipliedWithWeight_AccNormApplied_7405Evts.pdf}
 \includegraphics[width = 0.32 \textwidth]{/home/annik/Documents/Vub/PhD/ThesisSubjects/AnomalousCouplings/June2015_PtCutInfluence/Pos02/LLComparison_ChiSqCutsRVR_MGSamplePos02_SingleGausTF_10000Evts_WideRange_NoLowPtEvts_Cut30_CosThetaMultipliedWithWeight_AccNormApplied_2526Evts.pdf}
 \caption{Influence of the $\pT$-cut on the $\loglik$ distribution for increasing $\pT$-cut value (0 $\GeV$, 15 $\GeV$ and 30 $\GeV$) for a MG sample created with $\VR$ = 0.2 with $\csTh$ reweighting and acceptance normalisation applied (updated implementation).}
 \label{fig::AccNormCosPos02Update}
\end{figure}

\begin{figure}[h!t]
 \centering
 \includegraphics[width = 0.32 \textwidth]{/home/annik/Documents/Vub/PhD/ThesisSubjects/AnomalousCouplings/June2015_PtCutInfluence/Pos01/LLComparison_ChiSqCutsRVR_MGSamplePos01_SingleGausTF_10000Evts_WideRange_10000Evts.pdf}
 \includegraphics[width = 0.32 \textwidth]{/home/annik/Documents/Vub/PhD/ThesisSubjects/AnomalousCouplings/June2015_PtCutInfluence/Pos01/LLComparison_ChiSqCutsRVR_MGSamplePos01_SingleGausTF_10000Evts_WideRange_NoLowPtEvts_Cut15_CosThetaMultipliedWithWeight_AccNormApplied_7512Evts.pdf}
 \includegraphics[width = 0.32 \textwidth]{/home/annik/Documents/Vub/PhD/ThesisSubjects/AnomalousCouplings/June2015_PtCutInfluence/Pos01/LLComparison_ChiSqCutsRVR_MGSamplePos01_SingleGausTF_10000Evts_WideRange_NoLowPtEvts_Cut30_CosThetaMultipliedWithWeight_AccNormApplied_2664Evts.pdf}
 \caption{Influence of the $\pT$-cut on the $\loglik$ distribution for increasing $\pT$-cut value (0 $\GeV$, 15 $\GeV$ and 30 $\GeV$) for a MG sample created with $\VR$ = 0.1 with $\csTh$ reweighting and acceptance normalisation applied (updated implementation).}
 \label{fig::AccNormCosPos01Update}
\end{figure}

\begin{figure}[h!t]
 \centering
 \includegraphics[width = 0.32 \textwidth]{/home/annik/Documents/Vub/PhD/ThesisSubjects/AnomalousCouplings/June2015_PtCutInfluence/SM/LLComparison_ChiSqCutsRVR_MGSampleSMNew_SingleGausTF_10000Evts_WideRange_10000Evts.pdf}
 \includegraphics[width = 0.32 \textwidth]{/home/annik/Documents/Vub/PhD/ThesisSubjects/AnomalousCouplings/June2015_PtCutInfluence/SM/LLComparison_ChiSqCutsRVR_MGSampleSMNew_SingleGausTF_10000Evts_WideRange_NoLowPtEvts_Cut15_CosThetaMultipliedWithWeight_AccNormApplied_7425Evts.pdf}
 \includegraphics[width = 0.32 \textwidth]{/home/annik/Documents/Vub/PhD/ThesisSubjects/AnomalousCouplings/June2015_PtCutInfluence/SM/LLComparison_ChiSqCutsRVR_MGSampleSMNew_SingleGausTF_10000Evts_WideRange_NoLowPtEvts_Cut30_CosThetaMultipliedWithWeight_AccNormApplied_2542Evts.pdf}
 \caption{Influence of the $\pT$-cut on the $\loglik$ distribution for increasing $\pT$-cut value (0 $\GeV$, 15 $\GeV$ and 30 $\GeV$) for a MG sample created with $\VR$ = 0.0 with $\csTh$ reweighting and acceptance normalisation applied (updated implementation).}
 \label{fig::AccNormCosSMUpdate}
\end{figure}

\begin{figure}[h!t]
 \centering
 \includegraphics[width = 0.32 \textwidth]{/home/annik/Documents/Vub/PhD/ThesisSubjects/AnomalousCouplings/June2015_PtCutInfluence/Neg01/LLComparison_ChiSqCutsRVR_MGSampleNeg01_SingleGausTF_10000Evts_WideRange_10000Evts.pdf}
 \includegraphics[width = 0.32 \textwidth]{/home/annik/Documents/Vub/PhD/ThesisSubjects/AnomalousCouplings/June2015_PtCutInfluence/Neg01/LLComparison_ChiSqCutsRVR_MGSampleNeg01_SingleGausTF_10000Evts_WideRange_NoLowPtEvts_Cut15_CosThetaMultipliedWithWeight_AccNormApplied_7461Evts.pdf}
 \includegraphics[width = 0.32 \textwidth]{/home/annik/Documents/Vub/PhD/ThesisSubjects/AnomalousCouplings/June2015_PtCutInfluence/Neg01/LLComparison_ChiSqCutsRVR_MGSampleNeg01_SingleGausTF_10000Evts_WideRange_NoLowPtEvts_Cut30_CosThetaMultipliedWithWeight_AccNormApplied_2526Evts.pdf}
 \caption{Influence of the $\pT$-cut on the $\loglik$ distribution for increasing $\pT$-cut value (0 $\GeV$, 15 $\GeV$ and 30 $\GeV$) for a MG sample created with $\VR$ = -0.1 with $\csTh$ reweighting and acceptance normalisation applied (updated implementation).}
 \label{fig::AccNormCosNeg01Update}
\end{figure}

\begin{figure}[h!t]
 \centering
 \includegraphics[width = 0.32 \textwidth]{/home/annik/Documents/Vub/PhD/ThesisSubjects/AnomalousCouplings/June2015_PtCutInfluence/Neg02/LLComparison_ChiSqCutsRVR_MGSampleNeg02_SingleGausTF_10000Evts_WideRange_10000Evts.pdf}
 \includegraphics[width = 0.32 \textwidth]{/home/annik/Documents/Vub/PhD/ThesisSubjects/AnomalousCouplings/June2015_PtCutInfluence/Neg02/LLComparison_ChiSqCutsRVR_MGSampleNeg02_SingleGausTF_10000Evts_WideRange_NoLowPtEvts_Cut15_CosThetaMultipliedWithWeight_AccNormApplied_7506Evts.pdf}
 \includegraphics[width = 0.32 \textwidth]{/home/annik/Documents/Vub/PhD/ThesisSubjects/AnomalousCouplings/June2015_PtCutInfluence/Neg02/LLComparison_ChiSqCutsRVR_MGSampleNeg02_SingleGausTF_10000Evts_WideRange_NoLowPtEvts_Cut30_CosThetaMultipliedWithWeight_AccNormApplied_2548Evts.pdf}
 \caption{Influence of the $\pT$-cut on the $\loglik$ distribution for increasing $\pT$-cut value (0 $\GeV$, 15 $\GeV$ and 30 $\GeV$) for a MG sample created with $\VR$ = -0.2 with $\csTh$ reweighting and acceptance normalisation applied (updated implementation).}
 \label{fig::AccNormCosNeg02Update}
\end{figure}

\begin{figure}[h!t]
 \centering
 \includegraphics[width = 0.32 \textwidth]{/home/annik/Documents/Vub/PhD/ThesisSubjects/AnomalousCouplings/June2015_PtCutInfluence/Neg03/LLComparison_ChiSqCutsRVR_MGSampleNeg03_SingleGausTF_10000Evts_WideRange_10000Evts.pdf}
 \includegraphics[width = 0.32 \textwidth]{/home/annik/Documents/Vub/PhD/ThesisSubjects/AnomalousCouplings/June2015_PtCutInfluence/Neg03/LLComparison_ChiSqCutsRVR_MGSampleNeg03_SingleGausTF_10000Evts_WideRange_NoLowPtEvts_Cut15_CosThetaMultipliedWithWeight_AccNormApplied_7519Evts.pdf}
 \includegraphics[width = 0.32 \textwidth]{/home/annik/Documents/Vub/PhD/ThesisSubjects/AnomalousCouplings/June2015_PtCutInfluence/Neg03/LLComparison_ChiSqCutsRVR_MGSampleNeg03_SingleGausTF_10000Evts_WideRange_NoLowPtEvts_Cut30_CosThetaMultipliedWithWeight_AccNormApplied_2625Evts.pdf}
 \caption{Influence of the $\pT$-cut on the $\loglik$ distribution for increasing $\pT$-cut value (0 $\GeV$, 15 $\GeV$ and 30 $\GeV$) for a MG sample created with $\VR$ = -0.3 with $\csTh$ reweighting and acceptance normalisation applied (updated implementation).}
 \label{fig::AccNormCosNeg03Update}
\end{figure}

\begin{figure}[h!t]
 \centering
 \includegraphics[width = 0.32 \textwidth]{/home/annik/Documents/Vub/PhD/ThesisSubjects/AnomalousCouplings/June2015_PtCutInfluence/Neg05/LLComparison_ChiSqCutsRVR_MGSampleNeg05_SingleGausTF_10000Evts_WideRange_10000Evts.pdf}
 \includegraphics[width = 0.32 \textwidth]{/home/annik/Documents/Vub/PhD/ThesisSubjects/AnomalousCouplings/June2015_PtCutInfluence/Neg05/LLComparison_ChiSqCutsRVR_MGSampleNeg05_SingleGausTF_10000Evts_WideRange_NoLowPtEvts_Cut15_CosThetaMultipliedWithWeight_AccNormApplied_7488Evts.pdf}
 \includegraphics[width = 0.32 \textwidth]{/home/annik/Documents/Vub/PhD/ThesisSubjects/AnomalousCouplings/June2015_PtCutInfluence/Neg05/LLComparison_ChiSqCutsRVR_MGSampleNeg05_SingleGausTF_10000Evts_WideRange_NoLowPtEvts_Cut30_CosThetaMultipliedWithWeight_AccNormApplied_2752Evts.pdf}
 \caption{Influence of the $\pT$-cut on the $\loglik$ distribution for increasing $\pT$-cut value (0 $\GeV$, 15 $\GeV$ and 30 $\GeV$) for a MG sample created with $\VR$ = -0.5 with $\csTh$ reweighting and acceptance normalisation applied (updated implementation).}
 \label{fig::AccNormCosNeg05Update}
\end{figure}
 
\begin{figure}[h!t]
 \centering
 %Pos05
 \includegraphics[width = 0.24 \textwidth]{/home/annik/Documents/Vub/PhD/ThesisSubjects/AnomalousCouplings/June2015_PtCutInfluence/Pos05/LLComparison_ChiSqCutsRVR_MGSamplePos05_SingleGausTF_10000Evts_WideRange_NoLowPtEvts_Cut15_ApplyCosThetaReweighting_AccNormApplied_7572Evts.pdf}
 \includegraphics[width = 0.24 \textwidth]{/home/annik/Documents/Vub/PhD/ThesisSubjects/AnomalousCouplings/June2015_PtCutInfluence/Pos05/LLComparison_ChiSqCutsRVR_MGSamplePos05_SingleGausTF_10000Evts_WideRange_NoLowPtEvts_Cut30_ApplyCosThetaReweighting_AccNormApplied_2689Evts.pdf}
 \includegraphics[width = 0.24 \textwidth]{/home/annik/Documents/Vub/PhD/ThesisSubjects/AnomalousCouplings/June2015_PtCutInfluence/Pos05/LLComparison_ChiSqCutsRVR_MGSamplePos05_SingleGausTF_10000Evts_WideRange_NoLowPtEvts_Cut15_CosThetaMultipliedWithWeight_AccNormApplied_7572Evts.pdf}
 \includegraphics[width = 0.24 \textwidth]{/home/annik/Documents/Vub/PhD/ThesisSubjects/AnomalousCouplings/June2015_PtCutInfluence/Pos05/LLComparison_ChiSqCutsRVR_MGSamplePos05_SingleGausTF_10000Evts_WideRange_NoLowPtEvts_Cut30_CosThetaMultipliedWithWeight_AccNormApplied_2689Evts.pdf}
 %Pos02
 \includegraphics[width = 0.24 \textwidth]{/home/annik/Documents/Vub/PhD/ThesisSubjects/AnomalousCouplings/June2015_PtCutInfluence/Pos02/LLComparison_ChiSqCutsRVR_MGSamplePos02_SingleGausTF_10000Evts_WideRange_NoLowPtEvts_Cut15_ApplyCosThetaReweighting_AccNormApplied_7405Evts.pdf}
 \includegraphics[width = 0.24 \textwidth]{/home/annik/Documents/Vub/PhD/ThesisSubjects/AnomalousCouplings/June2015_PtCutInfluence/Pos02/LLComparison_ChiSqCutsRVR_MGSamplePos02_SingleGausTF_10000Evts_WideRange_NoLowPtEvts_Cut30_ApplyCosThetaReweighting_AccNormApplied_2526Evts.pdf}
 \includegraphics[width = 0.24 \textwidth]{/home/annik/Documents/Vub/PhD/ThesisSubjects/AnomalousCouplings/June2015_PtCutInfluence/Pos02/LLComparison_ChiSqCutsRVR_MGSamplePos02_SingleGausTF_10000Evts_WideRange_NoLowPtEvts_Cut15_CosThetaMultipliedWithWeight_AccNormApplied_7405Evts.pdf}
 \includegraphics[width = 0.24 \textwidth]{/home/annik/Documents/Vub/PhD/ThesisSubjects/AnomalousCouplings/June2015_PtCutInfluence/Pos02/LLComparison_ChiSqCutsRVR_MGSamplePos02_SingleGausTF_10000Evts_WideRange_NoLowPtEvts_Cut30_CosThetaMultipliedWithWeight_AccNormApplied_2526Evts.pdf}
 %Standard Model
 \includegraphics[width = 0.24 \textwidth]{/home/annik/Documents/Vub/PhD/ThesisSubjects/AnomalousCouplings/June2015_PtCutInfluence/SM/LLComparison_ChiSqCutsRVR_MGSampleSMNew_SingleGausTF_10000Evts_WideRange_NoLowPtEvts_Cut15_ApplyCosThetaReweighting_AccNormApplied_7425Evts.pdf}
 \includegraphics[width = 0.24 \textwidth]{/home/annik/Documents/Vub/PhD/ThesisSubjects/AnomalousCouplings/June2015_PtCutInfluence/SM/LLComparison_ChiSqCutsRVR_MGSampleSMNew_SingleGausTF_10000Evts_WideRange_NoLowPtEvts_Cut30_ApplyCosThetaReweighting_AccNormApplied_2542Evts.pdf}
 \includegraphics[width = 0.24 \textwidth]{/home/annik/Documents/Vub/PhD/ThesisSubjects/AnomalousCouplings/June2015_PtCutInfluence/SM/LLComparison_ChiSqCutsRVR_MGSampleSMNew_SingleGausTF_10000Evts_WideRange_NoLowPtEvts_Cut15_CosThetaMultipliedWithWeight_AccNormApplied_7425Evts.pdf}
 \includegraphics[width = 0.24 \textwidth]{/home/annik/Documents/Vub/PhD/ThesisSubjects/AnomalousCouplings/June2015_PtCutInfluence/SM/LLComparison_ChiSqCutsRVR_MGSampleSMNew_SingleGausTF_10000Evts_WideRange_NoLowPtEvts_Cut30_CosThetaMultipliedWithWeight_AccNormApplied_2542Evts.pdf}
 %Neg02
 \includegraphics[width = 0.24 \textwidth]{/home/annik/Documents/Vub/PhD/ThesisSubjects/AnomalousCouplings/June2015_PtCutInfluence/Neg02/LLComparison_ChiSqCutsRVR_MGSampleNeg02_SingleGausTF_10000Evts_WideRange_NoLowPtEvts_Cut15_ApplyCosThetaReweighting_AccNormApplied_7506Evts.pdf}
 \includegraphics[width = 0.24 \textwidth]{/home/annik/Documents/Vub/PhD/ThesisSubjects/AnomalousCouplings/June2015_PtCutInfluence/Neg02/LLComparison_ChiSqCutsRVR_MGSampleNeg02_SingleGausTF_10000Evts_WideRange_NoLowPtEvts_Cut30_ApplyCosThetaReweighting_AccNormApplied_2548Evts.pdf}
 \includegraphics[width = 0.24 \textwidth]{/home/annik/Documents/Vub/PhD/ThesisSubjects/AnomalousCouplings/June2015_PtCutInfluence/Neg02/LLComparison_ChiSqCutsRVR_MGSampleNeg02_SingleGausTF_10000Evts_WideRange_NoLowPtEvts_Cut15_CosThetaMultipliedWithWeight_AccNormApplied_7506Evts.pdf}
 \includegraphics[width = 0.24 \textwidth]{/home/annik/Documents/Vub/PhD/ThesisSubjects/AnomalousCouplings/June2015_PtCutInfluence/Neg02/LLComparison_ChiSqCutsRVR_MGSampleNeg02_SingleGausTF_10000Evts_WideRange_NoLowPtEvts_Cut30_CosThetaMultipliedWithWeight_AccNormApplied_2548Evts.pdf}
 %Neg05
 \includegraphics[width = 0.24 \textwidth]{/home/annik/Documents/Vub/PhD/ThesisSubjects/AnomalousCouplings/June2015_PtCutInfluence/Neg05/LLComparison_ChiSqCutsRVR_MGSampleNeg05_SingleGausTF_10000Evts_WideRange_NoLowPtEvts_Cut15_ApplyCosThetaReweighting_AccNormApplied_7488Evts.pdf}
 \includegraphics[width = 0.24 \textwidth]{/home/annik/Documents/Vub/PhD/ThesisSubjects/AnomalousCouplings/June2015_PtCutInfluence/Neg05/LLComparison_ChiSqCutsRVR_MGSampleNeg05_SingleGausTF_10000Evts_WideRange_NoLowPtEvts_Cut30_ApplyCosThetaReweighting_AccNormApplied_2752Evts.pdf}
 \includegraphics[width = 0.24 \textwidth]{/home/annik/Documents/Vub/PhD/ThesisSubjects/AnomalousCouplings/June2015_PtCutInfluence/Neg05/LLComparison_ChiSqCutsRVR_MGSampleNeg05_SingleGausTF_10000Evts_WideRange_NoLowPtEvts_Cut15_CosThetaMultipliedWithWeight_AccNormApplied_7488Evts.pdf}
 \includegraphics[width = 0.24 \textwidth]{/home/annik/Documents/Vub/PhD/ThesisSubjects/AnomalousCouplings/June2015_PtCutInfluence/Neg05/LLComparison_ChiSqCutsRVR_MGSampleNeg05_SingleGausTF_10000Evts_WideRange_NoLowPtEvts_Cut30_CosThetaMultipliedWithWeight_AccNormApplied_2752Evts.pdf}
 \caption{Influence of combining the $\csTh$ reweighting with the MadWeight probability on an event-by-event basis. From left to right the different plots always show the same distributions: first the 15 $\GeV$ case and 30 $\GeV$ case with the first $\csTh$ reweighting implemented and then with the updated definition (acceptance normalisation is applied each time). From top to bottom the different $\VR$ configurations which have been shown are $-0.5$, $-0.2$, $0.0$, $0.2$ and $0.5$.}
 \label{fig::CosThAccNormBothDefs}
 \end{figure}

It seems that this new implementation of the $\csTh$ reweighting actually has no influence at all on the $\loglik$ shape ...

\newpage
\section{Ensuring $\csTh$ normalisation}
In order to apply this $\csTh$ reweighting it has to be checked in detail whether the applied correction does not break the normalisation of the MadWeight probability\footnote{This normalisation was checked by calculating which likelihood value is obtained when the cross-section normalisation is not applied. This value should correspond to the MadWeight cross-section which is then proven to be the necessary normalisation factor needed to ensure a probability density.}.

%°°°°°°°°°°°°°°°°°°°°°°°°°°°°°°°°°°°°°°°°°°°°°°°°°°°°°°°°°°°°°°°°°°°°°°°°

%°°°°°°°°°°°°°°°°°°°°°°°°°°°°°°°°°°°°°°°°°°°°°°°°°°°°°°°°°°°°°°°°°°°°°°°°
%	CHAPTER: Effect of applying cut on FitDeviation in stead of $\chisq$
\chapter{Effect of applying cut on FitDeviation in stead of on $\chisq$}
The problem of applying a cut on the $\chisq$ of the $\fth$ order polynomial through the number of reduced points is that it is likely to not have the complete physical meaning due to the lack of uncertainties on the MadWeight probabilities. Hence the $\chisq$ calculation will need to choose an arbitrary value to use as uncertainty when calculating the $\chisq$ of the fitted function.
A possible solution can lie in using the ``FitDeviation'' variable which represents the total deviation of the measured points from the value suggested by the fit, relative to the value of the measured point.

\begin{equation}
 \textrm{FitDeviation} = \sum_{\textrm{bin}~i} \frac{\vert x_i - f(x_i) \vert}{x_i}
\end{equation}

This distribution should also be considered for ensuring that the number of excluded points is sufficient. This is the case when this distribution does not have a tail property anymore.

%°°°°°°°°°°°°°°°°°°°°°°°°°°°°°°°°°°°°°°°°°°°°°°°°°°°°°°°°°°°°°°°°°°°°°°°°

%°°°°°°°°°°°°°°°°°°°°°°°°°°°°°°°°°°°°°°°°°°°°°°°°°°°°°°°°°°°°°°°°°°°°°°°°
%	CHAPTER: Comparision between first and second polynomial fit
\chapter{Comparision between first and second polynomial fit}
Since the considered range is rather limited for the moment and the number of studied points is not much less than the number of degrees of freedom necessary to fit a decent polynomial of order 4, it should be ensured that a clear benefit is gained from applying this double-fit procedure.
Since for the second fit in this double-fit procedure 3 out of 9 points are rejected, only 6 points remain to fit a $\fth$ order polynomial which requires 5 points hence leaving only 1 single degree of freedom. This could possibly lead to problems, mainly because it is rather likely that good points are rejected and it is very difficult to check this from the individual $\loglik$ distributions.
\\

Therefore it should be checked whether the first fit could not be sufficient to obtain a smooth overall $\loglik$ distribution and whether the badly behaving events cannot be removed using the $\chisq$ or FitDeviation rejection as before.
As a first step the distributions without any influence of the $\pT$ cut will be considered, as given in Figure~\ref{fig::FirstVsSecondFitPos05} to \ref{fig::FirstVsSecondFitNeg05}. Here the sum of the first fit on the full range is given on the left while the second fit on the reduced number of events is given on the right.

\begin{figure}[h!t]
 \centering
 \includegraphics[width = 0.49 \textwidth]{/home/annik/Documents/Vub/PhD/ThesisSubjects/AnomalousCouplings/June2015_FirstOrSecondFit/Pos05/LLFirstFitComparison_ChiSqCutsRVR_MGSamplePos05_SingleGausTF_10000Evts_WideRange_10000Evts.pdf}
 \includegraphics[width = 0.49 \textwidth]{/home/annik/Documents/Vub/PhD/ThesisSubjects/AnomalousCouplings/June2015_FirstOrSecondFit/Pos05/LLSecondFitComparison_ChiSqCutsRVR_MGSamplePos05_SingleGausTF_10000Evts_WideRange_10000Evts.pdf}
 \caption{Overall $\loglik$ distribution for MG samples created with $\VR$ = 0.5 obtained from the first fit (left) and the second fit (right) prior to any $\pT$-cut requirement. The different $\chisq$ cuts applied correspond to percentages of about ....!!} 
 \label{fig::FirstVsSecondFitPos05}
\end{figure}
\begin{figure}[h!t]
 \centering
 \includegraphics[width = 0.49 \textwidth]{/home/annik/Documents/Vub/PhD/ThesisSubjects/AnomalousCouplings/June2015_FirstOrSecondFit/Pos03/LLFirstFitComparison_ChiSqCutsRVR_MGSamplePos03_SingleGausTF_10000Evts_WideRange_10000Evts.pdf}
 \includegraphics[width = 0.49 \textwidth]{/home/annik/Documents/Vub/PhD/ThesisSubjects/AnomalousCouplings/June2015_FirstOrSecondFit/Pos03/LLSecondFitComparison_ChiSqCutsRVR_MGSamplePos03_SingleGausTF_10000Evts_WideRange_10000Evts.pdf}
 \caption{Overall $\loglik$ distribution for MG samples created with $\VR$ = 0.3 obtained from the first fit (left) and the second fit (right) prior to any $\pT$-cut requirement. The different $\chisq$ cuts applied correspond to percentages of about ....!!} 
 \label{fig::FirstVsSecondFitPos03}
\end{figure}
\begin{figure}[h!t]
 \centering
 \includegraphics[width = 0.49 \textwidth]{/home/annik/Documents/Vub/PhD/ThesisSubjects/AnomalousCouplings/June2015_FirstOrSecondFit/Pos02/LLFirstFitComparison_ChiSqCutsRVR_MGSamplePos02_SingleGausTF_10000Evts_WideRange_10000Evts.pdf}
 \includegraphics[width = 0.49 \textwidth]{/home/annik/Documents/Vub/PhD/ThesisSubjects/AnomalousCouplings/June2015_FirstOrSecondFit/Pos02/LLSecondFitComparison_ChiSqCutsRVR_MGSamplePos02_SingleGausTF_10000Evts_WideRange_10000Evts.pdf}
 \caption{Overall $\loglik$ distribution for MG samples created with $\VR$ = 0.2 obtained from the first fit (left) and the second fit (right) prior to any $\pT$-cut requirement. The different $\chisq$ cuts applied correspond to percentages of about ....!!} 
 \label{fig::FirstVsSecondFitPos02}
\end{figure}
\begin{figure}[h!t]
 \centering 
 \includegraphics[width = 0.49 \textwidth]{/home/annik/Documents/Vub/PhD/ThesisSubjects/AnomalousCouplings/June2015_FirstOrSecondFit/Pos01/LLFirstFitComparison_ChiSqCutsRVR_MGSamplePos01_SingleGausTF_10000Evts_WideRange_10000Evts.pdf}
 \includegraphics[width = 0.49 \textwidth]{/home/annik/Documents/Vub/PhD/ThesisSubjects/AnomalousCouplings/June2015_FirstOrSecondFit/Pos01/LLSecondFitComparison_ChiSqCutsRVR_MGSamplePos01_SingleGausTF_10000Evts_WideRange_10000Evts.pdf}
 \caption{Overall $\loglik$ distribution for MG samples created with $\VR$ = 0.1 obtained from the first fit (left) and the second fit (right) prior to any $\pT$-cut requirement. The different $\chisq$ cuts applied correspond to percentages of about ....!!} 
 \label{fig::FirstVsSecondFitPos01}
\end{figure}
\begin{figure}[h!t]
 \centering 
 \includegraphics[width = 0.49 \textwidth]{/home/annik/Documents/Vub/PhD/ThesisSubjects/AnomalousCouplings/June2015_FirstOrSecondFit/SM/LLFirstFitComparison_ChiSqCutsRVR_MGSampleSMNew_SingleGausTF_10000Evts_WideRange_10000Evts.pdf}
 \includegraphics[width = 0.49 \textwidth]{/home/annik/Documents/Vub/PhD/ThesisSubjects/AnomalousCouplings/June2015_FirstOrSecondFit/SM/LLSecondFitComparison_ChiSqCutsRVR_MGSampleSMNew_SingleGausTF_10000Evts_WideRange_10000Evts.pdf}
 \caption{Overall $\loglik$ distribution for MG samples created with $\VR$ = 0.0 obtained from the first fit (left) and the second fit (right) prior to any $\pT$-cut requirement. The different $\chisq$ cuts applied correspond to percentages of about ....!!} 
 \label{fig::FirstVsSecondFitSM}
\end{figure}
\begin{figure}[h!t]
 \centering 
 \includegraphics[width = 0.49 \textwidth]{/home/annik/Documents/Vub/PhD/ThesisSubjects/AnomalousCouplings/June2015_FirstOrSecondFit/Neg01/LLFirstFitComparison_ChiSqCutsRVR_MGSampleNeg01_SingleGausTF_10000Evts_WideRange_10000Evts.pdf}
 \includegraphics[width = 0.49 \textwidth]{/home/annik/Documents/Vub/PhD/ThesisSubjects/AnomalousCouplings/June2015_FirstOrSecondFit/Neg01/LLSecondFitComparison_ChiSqCutsRVR_MGSampleNeg01_SingleGausTF_10000Evts_WideRange_10000Evts.pdf}
 \caption{Overall $\loglik$ distribution for MG samples created with $\VR$ = -0.1 obtained from the first fit (left) and the second fit (right) prior to any $\pT$-cut requirement. The different $\chisq$ cuts applied correspond to percentages of about ....!!} 
 \label{fig::FirstVsSecondFitNeg01}
\end{figure}
\begin{figure}[h!t]
 \centering 
 \includegraphics[width = 0.49 \textwidth]{/home/annik/Documents/Vub/PhD/ThesisSubjects/AnomalousCouplings/June2015_FirstOrSecondFit/Neg02/LLFirstFitComparison_ChiSqCutsRVR_MGSampleNeg02_SingleGausTF_10000Evts_WideRange_10000Evts.pdf}
 \includegraphics[width = 0.49 \textwidth]{/home/annik/Documents/Vub/PhD/ThesisSubjects/AnomalousCouplings/June2015_FirstOrSecondFit/Neg02/LLSecondFitComparison_ChiSqCutsRVR_MGSampleNeg02_SingleGausTF_10000Evts_WideRange_10000Evts.pdf}
 \caption{Overall $\loglik$ distribution for MG samples created with $\VR$ = -0.2 obtained from the first fit (left) and the second fit (right) prior to any $\pT$-cut requirement. The different $\chisq$ cuts applied correspond to percentages of about ....!!} 
 \label{fig::FirstVsSecondFitNeg02}
\end{figure}
\begin{figure}[h!t]
 \centering 
 \includegraphics[width = 0.49 \textwidth]{/home/annik/Documents/Vub/PhD/ThesisSubjects/AnomalousCouplings/June2015_FirstOrSecondFit/Neg03/LLFirstFitComparison_ChiSqCutsRVR_MGSampleNeg03_SingleGausTF_10000Evts_WideRange_10000Evts.pdf}
 \includegraphics[width = 0.49 \textwidth]{/home/annik/Documents/Vub/PhD/ThesisSubjects/AnomalousCouplings/June2015_FirstOrSecondFit/Neg03/LLSecondFitComparison_ChiSqCutsRVR_MGSampleNeg03_SingleGausTF_10000Evts_WideRange_10000Evts.pdf}
 \caption{Overall $\loglik$ distribution for MG samples created with $\VR$ = -0.3 obtained from the first fit (left) and the second fit (right) prior to any $\pT$-cut requirement. The different $\chisq$ cuts applied correspond to percentages of about ....!!} 
 \label{fig::FirstVsSecondFitNeg03}
\end{figure}
\begin{figure}[h!t]
 \centering 
 \includegraphics[width = 0.49 \textwidth]{/home/annik/Documents/Vub/PhD/ThesisSubjects/AnomalousCouplings/June2015_FirstOrSecondFit/Neg05/LLFirstFitComparison_ChiSqCutsRVR_MGSampleNeg05_SingleGausTF_10000Evts_WideRange_10000Evts.pdf}
 \includegraphics[width = 0.49 \textwidth]{/home/annik/Documents/Vub/PhD/ThesisSubjects/AnomalousCouplings/June2015_FirstOrSecondFit/Neg05/LLSecondFitComparison_ChiSqCutsRVR_MGSampleNeg05_SingleGausTF_10000Evts_WideRange_10000Evts.pdf}
 \caption{Overall $\loglik$ distribution for MG samples created with $\VR$ = -0.5 obtained from the first fit (left) and the second fit (right) prior to any $\pT$-cut requirement. The different $\chisq$ cuts applied correspond to percentages of about ....!!} 
 \label{fig::FirstVsSecondFitNeg05}
\end{figure}

%°°°°°°°°°°°°°°°°°°°°°°°°°°°°°°°°°°°°°°°°°°°°°°°°°°°°°°°°°°°°°°°°°°°°°°°°

%°°°°°°°°°°°°°°°°°°°°°°°°°°°°°°°°°°°°°°°°°°°°°°°°°°°°°°°°°°°°°°°°°°°°°°°°
%	CHAPTER: Results from gR coefficient
\chapter{Results from gR coefficient}
The benefit of looking also at the $\gR$ coefficient, and not only focussing on the $\VR$ one is that the relative changes in kinematics for this right-handed tensor coupling are expected to be much larger. This is explained by the way the coefficient enters the width formulas and the mixing which occurs with the $\VR$ coefficient.\\
This different sensitivity is clearly visible in Figure~\ref{fig::VRvsgR}. An additional bonus for the $\gR$ coefficient is that the distributions are not symmetrical compared to 0.0 allowing the use of a simple second-order polynomial instead of a more complex $\fth$ order one which is needed for the $\VR$ case.

\begin{figure}[h!t]
 \centering
 \includegraphics[width = 0.49 \textwidth]{/home/annik/Documents/Vub/PhD/ThesisSubjects/AnomalousCouplings/May2015_LikelihoodEvtSel/CosThetaVariation/CosThetaChange_RVRScan_FewerPoints.pdf}
 \includegraphics[width = 0.49 \textwidth]{/home/annik/Documents/Vub/PhD/ThesisSubjects/AnomalousCouplings/May2015_LikelihoodEvtSel/CosThetaVariation/RgRStudy/CosThetaVariation_RgRVariationNarrow.pdf}
 \caption{Stronger dependence of the $\csTh$ distribution on the $\gR$ coefficient than on the $\VR$ one. Therefore the $\gR$ coefficient will be measured in a more narrow range than the one used for the $\VR$ measurement.}
 \label{fig::VRvsgR}
\end{figure}

\section{Results prior to any pT-cuts}

The first results for this $\gR$ coefficient are given in Figure~\ref{fig::gRResultsNoCut} which shows the obtained likelihood distribution for most of the $\gR$ values in the studied range:
\begin{equation}
 \gR \in \left[-0.2, -0.15, -0.1, -0.05, 0.0, 0.05, 0.1, 0.15, 0.2 \right]
\end{equation}
Currently the result for $\gR$ = 0.1 is still missing together with the values at the outer edges of the considered range. These last two are missing since the results for $\gR$ $\pm$ 0.15 seem to suggest that values further away from the expected Standard Model configuration value do not agree anymore with the simulated value. For all the smaller $\gR$ values a nice agreement is found with the value used for generating the MadGraph sample.\\

\begin{figure}[h!t]
 \centering
 \includegraphics[width = 0.49 \textwidth]{/home/annik/Documents/Vub/PhD/ThesisSubjects/AnomalousCouplings/July2015_gRResults/Neg015/AllThreeDistributions_Acc_NoCuts_Neg015.pdf}
 \includegraphics[width = 0.49 \textwidth]{/home/annik/Documents/Vub/PhD/ThesisSubjects/AnomalousCouplings/July2015_gRResults/Neg01/AllThreeDistributions_Acc_NoCuts_Neg01.pdf}
 \includegraphics[width = 0.49 \textwidth]{/home/annik/Documents/Vub/PhD/ThesisSubjects/AnomalousCouplings/July2015_gRResults/Neg005/AllThreeDistributions_Acc_NoCuts_Neg005.pdf}
 \includegraphics[width = 0.49 \textwidth]{/home/annik/Documents/Vub/PhD/ThesisSubjects/AnomalousCouplings/July2015_gRResults/SM/AllThreeDistributions_Acc_NoCuts_SM.pdf}
 \includegraphics[width = 0.49 \textwidth]{/home/annik/Documents/Vub/PhD/ThesisSubjects/AnomalousCouplings/July2015_gRResults/Pos005/AllThreeDistributions_Acc_NoCuts_Pos005.pdf}
 \includegraphics[width = 0.49 \textwidth]{/home/annik/Documents/Vub/PhD/ThesisSubjects/AnomalousCouplings/July2015_gRResults/Pos015/AllThreeDistributions_Acc_NoCuts_Pos015.pdf}
 \caption{Obtained $\loglik$ distribution for MadGraph samples created with different $\gR$ values. From top left to bottom right the values used are -0.15, -0.01, -0.05, 0.0, 0.05 and 0.15 respectively.}
 \label{fig::gRResultsNoCut}
\end{figure}

From these $\loglik$ distributions can be concluded that the correct $\gR$ coefficient is recoverd for most of the considered MadGraph samples. However the deviation from the outer edges of the range are clearly visible and should be investigated further by looking at the result for a MadGraph sample created with $\gR$ = 0.1.\\
Also the presence of large deviations in the kinematics, even for low changes in $\gR$, opens the possibility to add the $\gR$ = 0.025 parameter to improve the accuracy close to the Standard Model expectation value.

\section{Results afer applying the event selection cuts}
Since it is possible that MadWeight and MadGraph have different event cleaning procedures for low-momentum events, it is interesting to study the likelihood distributions with a full event selection applied. The one which is currently used is the following:
\begin{table}[h!]
 \centering
 \caption{Event selection constraints applied for the MadGraph samples created.}\label{table::EvtSelCutsgR}
 \begin{tabular}{c|c|c|c}
		& pT-cut 	& $\eta$ 	& $\Delta$ R 	\\
  \hline
  jet 		& 30 $\GeV$ 	& 2.5 		& 0.3 		\\
  lepton 	& 26 $\GeV$ 	& 2.5 		& 0.3 		\\
  neutrino 	& 25 $\GeV$ 	& 2.5 		& 0.3
 \end{tabular}
\end{table}

Another important difference with respect to the $\loglik$ distributions shown before is the adapted range which is made significantly wider in order to capture the possible minima that might occur further away from the expected Standard Model value. The considered range is given below but for the moment only the $\loglik$ distributions for the cases $-0.15$, $0.0$ and $0.05$ have been calculated.
\begin{equation}\label{eq::FullgRRange}
 \left[-0.5, -0.3, -0.2, -0.15, -0.1, -0.05, -0.025, 0, 0.025, 0.05, 0.1, 0.15, 0.2, 0.3, 0.5 \right]
\end{equation}

The result for these three $\gR$ configurations is given in the Figure~\ref{fig::gRAllCuts}.
\begin{figure}[h!t]
 \centering
 \includegraphics[width=0.49 \textwidth]{/home/annik/Documents/Vub/PhD/ThesisSubjects/AnomalousCouplings/July2015_gRResults/CutsAppliedAlsoOnMET/Neg015/AllThreeDistributions_Acc_NoCuts_Neg015.pdf}
 \includegraphics[width=0.49 \textwidth]{/home/annik/Documents/Vub/PhD/ThesisSubjects/AnomalousCouplings/July2015_gRResults/CutsAppliedAlsoOnMET/SM/AllThreeDistributions_Acc_NoCuts_SM.pdf}
 \includegraphics[width=0.49 \textwidth]{/home/annik/Documents/Vub/PhD/ThesisSubjects/AnomalousCouplings/July2015_gRResults/CutsAppliedAlsoOnMET/Pos005/AllThreeDistributions_Acc_NoCuts_Pos005.pdf}
 \caption{Obtained $\loglik$ distributions from three configurations when the above-mentioned event selection constraints have been applied (Table~\ref{table::EvtSelCutsgR}). The final distribution is still lacking additional information from the extended range.}
 \label{fig::gRAllCuts}
\end{figure}

The biggest improvement is obtained for the configuration where $\gR$ = -0.15 since the applied event selection also ensures that the ``almost'' the correct minimum is retrieved. However it seems that also the enlarged range plays an important role.
\\
The Standard Model configuration does not correspond that well with the expected minimum but this is maybe caused by an issue with the cross-section calculated for $\gR$ = 0.05 which is for both the \textit{SM} as the \textit{Pos005} case larger than the surrounding configurations. \textbf{TO CHECK!}\\
\\
\textit{\textbf{Also mention something about the chi-sq distributions!}}

\subsection{Including $\csTh$ reweighting}
A next step in the event-selection calculations is applying the $\csTh$ reweighting which corrects the $\csTh$ distribution for the applied event selection constraints.
As before this weight is determined for each event by comparing the $\csTh$ value prior to the applied cuts with the value obtained after these event selection is applied.
\begin{equation}
 weight = \frac{\csTh_{All}}{\csTh_{Cut}}
\end{equation}

This results in a weight for each event but in order to apply the reweighting correctly to the MadWeight output this weight should be equal for each of the $\gR$ configurations. If this would not be the case the $\csTh$ reweighting should be determined individually for each of the considered $\gR$ configurations considered in the studied range. This would definitely complexify the application of this reweighting procedure.\\
Hence the obtained $\csTh$ fraction for each of the $\gR$ configurations has been plotted together in Figure~\ref{fig::CosThetagR_FullRange}, which shows a nice agreement between the different $\gR$ configurations although a clear discrepancy is visible for the so-called \textit{Neg05} case.
\begin{figure}[h!t]
 \centering
 \includegraphics[width = 0.44 \textwidth]{/home/annik/Documents/Vub/PhD/ThesisSubjects/AnomalousCouplings/July2015_gRResults/CutsAppliedAlsoOnMET/RelativeCutFunctionRgR_PtCutsApplied_AlsoOnMET.pdf}
 \includegraphics[width = 0.55 \textwidth]{/home/annik/Documents/Vub/PhD/ThesisSubjects/AnomalousCouplings/July2015_gRResults/CutsAppliedAlsoOnMET/WeightDistributionRgR_MGSamplePos015_CutsAppliedAlsoOnMET.pdf}
 \caption{Fraction of $\csTh$ distribution before and after event selection cuts have been applied for all the different $\gR$ configurations considered. They all have a similar distribution with the exeption of the \textit{Neg05} case. (left) Distribution of the obtained $\csTh$ weight for all the events in the \textit{Pos015} case. (right)} \label{fig::CosThetagR_FullRange}
\end{figure}

The weights obtained for the $\gR$ configuration are not yet normalised as can be seen from the title of the right-handed distribution in Figure~\ref{fig::CosThetagR_FullRange}. This makes the application of this $\csTh$ reweighting a bit more challenging since it implies that an additional normalisation factor should be introduced in order to ensure correct implementation of this reweighting procedure.\\
\textbf{\underline{Remark: }} Maybe interesting to check whether the weight remains similar (maybe also for the $\VR$ case) when only a reduced number of events is considered! This because the $\csTh$ weights are determined using 100 000 events but only 10 000 events or less are considered during the MadWeight calculations which might distort the normalisation of this reweighting procedure ...

For the moment the $\csTh$ reweighting is applied without altering the normalisation and the obtained distributions are given in Figure~\ref{fig::gRAllCuts_CosTheta}, again for the same limited number of $\gR$ configurations which is currently calculated by MadWeight. The calculations of the other $\gR$ parameters will be done as soon as possible in order to ensure a similar behaviour for configurations further away from the Standard Model.
\begin{figure}[h!t]
 \centering
 \includegraphics[width=0.49 \textwidth]{/home/annik/Documents/Vub/PhD/ThesisSubjects/AnomalousCouplings/July2015_gRResults/CutsAppliedAlsoOnMET/Neg015/AllThreeDistributions_Acc_NoCuts_Neg015_CosThetaReweightingApplied.pdf}
 \includegraphics[width=0.49 \textwidth]{/home/annik/Documents/Vub/PhD/ThesisSubjects/AnomalousCouplings/July2015_gRResults/CutsAppliedAlsoOnMET/SM/AllThreeDistributions_Acc_NoCuts_SM_CosThetaReweightingApplied.pdf}
 \includegraphics[width=0.49 \textwidth]{/home/annik/Documents/Vub/PhD/ThesisSubjects/AnomalousCouplings/July2015_gRResults/CutsAppliedAlsoOnMET/Pos005/AllThreeDistributions_Acc_NoCuts_Pos005_CosThetaReweightingApplied.pdf}
 \caption{Obtained $\loglik$ distributions from three configurations when the above-mentioned event selection constraints have been applied together with the $\csTh$ reweighting  (Table~\ref{table::EvtSelCutsgR}). The final distribution is still lacking additional information from the extended range.}
 \label{fig::gRAllCuts_CosTheta}
\end{figure}

The influence of this $\csTh$ reweighting procedure is extremely small and is almost only visible by the different range of the y-axis in the distributions in Figure~\ref{fig::gRAllCuts} and \ref{fig::gRAllCuts_CosTheta}. The current implementation of this reweighting is given in Equation (\ref{eq::CosThetaReweight}) and has an effect on both the MadWeight probability and the cross-section normalisation. The detailed influence of this $\csTh$ reweighting procedure is summarised in Figure~\ref{fig::CosThetaInfluence}, which contains the $\loglik$ distribution for the first polynomial fit before and after this reweighting is applied.

\begin{equation}\label{eq::CosThetaReweight}
 \mathcal{L}^{\csTh} = \sum (- \ln P^{MW}_{evt} + \ln \sigma) * weight_{\csTh,evt}
\end{equation}
\newpage

\begin{figure}[h!t]
 \centering
 \includegraphics[width = 0.32 \textwidth]{/home/annik/Documents/Vub/PhD/ThesisSubjects/AnomalousCouplings/July2015_gRResults/CutsAppliedAlsoOnMET/Neg015/CosThetaReweightingInfluence_Neg015_FullRange.pdf}
 \includegraphics[width = 0.32 \textwidth]{/home/annik/Documents/Vub/PhD/ThesisSubjects/AnomalousCouplings/July2015_gRResults/CutsAppliedAlsoOnMET/SM/CosThetaReweightingInfluence_SM_FullRange.pdf}
 \includegraphics[width = 0.32 \textwidth]{/home/annik/Documents/Vub/PhD/ThesisSubjects/AnomalousCouplings/July2015_gRResults/CutsAppliedAlsoOnMET/Pos005/CosThetaReweightingInfluence_Pos005_NarrowRange.pdf}
 \caption{Influence of the $\csTh$ reweighting procedure for the three $\gR$ configurations currently studied.} \label{fig::CosThetaInfluence}
\end{figure}

\textbf{\underline{Remark: }} Rather unexpected that the influence of this $\csTh$ reweighting procedure is so small, especially when keeping in mind the large shape difference obtained in the $\VR$ case. The implementation used seems to be correct since the goal of this reweighting is to include an event with weight $x$ $x$ times in the sum over the events. (\textit{Or should it also be inside the logarithm??})

\subsection{Adding additional constraint on slope of fit}
A second point which has been studied in order to further improve the $\loglik$ distributions for the different $\gR$ configurations was the value of the second derivative which gives an idea whether the $\loglik$ distribution has the desired minimum-like shape or the maximum-like shape. Since the $\gR$ distributions can be fitted with a simple $\scd$ polynomial fit it makes sense to require this second derivative to be positive.\\
When looking at all the events which have been studied rather a lot actually have this undesired maximum-like shape as can be seen from Figure~\ref{fig::ScdDerAllEvts_Neg015} which shows the value of this second derivative for the first polynomial fit. 

\begin{figure}[h!t]
 \centering
 \includegraphics[width = 0.49 \textwidth]{/home/annik/Documents/Vub/PhD/ThesisSubjects/AnomalousCouplings/July2015_gRResults/CutsAppliedAlsoOnMET/Neg015/ScdDerAllEvts_RgRNeg015.pdf}
 \caption{Distribution of second derivative of polynomial fit (Neg015 case).} \label{fig::ScdDerAllEvts_Neg015}
\end{figure}

Therefore the $\loglik$ distributions obtained when requiring this second derivative to be positive were created and compared to the ones obtained when combining all events. Strangely enough this requirement does not result in a nice minimum-like overall $\loglik$ distribution but gives a distribution which does not agree at all with the expectations. This is shown in Figure~\ref{fig::LogLik_PosSlope} which contains the $\loglik$ distributions for events with positive second derivative.
\\

\begin{figure}[h!t]
 \centering
 \includegraphics[width = 0.32 \textwidth]{/home/annik/Documents/Vub/PhD/ThesisSubjects/AnomalousCouplings/July2015_gRResults/CutsAppliedAlsoOnMET/Neg015/PosSlopeInfluence_Neg015.pdf}
 \includegraphics[width = 0.32 \textwidth]{/home/annik/Documents/Vub/PhD/ThesisSubjects/AnomalousCouplings/July2015_gRResults/CutsAppliedAlsoOnMET/SM/PosSlopeInfluence_SM.pdf}
 \includegraphics[width = 0.32 \textwidth]{/home/annik/Documents/Vub/PhD/ThesisSubjects/AnomalousCouplings/July2015_gRResults/CutsAppliedAlsoOnMET/Pos005/PosSlopeInfluence_Pos005.pdf}
 \caption{Comparison between the $\loglik$ distribution (first polynomial fit) when all events are used (red) and when only the events which have a positive second derivative are considered (blue). This selection requirement clearly does not result in the desired improvement ...} \label{fig::LogLik_PosSlope}
\end{figure}

So this requirement is clearly not sufficient and does not result in the desired improvement. Hence detailed study of the events removed by this requirement should be done since this result seems to suggest that in order to obtain a correct $\loglik$ minimum the so-called \textit{wrong} events have to be incorporated in order to end up with an overall shape which matches with expectation. It still is very strange that the obtained result is so sensitive to these type of constraints and therefore a sample with 50 000 instead of the currently studied 10 000 events is being processed by MadWeight. This will hopefully be able to help decide whether these large changes can be caused by a kind of statistical fluctuations or whether there is really a profound physics reason behind...
\\

Another variable which was considered in order to serve as a ``weight cleaning'' requirement was the \textit{slope steepness}. 
%Another observation that did not correspond with the expectations was the distribution of the ``steepness of the slope'' for all the events.  --> Combined all three normalisations ....
This variable looks at the difference in value between the outermost point and the expected minimum point such that it gives an idea of the sharpness of the individual $\loglik$. 
%For some strange reason this results in two separate peaks as can be seen from Figure~\ref{fig::SlopeSteepness_gR}. The peak positioned around 0 seems to suggest that a large portion of the events actually contain not much information and are almost flat while the events with a lot of information are represented by this second peak at higher values.
From Figure~\ref{fig::SlopeSteepness_gRSM} can be seen that the slope steepness does not have the same sign as the second derivative of the fit and can therefore maybe be combined with this previous constraint. Figure~\ref{fig::SlopeSteepness_gRNeg015} confirms that the same behaviour is recovered for the Standard Model case and the \textit{Neg015} one.

\begin{figure}[h!t]
 \centering
 %\includegraphics[width = 0.32 \textwidth]{/home/annik/Documents/Vub/PhD/ThesisSubjects/AnomalousCouplings/July2015_gRResults/CutsAppliedAlsoOnMET/Neg015/SlopeSteepness_RgR_MGSampleNeg015.pdf}
 \includegraphics[width = 0.32 \textwidth]{/home/annik/Documents/Vub/PhD/ThesisSubjects/AnomalousCouplings/July2015_gRResults/CutsAppliedAlsoOnMET/SM/SlopeSteepness_RgR_MGSampleSM.pdf}
 \includegraphics[width = 0.32 \textwidth]{/home/annik/Documents/Vub/PhD/ThesisSubjects/AnomalousCouplings/July2015_gRResults/CutsAppliedAlsoOnMET/SM/SlopeSteepness_PosScdDer_RgR_MGSampleSM.pdf}
 \includegraphics[width = 0.32 \textwidth]{/home/annik/Documents/Vub/PhD/ThesisSubjects/AnomalousCouplings/July2015_gRResults/CutsAppliedAlsoOnMET/SM/SlopeSteepness_NegScdDer_RgR_MGSampleSM.pdf}
 \caption{Steepness of the slope using MadWeight probabilities (so no fit information is used) for the Standard Model configuration.} \label{fig::SlopeSteepness_gRSM}
\end{figure}

\begin{figure}[h!t]
 \centering
 \includegraphics[width = 0.32 \textwidth]{/home/annik/Documents/Vub/PhD/ThesisSubjects/AnomalousCouplings/July2015_gRResults/CutsAppliedAlsoOnMET/Neg015/SlopeSteepness_RgR_MGSampleNeg015.pdf}
 \includegraphics[width = 0.32 \textwidth]{/home/annik/Documents/Vub/PhD/ThesisSubjects/AnomalousCouplings/July2015_gRResults/CutsAppliedAlsoOnMET/Neg015/SlopeSteepness_PosScdDer_RgR_MGSampleNeg015.pdf}
 \includegraphics[width = 0.32 \textwidth]{/home/annik/Documents/Vub/PhD/ThesisSubjects/AnomalousCouplings/July2015_gRResults/CutsAppliedAlsoOnMET/Neg015/SlopeSteepness_NegScdDer_RgR_MGSampleNeg015.pdf}
 \caption{Steepness of the slope using MadWeight probabilities (so no fit information is used) for the \textit{Neg015} configuration.} \label{fig::SlopeSteepness_gRNeg015}
\end{figure}

\textit{Rather strange that these two plots are so identical ... Would expect much less steeper slope in the ``Neg015'' case since the two points are located much closer together than in the ``SM'' case.}

But also this does not result in the desired improvement, it even seems that requiring this ``slope steepness'' to be positive again makes the position of the minimum deviate further from the expected minimum ... The example given in Figure~\ref{fig::LogLik_PosSlope_PosSteepness} is the distribution obtained for the Standard Model configuration. It contains the comparison between the original $\loglik$ distribution, the one where only the slope is required to be positive and finally the distribution where both slope and steepness are constrained.

\begin{figure}[h!t]
 \centering
 \includegraphics[width = 0.49 \textwidth]{/home/annik/Documents/Vub/PhD/ThesisSubjects/AnomalousCouplings/July2015_gRResults/CutsAppliedAlsoOnMET/SM/PosSlopeInfluence_AlsoSteepness_SM.pdf}
 \includegraphics[width = 0.49 \textwidth]{/home/annik/Documents/Vub/PhD/ThesisSubjects/AnomalousCouplings/July2015_gRResults/CutsAppliedAlsoOnMET/SM/PosSlopeInfluence_BothSteepness_SM.pdf} 
 \caption{Comparing the $\loglik$ distributions for events with a positive second derivative (obtained from fit) and a positive difference between the outermost $\gR$ point (-0.5) and the expected minimum value (0.0) for the Standard Model case. The second distribution also contains the additional requirement that the slope on the other side should be negative.} \label{fig::LogLik_PosSlope_PosSteepness}
\end{figure}


\section{Cross-section dependency}

As a test to determine how sensitive the obtained $\loglik$ distributions are to the used cross-section value for normalisation a sort of scaling of the cross-section has been applied. This is applied by multiplying the cross-section value for a specific $\gR$ configuration with the following function:
\begin{equation}
 f(\gR) = 1 + \gR *x ~ ~ ~ \textrm{with } x = \left\lbrace -0.1, -0.05, 0, 0.05, 0.1 \right\rbrace
\end{equation}

Since the cross-section normalisation term is a logarithmic term it was expected that even small changes in the cross-section can have enormeous effects, and this is indeed what is found and summarised in Figure~\ref{fig::XSScaling}. Strangely enough the event selection constraints applied for the bottom two configurations do not seem to reduce the dependency on this cross-section value. Since here the cross-section values are drastically reduced by these cuts it would seem more logical that the variations between the different scalings considered become less pronounced, but this is clearly not the case ...

\begin{figure}[h!t]
 \centering
 \includegraphics[width = 0.49 \textwidth]{/home/annik/Documents/Vub/PhD/ThesisSubjects/AnomalousCouplings/July2015_gRResults/XSScaling/MGSample_RgRNeg005_XSScaledComparison.pdf}
 \includegraphics[width = 0.49 \textwidth]{/home/annik/Documents/Vub/PhD/ThesisSubjects/AnomalousCouplings/July2015_gRResults/XSScaling/MGSample_RgRPos015_XSScaledComparison.pdf}
 \includegraphics[width = 0.49 \textwidth]{/home/annik/Documents/Vub/PhD/ThesisSubjects/AnomalousCouplings/July2015_gRResults/XSScaling/MGSample_RgRNeg015_CutsAppliedAlsoOnMET_XSScaledComparison.pdf}
 \includegraphics[width = 0.49 \textwidth]{/home/annik/Documents/Vub/PhD/ThesisSubjects/AnomalousCouplings/July2015_gRResults/XSScaling/MGSample_RgRPos005_CutsAppliedAlsoOnMET_XSScaledComparison.pdf}
 \caption{Influence of the considered cross-section scaling both with and without event selection constraints applied.} \label{fig::XSScaling}
\end{figure}





%°°°°°°°°°°°°°°°°°°°°°°°°°°°°°°°°°°°°°°°°°°°°°°°°°°°°°°°°°°°°°°°°°°°°°°°°

\end{document}
