\documentclass[a4paper,12pt]{report}
\usepackage[T1]{fontenc}
\usepackage[utf8]{inputenc}
\usepackage{lmodern}
\usepackage[margin=1in]{geometry}
\usepackage{graphicx}
\usepackage{lscape}
\usepackage{multirow}
\usepackage[labelfont=bf]{caption}
\usepackage{color}
%----------------------------------------------------------------------------------
\usepackage[centersections]{vubtitlepage}
\usepackage[title,titletoc,toc]{appendix}

\faculty{Faculteit Wetenschappen en bio-ingenieurswetenschappen}
\date{Started 4 April 2014}
\sectone{Anomalous Couplings Project}
\sectthree{Annik Olbrechts }
\secttwo{Progress Report and documentation of Tools }

\sectonesize{35pt}
\sectoneskip{40pt}
\secttwosize{30pt}
\secttwoskip{10pt}
\sectthreesize{24pt}
\sectthreeskip{28pt}
%----------------------------------------------------------------------------------

\usepackage{hyperref}
\begin{document}
\pagenumbering{roman}
\setcounter{secnumdepth}{3} %Number up to the subsubsections!
\setcounter{tocdepth}{2}    %Don't want the subsubsections appearing in the table of contents

\maketitle
\tableofcontents
\newpage

\pagenumbering{arabic}
%°°°°°°°°°°°°°°°°°°°°°°°°°°°°°°°°°°°°°°°°°°°°°°°°°°°°°°°°°°°°°°°°°°°°°°°°°°°°°°°°°°°°°°°°°°°°°°°°°°°°°°°°°°°°°°°°°°°°°°°°°°°°°°°°°°°°°°°°°°°°°°°°°°°°°°°°°°°°°°
\chapter{Pending issues}
\begin{itemize}
  \item Does this created FeynRules model still contain Effective Field Theory?
  \item If the kinematics doesn't change for coupling parameters larger than 1, how is then possible to differentiate the different configurations which will be studied?
  \item Access to Mathematica for Width studies ... ?
  \item XS values which are used are the ones calculated using the MadGraph$\_$v155 version. Should this be calculated again using the new MadGraphv5$\_$aMC@NLO since normally this shouldn't change ...?
\end{itemize}

%°°°°°°°°°°°°°°°°°°°°°°°°°°°°°°°°°°°°°°°°°°°°°°°°°°°°°°°°°°°°°°°°°°°°°°°°°°°°°°°°°°°°°°°°°°°°°°°°°°°°°°°°°°°°°°°°°°°°°°°°°°°°°°°°°°°°°°°°°°°°°°°°°°°°°°°°°°°°°°
\chapter{Used tools and techniques}
%°°°°°°°°°°°°°°°°°°°°°°°°°°°°°°°°°°°°°°°°°°°°°°°°°°°°°°°°°°°°°°°°°°°°°°°°°°°°°°°°°°°°°°°°°°°°°°°°°°°°°°°°°°°°°°°°°°°°°°°°°°°°°°°°°°°°°°°°°°°°°°°°°°°°°°°°°°°°°°

\section{MadGraph}

\subsection{MadGraph$\_$v155 version}
Currently the MadGraph version which should be used is the $MadGraph\_v155$ since this is the only one which was compatible with the used FeynRules version at the time of the creation of the model. According to a mail of Olivier Mattelaere on 6 February 2014 this issue should now be fixed.\\
\\
In this project MadGraph is only used to create .lhe files in order to understand the created model. Especially for Cross Section studies it is extremely useful.\\
Hence two python scripts have been created in the following directory:
\begin{eqnarray*}
  & & /user/aolbrech/AnomalousCouplings/MadGraph\_v155/MassiveLeptons/\\ & & MadGraph5\_v1\_5\_5/Wtb\_ttbarSemiElMinus/RVL\_RVR\_XSGrid003.py \\
  & & /user/aolbrech/AnomalousCouplings/MadGraph\_v155/MassiveLeptons/\\ & & MadGraph5\_v1\_5\_5/XSScript.py
\end{eqnarray*}
The first one automatically creates all the desired configurations, for different RVL, RVR, Lepton Pt Cut and Jet Pt Cut values. The desired output directory can be specified in this file as well.\\
In order to save some disk space the events.lhe file is deleted using this script and only the unweighted$\_$events.lhe are kept for analysis. These are the relevant events (\textit{according to Alexis so understand what is the difference}).\\
\\
The second file calculates the cross section of the desired configurations. It again loops over all the different variables and creates a .txt table and a .pdf table.\\
\textbf{Maybe this can be created automatically when executing the first python script! This to avoid copying the considered variables from one file to another!}

\subsection{New MadGraph5$\_$aMC@NLO version}
This new MadWeight version was released in the spring of 2014 and is accompanied with a new MadWeight version as well.\\
Normally the cross section calculations shouldn't differ between the different MadGraph versions, but for at least one of the decay channels this should be checked!\\
\\
In order to use MadWeight and calculate the cross section values for the desired processes and decay channels the following commands should be used:
\begin{eqnarray}
 ./bin/mg5_aMC \\
 import model MassiveLeptons \\
 generate p p > t t~ , ( t > b w+ , w+ > mu+ vm ) , ( t~ > b~ w- , w- > j j ) @1 QED = 2 \\
 output dirName \\
 exit
\end{eqnarray}
Once this directory is created the \textit{index.html} file contains all the Feynmann Diagrams which correspond to the considered process. This file can be opened using $firefox$ on mtop.\\
Since the FeynRules diagram is created in such a way that it contains a \textbf{NP}, \textbf{QCD}, \textbf{QED} and \textbf{TEST} variable representing the different interaction vertices. Hence asking the variable \textbf{QED} to be equal to $2$ results in the 16 independent diagrams which are required to represent the anomalies in the Wtb vertex. Currently the decay of the top quark is altered to contain all the different coupling constants while the other interaction vertices represent the Standard Model expectations.\\
\\
When the cross section should be calculated for a specific process the following configuration files in the Cards directory should be updated.
\begin{itemize}
 \item \textbf{run$\_$card.dat} where the number of events which should be generated is defined together with the beam energy of the considered collision process. Also should be ensured that no kinematic cuts are active on the particles since these cuts will have a different effect then the cuts applied in the event selection of the analysis. These cuts are applied on the generator level instead of the jet-level which results in much more stringent and tighter effects when using the same $p_{T}$ value.
 \item \textbf{param$\_$card.dat} where all the parameters of the considered model should be defined.
 \item \textbf{proc$\_$card.dat} contains the event which will be generated and the models which have been imported. This file shouldn't be changed but is particularly useful to ensure that the correct event is considered for generation.
\end{itemize}

All these files can still be changed once the \textit{./bin/generate$\_$events} command has been executed. The MadGraph software always asks whether one or more of the configuration files should be adapted before the command is submitted and allows a waiting time of 60s. Hence in order to run MadGraph continously using a script the \textbf{me5$\_$configuration.txt} file should be adapted to enable this waiting time since otherwise when using nohup to run the script results in a crash and the script is terminated.
%**************************************************

\section{MadWeight}

The running of the MadWeight event generator has to be done on localgrid since the events .lhco file has to be splitted in multiple jobs. The number of jobs which can be submitted is not limited, hence the events per job should be taken as low as possible. Running MadWeight with a large number of events per job implies that it will take a very long time since the submitted jobs have a long walltime which reduces their priority. \\
Another important factor for the MadWeight event generator is defining the number of interaction points which have to be considered. This number has to be rather large (around 10 000 - 30 000 according to Lieselotte \textbf{TO CHECK MYSELF}) to ensure the uncertainty on the weight to be smaller than the weight itself.\\
\\
Currently two different MadWeight event generator versions exist, and up to now (May 6 2014) only the oldest one has been tested and used extensively. The newer one was only installed beginning of May but should be less CPU intensive and allows the user to split the interaction points in two distinct steps. The versions are:
\begin{itemize}
  \item madweight\_mc\_perm
  \item MadGraph5\_aMC@NLO
\end{itemize}

The installation commands of these two versions are the following (the bzr command package has been installed on the m-machines, but not on mtop):
\begin{itemize}
  \item bzr branch lp:~maddevelopers/madgraph5/madweight 
  \item bzr branch lp:~maddevelopers/mg5amcnlo/madweight
\end{itemize}

Whenever MadWeight is first used a personal directory should be initalized. This can be done in the following way:
\begin{itemize}
  \item Import the created model (called MassiveLeptons) in the model directory of MadGraph
  \item Activating the MadWeight event program and initializing a personal directory.
  \begin{enumerate}
    \item ./bin/mg5$\_$aMC (The mg5 executable should not be used!)
    \item import model MassiveLeptons
    \item generate p p > t t~ , ( t > b w+ , w+ > mu+ vm ) , ( t~ > b~ w- , w- > j j ) @1 QED = 2 
    \item output madweight
    \item exit
  \end{enumerate}
\end{itemize}

\subsection{MC$\_$PERM MadWeight use}
The most important steps which have to be executed to use this MadWeight version are the following ones~\footnote{Full detailed explanation about the use of this madweight version can be found in the documentation of Bettina.}:
\begin{enumerate}
  \item Initialize MadWeight to run on localgrid!
  \item Update the MadWeight$\_$card.dat
  \item Set the correct transfer functions
  \item Run MadWeight
\end{enumerate}

For this first step the following two files have to be changed in the /blablaMadWeight/bin/internal directory:
\begin{itemize}
  \item madweight$\_$interface.py
  \item cluster.py
\end{itemize}

\subsection{Checking MadWeight on localgrid}
Different command which should be used to check whether jobs are running are running on localgrid and how they should be killed when something went wrong:
\begin{eqnarray}
 qstat \; \; @cream02 \; | \; grep \; \; aolbrech \\
 qdel \; \; 394402.cream02
\end{eqnarray}

\subsection{Influence of the used Transfer Function}

\subsection{Adapting MadWeight to run continously on localgrid}
A single test with the following configuration (5000 events - 20 events/job - 10 000 initPoints) took almost 16h to finish so the full generator event sample ($>$ 2 470 000 events) would take more than 329 days!\footnote{From this can be concluded that the events on generator level should never be all run. Only a limited selection of these events should be considered.}\\
Even the full reconstructed event sample ($>$ 210 000 events) requires more than 28 days of running this way.\\
\textbf{TODO: Time of running should be compared against the new MadWeight version!!}\\
\\
A possible solution to reduce the CPU time to process these events is in stead of sending 2000 jobs, waiting until they are all finished and only then submitting the next bunch of 2000 jobs trying to adapt the MadWeight script to send continously 2000 jobs. This would imply that the script should check whether one of the jobs have been finished and immediatly submitting a new one.\\
For this the above mentioned scripts should be adapted!\\
\\
According to Olivier (D) this can't be changed by the grid admins, but should be adapted by the MadGraph developers. 
%**************************************************

\section{MadAnalysis}
\subsection{How to run MadAnalysis in expert mode}
Version v112 should be used since the expert mode in this version works exactly as explained in the manual (arXiv: 1206.1599). 
In the more recent version v115 the expert mode doesn't work out-of-the-box and the python files need to be adapted ...\\
\\
Starting with a new analysis directory in MadAnalysis in expert mode can be done by typing the following command:
\begin{equation}
  ./bin/ma5 \; \; \; --expert
\end{equation}
In the questions asked by MadAnalysis after executing this command the name of the directory which needs to be created and the name of the analysis has to be given.\\
The name of this analysis shouldn't be made too complex since it has to be typed everytime when creating plots with MadAnalysis.\\
\\
After the correct directory is created, the $Name$/SampleAnalyzer directory should be initialized by executing the following two commands:
\begin{eqnarray}
  source \; \; setup.sh \\
  make
\end{eqnarray}

The actual analysis should be created in the Analysis directory, and a similar approach to user.cpp and user.h should be adopted.\\
Everytime a change has been made to these two files $make$ should be executed in the SampleAnalyzer directory in order to process these changes.\\
\\
The .lhe files (MadAnalysis cannot process .lhe.gz files so they should be unpacked using gunzip .lhe.gz) which should be considered should be wirtten down in a .txt files which is saved in the SampleAnalyzer directory. \\
\\
The actual running of MadAnalysis is done with the following command:
\begin{equation}
  ./SampleAnalyzer \; \; --analysis="Name ~ of ~ analysis" ~~~~ List.txt
\end{equation}

\subsection{Content of analysis file in MadAnalysis}
The latest analysis file which has been used with all the necessary information can be found in the following directory on the m-machines:
\begin{eqnarray*}
  AnomalousCouplings/MadAnalysis\_v112/Wtb\_PtCutInfluence/SampleAnalyzer/Analysis 
\end{eqnarray*}
The two analysis files with full detailed information are the LeptonPtCutInfluence.cpp and the JetPtCutInfluence.cpp files.
They both consist of two different functions, namely the $Execute$ and the $Finalize$ function. The first one allows to access the information of each event while the second one is entered for each file.
Therefore the particle content is reconstructed in the $Execute$ function and the histograms for all the considered files are constructed in the $Finalize$ function.
Currently these analyzer files look at 28 different kinematic variables. All the kinematic information of each of the particles present in the expected semi-leptonic $t\bar{t}$ event is created.\\
\\
In order to separate the two b-quarks in the event, the Particle Id of the leptonic top quark needs to be known. Therefore an integer $LeptonicTopPdgId$ is used and the kinematic information of the b-quarks can only be stored when this integer is different from zero.
This will normally not result in by-passing the b-quark information since the events in the .lhe files are read in in the same order as they are created by the MadGraph command. So first the top quarks are considered and only then the final state particles.\\
\\
\textit{? Isn't it safer to use the daughter information of the top quark? Because in the case of t t -> b jj b l v the lepton comes after the b-quark ... Should investigate what happens in this case ...} \textbf{TO CHECK: } I think some extra safety is incorporated and the actual loop of filling the event content is only done when the b's are reconstructed!\\
\\
In order to automatically create the 28 histograms for all the different files and distinguish the different RVR and RVL values, an automatic name-givng loop is used. The name of the content of the histogram is each of the time combined with the correct name of the RVL/RVR content of the .lhe file and the decayChannel.\\
\\
\textit{Currently the name of the considered .lhe files has to be adapted every time the files are changed. Maybe this should be made automatically read in from the .txt file to ease the processing of different configurations. If the .lhe files have a clear name where the RVL, RVR and PtCut values can be obtained from using a python script it shouldn't be too much work to save this in the .cpp file with this script. \textbf{TO DO!}}

\subsection{Analyzing the MadGraph files}\label{subsec:MadGraphFiles}
Currenlty the created model should be completely understood and the behavior of the model when the coupling coefficients are larger than $1$ should be investigated. Therefore new MadGraph files have been processed for the following configuration:
\begin{eqnarray*}
  Re(V_L) & \in & \left[  0.7, \; 1.3\right] \\
  Re(V_R) & \in & \left[ -0.3, \; 0.3\right]
\end{eqnarray*}
These files can be found in the following directory on the m-machines and contain 100 000 events.
\begin{eqnarray*}
  /user/aolbrech/AnomalousCouplings/MadGraph\_v155/MassiveLeptons/\\ MadGraph5\_v1\_5\_5/Wtb\_ttbarSemiElMinus/ResultsXSGrid003
\end{eqnarray*}

\subsection{Width of the decay}\label{subsec:DecayWidth}
%**************************************************

\section{FeynRules}

%°°°°°°°°°°°°°°°°°°°°°°°°°°°°°°°°°°°°°°°°°°°°°°°°°°°°°°°°°°°°°°°°°°°°°°°°°°°°°°°°°°°°°°°°°°°°°°°°°°°°°°°°°°°°°°°°°°°°°°°°°°°°°°°°°°°°°°°°°°°°°°°°°°°°°°°°°°°°°°
\chapter{Understanding results obtained with MadWeight}
%°°°°°°°°°°°°°°°°°°°°°°°°°°°°°°°°°°°°°°°°°°°°°°°°°°°°°°°°°°°°°°°°°°°°°°°°°°°°°°°°°°°°°°°°°°°°°°°°°°°°°°°°°°°°°°°°°°°°°°°°°°°°°°°°°°°°°°°°°°°°°°°°°°°°°°°°°°°°°°

\section{Comparing the two used MadWeight versions}
Since it was found that the latest MadWeight version (aMC@NLO) resulted in many events with weight equal to $0$ it was decided to compare this version with the previous one (mc$\_$perm). It is expected that both versions result in similar weights when identical events are considered, otherwise the used version of MadWeight would have a too large influence on the analysis result.\\
Therefore two events which could succesfully run in the newest MadWeight version were also calculated using the old version. The obtained weights and their uncertainty can be compared in the following table and all the relevant information can be found in:
\begin{eqnarray*}
 /home/annik/Documents/Vub/PhD/ThesisSubjects/AnomalousCouplings\\ /CompareMWVersions\_May2014
\end{eqnarray*}

\begin{table}[h!]
 \centering
 \begin{tabular}{|c|c|c|c|c|c|}
  \hline
  \multirow{2}{*}{Event Number} &  \multirow{2}{*}{$V_L$ value}	& \multicolumn{2}{|c|}{aMC@NLO version}		& \multicolumn{2}{|c|}{mc$\_$perm version}  	\\
				&  				& Weight		& Uncertainty		& Weight 		& Uncertainty  		\\
  \hline
  \multirow{3}{*}{1} 		& 1.5 				& $9.76 10^{-28}$	& $4.07 10^{-30}$	& $1.44 10^{-26}$	& $4.19 10^{-29}$	\\
				& 1.0				& $1.92 10^{-28}$	& $8.04 10^{-31}$	& $2.84 10^{-27}$	& $8.28 10^{-30}$	\\
				& 0.5				& $1.20 10^{-29}$	& $5.03 10^{-32}$	& $1.78 10^{-28}$	& $5.17 10^{-31}$	\\
  \hline 
  \multirow{3}{*}{2}	 	& 1.5 				& $1.86 10^{-23}$	& $1.15 10^{-25}$	& $1.77 10^{-24}$	& $1.23 10^{-26}$	\\
				& 1.0				& $3.65 10^{-24}$	& $2.25 10^{-26}$	& $3.57 10^{-25}$	& $2.75 10^{-27}$	\\
				& 0.5				& $2.27 10^{-25}$	& $1.47 10^{-27}$	& $2.23 10^{-26}$	& $1.72 10^{-28}$	\\
  \hline 
 \end{tabular} 
 \caption{Weight obtained from MadWeight for two specific ttbar semi-muonic (+) events. For these events the $V_R$ was fixed to its Standard Model expectation value, which is $0$, while the $V_L$ value was varied.} 
\end{table}

Comparing these values clearly shows that there is a significant difference between the two MadWeight versions which were considered in this analysis. With some effort a general difference of a facto 10 can be identified between the two versions, with a higher weight value for the older mc$\_$perm MadWeight version.\\
However when showing the relative differences between the different weights it can be seen that the behavior of these two MadWeight versions is actually very similar. Therefore the histograms below give firstly the actual weight value and secondly the weight value normalised to the weight corresponding to the coupling parameter $V_L$ = 0.
\begin{figure}[!h]
\includegraphics[width = 0.48 \textwidth]{/home/annik/Documents/Vub/PhD/ThesisSubjects/AnomalousCouplings/CompareMWVersions_May2014/FirstEvtCanvas.png}
\includegraphics[width = 0.48 \textwidth]{/home/annik/Documents/Vub/PhD/ThesisSubjects/AnomalousCouplings/CompareMWVersions_May2014/SecondEvtCanvas.png}\\
\includegraphics[width = 0.48 \textwidth]{/home/annik/Documents/Vub/PhD/ThesisSubjects/AnomalousCouplings/CompareMWVersions_May2014/FirstEvtCanvas_Rel.png}
\includegraphics[width = 0.48 \textwidth]{/home/annik/Documents/Vub/PhD/ThesisSubjects/AnomalousCouplings/CompareMWVersions_May2014/SecondEvtCanvas_Rel.png}
\caption{Distribution of the weights obtained from MadWeight for the two considered MadWeight versions (aMC@NLO and mc$\_$perm) for two specific ttbar semi-muonic (+) events.}
\end{figure}

The above histograms clearly indicate that however the actual values of the weight significantly differ, the normalized results are almost identical.\\
\textbf{This implies that the results obtained with MadWeight should always be normalized with respect to another MadWeight result} (obtained using the same MadWeight version of course). 
%**************************************************

\section{Individual weight distribution for condsidered gridpoints}
The first Likelihood distribution for the considered gridpoints gave rise to an unexpected result. In order to understand whether this strange behavior can be explained by a couple of events with a bad weight, the indiviudual weight distribution for a couple random events was studied. If these distributions indeed result in the correct behavior the events which influence the overal Likelihood distribution can be studied individually.\\
All the relevant information can be found in the following directory:
\begin{eqnarray*}
 /home/annik/Documents/Vub/PhD/ThesisSubjects/AnomalousCouplings/ \\ UnderstandLikelihoodDistr\_July2014
\end{eqnarray*}
And the creation of the MadWeight weights together with the python scripts making the corresponding histograms can be found in:
\begin{eqnarray*}
 /localgrid/aolbrech/madweight/ttbarSemiMuPlus\_QED2/Events
\end{eqnarray*}
The relevant python scripts are the following:
\begin{itemize}
 \item \textbf{CalculateLikelihood.py} which simultaneously creates the histograms for the XS distribution, the raw and normalized Likelihood distribution, and if required the individual weight distributions for the selected events. All these histograms are saved in the Histos.root file.
 \item \textbf{RemoveZeroWeightEvents.py} which removes the events with weight equal to zero from the list and saves the non-zero events in a new .out file. On the other hand the events which failed the MadWeight computation are saved on a new .lhco file together with one succesful control event and are send again through MadWeight for a new weight calculation. \textcolor{red}{This should be updated since the new computation of MadWeight also results in weights equal to zero for these failing events. So should be investigated what is different about these events ...}
\end{itemize}




%°°°°°°°°°°°°°°°°°°°°°°°°°°°°°°°°°°°°°°°°°°°°°°°°°°°°°°°°°°°°°°°°°°°°°°°°°°°°°°°°°°°°°°°°°°°°°°°°°°°°°°°°°°°°°°°°°°°°°°°°°°°°°°°°°°°°°°°°°°°°°°°°°°°°°°°°°°°°°°
\chapter{Analyzing created FeynRules model}
%°°°°°°°°°°°°°°°°°°°°°°°°°°°°°°°°°°°°°°°°°°°°°°°°°°°°°°°°°°°°°°°°°°°°°°°°°°°°°°°°°°°°°°°°°°°°°°°°°°°°°°°°°°°°°°°°°°°°°°°°°°°°°°°°°°°°°°°°°°°°°°°°°°°°°°°°°°°°°°

\section{Normalized coupling parameters}
In order to investigate the actual influence of the value of the coupling parameters on the kinematics of the event, the considered coupling parameters should be normalized to unitarity before any hard conclusions can be made.
Therefore the configurations which should be investigated in large detail are the ones for which the width of the decay remains unchanged. This is explained in detail in one of the previous sections (\ref{subsec:DecayWidth} on page \pageref{subsec:DecayWidth}).
%**************************************************

\section{Understanding parameters larger than 1}
Before starting to look at the ttbar Monte Carlo and reweighting the events with MadWeight, the created model in FeynRules should be completely understood.
Special care goes out to the behavior of the kinematic distributions for values of the coupling parameters larger than 1. Since the Standard Model expectation puts the real part of the left-handed vector coupling $V_L$ almost equal to 1, simulation should be available around this Standard Model expectation value.
Therefore the created model should be able to cope with coupling parameters larger than 1. \\
\\
For this reason .lhco files were generated with MadGraph with the following configuration, as mentioned in \ref{subsec:MadGraphFiles}:
\begin{eqnarray*}
  Re(V_L) & \in & \left[  0.7, \; 1.3\right] \\
  Re(V_R) & \in & \left[ -0.3, \; 0.3\right]
\end{eqnarray*}
For these generated events the main kinematic distributions have been investigated.
No clear difference between the behavior below and above 1 has been found.

\newpage
\subsection{Performed checks}
\subsubsection{Cross section change}
\begin{table}[h!] 
 \begin{tabular}{c|c c c c c c c} 
  RVR &  -0.30  &  -0.20  &  -0.10  &  0.00  &  0.10  &  0.20  &   0.30  \\  
  RVL & & & & & & & \\ 
  \hline 
  0.70  & 0.3775 &  0.3120  & 0.2724  & 0.3275  & 0.2632  & 0.2910 &  0.3436  \\ 
  0.80  & 0.5964 &  0.5097  & 0.4595  & 0.4385  & 0.4444  & 0.4816 &  0.5471  \\ 
  0.90  & 0.9011 &  0.7950  & 0.7308  & 0.7026  & 0.7103  & 0.3229 &  0.8356  \\ 
  1.00  & 1.3187 &  1.1874  & 1.1085  & 1.0711  & 1.0823  & 1.1317 &  1.2263  \\ 
  1.10  & 1.8700 &  1.7116  & 1.6154  & 1.5669  & 1.5763  & 1.6335 &  1.7522  \\ 
  1.20  & 2.5858 &  2.3996  & 2.2789  & 2.2200  & 2.2263  & 2.2983 &  2.4322  \\ 
  1.30  & 3.4896 &  3.2711  & 3.1278  & 3.0626  & 3.0655  & 3.1506 &  3.2983  \\ 
 \end{tabular} 
 \caption{Cross sections for the different RVR-RVL couplings normalized to the SemiElMinus Standard Model Cross section (8.261 pb)} 
\end{table}
From this table can be seen that the cross section increases when the real component of $V_L$ gets larger. The value of the right-handed vector coupling has only a minor influence on the cross section.

\subsubsection{Relative increase visible in XS, but not in kinematic distributions}
Since the observed model cannot represent physics at values larger than 1, one option is to look at specific fixed values of the real part of the left-handed and right-handed vector couplings. 
A proportional change in both of these coupling parameters should change the cross section values, but the kinematic should remain unchanged. Therefore the following configurations will be investigated:
\begin{eqnarray*}
  Re(V_L) = 0.5 ~~~ \& ~~~ Re(V_R) = 0.5 \rightarrow ~2.07115~ pb\\
  Re(V_L) = 1.0 ~~~ \& ~~~ Re(V_R) = 1.0 \rightarrow ~33.1479~ pb\\
  Re(V_L) = 2.0 ~~~ \& ~~~ Re(V_R) = 2.0 \rightarrow ~530.027~ pb
\end{eqnarray*}
From the above numbers is clear that the Cross section becomes very large when the two coupling parameters increase.
This can be understood quite easily since the second option allows much more decay options since the top quarks can decay both through the left-handed and the right-handed vector coupling side of the interaction vertex. The width of this configuration is not equal to the width of the Standard Model expectation and hence does not correspond to an actual physical solution. It is merely seen as a test of the model since the kinematics of the interaction should not differ.\\
\\
Looking at these plots clearly indicates that the kinematics doesn't change at all.\\
Hence the created FeynRules model is able to deal in a correct way with these coupling parameters larger than 1.\\
\\
These MadGraph files have been created and can be found in:
\begin{eqnarray*}
  /user/aolbrech/AnomalousCouplings/MadGraph\_v155/MassiveLeptons/\\ MadGraph5\_v1\_5\_5/Wtb\_ttbarSemiElMinus/RelativeChange
\end{eqnarray*}

The distributions shown in this subsection are for fixed Jet Pt Cut value, set to 0. Also no Pt cut on the lepton was applied. Both coupling parameters have been changed proportionally.
\begin{center}
\includegraphics[width = 0.32 \textwidth]{../April2014/KinematicPlots_RelativeChange/StackCanvas_CosTheta_JetPt0.png}
\includegraphics[width = 0.32 \textwidth]{../April2014/KinematicPlots_RelativeChange/StackCanvas_EventContent_JetPt0.png}
\includegraphics[width = 0.32 \textwidth]{../April2014/KinematicPlots_RelativeChange/StackCanvas_HadronicBMass_JetPt0.png}
\includegraphics[width = 0.32 \textwidth]{../April2014/KinematicPlots_RelativeChange/StackCanvas_HadronicBPt_JetPt0.png}
\includegraphics[width = 0.32 \textwidth]{../April2014/KinematicPlots_RelativeChange/StackCanvas_HadronicBR_JetPt0.png}
\includegraphics[width = 0.32 \textwidth]{../April2014/KinematicPlots_RelativeChange/StackCanvas_HadronicBTheta_JetPt0.png}
\includegraphics[width = 0.32 \textwidth]{../April2014/KinematicPlots_RelativeChange/StackCanvas_LeptonicBMass_JetPt0.png}
\includegraphics[width = 0.32 \textwidth]{../April2014/KinematicPlots_RelativeChange/StackCanvas_LeptonicBPt_JetPt0.png}
\includegraphics[width = 0.32 \textwidth]{../April2014/KinematicPlots_RelativeChange/StackCanvas_LeptonicBR_JetPt0.png}
\includegraphics[width = 0.32 \textwidth]{../April2014/KinematicPlots_RelativeChange/StackCanvas_LeptonicBTheta_JetPt0.png}
\includegraphics[width = 0.32 \textwidth]{../April2014/KinematicPlots_RelativeChange/StackCanvas_LeptonId_JetPt0.png}
\includegraphics[width = 0.32 \textwidth]{../April2014/KinematicPlots_RelativeChange/StackCanvas_LeptonMass_JetPt0.png}
\includegraphics[width = 0.32 \textwidth]{../April2014/KinematicPlots_RelativeChange/StackCanvas_LeptonPt_JetPt0.png}
\includegraphics[width = 0.32 \textwidth]{../April2014/KinematicPlots_RelativeChange/StackCanvas_LeptonR_JetPt0.png}
\includegraphics[width = 0.32 \textwidth]{../April2014/KinematicPlots_RelativeChange/StackCanvas_LeptonTheta_JetPt0.png}
\includegraphics[width = 0.32 \textwidth]{../April2014/KinematicPlots_RelativeChange/StackCanvas_LightAntiQuarkId_JetPt0.png}
\includegraphics[width = 0.32 \textwidth]{../April2014/KinematicPlots_RelativeChange/StackCanvas_LightAntiQuarkMass_JetPt0.png}
\includegraphics[width = 0.32 \textwidth]{../April2014/KinematicPlots_RelativeChange/StackCanvas_LightAntiQuarkPt_JetPt0.png}
\includegraphics[width = 0.32 \textwidth]{../April2014/KinematicPlots_RelativeChange/StackCanvas_LightAntiQuarkR_JetPt0.png}
\includegraphics[width = 0.32 \textwidth]{../April2014/KinematicPlots_RelativeChange/StackCanvas_LightAntiQuarkTheta_JetPt0.png}
\includegraphics[width = 0.32 \textwidth]{../April2014/KinematicPlots_RelativeChange/StackCanvas_LightQuarkId_JetPt0.png}
\includegraphics[width = 0.32 \textwidth]{../April2014/KinematicPlots_RelativeChange/StackCanvas_LightQuarkMass_JetPt0.png}
\includegraphics[width = 0.32 \textwidth]{../April2014/KinematicPlots_RelativeChange/StackCanvas_LightQuarkPt_JetPt0.png}
\includegraphics[width = 0.32 \textwidth]{../April2014/KinematicPlots_RelativeChange/StackCanvas_LightQuarkR_JetPt0.png}
\includegraphics[width = 0.32 \textwidth]{../April2014/KinematicPlots_RelativeChange/StackCanvas_LightQuarkTheta_JetPt0.png}
\includegraphics[width = 0.32 \textwidth]{../April2014/KinematicPlots_RelativeChange/StackCanvas_TopMass_JetPt0.png}
\includegraphics[width = 0.32 \textwidth]{../April2014/KinematicPlots_RelativeChange/StackCanvas_TopProductionId_JetPt0.png}
\includegraphics[width = 0.32 \textwidth]{../April2014/KinematicPlots_RelativeChange/StackCanvas_WBosonMass_JetPt0.png}
\end{center}

\subsubsection{Model plots for fixed Pt Cut}
The distributions shown in this subsection are for fixed Jet Pt Cut value, set to 0. Also no Pt cut on the lepton was applied. In this case the real part of the left-handed vector coupling has been varied between $0.7$ and $1.3$ in steps of $0.1$ while the right-handed parameter has been fixed to its Standard Model value ($0.0$).
\begin{center}
\includegraphics[width = 0.32 \textwidth]{../April2014/KinematicPlots_CoeffLargerThan1/StackCanvas_CosTheta_JetPt0.png}
\includegraphics[width = 0.32 \textwidth]{../April2014/KinematicPlots_CoeffLargerThan1/StackCanvas_EventContent_JetPt0.png}
\includegraphics[width = 0.32 \textwidth]{../April2014/KinematicPlots_CoeffLargerThan1/StackCanvas_HadronicBMass_JetPt0.png}
\includegraphics[width = 0.32 \textwidth]{../April2014/KinematicPlots_CoeffLargerThan1/StackCanvas_HadronicBPt_JetPt0.png}
\includegraphics[width = 0.32 \textwidth]{../April2014/KinematicPlots_CoeffLargerThan1/StackCanvas_HadronicBR_JetPt0.png}
\includegraphics[width = 0.32 \textwidth]{../April2014/KinematicPlots_CoeffLargerThan1/StackCanvas_HadronicBTheta_JetPt0.png}
\includegraphics[width = 0.32 \textwidth]{../April2014/KinematicPlots_CoeffLargerThan1/StackCanvas_LeptonicBMass_JetPt0.png}
\includegraphics[width = 0.32 \textwidth]{../April2014/KinematicPlots_CoeffLargerThan1/StackCanvas_LeptonicBPt_JetPt0.png}
\includegraphics[width = 0.32 \textwidth]{../April2014/KinematicPlots_CoeffLargerThan1/StackCanvas_LeptonicBR_JetPt0.png}
\includegraphics[width = 0.32 \textwidth]{../April2014/KinematicPlots_CoeffLargerThan1/StackCanvas_LeptonicBTheta_JetPt0.png}
\includegraphics[width = 0.32 \textwidth]{../April2014/KinematicPlots_CoeffLargerThan1/StackCanvas_LeptonId_JetPt0.png}
\includegraphics[width = 0.32 \textwidth]{../April2014/KinematicPlots_CoeffLargerThan1/StackCanvas_LeptonMass_JetPt0.png}
\includegraphics[width = 0.32 \textwidth]{../April2014/KinematicPlots_CoeffLargerThan1/StackCanvas_LeptonPt_JetPt0.png}
\includegraphics[width = 0.32 \textwidth]{../April2014/KinematicPlots_CoeffLargerThan1/StackCanvas_LeptonR_JetPt0.png}
\includegraphics[width = 0.32 \textwidth]{../April2014/KinematicPlots_CoeffLargerThan1/StackCanvas_LeptonTheta_JetPt0.png}
\includegraphics[width = 0.32 \textwidth]{../April2014/KinematicPlots_CoeffLargerThan1/StackCanvas_LightAntiQuarkId_JetPt0.png}
\includegraphics[width = 0.32 \textwidth]{../April2014/KinematicPlots_CoeffLargerThan1/StackCanvas_LightAntiQuarkMass_JetPt0.png}
\includegraphics[width = 0.32 \textwidth]{../April2014/KinematicPlots_CoeffLargerThan1/StackCanvas_LightAntiQuarkPt_JetPt0.png}
\includegraphics[width = 0.32 \textwidth]{../April2014/KinematicPlots_CoeffLargerThan1/StackCanvas_LightAntiQuarkR_JetPt0.png}
\includegraphics[width = 0.32 \textwidth]{../April2014/KinematicPlots_CoeffLargerThan1/StackCanvas_LightAntiQuarkTheta_JetPt0.png}
\includegraphics[width = 0.32 \textwidth]{../April2014/KinematicPlots_CoeffLargerThan1/StackCanvas_LightQuarkId_JetPt0.png}
\includegraphics[width = 0.32 \textwidth]{../April2014/KinematicPlots_CoeffLargerThan1/StackCanvas_LightQuarkMass_JetPt0.png}
\includegraphics[width = 0.32 \textwidth]{../April2014/KinematicPlots_CoeffLargerThan1/StackCanvas_LightQuarkPt_JetPt0.png}
\includegraphics[width = 0.32 \textwidth]{../April2014/KinematicPlots_CoeffLargerThan1/StackCanvas_LightQuarkR_JetPt0.png}
\includegraphics[width = 0.32 \textwidth]{../April2014/KinematicPlots_CoeffLargerThan1/StackCanvas_LightQuarkTheta_JetPt0.png}
\includegraphics[width = 0.32 \textwidth]{../April2014/KinematicPlots_CoeffLargerThan1/StackCanvas_TopMass_JetPt0.png}
\includegraphics[width = 0.32 \textwidth]{../April2014/KinematicPlots_CoeffLargerThan1/StackCanvas_TopProductionId_JetPt0.png}
\includegraphics[width = 0.32 \textwidth]{../April2014/KinematicPlots_CoeffLargerThan1/StackCanvas_WBosonMass_JetPt0.png}
\end{center}

\subsubsection{Model plots for varying Pt Cut}
Script adapted, but error obtained when running the python script ...\\
Worked when everything was copied to the TestDir ...\\
Maybe the use of nohup gives the problem ...

%°°°°°°°°°°°°°°°°°°°°°°°°°°°°°°°°°°°°°°°°°°°°°°°°°°°°°°°°°°°°°°°°°°°°°°°°°°°°°°°°°°°°°°°°°°°°°°°°°°°°°°°°°°°°°°°°°°°°°°°°°°°°°°°°°°°°°°°°°°°°°°°°°°°°°°°°°°°°°°
\chapter{Event selection}
%°°°°°°°°°°°°°°°°°°°°°°°°°°°°°°°°°°°°°°°°°°°°°°°°°°°°°°°°°°°°°°°°°°°°°°°°°°°°°°°°°°°°°°°°°°°°°°°°°°°°°°°°°°°°°°°°°°°°°°°°°°°°°°°°°°°°°°°°°°°°°°°°°°°°°°°°°°°°°°

\textcolor{red}{\textbf{Not correctly considering tau's for the moment ...}}
\section{Choice of b-tag requirements}

\subsection{Signal vs background comparison}
Since the complete event should be reconstructed as accurately as possible the 'signal' is represented by the case that all four particles are matched correctly while the 'background' events are the events for which at least one particle is matched wrongly.
Events which are not matched using the JetPartonMatching algorithm (currently ptOrderedMinDist with dR of 0.3 is used) are not included in either of these two variables and are shown separately in the tables found below.\\
First the different b-tag options are compared for the four possible combinations which are still allowed, namely the interchange of both the two light quarks and the two b-jets.

 \begin{table}[!h] 
 \begin{tabular}{c|c|c|c|c|c} 
 \textbf{Option} (no $\chi^{2}$ $m_{lb}$) & all 4 correct   & $\geq$ 1 wrong       & $\frac{s}{\sqrt{b}}$ & $\frac{s}{b}$ & non-matched \\ \hline 
2 L b-tags,                & 55967 & 50416 & 249.257 & 1.1101 & 225217 \\ 
2 M b-tags,              & 49983 & 32012 & 279.361 & 1.56138 & 146633 \\ 
2 M b-tags, light L-veto & 39661 & 27751 & 238.081 & 1.42917 & 118389 \\ 
2 T b-tags,              & 31444 & 16061 & 248.114 & 1.95779 & 78062 \\ 
2 T b-tags, light M-veto & 29160 & 15585 & 233.579 & 1.87103 & 73570 \\ 
2 T b-tags, light L-veto & 23159 & 14093 & 195.082 & 1.6433 & 59997 \\ 
 \end{tabular} 
 \caption{Overview of correct and wrong reconstructed events for the different b-tags without the use of a $\chi^{2}$ $m_{lb}$ - $m_{qqb}$ method} 
 \end{table} 
 
 \begin{table}[!h] 
 \begin{tabular}{c|c|c|c|c|c} 
 \textbf{Option} (no $\chi^{2}$ $m_{lb}$) & 2 b's good      & $\geq$ 1 b wrong     & $\frac{s}{\sqrt{b}}$ & $\frac{s}{b}$ & non-matched \\ \hline 
2 L b-tags,                & 78896 & 27487 & 475.873 & 2.8703 & 225217 \\ 
2 M b-tags,              & 70073 & 11922 & 641.765 & 5.87762 & 146633 \\ 
2 M b-tags, light L-veto & 58778 & 8634 & 632.57 & 6.80774 & 118389 \\ 
2 T b-tags,              & 43804 & 3701 & 720.036 & 11.8357 & 78062 \\ 
2 T b-tags, light M-veto & 41617 & 3128 & 744.11 & 13.3047 & 73570 \\ 
2 T b-tags, light L-veto & 34908 & 2344 & 721.018 & 14.8925 & 59997 \\ 
 \end{tabular} 
 \caption{Overview of correct and wrong reconstructed b-jets for the different b-tags without the use of a $\chi^{2}$ $m_{lb}$ - $m_{qqb}$ method} 
 \end{table} 
 
 \begin{table}[!h] 
 \begin{tabular}{c|c|c|c|c|c} 
 \textbf{Option} (no $\chi^{2}$ $m_{lb}$) & 2 light good    & $\geq$ 1 light wrong & $\frac{s}{\sqrt{b}}$ & $\frac{s}{b}$ & non-matched \\ \hline 
2 L b-tags,                & 58707 & 47676 & 268.869 & 1.23137 & 225217 \\ 
2 M b-tags,              & 50676 & 31319 & 286.351 & 1.61806 & 146633 \\ 
2 M b-tags, light L-veto & 40361 & 27051 & 245.398 & 1.49203 & 118389 \\ 
2 T b-tags,              & 31690 & 15815 & 251.993 & 2.00379 & 78062 \\ 
2 T b-tags, light M-veto & 29466 & 15279 & 238.382 & 1.92853 & 73570 \\ 
2 T b-tags, light L-veto & 23456 & 13796 & 199.7 & 1.7002 & 59997 \\ 
 \end{tabular} 
 \caption{Overview of correct and wrong reconstructed light jets for the different b-tags without the use of a $\chi^{2}$ $m_{lb}$ - $m_{qqb}$ method} 
 \end{table} 
 
Above tables show a clear improvement for the 2 Tight b-tag case since suddenly the $\frac{s}{b}$ value goes up to almost 2. An additional benefit of the 2 Tight bt-tag case is that it as well will take care of a large part of the process backgrounds. Hence the motivation for selecting this b-tag option.\\
Comparing the normal 2 Tight b-tag case, where the light jets are defined as not being Tight b-tagged jets, against the two possibilities using a light-jet veto indicates no motivation to go for the light-veto option.\\
\\
However the second table, showing only the reconstruction efficiency of the b-jets, shows an improvement when using a light-jet veto.\\
This means that, even if the number of selected events gets lower, the percentage of correct events does improve when asking for a light-jet veto since it ensures that mistagged b-jet events doesn't by mistake get identified as light jets. So events with a b-jet with a too low CSV discriminant now don't get selected anymore because these so-called light-jets don't survive the veto cut.\\
\\
But, as confirmed by the last table, the efficiency of the light-jet reconstruction shows no distinct improvement. So for some reason the light-jets which are chosen with this veto method are not by definition the actual light-jets.
%**************************************************

\section{Use of $m_{lb}$ $\chi^{2}$ method for selecting the correct b-jets}
 \begin{table}[!h] 
 \begin{tabular}{c|c|c|c|c|c} 
 \textbf{Option} (with $\chi^{2}$ $m_{lb}$) & all 4 correct      & $\geq$ 1 wrong  & $\frac{s}{\sqrt{b}}$ & $\frac{s}{b}$ & non-matched \\ \hline 
2 L b-tags,                & 51055 & 55328 & 217.053 & 0.92277 & 514406 \\ 
2 M b-tags,              & 45664 & 36331 & 239.572 & 1.25689 & 538794 \\ 
2 M b-tags, light L-veto & 36271 & 31141 & 205.539 & 1.16473 & 553377 \\ 
2 T b-tags,              & 28580 & 18925 & 207.752 & 1.51017 & 573284 \\ 
2 T b-tags, light M-veto & 26513 & 18232 & 196.355 & 1.4542 & 576044 \\ 
2 T b-tags, light L-veto & 21073 & 16179 & 165.673 & 1.30249 & 583537 \\ 
 \end{tabular} 
 \caption{Overview of correct and wrong reconstructed events for the different b-tags when a $\chi^{2}$ $m_{lb}$ - $m_{qqb}$ method is applied} 
 \end{table}
 
 \begin{table}[!h] 
 \begin{tabular}{c|c|c|c|c|c} 
 \textbf{Option} (with $\chi^{2}$ $m_{lb}$) & Correct b's & Wrong b's & $\frac{s}{\sqrt{b}}$ & $\frac{s}{b}$ & Correct option exists \\ \hline 
2 L b-tags,                & 66882 & 12014 & 610.19 & 5.56701 & 78896 \\ 
2 M b-tags,              & 59520 & 10553 & 579.395 & 5.6401 & 70073 \\ 
2 M b-tags, light L-veto & 49556 & 9222 & 516.04 & 5.37367 & 58778 \\ 
2 T b-tags,              & 37013 & 6791 & 449.146 & 5.4503 & 43804 \\ 
2 T b-tags, light M-veto & 35015 & 6602 & 430.94 & 5.3037 & 41617 \\ 
2 T b-tags, light L-veto & 29139 & 5769 & 383.64 & 5.05096 & 34908 \\ 
 \end{tabular} 
 \caption{Overview of the number of times the correct b-jet combination is chosen when using a $\chi^{2}$ $m_{lb}$ - $m_{qqb}$ method} 
 \end{table}
 
The two tables in this subsection require a different interpretation. The first one is actually a combined test of the $\chi^{2}$ $m_{lb}$ - $m_{qqb}$ method and the optimal b-tag choice while the second one is merely a performance check of the mlb method. \\
This second table had to be added since the first table can't be directly compared against the tables in the previous subsection since currently only one b-jet combination is left while in the previous case an iteration between the different b-jets was allowed. So the numbers will be lower by definition when the $\chi^{2}$ $m_{lb}$ - $m_{qqb}$ method is applied.\\
Therefore the second table is relevant in order to select whether some clear gain can be obtained when applying this method since it represents the number of times the correct b-jet combination. If this percentage is higher than 50$\%$, which is the case, an improvement is obtained compared to an iteration between the two possible combinations.\\
\\
The first table indicates again that no real difference is found between the different b-tag options, but that as soon as 2 Tight b-tags are applied the $\frac{s}{b}$ improves slightly. Also the second table shows no difference in efficiency between the different b-tag options, but shows however that the use of this $\chi^{2}$ $m_{lb}$ - $m_{qqb}$ method significantly enhances the correct choice of the b-jet combination. In about 84$\%$ of the cases the correct b-jet combination is chosen. 
%**************************************************

\section{Histograms for event selection choice}
\begin{center}

\begin{figure}[!h]
 \includegraphics[width = 0.32 \textwidth]{/home/annik/Documents/Vub/PhD/ThesisSubjects/AnomalousCouplings/EventSelectionChoice_June2014/StCosTheta_BeforeEvtSel.png}
\includegraphics[width = 0.32 \textwidth]{/home/annik/Documents/Vub/PhD/ThesisSubjects/AnomalousCouplings/EventSelectionChoice_June2014/StCosThetaNoBTag.png}
\includegraphics[width = 0.32 \textwidth]{/home/annik/Documents/Vub/PhD/ThesisSubjects/AnomalousCouplings/EventSelectionChoice_June2014/StCosThetaLCSV.png}
\includegraphics[width = 0.32 \textwidth]{/home/annik/Documents/Vub/PhD/ThesisSubjects/AnomalousCouplings/EventSelectionChoice_June2014/StCosThetaMCSV.png}
\includegraphics[width = 0.32 \textwidth]{/home/annik/Documents/Vub/PhD/ThesisSubjects/AnomalousCouplings/EventSelectionChoice_June2014/StCosThetaTCSV.png}
\caption{The $\cos \theta^{*}$ distribution for the different b-tag options (all of them imply double b-tag), which are not really influenced by the application of a b-tag. The only relevant distortion is caused by the event selection which is applied.}
\end{figure}

\begin{figure}[!h]
 \includegraphics[width = 0.32 \textwidth]{/home/annik/Documents/Vub/PhD/ThesisSubjects/AnomalousCouplings/EventSelectionChoice_June2014/CorrectBHadrCSVDiscr.png}
\includegraphics[width = 0.32 \textwidth]{/home/annik/Documents/Vub/PhD/ThesisSubjects/AnomalousCouplings/EventSelectionChoice_June2014/CorrectBLeptCSVDiscr.png}
\includegraphics[width = 0.32 \textwidth]{/home/annik/Documents/Vub/PhD/ThesisSubjects/AnomalousCouplings/EventSelectionChoice_June2014/CorrectQuark1CSVDiscr.png}
\includegraphics[width = 0.32 \textwidth]{/home/annik/Documents/Vub/PhD/ThesisSubjects/AnomalousCouplings/EventSelectionChoice_June2014/CorrectQuark2CSVDiscr.png}
\caption{Distribution of the CSV discriminant for the different correctly matched quark-jet pairs. The value $-2$ is used to represent a non-matched jet. As expected the b-jets have a large peak at $1$, so the Tight b-tag of 0.898 will not take away to many correct b-jets. The problem in the matching is clearly represented in distribution of the second quark which is only reconstructed in less than half of the cases.}
\end{figure}

\begin{figure}[!h]
\includegraphics[width = 0.32 \textwidth]{/home/annik/Documents/Vub/PhD/ThesisSubjects/AnomalousCouplings/EventSelectionChoice_June2014/CSVDiscrLCSVLightJets.png}
\caption{ Distribution of the CSV discriminant of the selected light jets (all of them). \textbf{Add same histograms for Medium and Tight option, this will show how many of the light jets actually have a large CSV discriminant (maybe only focus on the two/three leading light jets) }}
\end{figure}

\begin{figure}[!h]
\includegraphics[width = 0.32 \textwidth]{/home/annik/Documents/Vub/PhD/ThesisSubjects/AnomalousCouplings/EventSelectionChoice_June2014/JetTypeLCSVLightJets.png}
\caption{Jet type of the of the light jets (all of them). The value $25$ means that the found light jet couldn't be matched to a Parton witht he JetPartonMatching method. \textbf{Same for M and T ... ?} }
\end{figure}

\begin{figure}[!h]
\includegraphics[width = 0.32 \textwidth]{/home/annik/Documents/Vub/PhD/ThesisSubjects/AnomalousCouplings/EventSelectionChoice_June2014/JetTypeLCSV.png}
\includegraphics[width = 0.32 \textwidth]{/home/annik/Documents/Vub/PhD/ThesisSubjects/AnomalousCouplings/EventSelectionChoice_June2014/JetTypeMCSV.png}
\includegraphics[width = 0.32 \textwidth]{/home/annik/Documents/Vub/PhD/ThesisSubjects/AnomalousCouplings/EventSelectionChoice_June2014/JetTypeTCSV.png}
\caption{Jet type of the b-tagged jets (all of them) with the same convention for the non-matched jets. }
\end{figure}

\begin{figure}[!h]
\includegraphics[width = 0.32 \textwidth]{/home/annik/Documents/Vub/PhD/ThesisSubjects/AnomalousCouplings/EventSelectionChoice_June2014/MlbMass.png}
\includegraphics[width = 0.32 \textwidth]{/home/annik/Documents/Vub/PhD/ThesisSubjects/AnomalousCouplings/EventSelectionChoice_June2014/MqqbMass.png}
\includegraphics[width = 0.32 \textwidth]{/home/annik/Documents/Vub/PhD/ThesisSubjects/AnomalousCouplings/EventSelectionChoice_June2014/1July2014/MlbMqqbCorrectAll.png}
\caption{The first two histograms show the mass distribution for the correctly matched and reconstructed particles. A gaussian fit is applied in order to obtain both the $m_{lb}$ and $m_{qqb}$ mass and sigma. The last histogram shows the 2D behavior of these distributions. }
\end{figure}
\end{center}
%**************************************************

\newpage
\section{Considering 2 or 3 light jets}
In order to try to improve the signal efficiency it can be considered to add a third light jet to the particles which have to be taken into account for the jet selection. Adding this third jet will however result in 4 additional combinations which have to be considered so this method will only benefit when the $\chi^{2}$ $m_{lb}$ - $m_{qqb}$ method is applied. Since MadWeight uses so much CPU sending the $6$ possible combinations to MadWeight will not be beneficial.
\subsection{Event selection numbers comparison}
 \begin{table}[!h] 
 \begin{tabular}{c|c|c|c|c} 
\textbf{Option} (no $\chi^{2}$ $m_{lb}$) & chosen jets are correct ($\%$)       & $\frac{s}{b}$ & 3rd jet is correct ($\%$) \\ \hline 
 5 jet case, 2 T b-tags              & 76.2852 & 3.21677 & 70.244\\ 
 4 jet case, 2 T b-tags              & 66.9091 & 2.02198 & X \\ 
 \end{tabular} 
\caption{Overview of correct and wrong reconstructed events for the different b-tags without the use of a $\chi^{2}$ $m_{lb}$ - $m_{qqb}$ method} 
 \end{table} 
 
 \begin{table}[!h] 
 \begin{tabular}{c|c|c|c|c} 
\textbf{Option} (no $\chi^{2}$ $m_{lb}$) & 2 b's chosen correctly ($\%$)        & $\frac{s}{b}$ & 3rd jet is correct ($\%$) \\ \hline 
 5 jet case, 2 T b-tags              & 90.5014 & 9.52784 & 64.5863\\ 
 4 jet case, 2 T b-tags              & 92.3745 & 12.1138 & X \\ 
 \end{tabular} 
\caption{Overview of correct and wrong reconstructed b-jets for the different b-tags without the use of a $\chi^{2}$ $m_{lb}$ - $m_{qqb}$ method} 
 \end{table} 
 
 \begin{table}[!h] 
 \begin{tabular}{c|c|c|c|c} 
\textbf{Option} (no $\chi^{2}$ $m_{lb}$) & chosen light jets are correct ($\%$) & $\frac{s}{b}$ & 3rd jet is correct ($\%$) \\ \hline 
 5 jet case, 2 T b-tags              & 79.2892 & 3.8284 & 70.2241\\ 
 4 jet case, 2 T b-tags              & 67.4722 & 2.0743 & X \\ 
 \end{tabular} 
\caption{Overview of correct and wrong reconstructed light jets for the different b-tags without the use of a $\chi^{2}$ $m_{lb}$ - $m_{qqb}$ method} 
 \end{table} 

These three tables look at either a pure 5 jet case or a pure 4 jet case and comparing the numbers in each table with eachother is probably not extremely relevant. This because in the pure 5 jets case the matching requirement is loosened to two out of the three chosen light jets correctly matching with the partons. So in $1$ out of $3$ possibilities the so-called correct event will not be correct resulting in too high numbers for this case.\\
\\
The first column gives the percentage how often the chosen jets are indeed the correct partons, hence in the 5-jet case the number of times the 5 possible jets match with the 4 correct partons. In the 4-jet case it implies that the four chosen jets are matched correctly with the 4 partons. The second column gives a similar value since the signal is defined as the number of times the matching was done correctly for the four partons while the background stands for the events where one of the matching is not succesful. The third column checks how often the third jet is one of the two correct light jets and compares it against the number in the first column. So it represents the number of times adding the third jet results in an improvement of the event reconstruction. \\
\\
The numbers which are relevant in these tables are exactly these two last columns. These numbers show that in about $70 \%$ of the cases the third jet is actually one of the correct quarks. Therefore it can be decided from these numbers that in quite a lot of events, an improvement can be obtained when this third light jet is considered as well.

\subsection{Mlb-algorithm numbers comparison}
  \begin{table}[!h] 
 \begin{tabular}{c|c|c|c|c} 
\multirow{2}{*}{\textbf{Option} (with $\chi^{2}$ $m_{lb}$)} & 4 chosen jets & $\frac{s}{b}$ & 3rd jet is one of the & 3rd jet is chosen \\ & are correct ($\%$)    & 	             & 2 correct light jets ($\%$) &  and correct ($\%$)	  \\ \hline 
 5 jet case,      2 T b-tags              & 73.2413 & 2.73711 & 21.3059 & 89.2513 \\ 
 4 jet case,      2 T b-tags              & 76.9258 & 3.33384 & 0 & -nan \\ 
 Pure 5 jet case, 2 T b-tags              & 65.6683 & 1.91276 & 74.3984 & 89.2513 \\ 
 \end{tabular} 
 \caption{Overview of correct and wrong reconstructed events for the different b-tags when a $\chi^{2}$ $m_{lb}$ - $m_{qqb}$ method is applied} 
 \end{table} 
 
 \begin{table}[!h] 
 \begin{tabular}{c|c|c|c|c} 
 \textbf{Option} (with $\chi^{2}$ $m_{lb}$) & \% b's correct   & $\frac{s}{b}$ &  &  \\ \hline 
 5 jet case,      2 T b-tags              & 90.3229 & 9.33364 & 0 & 0 \\ 
 4 jet case,      2 T b-tags              & 90.7656 & 9.82911 & 0 & 0 \\ 
 Pure 5 jet case, 2 T b-tags              & 89.9057 & 8.90659 & 0 & 0 \\ 
 \end{tabular} 
 \caption{Overview of the number of times the correct b-jet combination is chosen when using a $\chi^{2}$ $m_{lb}$ - $m_{qqb}$ method} 
 \end{table} 
When the $m_{lb}$ method is applied, the two first columns in the given tables represent similar quantities with the only difference that now the 4 jets which are actually chosen by the $\chi^{2}$ $m_{lb}$ - $m_{qqb}$ method are considered. 
In this case the third column represents the number of times the third jet is chosen when the two light jets are matched correctly, implying that the third jet is one of the correct ones and that the second light jet is also correctly matched. The final column only looks at the third jet and puts no requirement on the matching of the second light jet. So it compares the number of times one of the chosen jets is the third jet and in how many cases this chosen third jet is one of the correct partons.
%**************************************************
 
\section{Studying optimal cut on $\chi^{2}$ value}
The tables shown here indicate that the application of a cut on the $\chi^{2}$ value of the $m_{lb}$ - $m_{qqb}$ method doesn't change the rates of good and wrong chosen events. Therefore it will only reduce the number of selected events, and hence reduce the needed CPU time, but still keep an as pure sample as obtained without any cut.\\
Since there is no real difference visible between setting the cut value to $3$ or $5$ it is advisable to use the cut value of $3$ to reduce the number of selected events.
 
\subsection{$\chi^{2}$ required to be smaller than $5$}
  \begin{table}[!h] 
 \begin{tabular}{c|c|c|c|c} 
\multirow{2}{*}{\textbf{Option} (with $\chi^{2}$ $m_{lb}$)} & 4 chosen jets & $\frac{s}{b}$ & 3rd jet is one of the & 3rd jet is chosen \\ & are correct ($\%$)    & 	             & 2 correct light jets ($\%$) &  and correct ($\%$)	  \\ \hline 
 5 jet case,      2 T b-tags              & 73.1485 & 2.72419 & 21.5302 & 89.0292 \\ 
 4 jet case,      2 T b-tags              & 76.89 & 3.32712 & 0 & -nan \\ 
 Pure 5 jet case, 2 T b-tags              & 65.5254 & 1.90068 & 74.5263 & 89.0292 \\ 
 \end{tabular} 
 \caption{Overview of correct and wrong reconstructed events for the different b-tags when a $\chi^{2}$ $m_{lb}$ - $m_{qqb}$ method is applied} 
 \end{table} 
 
 \begin{table}[!h] 
 \begin{tabular}{c|c|c|c|c} 
 \textbf{Option} (with $\chi^{2}$ $m_{lb}$) & \% b's correct   & $\frac{s}{b}$ &  &  \\ \hline 
 5 jet case,      2 T b-tags              & 90.3451 & 9.35743 & 0 & 0 \\ 
 4 jet case,      2 T b-tags              & 90.8573 & 9.93767 & 0 & 0 \\ 
 Pure 5 jet case, 2 T b-tags              & 89.8465 & 8.84884 & 0 & 0 \\ 
 \end{tabular} 
 \caption{Overview of the number of times the correct b-jet combination is chosen when using a $\chi^{2}$ $m_{lb}$ - $m_{qqb}$ method} 
 \end{table} 
 
 \subsection{$\chi^{2}$ required to be smaller than $3$}
 \begin{table}[!h] 
 \begin{tabular}{c|c|c|c|c} 
\multirow{2}{*}{\textbf{Option} (with $\chi^{2}$ $m_{lb}$)} & 4 chosen jets & $\frac{s}{b}$ & 3rd jet is one of the & 3rd jet is chosen \\ & are correct ($\%$)    & 	             & 2 correct light jets ($\%$) &  and correct ($\%$)	  \\ \hline 
 5 jet case,      2 T b-tags              & 72.8939 & 2.68921 & 21.7785 & 88.6241 \\ 
 4 jet case,      2 T b-tags              & 77.0226 & 3.3521 & 0 & -nan \\ 
 Pure 5 jet case, 2 T b-tags              & 64.7571 & 1.83745 & 74.192 & 88.6241 \\ 
 \end{tabular} 
 \caption{Overview of correct and wrong reconstructed events for the different b-tags when a $\chi^{2}$ $m_{lb}$ - $m_{qqb}$ method is applied} 
 \end{table} 
 
 \begin{table}[!h] 
 \begin{tabular}{c|c|c|c|c} 
 \textbf{Option} (with $\chi^{2}$ $m_{lb}$) & \% b's correct   & $\frac{s}{b}$ &  &  \\ \hline 
 5 jet case,      2 T b-tags              & 90.1586 & 9.16114 & 0 & 0 \\ 
 4 jet case,      2 T b-tags              & 90.8239 & 9.89789 & 0 & 0 \\ 
 Pure 5 jet case, 2 T b-tags              & 89.4472 & 8.47619 & 0 & 0 \\ 
 \end{tabular} 
 \caption{Overview of the number of times the correct b-jet combination is chosen when using a $\chi^{2}$ $m_{lb}$ - $m_{qqb}$ method} 
 \end{table} 
 
\subsection{$\chi^{2}$ required to be smaller than $1$}
 \begin{table}[!h] 
 \begin{tabular}{c|c|c|c|c} 
\multirow{2}{*}{\textbf{Option} (with $\chi^{2}$ $m_{lb}$)} & 4 chosen jets & $\frac{s}{b}$ & 3rd jet is one of the & 3rd jet is chosen \\ & are correct ($\%$)    & 	             & 2 correct light jets ($\%$) &  and correct ($\%$)	  \\ \hline 
 5 jet case,      2 T b-tags              & 72.4364 & 2.62798 & 21.6013 & 87.5833 \\ 
 4 jet case,      2 T b-tags              & 77.3908 & 3.42298 & 0 & -nan \\ 
 Pure 5 jet case, 2 T b-tags              & 62.9538 & 1.69934 & 73.7615 & 87.5833 \\ 
 \end{tabular} 
 \caption{Overview of correct and wrong reconstructed events for the different b-tags when a $\chi^{2}$ $m_{lb}$ - $m_{qqb}$ method is applied} 
 \end{table} 
 
 \begin{table}[!h] 
 \begin{tabular}{c|c|c|c|c} 
 \textbf{Option} (with $\chi^{2}$ $m_{lb}$) & \% b's correct   & $\frac{s}{b}$ &  &  \\ \hline 
 5 jet case,      2 T b-tags              & 90.233 & 9.23861 & 0 & 0 \\ 
 4 jet case,      2 T b-tags              & 90.9792 & 10.0854 & 0 & 0 \\ 
 Pure 5 jet case, 2 T b-tags              & 89.5385 & 8.55882 & 0 & 0 \\ 
 \end{tabular} 
 \caption{Overview of the number of times the correct b-jet combination is chosen when using a $\chi^{2}$ $m_{lb}$ - $m_{qqb}$ method} 
 \end{table} 
%**************************************************

\section{Influence of using $p_T$ cuts suggested by the TOP reference selection Twiki}

The TOP Reference Selection Twiki, and the different subgroup Twikis, suggest to use different event selection requirements than used before. The values suggested can be found in the table below, together with the values which were used for producing the tables and figures given above.\\
In this section the different results will be compared for these two event selections in order to ensure that the influence of this event selection can be ignored. If this is not the case, all the above tables have to be replaced and updated (which will however happen on a larger time-scale since these will be the correct values which will be used further in this analysis). The main goal is to quickly check whether the conclusion obtained from the above figures and tables still remain valid.
\begin{table}[!h]
 \centering
 \begin{tabular}{c|c|c}
                      & Old values & TOP RefSel values \\
   \hline
   selected jets      &   40 GeV   &       30 GeV      \\
   selected muons     &   25 GeV   &       26 GeV      \\
   selected electrons &   32 GeV   &       30 GeV      \\
   veto muons         &   10 GeV   &       10 GeV      \\
   veto electrons     &   10 GeV   &       20 GeV      
 \end{tabular}
\end{table}

In the following table the number of selected events after the different event selection cuts which are applied in this analysis can be found. The left-handed columns contain the information when the old $p_T$ cuts are applied while the right-handed columns gives the number of selected events when the recommendations of the Top Reference Selection Twiki are followed.\\
\\
From the comparison of the two columns can be seen that lowering the $p_T$ cut on the selected jets significantly improves the percentage of selected events, especially because of the improved selection efficiency for the third and fourth jet. This can easily be understood from the $p_T$ distribution histogram for the different leading jets which can be found below.
These distributions show that in the case of the fourth jet, moving the $p_T$ cut from $30$ GeV to $40$ GeV cuts away the largest percentage of this fourth jet since its distribution is peaked at a lower $p_T$ value. The influence is much lower for the first and second jet because their distrubition is maximal around $100$ GeV so changing this $p_T$ cut only affects the left side of the tail of the distribution.\\
\begin{table}
\caption{Event selection table before (left) and after (right) the pT cuts were updated to the ones suggested by the TOP reference twiki, for Semi-elec channel $t\bar{t}+jets$. ($19600.8~pb^{-1}$ of int. lumi.)}
\centering
\begin{tabular}{|c|c|c|c|c|}
\hline
				& \multicolumn{2}{|c|}{Old $p_T$ cuts}	& \multicolumn{2}{|c|}{New $p_T$ cuts}	\\
\hline
preselected			& 1.91516e+07	& 		 	& 1.91516e+07	&	 		\\

trigged				& 3.52179e+06	&	18.4 $\%$	& 3.52179e+06	&	18.4 $\%$	\\

Good PV				& 3.52179e+06	&	100 $\%$	& 3.52179e+06	&	100 $\%$	\\

1 selected electron		& 2.76014e+06	&	78.4 $\%$	& 2.85992e+06	&	\textcolor{red}{81.2 $\%$}	\\

Veto muon			& 2.75335e+06	&	99.8 $\%$	& 2.8529e+06	&	99.8 $\%$	\\

Veto 2nd electron from Z-decay	& 2.7483e+06	&	99.8 $\%$	& 2.84766e+06	&	99.8 $\%$	\\

Conversion veto			& 2.7483e+06	&	100 $\%$	& 2.84766e+06	&	100 $\%$	\\

$\geq$ 1 jets			& 2.72998e+06	&	99.3 $\%$	& 2.84406e+06	&	\textcolor{red}{99.9 $\%$}	\\

$\geq$ 2 jets			& 2.50625e+06	&	91.8 $\%$	& 2.77442e+06	&	\textcolor{red}{97.6 $\%$}	\\

$\geq$ 3 jets			& 1.74245e+06	&	69.5 $\%$	& 2.34978e+06	&	\textcolor{red}{84.7 $\%$}	\\

$\geq$ 4 jets			& 0.753281e+06	&	43.2 $\%$	& 1.38732e+06	&	\textcolor{red}{59.0 $\%$}	\\

\hline
\end{tabular}
\end{table}

\begin{figure}[!h]
\includegraphics[width = 0.45 \textwidth]{/home/annik/Documents/Vub/PhD/ThesisSubjects/AnomalousCouplings/EventSelectionChoice_June2014/UpdatedPtCuts_30July2014/PtDistributionFirstJet_NewPtCuts_El.png}
\includegraphics[width = 0.45 \textwidth]{/home/annik/Documents/Vub/PhD/ThesisSubjects/AnomalousCouplings/EventSelectionChoice_June2014/UpdatedPtCuts_30July2014/PtDistributionSecondJet_NewPtCuts_El.png}\\
\includegraphics[width = 0.45 \textwidth]{/home/annik/Documents/Vub/PhD/ThesisSubjects/AnomalousCouplings/EventSelectionChoice_June2014/UpdatedPtCuts_30July2014/PtDistributionThirdJet_NewPtCuts_El.png}
\includegraphics[width = 0.45 \textwidth]{/home/annik/Documents/Vub/PhD/ThesisSubjects/AnomalousCouplings/EventSelectionChoice_June2014/UpdatedPtCuts_30July2014/PtDistributionFourthJet_NewPtCuts_El.png}
\caption{$P_T$ distributions for the first, second, third and fourth jet respectively. All four histograms are for the semi-electronic decay channel. (Information about the Cross Section and number of events should be ignored for the moment. Correct Cross Section value is not yet used ...) }
\end{figure}

\newpage
The same table can be created for the semi-muonic decay channel which shows a similar result.\\
\begin{table}
\caption{Event selection table before (left) and after (right) the pT cuts were updated to the ones suggested by the TOP reference twiki, for Semi-muon channel $t\bar{t}+jets$. ($19600.8~pb^{-1}$ of int. lumi.)}
\centering
\begin{tabular}{|c|c|c|c|c|}
\hline
				& \multicolumn{2}{|c|}{Old $p_T$ cuts}	& \multicolumn{2}{|c|}{New $p_T$ cuts}	\\
\hline
preselected			& 1.91516e+07	& 		 	& 1.91516e+07	&	 		\\

trigged				& 4.01265e+06	&	20.9 $\%$	& 4.01265e+06	&	20.9 $\%$	\\

Good PV				& 4.01265e+06	&	 100 $\%$	& 4.01265e+06	&	100 $\%$	\\

1 selected muon			& 3.46903e+06	&	 86.4 $\%$	& 3.40822e+06	&	\textcolor{red}{84.9 $\%$}	\\

Veto 2nd muon			& 3.46369e+06	&	 99.8 $\%$	& 3.40291e+06	&	99.8 $\%$	\\

Veto electron			& 3.45244e+06	&	 99.7 $\%$	& 3.39184e+06	&	99.7$\%$	\\

$\geq$ 1 jets			& 3.4281e+06	&	 99.3 $\%$	& 3.38698e+06	&	\textcolor{red}{99.9 $\%$}	\\

$\geq$ 2 jets			& 3.14456e+06	&	 91.7 $\%$	& 3.30209e+06	&	\textcolor{red}{97.5 $\%$}	\\

$\geq$ 3 jets			& 2.19456e+06	&	 69.8 $\%$	& 2.80066e+06	&	\textcolor{red}{84.8 $\%$}	\\

$\geq$ 4 jets			& 944572	&	 43.0 $\%$	& 1.65345e+06	&	\textcolor{red}{59.0 $\%$}	\\
\hline
\end{tabular}
\end{table}

\newpage
\subsection{Influence on choice of b-tag option and use of $\chi^{2}$ $m_{lb}$ - $m_{qqb}$ method}
With these new values considered for the event selection requirements, the obtained percentages and $\frac{s}{b}$ values should be compared again.
The values wich are obtained using these new $p_T$ cuts can be found in the following tables.\\
The first three tables show the results before the use of a $\chi^{2}$ $m_{lb}$ - $m_{qqb}$ method while the last two tables give the obtained numbers when this $\chi^{2}$ $m_{lb}$ - $m_{qqb}$ method is applied.\\
\\
Analyzing the numbers in this tables clearly indicates that the obtained results for the old $p_T$ cuts correspond to the values obtained earlier, which implies that the obtained values can easily be compared against each other and stil represent the same variables. The small differences which are however visible can be easily explained by statistical deviations.\\
Comparing the results obtained using the new $p_T$ cut values with the ones using the old cut values clearly indicates that the old values resulted in slightly better selection efficiency and $\frac{s}{b}$ value. Both for the selection efficiency of the b-quark jets and the light jets a higher percentage is found when using the old $p_T$ cut values. \\
A positive remark corresponding to the new $p_T$ cut values is that the behavior of the different considered b-tag options is similar. The 2 Tight b-tag case without any veto on the light jets results in the highest selection efficiency and $\frac{b}{s}$ value. Hence the choice of the used b-tag option does not need to be updated.\\
\\
Even the use of a $\chi^{2}$ $m_{lb}$ - $m_{qqb}$ method doesn't improve the selection efficiency and $\frac{s}{b}$ values when changing to the newer $p_T$ cut values. However compared to the previous case when no $\chi^{2}$ $m_{lb}$ - $m_{qqb}$ method is applied the difference between the two $p_T$ cut options becomes slightly less significant (from 18 $\%$ to 13 $\%$ difference). Still no improvement can be found impying that lowering the $p_T$ cut on the jets only increases the number of selected events but doesn't insures selection more good events, on the contrary the selection efficiencies decreases even.
\begin{landscape}
\begin{table}[!h] 
 \begin{tabular}{c|c|c|c|c|c|c} 
& \textbf{Option} (no $\chi^{2}$ $m_{lb}$) & all 4 correct & $\geq$ 1 wrong & correct ($\%$)       & $\frac{s}{b}$ & non-matched \\ \hline 
\multirow{6}{*}{\textbf{New $p_T$ cuts}} 
& L b-tags                & 31797 & 36132 & 46.8092 & 0.880023 & 103988  \\ 
&2 M b-tags               & 28380 & 21485 & 56.9137 & 1.32092  & 63717  \\ 
&2 M b-tags, light L-veto & 21578 & 18821 & 53.4122 & 1.14649  & 50549  \\ 
&2 T b-tags               & 17858 & 10684 & 62.5674 & 1.67147  & 33551  \\ 
&2 T b-tags, light M-veto & 16574 & 10496 & 61.2264 & 1.57908  & 31778  \\ 
&2 T b-tags, light L-veto & 12664 & 9552  & 57.004  & 1.3258   & 25440  \\
\hline
\multirow{6}{*}{\textbf{Old $p_T$ cuts}} 
& 2 L b-tags              & 15285 & 13577 & 52.9589 & 1.1258 & 61307  \\ 
& 2 M b-tags              & 13480 & 8590 & 61.0784 & 1.56927 & 39800  \\ 
& 2 M b-tags, light L-veto & 10718 & 7461 & 58.9581 & 1.43654 & 32048 \\ 
& 2 T b-tags              & 8555 & 4231 & 66.9091 & 2.02198 & 21234  \\ 
& 2 T b-tags, light M-veto & 7915 & 4097 & 65.8924 & 1.9319 & 19946  \\ 
& 2 T b-tags, light L-veto & 6327 & 3701 & 63.0933 & 1.70954 & 16156  \\ 
 \end{tabular} 
\caption{Overview of correct and wrong reconstructed events for the different b-tags without the use of a $\chi^{2}$ $m_{lb}$ - $m_{qqb}$ method} 
 \end{table} 
 
 \begin{table}[!h] 
 \begin{tabular}{c|c|c|c|c|c|c} 
&\textbf{Option} (no $\chi^{2}$ $m_{lb}$) & 2 b's correct & $\geq$ 1 b wrong & b's correct ($\%$) & $\frac{s}{b}$ & non-matched \\ \hline 
\multirow{6}{*}{\textbf{New $p_T$ cuts}} 
& 2 L b-tags              & 47926 & 20003 & 70.5531 & 2.39594 & 103988  \\ 
& 2 M b-tags              & 42045 & 7820 & 84.3177 & 5.3766 & 63717  \\ 
& 2 M b-tags, light L-veto & 34622 & 5777 & 85.7001 & 5.99308 & 50549  \\ 
& 2 T b-tags              & 26117 & 2425 & 91.5037 & 10.7699 & 33551  \\ 
& 2 T b-tags, light M-veto & 24974 & 2096 & 92.2571 & 11.9151 & 31778  \\ 
& 2 T b-tags, light L-veto & 20585 & 1631 & 92.6584 & 12.6211 & 25440  \\ 
\hline
\multirow{6}{*}{\textbf{Old $p_T$ cuts}} 
& 2 L b-tags              & 21456 & 7406 & 74.34 & 2.89711 & 61307  \\ 
& 2 M b-tags              & 18863 & 3207 & 85.469 & 5.88182 & 39800  \\ 
& 2 M b-tags, light L-veto & 15859 & 2320 & 87.238 & 6.83578 & 32048  \\ 
& 2 T b-tags              & 11811 & 975 & 92.3745 & 12.1138 & 21234  \\ 
& 2 T b-tags, light M-veto & 11203 & 809 & 93.2651 & 13.848 & 19946  \\ 
& 2 T b-tags, light L-veto & 9424 & 604 & 93.9769 & 15.6026 & 16156  \\ 
 \end{tabular} 
\caption{Overview of correct and wrong reconstructed b-jets for the different b-tags without the use of a $\chi^{2}$ $m_{lb}$ - $m_{qqb}$ method} 
 \end{table} 
 
\begin{table}[!h] 
\begin{tabular}{c|c|c|c|c|c|c} 
&\textbf{Option} (no $\chi^{2}$ $m_{lb}$) & 2 light good  & $\geq$ 1 light wrong & light correct ($\%$) & $\frac{s}{b}$ & non-matched \\ \hline 
\multirow{6}{*}{\textbf{New $p_T$ cuts}} 
& 2 L b-tags              & 33731 & 34198 & 49.6563 & 0.986344 & 103988  \\ 
& 2 M b-tags              & 28893 & 20972 & 57.9424 & 1.37769 & 63717  \\ 
& 2 M b-tags, light L-veto & 22118 & 18281 & 54.7489 & 1.20989 & 50549  \\ 
& 2 T b-tags              & 18030 & 10512 & 63.1701 & 1.71518 & 33551  \\ 
& 2 T b-tags, light M-veto & 16788 & 10282 & 62.017 & 1.63276 & 31778  \\ 
& 2 T b-tags, light L-veto & 12868 & 9348 & 57.9222 & 1.37655 & 25440  \\ 
\hline
\multirow{6}{*}{\textbf{Old $p_T$ cuts}} 
& 2 L b-tags              & 16015 & 12847 & 55.4882 & 1.24659 & 61307  \\ 
& 2 M b-tags              & 13688 & 8382 & 62.0208 & 1.63302 & 39800  \\ 
& 2 M b-tags, light L-veto & 10938 & 7241 & 60.1683 & 1.51056 & 32048  \\ 
& 2 T b-tags              & 8627 & 4159 & 67.4722 & 2.0743 & 21234  \\ 
& 2 T b-tags, light M-veto & 8005 & 4007 & 66.6417 & 1.99775 & 19946  \\ 
& 2 T b-tags, light L-veto & 6409 & 3619 & 63.911 & 1.77093 & 16156  \\ 
 \end{tabular} 
\caption{Overview of correct and wrong reconstructed light jets for the different b-tags without the use of a $\chi^{2}$ $m_{lb}$ - $m_{qqb}$ method} 
 \end{table}

 \begin{table}[!h] 
 \begin{tabular}{c|c|c|c|c|c|c} 
&\textbf{Option} (with $\chi^{2}$ $m_{lb}$) & all 4 correct & $\geq$ 1 wrong & 4 chosen jets correct ($\%$) & $\frac{s}{b}$ & non-matched \\ \hline 
\multirow{6}{*}{\textbf{New $p_T$ cuts}}
& 2 L b-tags              & 24611 & 43318 & 77.4004 & 0.568147 & 103988 \\ 
& 2 M b-tags              & 21813 & 28052 & 76.8605 & 0.777592 & 63717 \\ 
& 2 M b-tags, light L-veto & 16731 & 23668 & 77.5373 & 0.706904 & 50549 \\ 
& 2 T b-tags              & 13681 & 14861 & 76.6099 & 0.920598 & 33551 \\ 
& 2 T b-tags, light M-veto & 12674 & 14396 & 76.4692 & 0.880383 & 31778 \\ 
& 2 T b-tags, light L-veto & 9757 & 12459 & 77.0452 & 0.783129 & 25440 \\ 
\hline
\multirow{6}{*}{\textbf{Old $p_T$ cuts}} 
& 2 L b-tags              & 11893 & 16969 & 77.8083 & 0.700866 & 61307 \\ 
& 2 M b-tags              & 10422 & 11648 & 77.3145 & 0.894746 & 39800 \\ 
& 2 M b-tags, light L-veto & 8369 & 9810 & 78.0836 & 0.853109 & 32048 \\ 
& 2 T b-tags              & 6581 & 6205 & 76.9258 & 1.0606 & 21234 \\ 
& 2 T b-tags, light M-veto & 6080 & 5932 & 76.8162 & 1.02495 & 19946 \\ 
& 2 T b-tags, light L-veto & 4903 & 5125 & 77.4933 & 0.956683 & 16156 \\ 
 \end{tabular} 
 \caption{Overview of correct and wrong reconstructed events for the different b-tags when a $\chi^{2}$ $m_{lb}$ - $m_{qqb}$ method is applied} 
 \end{table} 
 
 \begin{table}[!h] 
 \begin{tabular}{c|c|c|c|c|c|c} 
& \textbf{Option} (with $\chi^{2}$ $m_{lb}$) & Correct b's & Wrong b's & \% b's correct   & $\frac{s}{b}$ & Correct option exists \\ \hline 
\multirow{6}{*}{\textbf{New $p_T$ cuts}}
& 2 L b-tags              & 28190 & 3607 & 88.6562 & 7.81536 & 31797 \\ 
& 2 M b-tags              & 25143 & 3237 & 88.5941 & 7.76738 & 28380 \\ 
& 2 M b-tags, light L-veto & 19183 & 2395 & 88.9007 & 8.0096 & 21578 \\ 
& 2 T b-tags              & 15736 & 2122 & 88.1174 & 7.41565 & 17858 \\ 
& 2 T b-tags, light M-veto & 14586 & 1988 & 88.0053 & 7.33702 & 16574 \\ 
& 2 T b-tags, light L-veto & 11178 & 1486 & 88.266 & 7.52221 & 12664 \\ 
\hline
\multirow{6}{*}{\textbf{Old $p_T$ cuts}} 
& 2 L b-tags              & 13935 & 1350 & 91.1678 & 10.3222 & 15285 \\ 
& 2 M b-tags              & 12306 & 1174 & 91.2908 & 10.4821 & 13480 \\ 
& 2 M b-tags, light L-veto & 9808 & 910 & 91.5096 & 10.778 & 10718 \\ 
& 2 T b-tags              & 7765 & 790 & 90.7656 & 9.82911 & 8555 \\ 
& 2 T b-tags, light M-veto & 7172 & 743 & 90.6128 & 9.65276 & 7915 \\ 
& 2 T b-tags, light L-veto & 5743 & 584 & 90.7697 & 9.8339 & 6327 \\ 
 \end{tabular} 
 \caption{Overview of the number of times the correct b-jet combination is chosen when using a $\chi^{2}$ $m_{lb}$ - $m_{qqb}$ method} 
 \end{table} 

\end{landscape}

%°°°°°°°°°°°°°°°°°°°°°°°°°°°°°°°°°°°°°°°°°°°°°°°°°°°°°°°°°°°°°°°°°°°°°°°°°°°°°°°°°°°°°°°°°°°°°°°°°°°°°°°°°°°°°°°°°°°°°°°°°°°°°°°°°°°°°°°°°°°°°°°°°°°°°°°°°°°°°°
\chapter{Event corrections and reconstruction}
%°°°°°°°°°°°°°°°°°°°°°°°°°°°°°°°°°°°°°°°°°°°°°°°°°°°°°°°°°°°°°°°°°°°°°°°°°°°°°°°°°°°°°°°°°°°°°°°°°°°°°°°°°°°°°°°°°°°°°°°°°°°°°°°°°°°°°°°°°°°°°°°°°°°°°°°°°°°°°°

\section{Which event corrections should be applied?}

\subsection{Trigger choice}
Two possible triggers exist, namely the so-called SingleLepton triggers or the CrossTriggers. The former kind only uses the information of the lepton while the latter one combines the lepton information together with the jets present in the event. These second kind of triggers were developed since it was expected that the $p_T$ cut which had to be applied on this SingleLepton trigger would become too high to be physically relevant. However it has been found that the applied $p_T$ cut doesn't differ that much between these two triggers, especially in the muon case the difference is almost negligible.\\
\\
Since the scale factors which should be applied are much more complex in the case of the CrossTriggers preference is given to these SingleLepton triggers.\\
Minor disadvantage of these kind of triggers is that no information can be found on the TOP Twiki page (\url{https://twiki.cern.ch/twiki/bin/viewauth/CMS/TopTrigger}) and currently no other Twiki page has been found with similar information ... Trigger contacts of TOP group prefer the use of the CrossTriggers since a lot of time and effort has been put into the development of these triggers. Probably the reason why no clear documentation is found on the Top Twiki page.\\
But James is using the same triggers in his analysis so all these triggers, and the different versions, can be found in the following two analyzer files:
\begin{itemize}
 \item \url{https://github.com/TopBrussels/FourTops/blob/master/FourTop_EventSelection.cc} which contains information about the muon event selection and used triggers.
 \item \url{https://github.com/TopBrussels/FourTops/blob/master/FourTop_EventSelection_El.cc} which contains the same information but about the electron case.
\end{itemize}

\subsection{Lumi- or PileUp Reweighting}
The choice of the root file which should be used for the LumiReweighting is related to the Monte Carlo samples which will be used. If the Summer 12 MC samples, which describe the 8 TeV data, are used the \textit{S10} file should be used.\\
LumiReweighting is actually taking into account the influence of PileUp but can't be called this way when publishing results since a systematic influence of PileUp is no physical variable. However the luminosity of MinimumBias events is. This systematic influence can be calculated by comparing the influence of the up and down part.

\subsection{Lepton Scale Factors}
Lepton scale factors should be used and can be found in the code of James.\\
In the muon case the scale factors correspond to the entire dataset (ABCD) since different scale factors have been obtained for the different runs. Therefore caution should be applied when only running over a limited range of data in order to avoid to run only over the A part. However the influence shouldn't be too large ...\\
\\
In the electron case the scale factors have been hard coded and do not vary for different parts of the data sample. \textbf{Therefore it should be checked whether the hard-coded numbers which are used in James code are still the correct ones which should be used. Hence a twiki with this information should be found ...}

\subsection{Jet Energy Correction factors}
Information available in the analyzer of James is up-to-date and can be copied.


\subsection{Jet corrections (on the fly ...)}
In James analyzer code Jet corrections are applied on the fly which is no problems, since even if the considered sample has these corrections already applied, the method of incorporating these jet corrections uses the gen information. So this avoids that these jet corrections would be applied twice ...
%**************************************************

\section{Choice of Monte Carlo samples}
Same samples as James can be used, which are the latest Jan22 rereco samples. A full overview of these samples can be found on:
\url{https://docs.google.com/spreadsheet/ccc?key=0Apc0aJdnaVjSdFVaLVU2dlk4RDZHcjlaakE3NWIxTUE&usp=sharing_eil#gid=0}

 
\end{document}
