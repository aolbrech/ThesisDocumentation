From analyzing the result obtained after applying the double-fit procedure, it was clear that a significant discrepancy occurs between the generator-level result using the CMS Monte Carlo samples and the self-simulated MG samples.
This was clearly visible in Figure~\ref{fig::ChiSqOnLL} which shows the expected location of the minimum for the CMS generator-level samples but a completely wrong distribution for the MG sample created with $\VR$ $=$ $0.0$. Hence from this can be concluded that the created FeynRules model does not correspond with the actual ``Standard Model'' of MadGraph and more investigation in the matter is necessary.
\\

In order to be sure that the issue is caused by the created FeynRules model an additional test has been performed. For this a personal MG sample has been created using the ``Standard Model'' model existing in MadGraph, and also used within CMS, which was processed in the same way by MadWeight and the analysis scripts.
This sample resulted in almost exactly the same distributions as was obtained using the sample created with the personally created AnomCoup model, hence a minimum differing from the expected position. The distributions for this sample are given in Figure~\ref{fig::MGSampleLL}.
\begin{figure}[h!t]
 \centering
 \includegraphics[width = 0.49 \textwidth]{/home/annik/Documents/Vub/PhD/ThesisSubjects/AnomalousCouplings/June2015_SMComparisonMGModels/SecondPolAdd_Summed_SMMadGraphModel.pdf}
 \includegraphics[width = 0.49 \textwidth]{/home/annik/Documents/Vub/PhD/ThesisSubjects/AnomalousCouplings/June2015_SMComparisonMGModels/LLComparison_ChiSqCutsMGSample_RVR_10000Evts.pdf}
 \caption{$\loglik$ distribution obtained when the MadGraph sample is created using the SM FeynRules model}
 \label{fig::MGSampleLL}
\end{figure}

However the right-handed figure shows that the application of the $\chisq$ cut in a very tight way, removing about $15\%$ of the events, repositions the overall likelihood distribution to have a minimum at the correct position. This seems to suggest that in the CMS sample the few event selection cuts which have been applied probably influence the $\chisq$ distribution of the fit.
The percentages of event kept after applying the tighter cuts can be found in Table~\ref{table::tightChiSqMG}.
\begin{table}[h!t]
 \centering
 \caption{Number of selected events after applying the different $\chisq$-cuts considered.} 
 \label{table::tightChiSqMG}
 \begin{tabular}{c|c|c|c}
  Sample 								& $\chisq$ $<$ 0.0002 	& $\chisq$ $<$ 0.00001 	& $\chisq$ $<$ 0.000005 	\\
  \hline
  MG SM-model in $\left[-0.5, -0.3, -0.2, ..., 0.3, 0.5 \right]$ 	& 98.69 $\%$		& 84.46 $\%$		& 76.28 $\%$		
 \end{tabular}
\end{table}

So as can be understood from the obtained results, there seems to be no real difference between the two different FeynRules models. This is again summarized in Figure~\ref{fig::FRModelComp} where some of the more important kinematic properties are shown together for the two models.
\begin{figure}[h!t]
 \centering
 \includegraphics[width = 0.49 \textwidth]{/home/annik/Documents/Vub/PhD/ThesisSubjects/AnomalousCouplings/June2015_SMComparisonMGModels/SMComparison_CosThetaDist.pdf}
 \includegraphics[width = 0.49 \textwidth]{/home/annik/Documents/Vub/PhD/ThesisSubjects/AnomalousCouplings/June2015_SMComparisonMGModels/SMComparison_LeptMassDist.pdf}
 \includegraphics[width = 0.49 \textwidth]{/home/annik/Documents/Vub/PhD/ThesisSubjects/AnomalousCouplings/June2015_SMComparisonMGModels/SMComparison_TopQuarkMassDist.pdf}
 \includegraphics[width = 0.49 \textwidth]{/home/annik/Documents/Vub/PhD/ThesisSubjects/AnomalousCouplings/June2015_SMComparisonMGModels/SMComparison_bQuarkPtDist.pdf} 
 \caption{Selected distributions from the comparison between the SM FeynRules model and the AnomCoup one.}
 \label{fig::FRModelComp}
\end{figure}

Hence it seems that the difference originates from the light pre-selection which is applied for the CMS Monte Carlo $t\bar{t}$ sample and not for the personally created MadGraph samples. Hence it should be investigated whether the influence of the event selection actually has a positive effect on the shape of the $\loglik$ distribution, and not a negative one as was thought in the beginning.
\\
%But first, a final closure is given in Figure~\ref{fig::LLBiasTightCut} containing the difference between the expected minimum and the calculated minimum for MadGraph samples created with different $\VR$-values.
%\\

However the application of such a tight $\chisq$-cut clearly improved the obtained $\loglik$ shape for the Standard Model case, it seems to worsen the result for different $\VR$ values.
The influence of the $\chisq$ cut on the $\loglik$ shape for two $\VR$ values which were actually good for less tight cuts is given in Figure~\ref{fig::BadLLTight}.
\begin{figure}[h!t]
 \centering
 \includegraphics[width = 0.49 \textwidth]{/home/annik/Documents/Vub/PhD/ThesisSubjects/AnomalousCouplings/June2015_BiasTest/InfluenceTighterCuts/LLComparison_ChiSqCutsMGSample_RVR_10000Evts_CreatedWithNeg05.pdf}
 \includegraphics[width = 0.49 \textwidth]{/home/annik/Documents/Vub/PhD/ThesisSubjects/AnomalousCouplings/June2015_BiasTest/InfluenceTighterCuts/LLComparison_ChiSqCutsMGSample_RVR_10000Evts_CreatedWithPos03.pdf}
 \caption{$\loglik$ distribution for two MadGraph samples which actually had a good agreement with the expected shape when looser $\chisq$ cuts are applied, but become completely distorted in the case of the very tight cuts. The left one corresponds to a MG sample created with $\VR$ $=$ $-0.5$ and the right one with $\VR$ $=$ $0.3$.}
 \label{fig::BadLLTight}
\end{figure}

So from this small study can be decided that the problem which occurs for the MadGraph generator-level events is not caused by a wrong ``Standard Model'' implementation but can (hopefully) be explained by the minor event selection which is applied for the CMS generator-level sample.
Hence the next study will discuss the influence of the $\pT$-cuts  on the obtained $\loglik$ distribution.