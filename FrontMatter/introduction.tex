\chapter*{Introduction\markboth{\MakeUppercase{Introduction}}{}}

Achieving insight in the most fundamental aspects of our universe requires a detailed knowledge of the building blocks of all matter.
Hence a thorough understanding of all the fundamental particles and interactions amongst them is of crucial importance.
These concepts are studied meticulously in the field of elementary particle physics and have been summarised in a theory denoted the Standard Model.
This theory has been experimentally verified with great precision and all obtained result are in excellent agreement with the provided predictions.
One of the more outstanding achievements of the Standard Model was realised in the summer of 2012 with the observation of the Brout-Englert-Higgs boson, predicted in 1964.
Even though the Standard Model can be considered as a very successful theory, it still has a couple of short-comings and can not be considered as a ``theory of everything''. 
Hence one should continue to experimentally confirm the various predictions made by the Standard Model.

In order to achieve the challenging conditions required to fully test the Standard Model, high-energetic particle accelerators such as the Large Hadron Collider at CERN near Geneva are constructed.
This is a $27\km$ long circular particle accelerator designed to deliver proton-proton collisions at a record-breaking centre-of-mass energy of $14 \TeV$.
For the measurement discussed in this thesis, the considered top-quark pair events were produced during the 2012 run, which operated at a centre-of-mass energy of $8\TeV$, and were collected by the Compact Muon Solenoid experiment.

In this thesis, such a consistency test of the Standard Model will be performed since it will be investigated whether the interaction vertex of the top-quark pair decay corresponds with the predicted left-handed description and is not influenced by new-physics phenomena.
This will be achieved by directly measuring the best estimate of one of the coupling coefficients in this interaction vertex using a Matrix Element method.
%This requires to measure one of the actual coupling coefficients in this interaction vertex

In Chapter~\ref{chp::SM} an overview of the theoretical framework of the Standard Model will be given together with the specific details related to the top-quark interaction vertex.
%The measurement discussed in this thesis has been performed using $pp$ collisions provided by the Large Hadron Collider and recorded by the CMS detector, both located at CERN. 
The experimental setup of the CERN accelerator complex will be discussed in Chapter~\ref{chp:CERN}.
An overview of the generation and simulation of proton-proton collisions is given in Chapter~\ref{chp:SimReco} together with the specific reconstruction algorithms. These algorithms describe how the electronic signals of the CMS detector are translated into actual physical objects such as electrons and muons.
The selection of top-quark events and the reconstruction of the desired event topology is discussed in Chapter~\ref{ch::EvtSel}.
In Chapter~\ref{ch::MW} the technique use to measure the right-handed tensor coupling of the Wtb interaction is introduced and the necessary normalisations and calibrations required to correctly apply this technique are discussed.
In Chapter~\ref{ch::Analysis} the actual measurement is performed using this Matrix Element method and the relevant uncertainties are discussed.
Finally in Chapter~\ref{ch::Concl} the agreement of the obtained result with both the theoretical expectation and the currently available exclusion limits is given.