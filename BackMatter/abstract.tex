\chapter*{Abstract}

%The Large Hadron Collider is the first particle accelerator which is capable of providing an abundant amount of top-quark pair events at record-breaking centre-of-mass energies, allowing for a thorough investigation of physics at the TeV scale.
%The data collected by the CMS detector during the 2012 run of the LHC is considered in this thesis and the event signature of interest corresponds to the semi-muonic decay of top-quark pair events.
%These events have been used to determine whether the top-quark decay vertex is described purely by a left-handed vector coupling, as predicted by the Standard Model of elementary particle physics, or whether additional couplings occur due to new-physics phenomena. In this thesis the focus was on the estimation of the right-handed tensor coupling, denoted as $\gR$.


The study discussed in this thesis is the first direct measurement of the right-handed anomalous tensor coupling of the top-quark decay vertex.
This has been performed using proton-proton collisions produced by the Large Hadron Collider at a centre-of-mass energy of 8 TeV during the 2012 run and collected by the Compact Muon Solenoid experiment, both located at CERN near Geneva.
The measurement itself uses a Matrix Element method, which is has the potential of extracting the best estimate of any theoretical parameter from a sample of experimental events.
%This advanced analysis technique evaluates each event and calculate a corresponding event probability using a dedicated phase-space integration.
%As a result, applying such a Matrix Element method requires a significant computing time such that 
%It has been opted to apply a stringent event selection by for instance exploiting the specific characteristics of the b-quark jets in order to improve the reconstruction efficiency.
%Two such jets are expected in top-quark pair events since the top-quark decays almost exclusively into a b-quark and a W-boson. 
%Since this reduces the number of selected events that will be processed by the phase-space integrator the required processing time is also seriously decreased.
%In addition, the number of selected events used in the analysis are reduced implying that also the 
With $19.6 \fbI$ of collision data a right-handed tensor coupling value of
\begin{equation}
 \gR = -0.0071 \, \pm \, 0.0083 \, (\textrm{stat.}) \, \pm \, 0.0134  \, (\textrm{syst.}) = -0.0071 \, \pm \, 0.0160 \nonumber
\end{equation}
was measured, which is consistent with the prediction of the Standard Model ($\gR$ = 0).
Comparing the obtained result with the currently existing exclusion limits for this right-handed tensor coupling also indicates an excellent agreement.
%Hence it can be concluded that the Matrix Element method has been successfully applied on the reconstructed collision events recorded by the CMS experiment and has resulted in a rather accurate and first direct measurement of this anomalous coupling coefficient $\gR$.
\\
\\
De studie uitgevoerd in deze thesis is de eerste meting van de rechtshandige anomalous tensor koppeling in de top-quark interactie vertex die via een rechtstreekse methode is gebeurd.
Voor deze studie is gebruik gemaakt van proton-proton botsingen die geproduceerd werden door de Large Hadron Collider in 2012 bij een energie van 8 TeV en gedetecteerd werden door het Compact Muon Solenoid experiment, beide te CERN, Geneve.
Voor deze meting is een Matrix Element method gebruikt geworden, die het potentieel heeft om de best mogelijke schatter te bekomen voor elke theoretische parameter vertrekkende van een verzameling experimentele gebeurtenissen.
Voor een totaal van $19.6 \fbI$ aan botsingsgegevens werd een waarde voor de rechtshandige anomalous tensor koppeling van
\begin{equation}
 \gR = -0.0071 \, \pm \, 0.0083 \, (\textrm{stat.}) \, \pm \, 0.0134  \, (\textrm{syst.}) = -0.0071 \, \pm \, 0.0160 \nonumber
\end{equation}
gemeten, wat overeenkomt met de voorspelling van het Standaard Model ($\gR$ = 0).
Ook is deze waarde vergeleken geworden met de bestaande uitsluitings limieten voor deze rechtshandige tensor koppeling, wat wederom een consistent resultaat opleverde.
